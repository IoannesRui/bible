% 创建日期:2019‎年‎4‎月‎18‎日,‏‎15:31:24
% 书籍类型,默认字体大小为11pt,纸张大小为A5,单面打印
\documentclass[11pt, a5paper, oneside]{ctexbook} 

% TODO	
% 导入fonts文件夹下的字体,以解决不同操作系统下的字体问题。用到了宋体和楷体,也要考虑加粗字体的问题。也可以参考:https://yangyq.net/2022/05/set-font-in-latex.html

% 中文宏包
\usepackage{ctex}
% 图文混排,分栏使用了这个宏包
\usepackage{multicol}
% 使用\ding{*}命令设置上标
\usepackage{pifont}
% 超链接
\usepackage{hyperref}
% 使用\uline{*}命令增加下划线,并能自动换行
\usepackage{ulem}
\usepackage{CJKulem} 
% 跨页表格
\usepackage{longtable}
% 脚注每页重新编号
\usepackage[perpage]{footmisc} 
% 页边距
\usepackage{geometry}
\usepackage{CJK}
% 定制页面页眉和页脚
\usepackage{fancyhdr}
\geometry{left=1.2cm, right=1cm, top=1.6cm, bottom=1cm}
% 行间距
\linespread{1.3541667}
% 设置段间距和行间距相等
\setlength{\parskip}{0pt}
% 标题
\title{\kaishu \Huge 聖經}

\begin{document}
	
% 显示标题
\maketitle

% 将页码格式设置为大写罗马数字
\pagenumbering{Roman}
% 页码从1开始
\setcounter{page}{1}

% 页眉
\pagestyle{fancy}
% 去掉页眉下划线
\renewcommand{\headrulewidth}{0pt}
% 中间页眉显示页码
\chead{\thepage}
% \chapter 页面显示页眉
\fancypagestyle{plain}{
	\pagestyle{fancy}
}
% 取消页脚
\cfoot{}

% 换行控制
\newcommand{\myspace}[1]{\par\vspace{#1\baselineskip}}

% 前言
% 标题
\chapter*{前\quad言}

为了满足广大神长教友的需求,经中国天主教主教团批准,现将旧、新约合订本《圣经》印制发行。并再次感谢香港思高圣经学会的帮助。

《圣经》是一部在天主圣神默示下写成的圣书,天主的话是圣教会及众信友的生命泉源,是不可缺少的精神食粮,愿神长教友在圣教会的训导下,虔诚地、认真地阅读、默想《圣经》,聆听基督的圣训,在爱主爱人的道路上不断圣化自己,为主作证。

\rightline{中国天主教主教团}
\rightline{二〇〇九年二月}
\clearpage

% 序
% 标题
\chapter*{序}

% 右页眉
\rhead{序}

% TODO 正文

序序序序序序序序序序序序序序序序序序序序序序序序序序序序序序序序序序序序序序序序序序序序序序序序序序序序序序序序序序序序序序序序序序序序序序序序序序序序序序序序序序序序序序序序序序序序序序序序序序序序序序序序序序序序序序序序序序序序序序序序序序序序序序序序序序序序序序序序序序序序序序序序序序序序序序序序序序序序序序序序序序序序序序序序序序序序序序序序序序序序序序序序序序序序序序序序序序序序序序序序序序序序序序序序序序序序序序序序序序序序序序序序序序序序序序序序序序序序序序序序序序序序序序序序序序序序序序序序序序序序序序序序序序序序序序序序序序序序序序序序序序序序序序序序序序序序序序序序序序序序序序序序序序序序序序序序序序序序序序序序序序序序序序序序序序序序序序序序序序序序序序序序序序序序序序序序序序序序序序序序序序序序序序序序序序序序序序序序序序序序序序序序序序序序序序序序序序序序序序序序序序序序序序序序序序序序序序序序序序序序序序序序序序序序序序序序序序序序序序序序序序序序序序序序序序序序序序序序序序序序序序序序序序序序序序序序序序序序序序序序序序序序序序序序序序序序序序序序序序序序序序序序序序序序序序序序序序序序序序序序序序序序序序序序序序序序序序序序序序序序序序序序序序序序序序序序序序序序序序序序序序序序序序序序序序序序序序序序序序序序序序序序序序序序序序序序序序序序序序序序序序序序序序序序序序序序序序序序序序序序序序序序序序序序序序序序序序序序序序序序序序序序序序序序序序序序序序序序序序序序序序序序序序序序序序序序序序序序序序序序序序序序序序序序序序序序序序序序序序序序序序序序序序序序序序序序序序序序序序序序序序序序序序序序序序序序序序序序序序序序序序序序序序序序序序序序序序序序序序序序序序序序序序序序序序序序序序序序序序序序序序序序序序序序序序序序序序序序序序序序序序序序序序序序序序序序序序序序序序序序序序序序序序序序序序序序序序序序序序序序序序序序序序序序序序序序序序序序序序序序序序序序序序序序序序序序序序序序序序序序序序序序序序序序序序序序序序序序序序序序序序序序序序序序序序序序序序序序序序序序序序序序序序序序序序序序序序序序序序序序序序序序序序序序序序序序序序序序序序序序序序序序序序序序序序序序序序序序序序序序序序序序序序序序序序序序序序序序序序序序序序序序序序序序序序序序序序序序序序序序序序序序序序序序序序序序序序序序序序序序序序序序序序序序序序序序序序序序序序序序序序序序序序序序序序序序序序序序序序序序序序序序序序序序序序序序序序序序序序序序序序序序序序序序序序序序序序序序序序序序序序序序序序序序序序序序序序序序序序序序序序序序序序序序序序序序序序序序序序序序序序序序序序序序序序序序序序序序序序序序序序序序序序序序序序序序序序序序序序序序序序序序序序序序序序序序序序序序序序序序序序序序序序序序序序序序序序序序序序序序序序序序序序序序序序序序序序序序序序序序序序序序序序序序序序序序序序序序序序序序序序序序序序序序序序序序序序序序序序序序序序序序序序序序序序序序序序序序序序序序序序序序序序序序序序序序序序序序序序序序序序序序序序序序序序序序序序序序序序序序序序序序序序序序序序序序序序序序序序序序序序序序序序序序序序序序序序序序序序序序序序序序序序序序序序序序序序序序序序序序序序序序序序序序序序序序序序序序序序序序序序序序序序序序序序序序序序序序序序序序序序序序序序序序序序序序序序序序序序序序序序序序序序序序序序序序序序序序序序序序序序序序序序序序序序序序序序序序序序序序序序序序序序序序序序序序序序序序序序序序序序序序序序序序序序序序序序序序序序序序序序序序序序序序序序序序序序序序序序序序序序序序序序序序序序序序序序序序序序序序序序序序序序序序序序序序序序序序序序序序序序序序序序序序序序序序序序序序序序序序序序序序序序序序序序序序序序序序序序序序序序序序序序序序序序序序序序序序序序序序序序序序序序序序序序序序序序序序序序序序序序序序序序序序序序序序序序序序序序序序序序序序序序序序序序序序序序序序序序序序序序序序序序序序序序序序序序序序序序序序序序序序序序序序序序序序序序序序序序序序序序序序序序序序序序序序序序序序序序序序序序序序序序序序序序序序序序序序序序序序序序序序序序序序序序序序序序序序序序序序序序序序序序序序序序序序序序序序序序序序序序序序序序序序序序序序序序序序序序序序序序序序序序序序序序序序序序序序序序序序序序序序序序序序序序序序序序序序序序序序序序序序序序序序序序序序序序序序序序序序序序序序序序序序序序序序序序序序序序序序序序序序序序序序序序序序序序序序序序序序序序序序序序序序序序序序序序序序序序序序序序序序序序序序序序序序序序序序序序序序序序序序序序序序序序序序序序序序序序序序序序序序序序序序序序序序序序序序序序序序序序序序序序序序序序序序序序序序序序序序序序序序序序序序序序序序序序序序序序序序序序序序序序序序序序序序序序序序序序序序序序序序序序序序序序序序序序序序序序序序序序序序序序序序序序序序序序序序序序序序序序序序序序序序序序序序序序序序序序序序序序序序序序序序序序序序序序序序序序序序序序序序序序序序序序序序序序序序序序序序序序序序序序序序序序序序序序序序序序序序序序序序序序序序序序序序序序序序序序序序序序序序序序序序序序序序序序序序序序序序序序序序序序序序序序序序序序序序序序序序序序序序序序序序序序序序序序序序序序序序序序序序序序序序序序序序序序序序序序序序序序序序序序序序序序序序序序序序序序序序序序序序序序序序序序序序序序序序序序序序序序序序序序序序序序序序序序序序序序序序序序序序序序序序序序序序序序序序序序序序序序序序序序序序序序序序序序序序序序序序序序序序序序序序序序序序序序序序序序序序序序序序序序序序序序序序序序序序序序序序序序序序序序序序序序序序序序序序序序序序序序序序序序序序序序序序序序序序序序序序序序序序序序序序序序序序序序序序序序序序序序序序序序序序序序序序序序序序序序序序序序序序序序序序序序序序序序序序序序序序序序序序序序序序序序序序序序序序序序序序序序序序序序序序序序序序序序序序序序序序序序序序序序序序序序序序序序序序序序序序序序序序序序序序序序序序序序序序序序序序序序序序序序序序序序序序序序序序序序序序序序序序序序序序序序序序序序序序序序序序序序序序序序序序序序序序序序序序序序序序序序序序序序序序序序序序序序序序序序序序序序序序序序序序序序序序序序序序序序序序序序序序序序序序序序序序序序序序序序序序序序序序序序序序序序序序序序序序序序序序序序序序序序序序序序序序序序序序序序序序序序序序序序序序序序序序序序序序序序序序序序序序序序序序序序序序序序序序
\clearpage

% 凡例
% 标题
\chapter*{凡\quad例}

% 右页眉
\rhead{凡\quad例}

% TODO 正文

凡\quad例凡\quad例凡\quad例凡\quad例凡\quad例凡\quad例凡\quad例凡\quad例凡\quad例凡\quad例凡\quad例凡\quad例凡\quad例凡\quad例凡\quad例凡\quad例凡\quad例凡\quad例凡\quad例凡\quad例凡\quad例凡\quad例凡\quad例凡\quad例凡\quad例凡\quad例凡\quad例凡\quad例凡\quad例凡\quad例凡\quad例凡\quad例凡\quad例凡\quad例凡\quad例凡\quad例凡\quad例凡\quad例凡\quad例凡\quad例凡\quad例凡\quad例凡\quad例凡\quad例凡\quad例凡\quad例凡\quad例凡\quad例凡\quad例凡\quad例凡\quad例凡\quad例凡\quad例凡\quad例凡\quad例凡\quad例凡\quad例凡\quad例凡\quad例凡\quad例凡\quad例凡\quad例凡\quad例凡\quad例凡\quad例凡\quad例凡\quad例凡\quad例凡\quad例凡\quad例凡\quad例凡\quad例凡\quad例凡\quad例凡\quad例凡\quad例凡\quad例凡\quad例凡\quad例凡\quad例凡\quad例凡\quad例凡\quad例凡\quad例凡\quad例凡\quad例凡\quad例凡\quad例凡\quad例凡\quad例凡\quad例凡\quad例凡\quad例凡\quad例凡\quad例凡\quad例凡\quad例凡\quad例凡\quad例凡\quad例凡\quad例凡\quad例凡\quad例凡\quad例凡\quad例凡\quad例凡\quad例凡\quad例凡\quad例凡\quad例凡\quad例凡\quad例凡\quad例凡\quad例凡\quad例凡\quad例凡\quad例凡\quad例凡\quad例凡\quad例凡\quad例凡\quad例凡\quad例凡\quad例凡\quad例凡\quad例凡\quad例凡\quad例凡\quad例凡\quad例凡\quad例凡\quad例凡\quad例凡\quad例凡\quad例凡\quad例凡\quad例凡\quad例凡\quad例凡\quad例凡\quad例凡\quad例凡\quad例凡\quad例凡\quad例凡\quad例凡\quad例凡\quad例凡\quad例凡\quad例凡\quad例凡\quad例凡\quad例凡\quad例凡\quad例凡\quad例凡\quad例凡\quad例凡\quad例凡\quad例凡\quad例凡\quad例凡\quad例凡\quad例凡\quad例凡\quad例凡\quad例凡\quad例凡\quad例凡\quad例凡\quad例凡\quad例凡\quad例凡\quad例凡\quad例凡\quad例凡\quad例凡\quad例凡\quad例凡\quad例凡\quad例凡\quad例凡\quad例凡\quad例凡\quad例凡\quad例凡\quad例凡\quad例凡\quad例凡\quad例凡\quad例凡\quad例凡\quad例凡\quad例凡\quad例凡\quad例凡\quad例凡\quad例凡\quad例凡\quad例凡\quad例凡\quad例凡\quad例凡\quad例凡\quad例凡\quad例凡\quad例凡\quad例凡\quad例凡\quad例凡\quad例凡\quad例凡\quad例凡\quad例凡\quad例凡\quad例凡\quad例凡\quad例凡\quad例凡\quad例凡\quad例凡\quad例凡\quad例凡\quad例凡\quad例凡\quad例凡\quad例凡\quad例凡\quad例凡\quad例凡\quad例凡\quad例凡\quad例凡\quad例凡\quad例凡\quad例凡\quad例凡\quad例凡\quad例凡\quad例凡\quad例凡\quad例凡\quad例凡\quad例凡\quad例凡\quad例凡\quad例凡\quad例凡\quad例凡\quad例凡\quad例凡\quad例凡\quad例凡\quad例凡\quad例凡\quad例凡\quad例凡\quad例凡\quad例凡\quad例凡\quad例凡\quad例凡\quad例凡\quad例凡\quad例凡\quad例凡\quad例凡\quad例凡\quad例凡\quad例凡\quad例凡\quad例凡\quad例凡\quad例凡\quad例凡\quad例凡\quad例凡\quad例凡\quad例凡\quad例凡\quad例凡\quad例凡\quad例凡\quad例凡\quad例凡\quad例凡\quad例凡\quad例凡\quad例凡\quad例凡\quad例凡\quad例凡\quad例凡\quad例凡\quad例凡\quad例凡\quad例凡\quad例凡\quad例凡\quad例凡\quad例凡\quad例凡\quad例凡\quad例凡\quad例凡\quad例凡\quad例凡\quad例凡\quad例凡\quad例凡\quad例凡\quad例凡\quad例凡\quad例凡\quad例凡\quad例凡\quad例凡\quad例凡\quad例凡\quad例凡\quad例凡\quad例凡\quad例凡\quad例凡\quad例凡\quad例凡\quad例凡\quad例凡\quad例凡\quad例凡\quad例凡\quad例凡\quad例凡\quad例凡\quad例凡\quad例凡\quad例凡\quad例凡\quad例凡\quad例凡\quad例凡\quad例凡\quad例凡\quad例凡\quad例凡\quad例凡\quad例凡\quad例凡\quad例凡\quad例凡\quad例凡\quad例凡\quad例凡\quad例凡\quad例凡\quad例凡\quad例凡\quad例凡\quad例凡\quad例凡\quad例凡\quad例凡\quad例凡\quad例凡\quad例凡\quad例凡\quad例凡\quad例凡\quad例凡\quad例凡\quad例凡\quad例凡\quad例凡\quad例凡\quad例凡\quad例凡\quad例凡\quad例凡\quad例凡\quad例凡\quad例凡\quad例凡\quad例凡\quad例凡\quad例凡\quad例凡\quad例凡\quad例凡\quad例凡\quad例凡\quad例凡\quad例凡\quad例凡\quad例凡\quad例凡\quad例凡\quad例凡\quad例凡\quad例凡\quad例凡\quad例凡\quad例凡\quad例凡\quad例凡\quad例凡\quad例凡\quad例凡\quad例凡\quad例凡\quad例凡\quad例凡\quad例凡\quad例凡\quad例凡\quad例凡\quad例凡\quad例凡\quad例凡\quad例凡\quad例凡\quad例凡\quad例凡\quad例凡\quad例凡\quad例凡\quad例凡\quad例凡\quad例凡\quad例凡\quad例凡\quad例凡\quad例凡\quad例凡\quad例凡\quad例凡\quad例凡\quad例凡\quad例凡\quad例凡\quad例凡\quad例凡\quad例凡\quad例凡\quad例凡\quad例凡\quad例凡\quad例凡\quad例凡\quad例凡\quad例凡\quad例凡\quad例凡\quad例凡\quad例凡\quad例凡\quad例凡\quad例凡\quad例凡\quad例凡\quad例凡\quad例凡\quad例凡\quad例凡\quad例凡\quad例凡\quad例凡\quad例凡\quad例凡\quad例凡\quad例凡\quad例凡\quad例凡\quad例凡\quad例凡\quad例凡\quad例凡\quad例凡\quad例凡\quad例凡\quad例凡\quad例凡\quad例凡\quad例凡\quad例凡\quad例凡\quad例凡\quad例凡\quad例凡\quad例凡\quad例凡\quad例凡\quad例凡\quad例凡\quad例凡\quad例凡\quad例凡\quad例凡\quad例凡\quad例凡\quad例凡\quad例凡\quad例凡\quad例凡\quad例凡\quad例凡\quad例凡\quad例凡\quad例凡\quad例凡\quad例凡\quad例凡\quad例凡\quad例凡\quad例凡\quad例凡\quad例凡\quad例凡\quad例凡\quad例凡\quad例凡\quad例凡\quad例凡\quad例凡\quad例凡\quad例凡\quad例凡\quad例凡\quad例凡\quad例凡\quad例凡\quad例凡\quad例凡\quad例凡\quad例凡\quad例凡\quad例凡\quad例凡\quad例凡\quad例凡\quad例凡\quad例凡\quad例凡\quad例凡\quad例凡\quad例凡\quad例凡\quad例凡\quad例凡\quad例凡\quad例凡\quad例凡\quad例凡\quad例凡\quad例凡\quad例凡\quad例凡\quad例凡\quad例凡\quad例凡\quad例凡\quad例凡\quad例凡\quad例凡\quad例凡\quad例凡\quad例凡\quad例凡\quad例凡\quad例凡\quad例凡\quad例凡\quad例凡\quad例凡\quad例凡\quad例凡\quad例凡\quad例凡\quad例凡\quad例凡\quad例凡\quad例凡\quad例凡\quad例凡\quad例凡\quad例凡\quad例凡\quad例凡\quad例凡\quad例凡\quad例凡\quad例凡\quad例凡\quad例凡\quad例凡\quad例凡\quad例凡\quad例凡\quad例凡\quad例凡\quad例凡\quad例凡\quad例凡\quad例凡\quad例凡\quad例凡\quad例凡\quad例凡\quad例凡\quad例凡\quad例凡\quad例凡\quad例凡\quad例凡\quad例凡\quad例凡\quad例凡\quad例凡\quad例凡\quad例凡\quad例凡\quad例凡\quad例凡\quad例凡\quad例凡\quad例凡\quad例凡\quad例凡\quad例凡\quad例凡\quad例凡\quad例凡\quad例凡\quad例凡\quad例凡\quad例凡\quad例凡\quad例凡\quad例凡\quad例凡\quad例凡\quad例凡\quad例凡\quad例凡\quad例凡\quad例凡\quad例凡\quad例凡\quad例凡\quad例凡\quad例凡\quad例凡\quad例凡\quad例凡\quad例凡\quad例凡\quad例凡\quad例凡\quad例凡\quad例凡\quad例凡\quad例凡\quad例凡\quad例凡\quad例凡\quad例凡\quad例凡\quad例凡\quad例凡\quad例凡\quad例凡\quad例凡\quad例凡\quad例凡\quad例凡\quad例凡\quad例凡\quad例凡\quad例凡\quad例凡\quad例凡\quad例凡\quad例凡\quad例凡\quad例凡\quad例凡\quad例凡\quad例凡\quad例凡\quad例凡\quad例凡\quad例凡\quad例凡\quad例凡\quad例凡\quad例凡\quad例凡\quad例凡\quad例凡\quad例凡\quad例凡\quad例凡\quad例凡\quad例凡\quad例凡\quad例凡\quad例凡\quad例凡\quad例凡\quad例凡\quad例凡\quad例凡\quad例凡\quad例凡\quad例凡\quad例凡\quad例凡\quad例凡\quad例凡\quad例凡\quad例凡\quad例凡\quad例凡\quad例凡\quad例凡\quad例凡\quad例凡\quad例凡\quad例凡\quad例凡\quad例凡\quad例凡\quad例凡\quad例凡\quad例凡\quad例凡\quad例凡\quad例凡\quad例凡\quad例凡\quad例凡\quad例凡\quad例凡\quad例凡\quad例凡\quad例凡\quad例凡\quad例凡\quad例凡\quad例凡\quad例凡\quad例凡\quad例凡\quad例凡\quad例凡\quad例凡\quad例凡\quad例凡\quad例凡\quad例凡\quad例凡\quad例凡\quad例凡\quad例凡\quad例凡\quad例凡\quad例凡\quad例凡\quad例凡\quad例凡\quad例凡\quad例凡\quad例凡\quad例凡\quad例凡\quad例凡\quad例凡\quad例凡\quad例凡\quad例凡\quad例凡\quad例凡\quad例凡\quad例凡\quad例凡\quad例凡\quad例凡\quad例凡\quad例凡\quad例凡\quad例凡\quad例凡\quad例凡\quad例凡\quad例凡\quad例凡\quad例凡\quad例凡\quad例凡\quad例凡\quad例凡\quad例凡\quad例凡\quad例凡\quad例凡\quad例凡\quad例凡\quad例凡\quad例凡\quad例凡\quad例凡\quad例凡\quad例凡\quad例凡\quad例凡\quad例凡\quad例凡\quad例凡\quad例凡\quad例凡\quad例凡\quad例凡\quad例凡\quad例凡\quad例凡\quad例凡\quad例凡\quad例凡\quad例凡\quad例凡\quad例凡\quad例凡\quad例凡\quad例凡\quad例凡\quad例凡\quad例凡\quad例凡\quad例凡\quad例凡\quad例凡\quad例凡\quad例凡\quad例凡\quad例凡\quad例凡\quad例凡\quad例凡\quad例凡\quad例凡\quad例凡\quad例凡\quad例凡\quad例凡\quad例凡\quad例凡\quad例凡\quad例凡\quad例凡\quad例凡\quad例凡\quad例凡\quad例凡\quad例凡\quad例凡\quad例凡\quad例凡\quad例凡\quad例凡\quad例凡\quad例凡\quad例凡\quad例凡\quad例凡\quad例凡\quad例凡\quad例凡\quad例凡\quad例凡\quad例凡\quad例凡\quad例凡\quad例凡\quad例凡\quad例凡\quad例凡\quad例凡\quad例凡\quad例凡\quad例凡\quad例凡\quad例凡\quad例凡\quad例凡\quad例凡\quad例凡\quad例凡\quad例凡\quad例凡\quad例凡\quad例凡\quad例凡\quad例凡\quad例凡\quad例凡\quad例凡\quad例凡\quad例凡\quad例凡\quad例凡\quad例凡\quad例凡\quad例凡\quad例凡\quad例凡\quad例凡\quad例凡\quad例凡\quad例凡\quad例凡\quad例凡\quad例凡\quad例凡\quad例凡\quad例凡\quad例凡\quad例凡\quad例凡\quad例凡\quad例凡\quad例凡\quad例凡\quad例凡\quad例凡\quad例凡\quad例凡\quad例凡\quad例凡\quad例凡\quad例凡\quad例凡\quad例凡\quad例凡\quad例凡\quad例凡\quad例凡\quad例凡\quad例凡\quad例凡\quad例凡\quad例凡\quad例凡\quad例凡\quad例凡\quad例凡\quad例凡\quad例凡\quad例凡\quad例凡\quad例凡\quad例凡\quad例凡\quad例凡\quad例凡\quad例凡\quad例凡\quad例凡\quad例凡\quad例凡\quad例凡\quad例凡\quad例凡\quad例凡\quad例凡\quad例凡\quad例凡\quad例凡\quad例凡\quad例凡\quad例凡\quad例凡\quad例凡\quad例凡\quad例凡\quad例凡\quad例凡\quad例凡\quad例凡\quad例凡\quad例凡\quad例凡\quad例凡\quad例凡\quad例凡\quad例凡\quad例凡\quad例凡\quad例凡\quad例凡\quad例凡\quad例凡\quad例凡\quad例凡\quad例凡\quad例凡\quad例凡\quad例凡\quad例凡\quad例凡\quad例凡\quad例凡\quad例凡\quad例凡\quad例凡\quad例凡\quad例凡\quad例凡\quad例凡\quad例凡\quad例凡\quad例凡\quad例凡\quad例凡\quad例凡\quad例凡\quad例凡\quad例凡\quad例凡\quad例凡\quad例凡\quad例凡\quad例凡\quad例凡\quad例凡\quad例凡\quad例凡\quad例凡\quad例凡\quad例凡\quad例凡\quad例凡\quad例凡\quad例凡\quad例凡\quad例凡\quad例凡\quad例凡\quad例凡\quad例凡\quad例凡\quad例凡\quad例凡\quad例凡\quad例凡\quad例凡\quad例凡\quad例凡\quad例凡\quad例凡\quad例凡\quad例凡\quad例凡\quad例凡\quad例凡\quad例凡\quad例凡\quad例凡\quad例凡\quad例凡\quad例凡\quad例凡\quad例凡\quad例凡\quad例凡\quad例凡\quad例凡\quad例凡\quad例凡\quad例凡\quad例凡\quad例凡\quad例凡\quad例凡\quad例凡\quad例凡\quad例凡\quad例凡\quad例凡\quad例凡\quad例凡\quad例凡\quad例凡\quad例凡\quad例凡\quad例凡\quad例凡\quad例凡\quad例凡\quad例凡\quad例凡\quad例凡\quad例凡\quad例凡\quad例凡\quad例凡\quad例凡\quad例凡\quad例凡\quad例凡\quad例凡\quad例凡\quad例凡\quad例凡\quad例凡\quad例凡\quad例凡\quad例凡\quad例凡\quad例凡\quad例凡\quad例凡\quad例凡\quad例凡\quad例凡\quad例凡\quad例凡\quad例凡\quad例凡\quad例凡\quad例凡\quad例凡\quad例凡\quad例凡\quad例凡\quad例凡\quad例凡\quad例凡\quad例凡\quad例凡\quad例凡\quad例凡\quad例凡\quad例凡\quad例凡\quad例凡\quad例凡\quad例凡\quad例凡\quad例凡\quad例凡\quad例凡\quad例凡\quad例凡\quad例凡\quad例凡\quad例凡\quad例凡\quad例凡\quad例凡\quad例凡\quad例凡\quad例凡\quad例凡\quad例凡\quad例凡\quad例凡\quad例凡\quad例凡\quad例凡\quad例凡\quad例凡\quad例凡\quad例凡\quad例凡\quad例凡\quad例凡\quad例凡\quad例凡\quad例凡\quad例凡\quad例凡\quad例凡\quad例凡\quad例凡\quad例凡\quad例凡\quad例凡\quad例凡\quad例凡\quad例凡\quad例凡\quad例凡\quad例凡\quad例凡\quad例凡\quad例凡\quad例凡\quad例凡\quad例凡\quad例凡\quad例凡\quad例凡\quad例凡\quad例凡\quad例凡\quad例凡\quad例凡\quad例凡\quad例凡\quad例凡\quad例凡\quad例凡\quad例凡\quad例凡\quad例凡\quad例凡\quad例凡\quad例凡\quad例凡\quad例凡\quad例凡\quad例凡\quad例凡\quad例凡\quad例凡\quad例凡\quad例凡\quad例凡\quad例凡\quad例凡\quad例凡\quad例凡\quad例
\clearpage

% 目录
% 右页眉
\rhead{}			
% 显示目录
\tableofcontents

\newpage

% 将页码格式设置为阿拉伯数字
\pagenumbering{arabic}
% 页码从1开始
\setcounter{page}{1}

\part{\kaishu 旧\quad约}

\newpage

% 新旧约全书总论
% 自定义目录
\phantomsection
\addcontentsline{toc}{chapter}{新旧约全书总论}
% 标题
\chapter*{新旧约全书总论}

% 右页眉
\rhead{新旧约全书总论}

《新旧约全书》,是数十卷经书的总集。这些经书的特点,在于它们的写成有超乎自然之处,因为这些经书都是在天主圣神默感下写成,赐予天主的之民——教会——的礼物。

圣教会自古以来,一致主张这部总集包括《旧约》46卷,《新约》27卷,共计73卷。但大多数的\uline{基督教}派,由于只相信以\uline{希伯来}文写成的书才为圣经,因此现今只有\uline{希腊}原文的《巴路克》、《多俾亚传》、《友弟德传》、《玛加伯》上下、《智慧篇》和《德训篇》7卷,未著录在他们的圣经书目内。而天主教会自古即以\uline{希腊}文七十贤士本为圣经,因而对上述7卷也一律认为是圣经。

这些经书称为“约”,因为其中心思想,是天主与人类所立的盟约。天主与\uline{以色列}民族在\uline{西乃}山上所立的盟约,称为“旧约”;\uline{耶稣}以自己的圣血和圣死为全人类所立的永远盟约,称为“新约”。这些经书又称为“圣经”,是为表示这些书所具有的独特地位和神圣权威。书中所记述的一切,是吾人信仰及道德的大经,又为吾人立身经世的大道。

《旧约》经书的原文,除几卷和几小段外,大都以\uline{希伯来}文写成。后来侨居\uline{北非}受了\uline{希腊}文化影响的\uline{犹太}人,因多不谙\uline{希伯来}文,\uline{犹太}人遂在公元前三至二世纪,将《旧约》各书译为\uline{希腊}文,即今所称的“七十贤士译本”。以后\uline{希腊}语文也成了\uline{罗马}帝国的通用语言,宗徒们在各地宣讲福音,为了方便起见,即时常利用这部\uline{希腊}文圣经,为此这部\uline{希腊}文圣经(包括46卷)自初即为教会所尊重,并具有极大的权威。

《新约》各书,全部是以\uline{希腊}文写成,只有《玛窦福音》,原文虽为\uline{阿辣美}文,但很早即已失传;今所留传的,只有\uline{希腊}文本。

《旧约全书》的写成,凡经一千余年(约由公元前1300至100年),而逐渐汇为一集。《新约全书》是公元初世纪宗徒时代的作品。

《新旧约全书》,通常分为三大类:即历史书、先知书和智慧书(或训诲书),这是很广泛的分类。至于作者,《旧约》大多出于先知及其他贤哲的手笔,《新约》是宗徒和宗徒弟子的写作。但因全部圣经都是“因天主的默感写成的”(\uwave{第}后3:16),经内的话是“由天主所派遣的圣人,在圣神推动之下说出来的”(\uwave{伯}后1:21),为此我们不得不承认圣经的首要作者是天主。所谓“默感”,即是说:圣经的作者与编者(人),在天主的灵性感动之下,写下天主愿向他的子民(旧约与新约的教会)所要说的话,记下天主要他们记述的史事。有时天主也曾向他们透露某些重要的事迹,或直接向他们说话;这样,作者不仅获得了“默感”,同时也获得了“启示”。既然天主是圣经的首要作者,那么圣经上所记载的即是天主的话,即是天主的“圣言”。既然是天主的话,那么圣经上所载的一切,句句都“真实无误”。就是说:圣经作者在天主默感下所愿表示出来的意义,是不会错误的。但为了解作者所要表达的本意,必须先注意经书中每部书的文体和体裁:是散文或是诗体?是历史或是传奇?是寓言或是训诲?因为每种文体有其独特的意义。同时还应注意作者或编者的时代背景,因为时代不同,论事的观点也各有异。比如古代民族,尤其\uline{以色列}人对历史的观点,和今日的史学家的观点,有绝大的不同。尤其圣经的作者或编者,是本着宗教观点来编述历史的过程。他们看历史时,常着眼于天主为历史的推定者和支配者;人民的盛衰兴亡,常系之于他们是否遵守天主的法律。

另一个极重要的问题,是圣经与科学。圣经的作者决无意以教授自然科学(如宇宙学、天文学、生物学、人类学等)为写作的目的。圣经作者的目的,是在于启迪人类“获得拯救的智慧”(\uwave{第}后3:15);为此他们无意研究自然界的进化和人体的构造。其用意只在说明自然界和人类与天主的关系,教导世人,天地万物都来自天主,一切都因天主的照顾而生存,最后又归于天主。

还应当注意的是:为适当地研究圣经和解释经意,人人必须先有信仰,并甘心接受圣教会的指导,因为天主把圣经委托给教会保管,因此只有教会才有解释圣经的特权。

圣经中所记载的都是些最重要的真理,教父多称圣经为天主给流徒的世人寄来的家书或天书。在这部天书内,天主先将自己启示给世人并告知世人,天地万物的来历和目的,告诉我们天主原先怎样给人类备下了幸福,现今的痛苦、患难和死亡又怎样来的;天主在漫长的历史过程中,怎样逐步实现了他救赎人类的大计划。《旧约》所载,即是天主为\uline{以色列}民族所行的大事,所定的法律,所发的劝言和警告。这一切都是他为完成救赎人类工程的准备,甚至\uline{以}民的被选,也是为准备万民获得救恩。《新约》则是全人类得救的大喜讯,

记载人类唯一的救主\uline{耶稣}\uline{基督}救赎人类的大事;因为\uline{耶稣}是天主第二位圣子,只有他才能把天主性体的真理启示给人类。他降生成人,籍着自己的人性,完成了天主慈爱的计划,使人与天主重归于好,且提高了人类的地位,使人分享天主性的永福。

由此看来,\uline{耶稣}\uline{基督}实为《旧新》二约的中心,是“法律”《旧约》的终向(\uwave{罗}10:4),是“新约的中保”(\uwave{希}8:6)。《旧新》二约各经卷的最后目的,就是叫人准备期待“我们伟大的天主及救主耶稣基督的光荣显现”(\uwave{铎}2:13)。

圣经为人类得救既有如此重要的关系,因此圣经对圣教会,对于一切基督信徒,对全人类,的确是举世无双的无价宝书。圣\uline{保禄}论《旧约》说:“凡所写的,都是为教训我们写的”(\uwave{罗}15:4),“为教训,为督责,为矫正,为教导人学正义,都是有益的”(\uwave{第后3:16})。这些话对《新约》来说,更为恰当;而且可说,如果没有圣经,我们无论对天主,或是对人,不会有一个正确的认识,因为只凭理性的自然神学是不够的,决不能打动人心,唯有研读圣经才能触及我们灵魂的深处,使我们听得见“生活天主”的话,领略天主威爱兼有的声音,洞见他全能的伟大化工,明白他怎样生养保存万物,怎样以他至高无上的主权宰治一切,裁判一切;又怎样以他慈父心肠,导引迷途的荡子,回归父家。无怪乎教宗\uline{良}十三称圣经为“神学的灵魂”。

诚然,一个怀有信德的教友,在恭读默思圣经时,应觉得是与天主会晤,是在静听天主的劝导,是在听他在天之父的慈音。当他心有所得,情有所动时,自然就向天主说话,这即是祈祷。无怪乎圣教会自古即以圣经为赞美、祈祷、默想最好的宝书。信友如能日日如此读经,与天主互诉衷曲,在日常的生活上或工作上,必能时时对越天主,承行他的圣意,臻于圣化一切的至境。为此教会不断劝勉信友多读圣经,尤其这次大公会议,对圣经研究与圣经诵读特予强调。愿我信友善体慈母教会的劝告,勉力天天去阅读这部天赐宝书。
\clearpage
% 梅瑟五书引论
\phantomsection
\addcontentsline{toc}{chapter}{梅瑟五书引论}
% 标题
\chapter*{梅瑟五书引论}

% 右页眉
\rhead{梅瑟五书引论}

《旧约全书》前五卷,通称“梅瑟五书”,或简称“五书”。因为在这五卷书内,包含着《旧约》中最重要的一部分,即\uline{梅瑟}给\uline{以色列}人所宣布的法律,为此圣经上多次称《五书》为“法律”。\uline{希伯来}人称之为“托辣”。

《梅瑟五书》虽然在纪元前已有如此的分划:即《创世纪》、《出谷纪》、《肋未纪》、《户籍纪》、《申命纪》,但自古以来这五卷书常视为一部,且是一部世界文学上的杰作。

如果说全部圣经的主题是阐述人类的救赎史,那么“五书”即是记述这救赎史的开端。作者从天地和人类的创造开始,说到人类因违背天主的命令,而失掉原有的幸福,再扼要地叙述各民族的太古史,继而只着重于\uline{以色列}民族的起源,及其成为天主选民的历史。这历史的中心即在于天主与选民在\uline{西乃}山上所结的盟约。天主很早即对\uline{以色列}人的先祖再三地许诺,要以特殊照顾和非凡的奇事,准备\uline{以}民的心灵,使他们对于天主养成坚定不移的信仰,以后好籍着\uline{梅瑟}选立他们为自己的国民,颁下当遵行的法律,在世上建立起神权政体的神国。以后又在旷野四十年之久,以种种试探考验了他们的忠诚和信心;最后引他们到了\uline{约但}河东岸,在那里又籍着\uline{梅瑟}劝告他们,重述以前教导过他们的一切,准备他们进占已预许给他们祖先的福地。所以从历史方面来看,“五书”有其统一的目标,实是一部上下一贯的著述。

按古来一致的传授,“五书”的作者是\uline{梅瑟}。称他为作者,并不是说全书每字每句都出于他一个人的手笔,而是说他曾搜集了不少当时所能找得到的史料、文献和法律。且在他死后,有许多历史或法律部分是后人增补的,因为“五书”原是\uline{以色列}人宗教、政治、社会生活的法典,所以常有随时增添一些解释的必要,为使\uline{梅瑟}法律能随历史的演变,而适应时代的环境。

从以上所述,可知“五书”为\uline{以色列}人具有多么重要的关系。如果我们对“五书”没有认识,便不能明了\uline{以色列}子民的历史,因为他们生活在一个神权政体的制度之下,他们的存亡盛衰,全系于他们是否忠实履行天主籍\uline{梅瑟}所颁布的法律。在《旧约》其他经书内,常不断指出法律的这种重要关系,并依法律为原则,来批判一切历史的得失。但这法律的最终目的,诚如圣\uline{保禄}所说:“法律的终向本来是基督”(\uwave{罗}10:4)。为此,法律为\uline{以色列}人,好像是“归于基督的启蒙师”(\uwave{迦}3:24)。换句话说,法律应领导\uline{以色列}人,认识并信仰将要来临的默西亚。

当默西亚\uline{耶稣}\uline{基督}一降生,法律的使命就算完结,\uline{耶稣}所宣讲的“爱的诫命”,满全了整个法律(\uwave{罗}13:10)。虽然如此,“五书”为《新约》的教会,仍未失其重要性,因为本书含有永生天主的启示,以及教会信仰的基础。
\clearpage

% 创世纪引言
\phantomsection
\addcontentsline{toc}{chapter}{创世纪引言}
\chapter{创世纪引言}

“梅瑟五书”每部的名称,\uline{犹太}人皆以每书的首句首字为名。自\uline{希腊}七十贤士译经以来,皆以每书的内容大意命名。“梅瑟五书”的第一部名为《创世纪》,因为本书所记载的是有关天地万物的创造,人类的太古史和\uline{以色列}民族的起源。但本书并不是以科学的论点和近代史学家的方法来记述,而是本着宗教的观点来说明救赎史的开端。在这救赎史中,依照天主的计划,\uline{以色列}民族在万民中占着最重要的角色,因此作者也只着重于这个民族的历史。

本书的前编(1-11章),是救世史的前导,说明天主是整个天地万物的创造者,和全人类历史的领导者;并指出\uline{以色列}人与其他民族的关系。原祖父母虽然背命,惹下了滔天大祸,后继的人们也多半背弃了天主,如洪水和\uline{巴贝耳}塔时代的人,但因为人是按天主的肖像造成的,天主决不愿将全人类完全抛弃,所以在后编内(12-50章),作者便记述天主怎样拣选了一位信仰坚定,服从听命的人——\uline{亚巴郎},怎样向他起誓,立他为一个新民族的始祖,即将来要成为天主选民的民族的始祖,许下因他和他的后裔,天下万民将要获得祝福(22:18),由他的后裔中要生出一位“应得权杖,万民都要归顺他”(49:10),他要使“救恩达于地极”的后裔(\uwave{依}49:6)。在记述\uline{亚巴郎}、\uline{依撒格}、\uline{雅各伯}和\uline{若瑟}的事迹中,作者一再证明天主怎样特殊地照顾了他们,以准备救赎人类的道路。
\clearpage
% 创世纪(创)
\phantomsection
\addcontentsline{toc}{chapter}{创世纪}
\chapter{创世纪(创)}

\begin{center}
	\textbf{前编 }
	\textbf{太古史(1-11)}
\end{center}

\textbf{第一章 }
\textbf{天地万物的创造 }
\textsuperscript{1}
在起初天主创造了天地。
\textsuperscript{2}  
大地还是混沌空虚,深渊上还是一团黑暗,天主的神在水面上运行。
\renewcommand\thefootnote{\ding{\numexpr171+\value{footnote}}}
\footnote{“在起初……”一语,暗示创造万物之时,除天主外,一无所有。“天地”二字此处有宇宙万物之意。作者用诗人的想象力描写天主好似一个工程师,在六天以内创造了万物,到第七天休息。首先所创造的是混沌的无生之物,后将这混沌之物分成天、地、海三大部分,然后以日月、星辰、草木、飞禽、走兽等来点缀天地海洋。最后天主照自己的肖像造了人。作者从创造混沌之物说起,到创造人,表示人是万物之灵,应效法造物主工作和守安息日。此开宗明义第一章是远古时代的文学杰作,是一篇宗教的重要文告,并不是自然科学的论著。按古代各民族对开天辟地,人类诞生的传说,没有可与《创世纪》第一章相比拟的。“天主的神”指施生命之神力,但若通观新旧二约的全部启示,此处也指赐生命的“天主圣神”。}
\textsuperscript{3}
天主说:“有光!”就有了光。
\textsuperscript{4}
天主见光好,就将光与黑暗分开。
\textsuperscript{5}
天主称光为“昼”,称黑暗为“夜”。过了晚上,过了早晨,这是第一天。

\textsuperscript{6}
天主说:“在水与水之间要有穹苍,将水分开!”事就这样成了。
\textsuperscript{7}
天主造了穹苍,分开了穹苍以下的水和穹苍以上的水。
\textsuperscript{8}
天主称穹苍为“天”,天主看了认为好。过了晚上,过了早晨,这是第二天。

\textsuperscript{9}
天主说:“天下的水应聚在一处,使旱地出现!”事就这样成了。
\textsuperscript{10}
天主称旱地为“陆地”,称水汇合处为“海洋”。天主看了认为好。
\textsuperscript{11}
天主说:“在陆地上,土地要生出青草、结种子的蔬菜和结果子的果树,各按照在它内的种子的种类!”事就这样成了。
\textsuperscript{12}
土地就生出了青草、结种子的蔬菜,各按其类,和结果子的树木,各按照在它内的种子的种类。天主看了认为好。
\textsuperscript{13}
过了晚上,过了早晨,这是第三天。

\textsuperscript{14}
天主说:“在天空中要有光体,以分别昼夜,作为规定时节和年月日的记号。
\textsuperscript{15}
要在天空中放光,照耀大地!”事就这样成了。
\textsuperscript{16}
天主于是造了两个大光体:较大的控制白天,较小的控制黑夜,并造了星宿。
\textsuperscript{17}
天主将星宿摆列在天空,照耀大地,
\textsuperscript{18}
控制昼夜,分别明与暗。天主看了认为好。
\textsuperscript{19}
过了晚上,过了早晨,这是第四天。

\textsuperscript{20}
天主说:“水中要繁生蠕动的生物,地面上、天空中要有鸟飞翔!”事就这样成了。
\textsuperscript{21}
天主于是造了大鱼和所有在水中孳生的蠕动生物,各按其类,以及各种飞鸟,各按其类。天主看了认为好。
\textsuperscript{22}
遂祝福它们说:“你们要孳生繁殖,充满海洋;飞鸟也要在地上繁殖!”
\textsuperscript{23}
过了晚上,过了早晨,这是第五天。

\textsuperscript{24}
天主说:“地上要生出生物,各按其类;走兽、爬虫和地上的各种生物,各按其类!”事就这样成了。
\textsuperscript{25}
天主于是造了地上的生物,各按其类;各种走兽,各按其类;以及地上所有的爬虫,各按其类。天主看了认为好。
\textsuperscript{26}
天主说:“让我们照我们的肖像,按我们的模样造人,叫他管理海中的鱼、天空的飞鸟、牲畜、各种野兽、在地上爬行的各种爬虫。”
\textsuperscript{27}
天主于是照自己的肖像造了人,就是照天主的肖像造了人:造了一男一女。
\textsuperscript{28}
天主祝福他们说:“你们要生育繁殖,充满大地,治理大地,管理海中的鱼、天空的飞鸟、各种在地上爬行的生物!”
\footnote{“人”按原文有红土或黄土的意思,是说人是属于土的造物。“我们”(26节)按古\uline{犹太}经师的解释,是指天主和天使,好似天主同天使商量;但有些学者主张为“威严复数”或“议决复数”。教父和神学家多以为此复数暗示天主圣三的奥理。此说若照启示的演进说是对的。人相似天主是按灵魂说的,相似天主有理智、意志和记忆。论人的肉身,当天主造\uline{亚当}时,已预见作\uline{亚当}第二的\uline{基督}(\uwave{罗}5:14)。“造了一男一女”,指婚姻一夫一妻制和不可分离性(\uwave{玛}19:1-6;\uwave{拉}2:15、16)。天主祝福原祖生育繁殖的话,说明婚姻的首要目的是生养教育子女(8:17;\uwave{咏}127:3、4)。}
\textsuperscript{29}
天主又说:“看,全地面上结种子的各种蔬菜,在果内含有种子的各种果树,我都给你们作食物;
\textsuperscript{30}
至于地上的各种野兽,天空中的各种飞鸟,在地上爬行有生魂的各种动物,我把一切青草给它们作食物。”事就这样成了。
\textsuperscript{31}
天主看了他造的一切,认为样样都很好。过了晚上,过了早晨,这是第六天。
\footnote{天主造了原祖,也赐给了他们和他们传生的人类的食物,并将普世交给他们统治。所造的万物样样都好,是说万物都合天主的旨意,都为他所喜爱。参阅\uwave{咏}19:1-6,104,145,148,150。}

\textbf{第二章 }
\textbf{安息日 }
\textsuperscript{1}
这样,天地和天地间的一切点缀都完成了。
\textsuperscript{2}
到第七天天主造物的工程已完成,就在第七天休息,停止了所作的一切工程。
\textsuperscript{3}
天主祝福了第七天,定为圣日,因为这一天,天主停止了他所行的一切创造工作。
\footnote{1-3节属前章,劝人守安息日为圣日。守安息日的原因与目的,见\uwave{出}23:12;\uwave{申}5:12-15。}

\textbf{人与乐园 }
\textsuperscript{4}
这是创造天地的来历:在上主天主创造天地时,
\textsuperscript{5}
地上还没有灌木,田间也没有生出蔬菜,因为上主天主还没有使雨降在地上,也没有人耕种土地,
\textsuperscript{6}
有从地下涌出的水浸润所有地面。
\textsuperscript{7}
上主天主用地上的灰土形成了人,在他鼻孔内吹了一口生气,人就成了一个有灵的生物。
\textsuperscript{8}
上主天主在\uline{伊甸}东部种植了一个乐园,就将他形成的人安置在里面。
\textsuperscript{9}
上主天主使地面生出各种好看好吃的果树,生命树和知善恶树在乐园中央。
\footnote{2:4-3:24为创造天地万物的另一记载。原来在这记载中只用了天主(雅威)的名词,但将这个记载与上章的记载编在一起,补入了“天主”的名词。这记载的中心为人:天主对人,人对天主的态度。关于人的来历和本性,作者用简略的话,教训人一端论宗教和文化的最高深的道理:人肉身的形成,好像其他的动物,是由尘土造成的,但对于灵魂却有极大的区别,它是直接由天主所造。\uline{伊甸}乐园位于何处,人不得而知。乐园是天主考验人的地方。“生命树”所象征的是天主愿意赐给人的“不死”之恩。“知善恶的树”,是试探人的工具。“知善恶”的意思,大概是说:人一犯天主的禁令,就知道所失去的超性恩宠——真善,是多么美善,所犯的罪恶——真恶,是如何凶恶。}
\textsuperscript{10}
有一条河由\uline{伊甸}流出灌溉乐园,由那里分为四支:
\textsuperscript{11}
第一支名叫\uline{丕雄},环流产金的\uline{哈威拉}全境;
\textsuperscript{12}
那地方的金子很好,那里还产珍珠和玛瑙;
\textsuperscript{13}
第二支河名叫\uline{基红},环流\uline{雇士}全境;
\textsuperscript{14}
第三支河名叫\uline{底格里斯},流入\uline{亚述}东部;第四支河即\uline{幼发拉的}。
\textsuperscript{15}
上主天主将人安置在\uline{伊甸}的乐园内,叫他耕种,看守乐园。
\footnote{说明人犯罪以前,天主已叫人应该工作。}
\textsuperscript{16}
上主天主给人下令说:“乐园中各树上的果子,你都可吃,
\textsuperscript{17}
只有知善恶树上的果子你不可吃,因为那一天你吃了,必定要死。”

\textbf{造女人立婚姻 }
\textsuperscript{18}
上主天主说:“人单独不好,我要给他造个与他相称的助手。”
\textsuperscript{19}
上主天主用尘土造了各种野兽和天空中的各种飞鸟,都引到人面前,看他怎样起名;凡人给生物起的名字,就成了那生物的名字。
\textsuperscript{20}
人遂给各种畜牲、天空中的各种飞鸟和各种野兽起了名字;但他没有找着一个与自己相称的助手。
\textsuperscript{21}
上主天主遂使人熟睡,当他睡着了,就取出了他的一根肋骨,再用肉补满原处。
\textsuperscript{22}
然后上主天主用那由人取来的肋骨,形成了一个女人,引她到人前,
\textsuperscript{23}
人遂说:“这才真是我的骨中之骨,肉中之肉,她应称为“女人”,因为是由男人取出的。”
\textsuperscript{24}
为此人应离开自己的父母,依附自己的妻子,二人成为一体。
\textsuperscript{25}
当时,男女二人都赤身露体,并不害羞。
\footnote{本段的要义有二:一、人给动物命名,是表示人受有统治一切造物的权柄;二、从\uline{亚当}的肉身形成了第一个女人,是指女人同他有一样的人性,像\uline{亚当}一样是照天主的肖像受造的。夫妇结为一体,表示婚姻的结合是天主制定的,人不能拆散(\uwave{玛}19:5、6)。赤身不害羞,是说原祖未犯罪前纯洁无罪的状态,还未体验到罪过的恶果。}

\textbf{第三章 }
\textbf{原祖违命 }
\textsuperscript{1}
在上主天主所造的一切野兽中,蛇是最狡猾的。蛇对女人说:“天主真说了,你们不可吃乐园中任何树上的果子吗?”
\textsuperscript{2}
女人对蛇说:“乐园中树上的果子,我们都可吃;
\textsuperscript{3}
只有乐园中央那颗树上的果子,天主说过,你们不可以吃,也不可摸,免得死亡。”
\textsuperscript{4}
蛇对女人说:“你们决不会死!
\textsuperscript{5}
因为天主知道,你们那天吃了这果子,你们的眼就会开了,将如同天主一样知道善恶。”
\textsuperscript{6}
女人看那棵果树实在好吃好看,令人羡慕,且能增加智慧,遂摘下一个果子吃了,又给了她的男人一个,他也吃了。
\textsuperscript{7}
于是二人的眼立即开了,发觉自己赤身露体,遂用无花果树叶,编了个裙子围身。
\textsuperscript{8}
当\uline{亚当}和他的妻子听见了上主天主趁晚凉在乐园中散步的声音,就躲藏在乐园的树林中,怕见上主天主的面。
\footnote{本章记的蛇就是魔鬼。他籍蛇形诱惑了\uline{厄娃}(\uwave{智}2:23、24;\uwave{若}8:44;\uwave{默}12:9,20:2)。原祖所犯是骄傲背命的罪。“发觉自己赤身”,是指失去天主的宠爱和原始的纯洁。}
\textsuperscript{9}
上主天主呼唤\uline{亚当}对他说:“你在哪里?”
\textsuperscript{10}
他答说:“我在乐园中听到了你的声音,就害怕起来,因为我赤身露体,遂躲藏了。”
\textsuperscript{11}
天主说:“谁告诉了你,赤身露体?莫非你吃了我禁止你吃的果子?”
\textsuperscript{12}
\uline{亚当}说:“是你给我作伴的那个女人给了我那树上的果子,我才吃了。”
\textsuperscript{13}
上主天主遂对女人说:“你为什么作了这事?”女人答说:“是蛇哄骗了我,我才吃了。”
\footnote{天主询问时,没有询问魔鬼,只询问了亚当\uline{亚当} \uline{厄娃};但惩罚时却按罪过的原因和轻重:先是魔鬼,后是\uline{厄娃},最后是\uline{亚当}(\uwave{第}前2:13-15)。}

\textbf{处罚与预许 }
\textsuperscript{14}
上主天主对蛇说:“因你做了这事,你在一切畜牲和野兽中,是可咒骂的;你要用肚子爬行,毕生日日吃土。
\textsuperscript{15}
我要把仇恨放在你和女人,你的后裔和她的后裔之间,她的后裔要踏碎你的头颅,你要伤害他的脚跟。”
\textsuperscript{16}
后对女人说:“我要增加你怀孕的苦楚,在痛苦中生子;你要依恋你的丈夫,也要受他的管辖。”
\textsuperscript{17}
后对\uline{亚当}说:“因为你听了你妻子的话,吃了我禁止你吃的果子,为了你的缘故,地成了可咒骂的;你一生日日劳苦才能得到吃食。
\textsuperscript{18}
地要给你生出荆棘和蒺藜,你要吃田间的蔬菜;
\textsuperscript{19}
你必须汗流满面,才有饭吃,直到你归于土中,因为你是由土来的;你既是灰土,你还要归于灰土。”
\footnote{天主的仁慈即超过了他的公义,故此他在义怒中给人类预许了人终要得胜魔鬼的诺言;因这许诺,3:15称为“原始福音”。大义是:踏碎蛇头是得胜魔鬼的象征;“女人的后裔”虽然也指犯罪败坏的人类,但在此特别指拯救人类的新元首\uline{基督}(\uwave{哥}1:15-18),只有他打败了魔鬼;故此圣\uline{保禄}称他为“新亚当”(\uwave{罗}5:12-15)。魔鬼同\uline{厄娃}的对白,与天使同\uline{玛利亚}的对白恰恰相反:一是诱惑的对白,一是商讨救赎的对白;因此教父由第2世纪起即称\uline{玛利亚}为“新\uline{厄娃}”。又因她与\uline{基督}的密切结合,她也踏碎了魔鬼的头颅。圣母始胎无玷的道理,由此处已露曙光(\uwave{路}1:26-38;\uwave{默}12)。}

\textbf{被逐出乐园 }
\textsuperscript{20}
\uline{亚当}给自己的妻子起名叫\uline{厄娃},因为她是众生的母亲。
\textsuperscript{21}
上主天主为\uline{亚当}和他的妻子做了件皮衣,给他们穿上;
\textsuperscript{22}
然后上主天主说:“看,人已相似我们中的一个,知道了善恶;如今不要让他伸手再摘取生命树上的果子,吃了活到永远。”
\textsuperscript{23}
上主天主遂把他赶出\uline{伊甸}乐园,叫他耕种他所由出的土地。
\textsuperscript{24}
天主将\uline{亚当}逐出了以后,就在\uline{伊甸}乐园的东面,派了“革鲁宾”和刀光四射的火剑,防守到生命树去的路。
\footnote{“革鲁宾”按\uline{巴比伦}语有“保护者”之意(\uwave{出}25:18-22;\uwave{则}1:11)。}

\textbf{第四章 }
\textbf{杀弟之罪 }
\textsuperscript{1}
\uline{亚当}认识了自己的妻子\uline{厄娃},\uline{厄娃}怀了孕,生了\uline{加音}说:“我赖上主获得了一个人。”
\textsuperscript{2}
以后她生了\uline{加音}的弟弟\uline{亚伯尔};\uline{亚伯尔}牧羊,\uline{加音}耕田。
\textsuperscript{3}
有一天,\uline{加音}把田地的出产作祭品献给天主;
\textsuperscript{4}
同时\uline{亚伯尔}献上自己羊群中最肥美而又是首生的羊;上主惠顾了\uline{亚伯尔}和他的祭品,
\footnote{“认识妻子”是表示夫妻结合的委婉语。\uline{厄娃}生子后说的话意义深奥,说明天主特籍为人母者广传肖似天主的人类(\uwave{加}下7:26-29)。由本章证明献祭从人类起初即有了;献祭的真正价值是在于人的敬心诚意(\uwave{希}11:4)。}
\textsuperscript{5}
却没有惠顾\uline{加音}和他的祭品;因此\uline{加音}大怒,垂头丧气。
\textsuperscript{6}
上主对\uline{加音}说:“你为什么发怒?为什么垂头丧气?
\textsuperscript{7}
你若做得好,岂不也可仰起头来?你若做得不好,罪恶就伏在你门前,企图对付你,但你应制服它。”

\textsuperscript{8}
事后\uline{加音}对他弟弟\uline{亚伯尔}说:“我们到田间去!”当他们在田间的时候,\uline{加音}就袭击了弟弟\uline{亚伯尔},将他杀死。
\textsuperscript{9}
上主对\uline{加音}说:“你弟弟\uline{亚伯尔}在哪里?”他答说:“我不知道,难道我是看守我弟弟的人?”
\textsuperscript{10}
上主说:“你作了什么事?听!你弟弟的血由地上向我喊冤。
\textsuperscript{11}
你现在是地上所咒骂的人,地张开口由你手中接收了你弟弟的血,
\textsuperscript{12}
从此你即使耕种,地也不会给你出产;你在地上要成个流离失所的人。”
\textsuperscript{13}
\uline{加音}对上主说:“我的罪罚太重,无法承担。
\textsuperscript{14}
看你今天将我由这地面上驱逐,我该躲避你的面,在地上成了个流离失所的人;那么凡遇见我的,必要杀我。”
\textsuperscript{15}
上主对他说:“决不这样,凡杀\uline{加音}的人,一定要受七倍的罚。”上主遂给\uline{加音}一个记号,以免遇见他的人击杀他。
\textsuperscript{16}
\uline{加音}就离开上主的面,住在\uline{伊甸}东方的\uline{诺得}地方。
\footnote{杀兄弟的暴行为原祖犯罪的恶果。“凡遇见我的”一句,假定除\uline{加音}、\uline{亚伯尔}、\uline{舍特}三人外,\uline{亚当}还生了别的一些儿女。圣经只记载此三人,因为他们的命运为叙述救赎史已够了。初民为遵从天主叫人传生人类的命令,不能不兄妹结合。但日后人类增多了,兄妹的结合为宗教与礼法所禁止。}

\textbf{加音的后代 }
\textsuperscript{17}
\uline{加音}认识了自己的妻子,她怀了孕,生了\uline{哈诺客}。\uline{加音}建筑了一座城,即以他儿子的名字,给这城起名叫“\uline{哈诺客}”。
\textsuperscript{18}
\uline{哈诺客}生了\uline{依辣得};\uline{依辣得}生了\uline{默胡雅耳};\uline{默胡雅耳}生了\uline{默突沙耳};\uline{默突沙耳}生了\uline{拉默客}。
\textsuperscript{19}
\uline{拉默客}娶了两个妻子:一个名叫\uline{阿达},一个名叫\uline{漆拉}。
\textsuperscript{20}
\uline{阿达}生了\uline{雅巴耳},他是住在帐幕内畜牧者的始祖。
\textsuperscript{21}
他的弟弟名叫\uline{犹巴耳},他是所有弹琴吹箫者的始祖。
\textsuperscript{22}
同时\uline{漆拉}也生了\uline{突巴耳}\uline{加音},他是制造各种铜铁器具的匠人。\uline{突巴耳}\uline{加音}有个姊妹名叫\uline{纳阿玛}。
\textsuperscript{23}
\uline{拉默客}对自己的妻子说:“\uline{阿拉}和\uline{漆拉}倾听我的声音,\uline{拉默客}的妻子,静聆我的言语:因我受伤,杀了一成年;因我受损,杀了一青年;
\textsuperscript{24}
杀\uline{加音}的受罚是七倍,杀\uline{拉默客}的是七十七倍。”
\footnote{此段略记\uline{加音}的后代子孙,和他们的发明以及文化的初步演进。由此可知在洪水之前文明已达到了相当的程度。\uline{巴比伦}史家亦主此说。\uline{拉默客}是违犯一夫一妻制的第一人,违犯了婚姻一夫一妻的理想。}

\textbf{舍特的子孙 }
\textsuperscript{25}
\uline{亚当}又认识了自己的妻子,她生了个儿子,给他起名叫\uline{舍特}说:“天主又赐给了我一个儿子,代替\uline{加音}杀了的\uline{亚伯尔}。”
\textsuperscript{26}
\uline{舍特}也生了一个儿子,给他起名叫\uline{厄诺士}。那时人才开始呼求上主的名。
\footnote{从\uline{亚当}到\uline{厄诺士},人在祈祷和祭献时一定呼求天主助佑。但由\uline{厄诺士}开始举行公众崇拜天主的敬礼。}

\textbf{第五章 }
\textbf{洪水前亚当的后代 }
\textsuperscript{1}
以下是\uline{亚当}后裔的族谱:当天主造人的时候,是按天主的肖像造的,
\textsuperscript{2}
造了一男一女,且在造他们的那一天,祝福了他们,称他们为“人”。
\textsuperscript{3}
\uline{亚当}一百三十岁时,生了一个儿子,也像自己的模样和肖像,给他起名叫\uline{舍特}。
\footnote{\uline{亚当}生了相似自己的子女,因为他是照天主的肖像受造的,他生儿养女,即是传生天生至尊贵的肖像于万世万代的人类。}
\textsuperscript{4}
\uline{亚当}生\uline{舍特}后,还活了八百年,生了其他的儿女。
\textsuperscript{5}
\uline{亚当}共活了九百三十岁死了。
\textsuperscript{6}
\uline{舍特}一百零五岁时,生了\uline{厄诺士}。
\textsuperscript{7}
\uline{舍特}生\uline{厄诺士}后,还活了八百零七年,生了其他的儿女。
\footnote{本章所记为洪水前的十位祖宗,他们的长寿若与其他民族传说的古人比较,所记的年龄还不算太大。虽然如此,有关十位祖宗的年龄,是不易解决的难题。——圣经上说的年是指十二个月的年,月指廿九或三十日的月。——上古人类是否能享如此的长寿,考古人类学至今尚未有一圆满的答案。教父和神学家提出了两个理由来解释原始人的长寿原因:一、自然环境的优良条件:即在人犯原罪之后,仍未丧失天主在造人时所赋的优良人性;二、长寿的主因是天主上智的措施,使人能迅速繁殖,并使人将天主的原始启示传于后代子孙。}
\textsuperscript{8}
\uline{舍特}共活了九百一十二岁死了。
\textsuperscript{9}
\uline{厄诺士}九十岁时生了\uline{刻南}。
\textsuperscript{10}
\uline{厄诺士}生\uline{刻南}后,还活了八百一十五年,生了其他的儿女。
\textsuperscript{11}
\uline{厄诺士}共活了九百零五岁死了。
\textsuperscript{12}
\uline{刻南}七十岁时,生了\uline{玛拉肋耳}。
\textsuperscript{13}
\uline{刻南}生\uline{玛拉肋耳}后,还活了八百四十年,生了其他的儿女。
\textsuperscript{14}
\uline{刻南}共活了九百一十岁死了。
\textsuperscript{15}
\uline{玛拉肋耳}六十五岁时,生了\uline{耶勒得}。
\textsuperscript{16}
\uline{玛拉肋耳}生\uline{耶勒得}后,还活了八百三十年,生了其他的儿女。
\textsuperscript{17}
\uline{玛拉肋耳}共活了八百九十五岁死了。
\textsuperscript{18}
\uline{耶勒得}一百六十二岁时,生了\uline{哈诺客}。
\textsuperscript{19}
\uline{耶勒得}生\uline{哈诺客}后。还活了八百年,生了其他的儿女。
\textsuperscript{20}
\uline{耶勒得}共活了九百六十二岁死了。
\textsuperscript{21}
\uline{哈诺客}六十五岁时,生了\uline{默突舍拉}。
\textsuperscript{22}
\uline{哈诺客}常与天主往来。\uline{哈诺客}生\uline{默突舍拉}后,还活了三百年,生了其他的儿女。
\textsuperscript{23}
\uline{哈诺客}共活了三百六十五岁。
\textsuperscript{24}
\uline{哈诺客}时与天主往来,然后就不见了,因为天主将他提去。
\footnote{论\uline{哈诺客}的事,见\uwave{德}44:16;\uwave{犹}14、15;\uwave{希}11:5。}
\textsuperscript{25}
\uline{默突舍拉}一百八十七岁时,生了\uline{拉默客}。
\textsuperscript{26}
\uline{默突舍拉}生了\uline{拉默客}后,还活了七百八十二年,生了其他的儿女。
\textsuperscript{27}
\uline{默突舍拉}共活了九百六十九岁死了。
\textsuperscript{28}
\uline{拉默客}一百八十二岁时,生了一个儿子,
\textsuperscript{29}
给他起名叫\uline{诺厄}说:“这孩子要使我们在上主诅咒的地上,在我们做的工作和劳苦上,获得欣慰!”
\textsuperscript{30}
\uline{拉默客}生\uline{诺厄}后,还活了五百九十五年,生了其他的儿女。
\textsuperscript{31}
\uline{拉默客}共活了七百七十七岁死了。
\textsuperscript{32}
\uline{诺厄}五百岁时,生了\uline{闪}、\uline{含}和\uline{耶斐特}。
\footnote{\uline{拉默客}对\uline{诺厄}的祝福是一预言。此预言在8:22,9:8-17实现了。}

\textbf{第六章 }
\textbf{人类的败坏 }
\textsuperscript{1}
当人在地上开始繁殖,生养女儿时,
\textsuperscript{2}
天主的儿子见人的女儿美丽,就随意选取,作为妻子。
\textsuperscript{3}
上主于是说:“因为人即属于血肉,我的神不能常在他内;他的寿数只可到一百二十岁。”
\textsuperscript{4}
当天主的儿子与人的儿女结合生子时,在地上已有一些巨人,(以后也有),他们就是古代的英雄,著名的人物。
\footnote{洪水之罚是人类的败坏所引起的,这败坏的近因是因天主的儿子们娶了人的女儿们。所谓天主的儿子即恭敬天主的\uline{舍特}的子孙;人的女儿即指\uline{加音}不恭敬天主的子女。“我的神”此处是指天主赋于人的生活之力(2:7)。“血肉”即指易于沉湎于肉身之乐的人性。“巨人”的来历无法考定。巨人的事迹,多见于《旧约》中(\uwave{户}13:33;\uwave{申}3:11;\uwave{撒}上17;\uwave{巴}3:26-28等处)。此处作者并非说巨人是由天主的儿子和人的女儿生的,而只是说当天主的儿子和人的女儿结合时,地上已有巨人。这些巨人相似那些强悍善战,不认识智慧之道的巨人(\uwave{巴}3:26、27)。}

\textbf{上主决意消灭世界 }
\textsuperscript{5}
上主见人在地上的罪恶重大,人心天天所思念的无非是邪恶;
\textsuperscript{6}
上主遂后悔在地上造了人,心中很是悲痛。
\textsuperscript{7}
上主于是说:“我要将我所造的人,连人带野兽、爬虫和天空的飞鸟,都由地面上消灭,因为我后悔造了他们。”
\textsuperscript{8}
惟有\uline{诺厄}在上主眼中蒙受恩爱。
\footnote{有关洪水的记载(6-8章),大概来自两种有关洪水的记述。近东古代史家编纂史书,多只穿插古文件,而对文件中互异之处,多不加修改。有关洪水的传说,古代民族大都有所记载。本书所记就结构和体裁而言,与\uwave{叔默尔}和\uwave{巴比伦}的洪水神话有很多类似之处,但根本的区别很大,因本书中决无多神的不经之论;且本书所记是在教训世人几端道德和宗教的高深道理,如天主的正义、仁慈、召选、救恩和盟约的道理(9:1-17)。《旧约》的作者多以\uwave{诺厄}和洪水的事为天主施恩和惩罚的预象(\uwave{依}54:7-10;\uwave{德}44:17-19;\uwave{智}10:4)。《新约》多以洪水的事为公审判(\uwave{玛}24:37-39),或圣洗圣事的预象(\uwave{伯}前3:18-22)。}

\textbf{诺厄建造方舟 }
\textsuperscript{9}
以下是\uline{诺厄}的小史:\uline{诺厄}是他同时代唯一正义齐全的人,常同天主往来。
\textsuperscript{10}
他生了三个儿子:就是\uline{闪}、\uline{含}、和\uline{耶斐特}。
\textsuperscript{11}
大地已在天主面前败坏,到处充满了强暴。
\textsuperscript{12}
天主见大地已败坏,因为凡有血肉的人,品行在地上全败坏了,
\textsuperscript{13}
天主遂对\uline{诺厄}说:“我已决定要结果一切有血肉的人,因为他们使大地充满了强暴,我要将他们由大地上消灭。
\textsuperscript{14}
你要用柏木造一只方舟,舟内建造一些舱房,内外都涂上沥青。
\textsuperscript{15}
你要这样建造:方舟要有三百肘长,五十肘宽,三十肘高。
\textsuperscript{16}
方舟上层四面做上窗户,高一肘;门要安在侧面;方舟要分为上中下三层。
\textsuperscript{17}
看我要使洪水在地上泛滥,消灭天下一切有生气的血肉;凡地上所有的都要灭亡。
\textsuperscript{18}
但我要与你立约,你以及你的儿子、妻子和儿媳,要与你一同进入方舟。
\textsuperscript{19}
你要由一切有血肉的生物中,各带一对,即一公一母,进入方舟,与你一同生活;
\textsuperscript{20}
各种飞鸟、各种牲畜、地上所有的各种爬虫,皆取一对同你进去,得以保存生命。
\textsuperscript{21}
此外,你还应带上各种吃用的食物,贮存起来,作你和他们的食物。”
\textsuperscript{22}
\uline{诺厄}全照办了;天主怎样吩咐了他,他就怎样做了。

\textbf{第七章 }
\textbf{洪水灭世 }
\textsuperscript{1}
上主对\uline{诺厄}说:“你和你全家进入方舟,因为在这一世代,我看只有你在我面前正义。
\textsuperscript{2}
由一切洁净牲畜中,各取公母七对;由那些不洁净的牲畜中,各取公母一对;
\textsuperscript{3}
由天空的飞鸟中,也各取公母七对;好在全地面上传种。
\textsuperscript{4}
因为还有七天,我要在地上降雨四十天四十夜,消灭我在地面上所造的一切生物。”
\textsuperscript{5}
\uline{诺厄}全照上主吩咐他的做了。
\footnote{带进方舟的牲畜,自然洁净的多于不洁净的,因为洁净的可为食用,又可为祭献天主之用。见8:20-22;\uwave{肋}11。}
\textsuperscript{6}
当洪水在地上泛滥时,\uline{诺厄}已六百岁。
\textsuperscript{7}
\uline{诺厄}和他的儿子,他的妻子和他的儿媳,同他进了方舟,为躲避洪水。
\textsuperscript{8}
洁净的牲畜和不洁净的牲畜,飞鸟和各种在地上爬行的动物,
\textsuperscript{9}
一对一对地同\uline{诺厄}进了方舟;都是一公一母,照天主对他所吩咐的。
\textsuperscript{10}
七天一过,洪水就在地上泛滥。
\textsuperscript{11}
\uline{诺厄}六百岁那一年,二月十七日那天,所有深渊的泉水都冒出,天上的水闸都开放了;
\textsuperscript{12}
大雨在地上下了四十天四十夜。
\textsuperscript{13}
正在这一天,\uline{诺厄}和他的儿子\uline{闪}、\uline{含}、\uline{耶斐特},他的妻子和他的三个儿媳,一同进了方舟。
\textsuperscript{14}
他们八口和所有的野兽、各种牲畜、各种在地上爬行的爬虫、各种飞禽,
\textsuperscript{15}
一切有生气有血肉的,都一对一对地同\uline{诺厄}进了方舟。
\textsuperscript{16}
凡有血肉的,都是一公一母地进了方舟,如天主对\uline{诺厄}所吩咐的。随后上主关了门。

\textsuperscript{17}
洪水在地上泛滥了四十天;水不断增涨,浮起了方舟,方舟遂由地面上升起。
\textsuperscript{18}
洪水汹涌,在地上猛涨,方舟飘浮在水面上。
\textsuperscript{19}
洪水在地上一再猛涨,天下所有的高山都没了顶;
\textsuperscript{20}
洪水高出淹没的群山十有五肘。
\textsuperscript{21}
凡地上行动而有血肉的生物:飞禽、牲畜、野兽、在地上爬行的爬虫,以及所有的人全灭亡了;
\textsuperscript{22}
凡在旱地上以鼻呼吸的生灵都死了。
\textsuperscript{23}
这样,天主消灭了在地面上的一切生物,由人以至于牲畜、爬虫以及天空中的飞鸟,这一切都由地上消灭了,只剩下\uline{诺厄}和同他在方舟内的人物。
\textsuperscript{24}
洪水在地上泛滥了一百五十天。
\footnote{据古\uwave{希伯来}人的宇宙观:天的上边(1:7;\uwave{咏}104:3-13,148:4),地的下面都为水所包围;地下的水也叫深渊(\uwave{依}51:10;\uwave{咏}36:6;\uwave{亚}7:4)。关于洪水泛滥的日期,17节为四十天,24节为一百五十天,大概由两种不同的文献而来。}

\textbf{第八章 }
\textbf{洪水退落 }
\textsuperscript{1}
天主想起了\uline{诺厄}和同他在方舟内的一切野兽和牲畜,遂使风吹过大地,水渐渐退落;
\textsuperscript{2}
深渊的泉源和天上的水闸已关闭,雨也由天上停止降落,
\textsuperscript{3}
于是水逐渐由地上退去;过了一百五十天,水就低落了。
\textsuperscript{4}
七月十七日,方舟停在\uline{阿辣}\uline{辣特}山上。
\textsuperscript{5}
洪水继续减退,直到十月;十月一日,许多山顶都露出来。
\textsuperscript{6}
过了四十天,\uline{诺厄}开了在方舟上做的窗户,
\textsuperscript{7}
放了一只乌鸦;乌鸦飞去又飞回,直到地上的水都干了。
\textsuperscript{8}
\uline{诺厄}等待了七天,又放出了一只鸽子,看看水是否已由地面退尽。
\textsuperscript{9}
但是,因为全地面上还有水,鸽子找不着落脚的地方,遂飞回方舟;\uline{诺厄}伸手将它接入方舟内。
\textsuperscript{10}
再等了七天,他由方舟中又放出一只鸽子,
\textsuperscript{11}
傍晚时,那只鸽子飞回他那里,看,嘴里衔着一根绿的橄榄树枝;\uline{诺厄}于是知道,水已由地上退去。
\textsuperscript{12}
\uline{诺厄}又等了七天再放出一只鸽子;这只鸽子没有回来。
\footnote{通观有关洪水的记载(6:5-7,7:1-8:12),似乎全世界都为洪水所淹没,人类除\uline{诺厄}一家外全都消灭了。但若注意近东古代史家的渲染夸大的作风,“全地”、“天下”或类似的词句,仅指作者所知道的地方(\uwave{创}41:54、57;\uwave{宗}2:5等)。由此可知洪水的泛滥仅是局部的,而未遍及于全世界。淹死的人也只是作者所知道的人民,而不是全人类。细察本书作者的目的,只是记载启示的历史,或天主在世建立神国的历史,所以与启示或与\uline{以色列}人无关系的历史与人物一概不提。}

\textbf{诺厄出方舟 }
\textsuperscript{13}
\uline{诺厄}六百零一岁,正月初一,地上的水都干了,\uline{诺厄}就撤开方舟的顶观望,看见地面已干。
\textsuperscript{14}
二月二十七日,大地全干了。
\textsuperscript{15}
天主于是吩咐\uline{诺厄}说:
\textsuperscript{16}
“你和你的妻子、儿子及儿媳,同你由方舟出来;
\textsuperscript{17}
所有同你在方舟内的有血肉的生物:飞禽、牲畜和各种地上的爬虫,你都带出来,叫他们在地上滋生,在地上生育繁殖。”
\textsuperscript{18}
\uline{诺厄}遂同他的儿子、妻子及儿媳出来;
\textsuperscript{19}
所有的爬虫、飞禽和地上所有的动物,各依其类出了方舟。
\textsuperscript{20}
\uline{诺厄}给上主筑了一座祭坛,拿各种洁净的牲畜和洁净的飞禽,献在祭坛上,作为全燔祭。
\textsuperscript{21}
上主闻到了馨香,心里说:“我再不为人的缘故咒骂大地,因为人心的思念从小就邪恶;我也不再照我所作的打击一切生物了,
\textsuperscript{22}
只愿大地存在之日,稼穑寒暑,冬夏昼夜,循环不息。”
\footnote{由洪水之罚,作者教训人几端重要的道理:罪恶使大地回到了原始的混沌状态;罪恶连累了无灵的受造之物(6:13;\uwave{罗}8:19-22);方舟为圣教会的预象(\uwave{伯}前3:20、21);祭献的举行使人再蒙受天主的祝福;生命之可贵;天主同\uline{诺厄}所立的盟约也及于天地万物。}

\textbf{第九章 }
\textbf{人类复兴 }
\textsuperscript{1}
天主祝福\uline{诺厄}和他的儿子们说:“你们要滋养繁殖,充满大地。
\textsuperscript{2}
地上的各种野兽,天空的各种飞鸟,地上的各种爬虫和水中的各种游鱼,都要对你们表示惊恐畏惧:这一切都已交在你们手中。
\textsuperscript{3}
凡有生命的动物,都可作你们的食物;我将这一切赐给你们,有如以前赐给你们蔬菜一样;
\textsuperscript{4}
凡有生命,带血的肉,你们不可吃;
\textsuperscript{5}
并且,我要追讨害你们生命的血债:向一切野兽追讨,向人,向为弟兄的人,追讨人命。
\textsuperscript{6}
凡流人血的,他的血也要为人所流,因为人是造天主的肖像造的。
\textsuperscript{7}
你们要生育繁殖,在地上滋生繁衍。”
\footnote{天主祝福他们传生人类的话,像祝福原祖一样(1:28-30)。起初天主似乎禁止人吃肉,只准吃蔬菜果品(1:29),现今都准许了,但不准吃带血的肉。按古人的思想,血是生命之所在,是生活的动力。这生命直接来自天主(\uwave{申}12:16、23,15:23;\uwave{肋}3:17,7:26,17:10-14;\uwave{宗}15:29),为此禁止吃血。5、6两节为日后报复法的根据(\uwave{户}35:19;\uwave{出}21:23-25;\uwave{申}19:18-21)。}

\textbf{天主与诺厄立约 }
\textsuperscript{8}
天主对\uline{诺厄}和他的儿子们说:
\textsuperscript{9}
“看,我现在与你们和你们未来的后裔立约,
\textsuperscript{10}
并与同你们在一起的一切生物:飞鸟、牲畜和一切地上野兽,即凡由方舟出来的一切地上生物立约。
\textsuperscript{11}
我与你们立约:凡有血肉的,以后决不再受洪水湮灭,再没有洪水来毁灭大地。”
\textsuperscript{12}
天主说:“这是我在我与你们以及同你们在一起的一切生物之间,立约的永远标记:
\textsuperscript{13}
我把虹霓放在云间,作我与大地之间立约的标记。
\textsuperscript{14}
几时我兴云遮盖大地,云中要出现虹霓,
\textsuperscript{15}
那时我便想起我与你们以及各种属血肉的生物之间所立的盟约:这样水就不会再成为洪水,毁灭一切血肉的生物。
\textsuperscript{16}
几时虹霓在云间出现,我一看见,就想起在天主与地上各种属血肉的生物之间所立的永远盟约。”
\textsuperscript{17}
天主对\uline{诺厄}说:“这就是我在我与地上一切有血肉的生物之间,所立的盟约的标记。”
\footnote{此段所说的盟约原是天主无条件赐恩的许诺;天主以此诺言保证自然界从今以后不再受洪水之害。虹霓的自然现象在洪水前虽已有,但从今以后当作天主诺言的保证。}

\textbf{诺厄的诅咒与祝福 }
\textsuperscript{18}
\uline{诺厄}的儿子由方舟出来的,有\uline{闪}、\uline{含}、和\uline{耶斐特}。含是\uline{客纳罕}的父亲。
\textsuperscript{19}
这三人是\uline{诺厄}的儿子;人类就由这三人分布天下。
\textsuperscript{20}
\uline{诺厄}原是农夫,遂开始种植葡萄园。
\textsuperscript{21}
一天他喝酒喝醉了,就在自己的帐幕内脱去了衣服。
\textsuperscript{22}
\uline{客纳罕}的父亲\uline{含}看见了父亲赤身露体,遂去告诉外面的两个兄弟。
\textsuperscript{23}
\uline{闪}和\uline{耶斐特}二人于是拿了件外衣,搭在肩上,倒退着走进去,盖上父亲的裸体。他们的脸背着,没有看见父亲的裸体。
\textsuperscript{24}
\uline{诺厄}醒了后,知道了小儿对他作的事,
\textsuperscript{25}
就说:“\uline{客纳罕}是可咒骂的,给兄弟当最下贱的奴隶。”
\footnote{\uline{诺厄}诅咒了\uline{客纳罕}而未诅咒\uline{含},一、因天主早祝福了\uline{含}(1节);二、因\uline{客纳罕}的后裔日后放荡无耻。\uline{闪}特受祝福,因他是\uline{亚巴郎}的祖宗,因\uline{亚巴郎}的后裔(基督),万民将获得祝福。\uline{闪}受的祝福也及于\uline{耶斐特},到新约时代也及于\uline{含}和普世万民(\uwave{宗}2:5、9-11;\uwave{路}3:6)。}
\textsuperscript{26}
又说:“上主,闪的天主,应受赞美,\uline{客纳罕}应作他的奴隶。
\textsuperscript{27}
愿天主扩展\uline{耶斐特},使他住在\uline{闪}的帐幕内;\uline{客纳罕}应作他的奴隶。”
\textsuperscript{28}
洪水以后,\uline{诺厄}又活了三百五十年。
\textsuperscript{29}
\uline{诺厄}共活了九百五十岁死了。

\textbf{第十章 }
\textbf{诺厄三子的后裔 }
\textsuperscript{1}
以下是\uline{诺厄}的儿子\uline{闪}、\uline{含}、和\uline{耶斐特}的后裔。洪水以后,他们都生了子孙。

\textsuperscript{2}
\uline{耶斐特}的之孙:\uline{哥默尔}、\uline{玛哥格}、\uline{玛待}、\uline{雅汪}、\uline{突巴耳}、\uline{默舍客}和\uline{提辣斯}。
\textsuperscript{3}
\uline{哥默尔}的子孙:\uline{阿市}\uline{革纳次}、\uline{黎法特}和\uline{托加尔玛}。
\textsuperscript{4}
\uline{雅汪}的子孙:\uline{厄里沙}、\uline{塔尔史士}、\uline{基延}和\uline{多丹}。
\textsuperscript{5}
那些分布于岛上的民族,就是出于这些人:以上这些人按疆域、语言、宗教和国籍,都属\uline{耶斐特}的子孙。

\textsuperscript{6}
\uline{含}的子孙:\uline{雇士}、\uline{米兹辣殷}、\uline{普特}、和\uline{客纳罕}。
\textsuperscript{7}
\uline{雇士}的子孙:\uline{色巴}、\uline{哈威拉}、\uline{撒贝拉}、\uline{辣阿玛}和\uline{撒贝特加}。\uline{辣阿玛}的之孙:\uline{舍巴}和\uline{德丹}。
\footnote{古地理学家多以本章民族的名单为一种极宝贵的文献。各民族的分布是:北有\uline{耶斐特}的后裔,南有\uline{含}的后裔,在二者之间为\uline{闪}的后裔。有许多名字至今已不可考。}
\textsuperscript{8}
\uline{雇士}生\uline{尼默洛得},他是世上第一个强人。
\textsuperscript{9}
他在上主面前是个有本事的猎人,为此有句俗话说:“如在上主面前,有本领的猎人\uline{尼默洛得}。”
\textsuperscript{10}
他开始建国于\uline{巴比伦}、\uline{厄勒客}和\uline{阿加得},都在\uline{史纳尔}地域。
\textsuperscript{11}
他由那地方去了\uline{亚述},建设了\uline{尼尼微}、\uline{勒曷波特}城、\uline{加拉}
\textsuperscript{12}
和在\uline{尼尼微}与\uline{加拉}之间的\uline{勒森}(\uline{尼尼微}即是那大城)。
\footnote{\uline{尼默洛得}(实体书上是:尼默洛特,有误?)的故事,是上古巨人的故事之一,参见6章注一。}
\textsuperscript{13}
\uline{米兹辣殷}生\uline{路丁}人、\uline{阿纳明}人、\uline{肋哈宾}人、\uline{纳斐突歆}人、
\textsuperscript{14}
\uline{帕特洛斯}人、\uline{加斯路}人和\uline{加非}\uline{托尔}人。\uline{培肋}\uline{舍特}人即出自此族。
\textsuperscript{15}
\uline{客纳罕}生长子\uline{漆冬},以后生\uline{赫特}、
\textsuperscript{16}
\uline{耶步斯}人、\uline{阿摩黎}人、\uline{基尔加}\uline{士}人、
\textsuperscript{17}
\uline{希威}人、\uline{阿尔克}人、\uline{息尼}人、
\textsuperscript{18}
\uline{阿尔瓦得}人、\uline{责玛}\uline{黎}人和\uline{哈玛}\uline{特}人;以后,\uline{客纳罕}的宗教分散了,
\textsuperscript{19}
一致\uline{客纳罕}人的边疆,自\uline{漆冬}经过\uline{革辣尔}直到\uline{迦萨},又经过\uline{索多玛}、\uline{哈摩辣}、\uline{阿德玛}和\uline{责波殷},直到\uline{肋沙}。
\textsuperscript{20}
以上这些人按疆域、语言、宗族和国籍,都属\uline{含}的子孙。

\textsuperscript{21}
\uline{耶斐特}的长兄,即\uline{厄贝尔}所有子孙的祖先\uline{闪},也生了儿子。
\textsuperscript{22}
\uline{闪}的子孙:\uline{厄蓝}、\uline{亚述}、\uline{阿帕革}\uline{沙得}、\uline{路得}和\uline{阿兰}。
\textsuperscript{23}
\uline{阿兰}的之孙:\uline{伍兹}、\uline{胡耳}、\uline{革特尔}和\uline{玛士}。
\textsuperscript{24}
\uline{阿帕革}\uline{沙得}生\uline{舍拉};\uline{舍拉}生\uline{厄贝尔}。
\textsuperscript{25}
\uline{厄贝尔}生了两个儿子:一个名叫\uline{培肋格},因为在他的时代世界分裂了;他的兄弟名叫\uline{约刻堂}。
\textsuperscript{26}
\uline{约刻堂}生\uline{阿耳}\uline{摩达得}、\uline{舍肋夫}、\uline{哈匝玛委特}、\uline{耶辣}、
\textsuperscript{27}
\uline{哈多兰}、\uline{乌匝耳}、\uline{狄刻拉}、
\textsuperscript{28}
\uline{敖巴耳}、\uline{阿彼玛耳}、\uline{舍巴}、
\textsuperscript{29}
\uline{敖非尔}、\uline{哈威拉}和\uline{约巴布}:以上都是\uline{约刻堂}的子孙。
\textsuperscript{30}
他们居住的地域,从\uline{默沙}经过\uline{色法尔}直到东面的山地:
\textsuperscript{31}
以上这些人按疆域、语言、宗族和国籍,都属\uline{闪}的子孙:
\textsuperscript{32}
以上这些人按他们的出身和国籍,都是\uline{诺厄}子孙的家族;洪水以后,地上的民族都是由他们分出来的。
\footnote{本章的主旨有二:一、指出\uline{希伯来}人所知道的主要民族和他们的关系;二、说明\uline{以色列}人在他们中的地位。此外值得注意的是:作者将民族的分布情况说明之后,将范围逐渐缩小,一直缩到\uline{以色列}人的历史。作者列出\uline{诺厄}的后代之后,只说\uline{闪}的嫡系,再后只说\uline{特辣黑}的一支。选民的始祖\uline{亚巴郎}即由此支而来。}

\textbf{第十一章 }
\textbf{巴贝耳塔 }
\textsuperscript{1}
当时全世界只有一种语言和一样的话。
\textsuperscript{2}
当人们由东方迁移的时候,在\uline{史纳尔}地方找到了一块平原,就在那里住下了。
\textsuperscript{3}
他们彼此说:“来,我们这砖,用火烧透。”他们遂拿砖当石,拿沥青代灰泥。
\textsuperscript{4}
然后彼此说:“来,让我们建造一城一塔,塔顶摩天,好给我们作纪念,免得我们在全地面上分散了!”
\textsuperscript{5}
上主遂下来,要看看世人所造的城和塔。
\textsuperscript{6}
上主说:“看,他们都是一个民族,都说一样的语言。他们如今就开始做这事;以后他们所想做的,就没有不成功的了。
\textsuperscript{7}
来,我们下去,混乱他们的语言,使他们彼此语言不通。”
\textsuperscript{8}
于是上主将他们分散到全地面,他们遂停止建造那城。
\textsuperscript{9}
为此人称那地为“巴贝耳”,因为上主在那里混乱了全地的语言,且从那里将他们分山到全地面。
\footnote{原祖由于骄傲违背了天主的命令,洪水以后的人类也犯了同样的罪,因此也受了天主的惩罚。再说,人由于骄傲,彼此不和,必分离四散;分离即久,语言必渐分歧。人类的合一,只有赖\uline{基督}的神国方可实现(\uwave{宗}2:5-21;\uwave{默}7:9、10;\uwave{若}11:52)。古人所建的城和塔。即含有骄傲的意思(4-7节),因此日后的先知多以此城象征世上的恶势力(\uwave{哈}1:11;\uwave{依}11:11;\uwave{达}1,2)。}

\textbf{闪族的家谱 }
\textsuperscript{10}
以下是\uline{闪}的后裔:洪水后两年,\uline{闪}正一百岁,生了\uline{阿帕革}\uline{沙得};
\textsuperscript{11}
生\uline{阿帕革}\uline{沙得}后,\uline{闪}还活了五百年,也生了其他的儿女。
\textsuperscript{12}
\uline{阿帕革}\uline{沙得}三十五岁时,生了\uline{舍拉};
\textsuperscript{13}
生\uline{舍拉}后,\uline{阿帕革}\uline{沙得}还活了四百零三年,也生了其他的儿女。
\textsuperscript{14}
\uline{舍拉}三十岁时,生了\uline{厄贝尔};
\textsuperscript{15}
生\uline{厄贝尔}后,\uline{舍拉}还活了四百零三年,也生了其他的儿女。
\textsuperscript{16}
\uline{厄贝尔}三十四岁时,生了\uline{培肋格};
\textsuperscript{17}
生\uline{培肋格}后,\uline{厄贝尔}还活了四百三十年,也生了其他的儿女。
\textsuperscript{18}
\uline{培肋格}三十岁时,生了\uline{勒伍};
\textsuperscript{19}
生\uline{勒伍}后,\uline{培肋格}还活了二百零九年,也生了其他的儿女。
\textsuperscript{20}
\uline{勒伍}三十二岁时,生了\uline{色鲁格};
\textsuperscript{21}
生\uline{色鲁格}后,\uline{勒伍}还活了二百零七年,也生了其他的儿女。
\textsuperscript{22}
\uline{色鲁格}三十岁时,生了\uline{纳曷尔};
\textsuperscript{23}
生\uline{纳曷尔}后,\uline{色鲁格}还活了二百零七年,也生了其他的儿女。
\textsuperscript{24}
\uline{纳曷尔}活到二十九岁时,生了\uline{特辣黑};
\textsuperscript{25}
生\uline{特辣黑}后,\uline{纳曷尔}还活了一百一十九年,也生了其他的儿女。
\textsuperscript{26}
\uline{特辣黑}七十岁时,生了\uline{亚巴郎}、\uline{纳曷尔}、\uline{哈朗}。
\footnote{\uline{闪}族的十大祖先,相似洪水前的十大祖先(5章)。这个族谱,决不是全人类的整个历史,而只是启示历史的系统而已。}

\textbf{特辣黑的后裔 }
\textsuperscript{27}
以下是\uline{特辣黑}的后裔:\uline{特辣黑}生了\uline{亚巴郎}、\uline{纳曷尔}、\uline{哈朗};\uline{哈朗}生了\uline{罗特}。
\textsuperscript{28}
\uline{哈朗}在他的出生地,\uline{加色丁}人的\uline{乌尔},死在他父亲\uline{特辣黑}面前。
\textsuperscript{29}
\uline{亚巴郎}和\uline{纳曷尔}都娶了妻子,\uline{亚巴郎}的妻子名叫\uline{撒辣依};\uline{纳曷尔}的妻子名叫\uline{米耳加},她是\uline{哈朗}的女儿;\uline{哈朗}是\uline{米耳加}和\uline{依色加}的父亲。
\textsuperscript{30}
\uline{撒辣依}不生育,没有子女。
\textsuperscript{31}
\uline{特辣黑}带了自己的儿子\uline{亚巴郎}和孙子,即\uline{哈朗}的儿子\uline{罗特},并儿媳,即\uline{亚巴郎}的妻子\uline{撒辣依},一同由\uline{加色丁}的\uline{乌尔}出发,往\uline{客纳罕}地去;他们到了\uline{哈兰},就在那里住下了。
\textsuperscript{32}
\uline{特辣黑}死于\uline{哈兰},享寿二百零五岁。
\footnote{按作者的意思,真宗教的信仰只保存在\uline{闪}族的伟大后裔\uline{亚巴郎}那一支内,他是选民的始祖,众信友之父(\uwave{世}32:13;\uwave{依}51:1、2;\uwave{罗}4:11)。}

\begin{center}
	\textbf{后编 }
	\textbf{圣组史(12-50)}
\end{center}

\textbf{第十二章 }
\textbf{亚巴郎蒙召 }
\textsuperscript{1}
上主对\uline{亚巴郎}说:“离开你的故乡、你的家族和父家,往我指给你的地方去。
\textsuperscript{2}
我要使你成为一个大民族,我必祝福你,使你成名,成为一个福源。
\textsuperscript{3}
我要祝福那祝福你的人,咒骂那咒骂你的人;地上万民都要因你获得祝福。”
\textsuperscript{4}
\uline{亚巴郎}遂照上主的吩咐起了身,\uline{罗特}也同他一起走了。\uline{亚巴郎}离开\uline{哈兰}时,已七十五岁。
\textsuperscript{5}
他带了妻子\uline{撒辣依}、他兄弟的儿子\uline{罗特}和他在\uline{哈兰}积蓄的财物,获得的仆婢,一同往\uline{客纳罕}地去,终于到了\uline{客纳罕}地。
\textsuperscript{6}
\uline{亚巴郎}经过那地,直到了\uline{舍根}地\uline{摩勒}橡树区;当时\uline{客纳罕}人尚住在那地方。
\textsuperscript{7}
上主显现给\uline{亚巴郎}说:“我要将这地方赐给你的后裔。”\uline{亚巴郎}就在那里给显现于他的上主,筑了一座祭坛。
\textsuperscript{8}
从那里又迁移到\uline{贝特耳}东面山区,在那里搭了帐幕,西有\uline{贝特耳},东有\uline{哈依};他在那里又为上主筑了一座祭坛,呼求上主的名。
\textsuperscript{9}
以后\uline{亚巴郎}渐渐移往\uline{乃革布}区。
\footnote{\uline{亚巴郎}因坚信天主的话,遵命起身,遂蒙了天主祝福,使他的子孙成为一大民族,而且因了他子孙中所出生的\uline{默西亚},为了那些效法他信德的人,\uline{亚巴郎}成了他们幸福的泉源,因为他保存了真宗教和对\uline{默西亚}的希望,因而万民籍\uline{默西亚}得到了救赎,即如圣\uline{保禄}所说:“\uline{亚巴郎}的祝福在\uline{基督}\uline{耶稣}内普及于万民”(\uwave{迦}3:14;\uwave{希}11:8-12)。}

\textbf{亚巴郎去埃及 }
\textsuperscript{10}
其时那地方起了饥荒,\uline{亚巴郎}遂下到埃及,寄居在那里,因为那地方饥荒十分严重。
\textsuperscript{11}
当他要进\uline{埃及}时,对妻子\uline{撒辣依}说:“我知道你是个貌美的女人;
\textsuperscript{12}
\uline{埃及}人见了你,必要说:这是他的妻子;他们定要杀我,让你活着。
\textsuperscript{13}
所以请你说:你是我的妹妹,这样我因了你而必获优待,赖你的情面,保全我的生命。”
\textsuperscript{14}
果然,当\uline{亚巴郎}一到了\uline{埃及},\uline{埃及}人就注意了这女人实在美丽。
\textsuperscript{15}
法郎的朝臣也看见了她,就在法郎前赞她美丽;这女人就被带入法郎的宫中。
\textsuperscript{16}
\uline{亚巴郎}因了她果然蒙了优待,得了些牛羊、公驴、仆婢、母驴和骆驼。
\textsuperscript{17}
但是,上主为了\uline{亚巴郎}的妻子\uline{撒辣依}的事,降下大难打击了法郎和他全家。
\footnote{\uline{亚巴郎}为保护自己的生命财产的所言所行,虽不可取法,但由此可知,圣经中不拘善恶都记载下来,连圣祖的毛病罪过都一一记叙;而且圣经不但对启示的道理有所进展,而且对道德的概念也有所演进。}
\textsuperscript{18}
法郎遂叫\uline{亚巴郎}来说:“你对我作的是什么事?为什么你没有告诉我,她是你的妻子?
\textsuperscript{19}
为什么你说:她是我的妹妹,以致我娶了她做我的妻子?现在,你的妻子在这里,你带她去吧!”
\textsuperscript{20}
法郎于是吩咐人送走了\uline{亚巴郎}和他的妻子以及他所有的一切。

\textbf{第十三章 }
\textbf{亚巴郎回贝特耳 }
\textsuperscript{1}
\uline{亚巴郎}带了妻子和他所有的一切,与\uline{罗特}一同由\uline{埃及}上来,往\uline{乃革布}去。
\textsuperscript{2}
\uline{亚巴郎}有许多牲畜和金银。
\textsuperscript{3}
他由\uline{乃革布}逐渐往\uline{贝特耳}移动帐幕,到了先前他在\uline{贝特耳}与\uline{哈依}之间,支搭帐幕的地方,
\textsuperscript{4}
亦即他先前筑了祭坛,呼求上主之名的地方。

\textbf{亚巴郎与罗特分离 }
\textsuperscript{5}
与\uline{亚巴郎}同行的\uline{罗特},也有羊群、牛群和帐幕,
\textsuperscript{6}
那地方容不下他们住在一起,因为他们的产业太多,无法住在一起。
\textsuperscript{7}
牧放\uline{亚巴郎}牲畜的人与牧放\uline{罗特}牲畜的人,时常发士口角,——当时\uline{客纳罕}人和\uline{培黎齐}人尚住在那里。
\textsuperscript{8}
\uline{亚巴郎}遂对\uline{罗特}说:“在我与你,我的牧人与你的牧人之间,请不要发士口角,因为我们是至亲。
\textsuperscript{9}
所有的地方不都是在你面前吗?请你与我分开。你若往左,我就往右;你若往右,我就往左。”
\textsuperscript{10}
\uline{罗特}举目看见\uline{约但}河整个平原,直到\uline{左哈尔}一带全有水灌溉,——这是在上主消灭\uline{索多玛}和\uline{哈摩辣}以前的事,——有如上主的乐园,有如\uline{埃及}地。
\textsuperscript{11}
\uline{罗特}选了\uline{约但}河的整个平原,遂向东方迁移;这样,他们就彼此分开了:
\textsuperscript{12}
\uline{亚巴郎}住在\uline{客纳罕}地;\uline{罗特}住在平原的城市中,渐渐移动帐幕,直到\uline{索多玛}。
\textsuperscript{13}
\uline{索多玛}人在上主面前罪大恶极。

\textsuperscript{14}
\uline{罗特}与\uline{亚巴郎}分离以后,上主对\uline{亚巴郎}说:“请你举起眼来,由你所在的地方,向东西南北观看;
\textsuperscript{15}
凡你看见的地方,我都要永远赐给你和你的后裔。
\textsuperscript{16}
我要使你的后裔有如地上的灰尘;如果人能数清地上的灰尘,也能数清你的后裔。
\textsuperscript{17}
你起来,纵横走遍这地,因为我要将这地赐给你。”
\textsuperscript{18}
于是\uline{亚巴郎}移动了帐幕,来到\uline{赫贝龙}的\uline{玛默勒}橡树区居住,在那里给上主筑了一座祭坛。
\footnote{论及\uline{亚巴郎}的宽宏大量,金口圣\uline{若望}说:“长辈对晚辈,老者对少年的\uline{罗特},叔父对侄子,言谈有如兄弟,如平辈,让侄子任意选择。”本章给予的教训是:\uline{罗特}喜爱优裕的生活,没有躲避恶劣的环境,而遭受了重罚;\uline{亚巴郎}的慷慨却受到了厚报。当知本章为18:20、21,19:4-19的前奏。}

\textbf{第十四章 }
\textbf{亚巴郎突袭获胜 }
\textsuperscript{1}
那时\uline{史纳尔}王\uline{阿默}\uline{辣斐耳},\uline{厄拉}\uline{撒尔}王\uline{阿黎约客},\uline{厄蓝}王\uline{革多尔}\uline{老默尔},\uline{哥因}王\uline{提达耳},
\textsuperscript{2}
兴兵攻击\uline{索多玛}王\uline{贝辣},\uline{哈摩辣}王\uline{彼尔沙},\uline{阿德玛}王\uline{史纳布},\uline{责波殷}王\uline{舍默}\uline{贝尔}及\uline{贝拉}即\uline{左哈尔}王。
\textsuperscript{3}
那些王子会合于\uline{息丁}山谷,即今日的\uline{盐海}。
\textsuperscript{4}
他们十二年之久隶属于\uline{革多尔}\uline{老默尔},在十三年上就背叛了。
\textsuperscript{5}
在十四年上,\uline{革多尔}\uline{老默尔}率领与他联盟的君王前来,在\uline{阿市}\uline{塔特}\uline{卡尔}\uline{纳殷}击败了\uline{勒法因},在\uline{哈木}击败了\uline{组斤},在\uline{克黎}\uline{雅塔殷}平原击败了\uline{厄明},
\textsuperscript{6}
在\uline{曷黎}人的\uline{色依尔}山击败了\uline{曷黎}人,一直杀到靠近旷野的\uline{厄耳帕兰};
\textsuperscript{7}
然后回军转到\uline{恩米市}\uline{帕特},即\uline{卡德士},征服了\uline{阿玛肋克}人的全部领土,也征服了住在\uline{哈匝宗}\uline{塔玛尔}的\uline{阿摩黎}人。
\textsuperscript{8}
\uline{索多玛}王\uline{哈摩辣}王,\uline{阿德玛}王,\uline{责波殷}王和\uline{贝拉}即\uline{左哈尔}王,于是出来,在\uline{息丁}山谷列阵,
\textsuperscript{9}
与\uline{厄蓝}王\uline{革多尔}\uline{老默尔},\uline{哥因}王\uline{提达耳},\uline{史纳尔}王\uline{阿默}\uline{辣斐耳}和\uline{厄拉}\uline{撒尔}王\uline{阿黎约客}交战:四个王子敌对五个王子。
\textsuperscript{10}
\uline{息丁}山谷遍地都是沥青坑;\uline{索多玛}王和\uline{哈摩辣}王逃跑时都跌在坑里;其余的人都逃到山里去了。
\textsuperscript{11}
那四个王子劫走了\uline{索多玛}和\uline{哈摩辣}所有的财物和一切食粮,
\textsuperscript{12}
连\uline{亚巴郎}兄弟的儿子\uline{罗特}和他的财物也带走了,因为那时他正住在\uline{索多玛}。
\textsuperscript{13}
有个逃出的人跑来,将这事告诉了\uline{希伯来}人\uline{亚巴郎},他那时住在\uline{阿摩黎}人\uline{玛默勒}的橡树区;这\uline{阿摩黎}人原是\uline{亚巴郎}的盟友\uline{厄市}\uline{苛耳}和\uline{阿乃尔}的兄弟。
\textsuperscript{14}
\uline{亚巴郎}一听说他的亲人被人掳去,遂率领家中的步兵三百一十八人,直追至\uline{丹};
\textsuperscript{15}
夜间又和自己的仆人分队袭击,将他们击败,直追至\uline{大马}\uline{士革}以北的\uline{曷巴},
\textsuperscript{16}
夺回了所有的财物,连他的亲属\uline{罗特}和他的财物,以及妇女和人民都夺回来了。
\footnote{本段所记一定是根据了一段很古的文献。从前的史学家多以为\uline{阿默}\uline{辣斐耳}是古\uline{巴比伦}著名的帝王\uline{哈慕辣彼};但近来的史家已不主张此说。史家都以为本章所述的史事,很适合公元前十九世纪近东各国的情况。\uwave{创}的作者把此段插入本章,愿意显示\uline{亚巴郎}在\uline{客纳罕}已蒙受了天主的祝福,成为一位有力量的酋长;愿说明他对\uline{罗特}的宽宏大量,尤其愿说明他要成为众信友之父,同未来真宗教的中心\uline{耶路撒冷}所有的关系。为了这些原故,他受了\uline{耶路撒冷}王兼司祭的祝福。}

\textbf{默基瑟德祝福亚巴郎 }
\textsuperscript{17}
\uline{亚巴郎}击败\uline{革多尔}\uline{老默尔}和与他联盟的王子回来时,\uline{索多玛}王出来,到沙委山谷,即“君王山谷”迎接他。
\textsuperscript{18}
\uline{撒冷}王\uline{默基}\uline{瑟德}也带了饼酒来,他是至高者天主的司祭,
\footnote{\uline{撒冷}即\uline{耶路撒冷}(\uwave{咏}76:3,110:4)。\uline{默基}\uline{瑟德}虽不属于\uline{亚巴郎}的家族,但和他有同样的信仰。身兼君王和司祭的\uline{默基}\uline{瑟德}是为君王为司祭的\uline{默西亚}的预象,因他所献的饼酒预兆了新约的圣经祭献(\uwave{希}7:1-28)。}
\textsuperscript{19}
祝福他说:“愿\uline{亚巴郎}蒙受天地的主宰,至高者天主的祝福!
\textsuperscript{20}
愿将你的敌人交于你手中的至高者的天主受赞美!”\uline{亚巴郎}遂将所得的,拿出十分之一,给了\uline{默基}\uline{瑟德}。
\textsuperscript{21}
\uline{索多玛}王对\uline{亚巴郎}说:“请你将人交给我,财物你都拿去罢!”
\textsuperscript{22}
\uline{亚巴郎}却对\uline{索多玛}王说:“我向上天、至高者天主、天地的主宰举手起誓:
\textsuperscript{23}
凡属于你的,连一根线,一根鞋带,我也不拿,免得你说:我使\uline{亚巴郎}发了财。
\textsuperscript{24}
除了仆从吃用了的以外,我什么也不要;至于与我同行的人\uline{阿乃尔}、\uline{厄市}\uline{苛耳}和\uline{玛默勒}所应得的一分,应让他们拿去。”
\clearpage

% 出谷纪引言
\phantomsection
\addcontentsline{toc}{chapter}{出谷纪引言}
\chapter{出谷纪引言}

“五书”的第二部紧接《创世纪》,继续记述\uline{以色列}子孙的事迹。我国译本名为《出埃及记》,或《出谷纪》。后一译名似乎更合适本书的深意,因为天主将\uline{以色列}子民由\uline{埃及}救出的事实,实是他要把全人类由罪恶的深渊中,救出来的预象和初步实现。

本书上编(1-18章)记述\uline{以色列}(雅各伯)的子孙,在\uline{埃及}国所受的压迫。但天主决未忘却他向\uline{亚巴郎}和其他圣祖所许的诺言,所以选拔了\uline{梅瑟}为民族的救星,叫他领导自己的同胞出离\uline{埃及},以伟大的奇迹,救他们脱离奴隶的生活,领他们来到\uline{西乃}旷野,一路上使他们经验了天主大能的呵护,令他们坚心信赖天主的照顾。

中编(19-24章)和下编(25-40章)可称为全旧约的中心,记载天主在\uline{西乃}山上,将自己启示给\uline{以色列}子民,给他们颁布了十诫和法律,籍\uline{梅瑟}与他们立了盟约,使他们成为特选的民族,成为神权政体的国民。从那时起,天主自己作他们的君王和领袖,住在百姓中的帐幕内;并任命\uline{肋未}的子孙,代百姓在会幕内服役,行祭献天主的大礼。

本书虽可说是\uline{以色列}子民,建国立宪的一部极有关系的史书,但因为没有提及\uline{埃及}王朝或君王的名号,故此本书的史事,究竟发生在何时,无法确定。据一般经学家的推究,大约是在纪元前十三世纪中叶。

本书就神学观点来说,具有极崇高的价值,给人类启示了天主的超然存在,和他的唯一性以及至圣性;同时也显示了他对自己百姓的慈爱和照顾。至于他向人类要求的,是对他应怀有赤诚的信赖,以及知恩报爱等美德。由于本书所记载的史事,大都含有预象的意义,为此为新约教会的生活和礼仪,也有其特殊的价值。
\clearpage
% 出谷纪(出)
\phantomsection
\addcontentsline{toc}{chapter}{出谷纪}
\chapter{出谷纪(出)}

\clearpage

% 肋未纪引言
\phantomsection
\addcontentsline{toc}{chapter}{肋未纪引言}
% 标题
\chapter*{}

% 右页眉
\rhead{}
\clearpage
% 肋未纪(肋)
\phantomsection
\addcontentsline{toc}{chapter}{肋未纪}
% 分栏
\setlength\columnsep{0.8cm}
\begin{multicols}{2}

% 标题
\chapter*{肋未纪}

% 右页眉
\rhead{}

% 正文

\end{multicols}
\clearpage

% 户籍纪引言
\phantomsection
\addcontentsline{toc}{chapter}{户籍纪引言}
% 标题
\chapter*{}

% 右页眉
\rhead{}
\clearpage
% 户籍纪(户)
\phantomsection
\addcontentsline{toc}{chapter}{户籍纪}
% 分栏
\setlength\columnsep{0.6cm}
\begin{multicols}{2}

% 标题
\chapter*{户籍纪}

% 右页眉
\rhead{}

% 正文

\end{multicols}
\clearpage

% 申命纪引言
\phantomsection
\addcontentsline{toc}{chapter}{申命纪引言}
% 标题
\chapter*{}

% 右页眉
\rhead{}
\clearpage
% 申命纪(申)
\phantomsection
\addcontentsline{toc}{chapter}{申命纪}
\chapter{申命纪}
\clearpage

% 若苏厄书引言
\phantomsection
\addcontentsline{toc}{chapter}{若苏厄书引言}
\input{introduction/08_Introduction_to_the_Joshua}
\clearpage
% 若苏厄书(苏)
\phantomsection
\addcontentsline{toc}{chapter}{若苏厄书}
% 分栏
\setlength\columnsep{0.8cm}
\begin{multicols}{2}

% 标题
\chapter*{若苏厄书}

% 右页眉
\rhead{}

% 正文

\end{multicols}
\clearpage

% 民长纪引言
\phantomsection
\addcontentsline{toc}{chapter}{民长纪引言}
\input{introduction/09_Introduction_to_the_Judges}
\clearpage
% 民长纪(民)
\phantomsection
\addcontentsline{toc}{chapter}{民长纪}
民长纪

民
\clearpage

% 卢德传引言
\phantomsection
\addcontentsline{toc}{chapter}{卢德传引言}
\input{introduction/10_Introduction_to_the_Ruth}
\clearpage
% 卢德传(卢)
\phantomsection
\addcontentsline{toc}{chapter}{卢德传}
卢德传

卢
\clearpage

% 撒慕尔纪引言
\phantomsection
\addcontentsline{toc}{chapter}{撒慕尔纪引言}
\input{introduction/11_Introduction_to_the_Samuel}
\clearpage
% 撒慕尔纪上(撒上)
\phantomsection
\addcontentsline{toc}{chapter}{撒慕尔纪上}
撒慕尔纪上(撒上)
\clearpage
% 撒慕尔纪下(撒下)
\phantomsection
\addcontentsline{toc}{chapter}{撒慕尔纪下}
% 分栏
\setlength\columnsep{0.6cm}
\begin{multicols}{2}

% 标题
\chapter*{撒慕尔纪下}

% 右页眉
\rhead{}

% 正文

\end{multicols}
\clearpage

% 列王纪引言
\phantomsection
\addcontentsline{toc}{chapter}{列王纪引言}
\input{introduction/12_Introduction_to_the_Kings}
\clearpage
% 列王纪上(列上)
\phantomsection
\addcontentsline{toc}{chapter}{列王纪上}
% 分栏
\setlength\columnsep{0.8cm}
\begin{multicols}{2}

% 标题
\chapter*{列王纪上}

% 右页眉
\rhead{}

% 正文

\end{multicols}
\clearpage
% 列王纪下(列下)
\phantomsection
\addcontentsline{toc}{chapter}{列王纪下}
% 分栏
\setlength\columnsep{0.8cm}
\begin{multicols}{2}

% 标题
\chapter*{列王纪下}

% 右页眉
\rhead{}

% 正文

\end{multicols}
\clearpage

% 编年纪引言
\phantomsection
\addcontentsline{toc}{chapter}{编年纪引言}
\input{introduction/13_Introduction_to_the_Chronicles}
\clearpage
% 编年纪上(编上)
\phantomsection
\addcontentsline{toc}{chapter}{编年纪上}
编年纪上(编上)
\clearpage
% 编年纪下(编下)
\phantomsection
\addcontentsline{toc}{chapter}{编年纪下}
% 标题
\chapter*{编年纪下}

% 右页眉
\rhead{}

% 正文

\clearpage

% 厄斯德拉引言
\phantomsection
\addcontentsline{toc}{chapter}{厄斯德拉引言}
\input{introduction/14_Introduction_to_the_Ezra}
\clearpage
% 厄斯德拉上(厄上)
\phantomsection
\addcontentsline{toc}{chapter}{厄斯德拉上}
厄斯德拉上(厄上)
\clearpage
% 厄斯德拉下(厄下)
\phantomsection
\addcontentsline{toc}{chapter}{厄斯德拉下(亦称“乃赫米雅”)}
\chapter{厄斯德拉下}
\clearpage

% 多俾亚传引言
\phantomsection
\addcontentsline{toc}{chapter}{多俾亚传引言}
\input{introduction/15_Introduction_to_the_Tobit}
\clearpage
% 多俾亚传(多)
\phantomsection
\addcontentsline{toc}{chapter}{多俾亚传}
% 分栏
\setlength\columnsep{0.6cm}
\begin{multicols}{2}

% 标题
\chapter*{多俾亚传}

% 右页眉
\rhead{}

% 正文

\end{multicols}
\clearpage

% 友弟德传引言
\phantomsection
\addcontentsline{toc}{chapter}{友弟德传引言}
\input{introduction/16_Introduction_to_the_Judith}
\clearpage
% 友弟德传(友)
\phantomsection
\addcontentsline{toc}{chapter}{友弟德传}
% 标题
\chapter*{友弟德传}

% 右页眉
\rhead{}

% 正文

\clearpage

% 艾斯德尔传引言
\phantomsection
\addcontentsline{toc}{chapter}{艾斯德尔传引言}
\input{introduction/17_Introduction_to_the_Esther}
\clearpage
% 艾斯德尔传(艾)
\phantomsection
\addcontentsline{toc}{chapter}{艾斯德尔传}
艾斯德尔传(艾)
\clearpage

% 玛加伯上下引言
\phantomsection
\addcontentsline{toc}{chapter}{玛加伯上下引言}
\input{introduction/18_Introduction_to_the_Maccabees}
\clearpage
% 玛加伯上(加上)
\phantomsection
\addcontentsline{toc}{chapter}{玛加伯上}
% 分栏
\setlength\columnsep{0.8cm}
\begin{multicols}{2}

% 标题
\chapter*{玛加伯上}

% 右页眉
\rhead{}

% 正文

\end{multicols}
\clearpage
% 玛加伯下(加下)
\phantomsection
\addcontentsline{toc}{chapter}{玛加伯下}
\chapter{玛加伯下}
\clearpage

% 智慧书概论
\phantomsection
\addcontentsline{toc}{chapter}{智慧书概论}
\input{introduction/19_Introduction_to_Wisdom_Books}
\clearpage
% 约伯传引言
\phantomsection
\addcontentsline{toc}{chapter}{约伯传引言}
\input{introduction/20_Introduction_to_the_Job}
\clearpage
% 约伯传(约)
\phantomsection
\addcontentsline{toc}{chapter}{约伯传}
% 分栏
\setlength\columnsep{0.8cm}
\begin{multicols}{2}

% 标题
\chapter*{约伯传}

% 右页眉
\rhead{}

% 正文

\end{multicols}
\clearpage

% 圣咏集引言
\phantomsection
\addcontentsline{toc}{chapter}{圣咏集引言}
\input{introduction/21_Introduction_to_the_Psalms}
\clearpage
% 圣咏集(咏)
\phantomsection
\addcontentsline{toc}{chapter}{圣咏集}
% 标题
\chapter*{圣咏集}

% 右页眉
\rhead{}

% 正文

\clearpage

% 箴言引言
\phantomsection
\addcontentsline{toc}{chapter}{箴言引言}
\input{introduction/22_Introduction_to_the_Proverbs}
\clearpage
% 箴言(箴)
\phantomsection
\addcontentsline{toc}{chapter}{箴言}
箴言

箴
\clearpage

% 训道篇引言
\phantomsection
\addcontentsline{toc}{chapter}{训道篇引言}
\input{introduction/23_Introduction_to_the_Ecclesiastes}
\clearpage
% 训道篇(训)
\phantomsection
\addcontentsline{toc}{chapter}{训道篇}
% 标题
\chapter*{训道篇}

% 右页眉
\rhead{}

% 正文

\clearpage

% 雅歌引言
\phantomsection
\addcontentsline{toc}{chapter}{雅歌引言}
\input{introduction/24_Introduction_to_the_Song_of_Solomon}
\clearpage
% 雅歌(歌)
\phantomsection
\addcontentsline{toc}{chapter}{雅歌}
% 分栏
\setlength\columnsep{0.8cm}
\begin{multicols}{2}

% 标题
\chapter*{雅歌}

% 右页眉
\rhead{}

% 正文

\end{multicols}
\clearpage

% 智慧篇引言
\phantomsection
\addcontentsline{toc}{chapter}{智慧篇引言}
\input{introduction/25_Introduction_to_the_Wisdom}
\clearpage
% 智慧篇(智)
\phantomsection
\addcontentsline{toc}{chapter}{智慧篇}
% 分栏
\setlength\columnsep{0.8cm}
\begin{multicols}{2}

% 标题
\chapter*{智慧篇}

% 右页眉
\rhead{}

% 正文

\end{multicols}
\clearpage

% 德训篇引言
\phantomsection
\addcontentsline{toc}{chapter}{德训篇引言}
\input{introduction/26_Introduction_to_the_Sirach}
\clearpage
% 德训篇(德)
\phantomsection
\addcontentsline{toc}{chapter}{德训篇}
德训篇(德)
\clearpage

% 先知书总论
\phantomsection
\addcontentsline{toc}{chapter}{先知书总论}
\input{introduction/27_Introduction_to_the_Prophets}
\clearpage

% 依撒意亚引言
\phantomsection
\addcontentsline{toc}{chapter}{依撒意亚引言}
\input{introduction/28_Introduction_to_the_Isaiah}
\clearpage
% 依撒意亚(依)
\phantomsection
\addcontentsline{toc}{chapter}{依撒意亚}
依撒意亚(依)
\clearpage

% 耶肋米亚引言
\phantomsection
\addcontentsline{toc}{chapter}{耶肋米亚引言}
\input{introduction/29_Introduction_to_the_Jeremiah}
\clearpage
% 耶肋米亚(耶)
\phantomsection
\addcontentsline{toc}{chapter}{耶肋米亚}
\chapter{耶肋米亚}
\clearpage

% 哀歌引言
\phantomsection
\addcontentsline{toc}{chapter}{哀歌引言}
\input{introduction/30_Introduction_to_the_Lamentations}
\clearpage
% 哀歌(哀)
\phantomsection
\addcontentsline{toc}{chapter}{哀歌}
\chapter{哀歌}
\clearpage

% 巴路克书引言
\phantomsection
\addcontentsline{toc}{chapter}{巴路克书引言}
\input{introduction/31_Introduction_to_the_Baruch}
\clearpage
% 巴路克(巴)
\phantomsection
\addcontentsline{toc}{chapter}{巴路克}
% 分栏
\setlength\columnsep{0.6cm}
\begin{multicols}{2}

% 标题
\chapter*{巴路克}

% 右页眉
\rhead{}

% 正文

\end{multicols}
\clearpage

% 厄则克耳引言
\phantomsection
\addcontentsline{toc}{chapter}{厄则克耳引言}
\input{introduction/32_Introduction_to_the_Ezekiel}
\clearpage
% 厄则克耳(则)
\phantomsection
\addcontentsline{toc}{chapter}{厄则克耳}
厄则克耳

则
\clearpage

% 达尼尔书引言
\phantomsection
\addcontentsline{toc}{chapter}{达尼尔书引言}
\input{introduction/33_Introduction_to_the_Daniel}
\clearpage
% 达尼尔(达)
\phantomsection
\addcontentsline{toc}{chapter}{达尼尔}
\chapter{达尼尔}
\clearpage


% 十二小先知总论
\phantomsection
\addcontentsline{toc}{chapter}{十二小先知总论}
\input{introduction/34_Introduction_to_Twelve_Little_Prophets}
\clearpage

% 欧瑟亚(欧)
\phantomsection
\addcontentsline{toc}{chapter}{欧瑟亚}
\chapter{欧瑟亚}
\clearpage

% 岳厄尔(岳)
\phantomsection
\addcontentsline{toc}{chapter}{岳厄尔}
% 分栏
\setlength\columnsep{0.6cm}
\begin{multicols}{2}

% 标题
\chapter*{岳厄尔}

% 右页眉
\rhead{}

% 正文

\end{multicols}
\clearpage

% 亚毛斯(亚)
\phantomsection
\addcontentsline{toc}{chapter}{亚毛斯}
\chapter{亚毛斯}
\clearpage

% 亚北底亚(北)
\phantomsection
\addcontentsline{toc}{chapter}{亚北底亚}
% 标题
\chapter*{亚北底亚}

% 右页眉
\rhead{}

% 正文

\clearpage

% 约纳(纳)
\phantomsection
\addcontentsline{toc}{chapter}{约纳}
% 分栏
\setlength\columnsep{0.6cm}
\begin{multicols}{2}

% 标题
\chapter*{约纳}

% 右页眉
\rhead{}

% 正文

\end{multicols}
\clearpage

% 米该亚(米)
\phantomsection
\addcontentsline{toc}{chapter}{米该亚}
米该亚

米
\clearpage

% 纳鸿(鸿)
\phantomsection
\addcontentsline{toc}{chapter}{纳鸿}
纳鸿

鸿
\clearpage

% 哈巴谷(哈)
\phantomsection
\addcontentsline{toc}{chapter}{哈巴谷}
% 分栏
\setlength\columnsep{0.6cm}
\begin{multicols}{2}

% 标题
\chapter*{哈巴谷}

% 右页眉
\rhead{}

% 正文

\end{multicols}
\clearpage

% 索福尼亚(索)
\phantomsection
\addcontentsline{toc}{chapter}{索福尼亚}
\chapter{索福尼亚}
\clearpage

% 哈盖(盖)
\phantomsection
\addcontentsline{toc}{chapter}{哈盖}
% 分栏
\setlength\columnsep{0.6cm}
\begin{multicols}{2}

% 标题
\chapter*{哈盖}

% 右页眉
\rhead{}

% 正文

\end{multicols}
\clearpage

% 匝加利亚(匝)
\phantomsection
\addcontentsline{toc}{chapter}{匝加利亚}
\chapter{匝加利亚}
\clearpage

% 玛拉基亚(拉)
\phantomsection
\addcontentsline{toc}{chapter}{玛拉基亚}
% 分栏
\setlength\columnsep{0.8cm}
\begin{multicols}{2}

% 标题
\chapter*{玛拉基亚}

% 右页眉
\rhead{}

% 正文

\end{multicols}
\clearpage

\part{\kaishu 新\quad约}

% 新约全书导论
\phantomsection
\addcontentsline{toc}{chapter}{新约全书导论}
% 标题
\chapter*{新约全书导论}

% 右页眉
\rhead{新约全书导论}

“新约全书”是\uline{耶稣}死后,由其宗徒弟子,在天主圣神的默感与引导之下,所写成的经典汇集。此汇集由第二世纪起即称为《新约书》,或简称《新约》。称之为“约”,因为其中所讲论的,是天主与人类所立的盟约;称之为“新”,以别于“旧约”。“旧约”是天主与\uline{以}民在\uline{西乃}山上所立的圣约,而“新约”是\uline{基督}以自己的圣血与圣死,在天主与人间,所建立的救恩圣约(参阅\uwave{玛}26:28;\uwave{谷}14:24等处)。

“新约全书”,按圣教会古老的传授,共计二十七卷。

《历史书》五卷:《玛窦福音》、《马尔谷福音》、《路加福音》、《若望福音》和《宗徒大事录》。

《训诲书》二十一卷:圣\uline{保禄}书信十四封:《罗马书》、《格林多》前后二书、《迦拉达书》、《厄弗所书》、《斐理伯书》、《哥罗森书》、《得撒洛尼》前后二书、《弟茂德》前后二书、《弟铎书》、《费肋孟书》和《希伯来书》;公函七封:《雅各伯书》、《伯多禄》前后二书、《若望》一、二、三书并《犹达书》。

《先知书》一卷:《若望默示录》。

《新约全书》,除《玛窦福音》的原文为\uline{阿剌美}文外,都是用\uline{希腊}文写成的。这些似乎有些奇怪,因为按当时\uline{耶稣}在世时,和宗徒最初讲道时所用的语言,本来都是\uline{阿剌美}语,并且全部《新约》作者,除圣\uline{路加}外,又都是\uline{犹太}人;那么为什么不用本国文字编写呢?其理由是因为只有《玛窦福音》是写给\uline{巴力斯坦}的\uline{犹太}人,而其余的书都是写给说\uline{希腊}话的基督徒,其中很少有通晓\uline{阿剌美}语的;更何况《新约》又是向天下万民所公布的;因此以当时\uline{罗马}帝国内所通行的\uline{希腊}语编写,是很自然的事。

《新约全书》(或《新经》),就宗教方面来说,远远超过《旧约全书》(或《古经》),因为天主在旧约时代只是“多次并以多种方式,藉着先知对祖先说过话”;然而在新约时代却是“藉着子对我们说了话”(\uwave{希}1:1)。如此,旧约的启示在新约内才得以圆满;旧约的预许在新约内才得以实现。所以吾人除非认识《新约》,决不能完全明了《旧约》;为此,可说《新约全书》实是世界上最重要和最宝贵的作品。
\clearpage

% 福音总论
\phantomsection
\addcontentsline{toc}{chapter}{福音总论}
% 标题
\chapter*{福音总论}

% 右页眉
\rhead{福音总论}

“福音”一词,按其字意,原指“喜讯”;但按《新约》作者采用此词的意义来说,乃是指天主子\uline{耶稣}降生为人,从天上给人类带来的启示,和在他完成救赎工程以后,诸宗徒向万民所宣布的得救喜讯。

这喜讯的传报,最初只靠口头的宣讲,稍后才有不少人士把\uline{耶稣}的生平与宣讲笔之于书,因而产生了“福音”的著作。按\uwave{路}1:1的记载,这样的著作在当时已为数不少,可是圣教会自初只承认《玛窦》、《马尔谷》、《路加》、《若望》这四部《福音》为受默感而写的经典,并著录在正经书目内,其他名为“福音”的著作,概著录为伪经。

“福音”书虽有四部,但所传述的“福音”却只是一个,因为四圣史所撰述的是同一的喜讯,只是在所采用形式上有所不同而已。前三部《福音》,无论是在取材和结构上,或在用字上,大致可说相同,甚至可并列对照,一望而知彼此间所有的关系,因而有“对观福音”之称。这三部《福音》之所以如此相同,是因为前三圣史记述了大体相同的“宗徒教理讲授”:\uline{玛窦}记述了\uline{耶路撒冷}教会的传授,\uline{马尔谷}记述了\uline{罗马}教会的传授,\uline{路加}记述了\uline{安提约基}\uline{雅}教会的传授。\uline{若望}因见前三《福音》已流传于世,没有重述的必要,遂由自己记忆所及,采取了一些有关的材料,在第一世纪末叶,针对当时人事环境的需要,编著了自己的《福音》,其目的是在攻击方兴的异端邪说。

四《福音》虽然不是狭义的史书,但就信实性来说:世界上没有一部史书可与之相比,因为各位作者,或是目睹所述之事的宗徒(\uline{玛窦}、\uline{若望}),或是宗徒的亲传弟子(\uline{马尔谷}、\uline{路加}),他们所依据的,全是亲历其事人物的口述;况且《福音》成书时,尚有不少耳闻目睹的证人生存于世。

四《福音》内不但包含了有关信仰绝对重要的道理,而且也给世人提示了诸德的完美模范,基督徒成全的最高理想:即为我们降生成人的天主圣子。
\clearpage

% 玛窦福音引言
\phantomsection
\addcontentsline{toc}{chapter}{玛窦福音引言}
% 标题
\chapter*{玛窦福音引言}

% 右页眉
\rhead{玛窦福音(玛)引言}

第一部《福音》的作者是圣\uline{玛窦}宗徒。\uline{玛窦}又名\uline{肋未},是\uline{阿耳斐}的儿子(\uwave{谷}2:14)。他在\uline{耶稣}召叫之前,曾在\uline{葛法翁}作过税吏。他一被召,即刻舍弃一切,跟随了\uline{耶稣}(\uwave{玛}9:9;\uwave{谷}2:13、14;\uwave{路}5:27、28)。\uline{耶稣}升天后,他先在\uline{巴力斯坦}一带,给自己的同胞宣讲福音多年,然后动身往外方传教去了。最后死在何处何时,史无确证。圣教会从古以来,即认他为一位为主殉道的宗徒,每年九月二十一日庆祝他的瞻礼。

据最古的传授,圣教会始终认为圣\uline{玛窦}是第一部《福音》的作者;这也可由《福音》书内的暗示得到证明:例如\uline{马尔谷}与\uline{路加}记载十二位宗徒名单时,只记了\uline{玛窦}的名字,然而在第一部《福音》内,于“\uline{玛窦}”名字前却加上了受人歧视的“税吏”头衔,可知原作者对自己的职位,毫不避讳。

《玛窦福音》的原著为\uline{阿剌美}文,因为是为\uline{巴力斯坦}的\uline{犹太}人写的,这是自古以来圣教会一致公认的事。此书后来不知由何人译为\uline{希腊}文。本《福音》因为是写给归化的\uline{犹太}人,因此特别力证\uline{耶稣}\uline{基督}即是天主所预许及先知所预言的“默西亚”。虽然大多数\uline{犹太}人否认\uline{耶稣}为默西亚,并把他置于死地:然而他却由死者中光荣复活,并建立了自己的教会作为天国在世上的开端,继续他救世的使命。由于这个特殊的目的,\uline{玛窦}比其他三位圣史,更强调先知们的预言在\uline{耶稣}身上全应验了。

本书的著作地点,大概是\uline{耶路撒冷}。至于著作时期,原文可说是写于其他《福音》之前,大约著于公元50年左右;现行的\uline{希腊}译本,大概是成于《马尔谷》和《路加》两福音之后,约在公元70年左右。

本书记述\uline{耶稣}言行,并未全按编年的次第,而是出于作者的匠心独运。他把\uline{耶稣}公开传教的整个生活分作五段,每段先记事,后记言。此五段即是:(一)3-7;(二)8-10;(三)11-13:53;(四)13:54-18;(五)19-25。

本《福音》因是四《福音》中材料最丰富的一部,在结构上又是最有系统的一部,为此本《福音》在教会内应用最广,引用最多。
\clearpage
% 玛窦福音(玛)
\phantomsection
\addcontentsline{toc}{chapter}{玛窦福音}
% 分栏
\setlength\columnsep{0.8cm}
\begin{multicols}{2}

% 标题
\chapter*{玛窦福音}

% 右页眉
\rhead{}

% 正文

\end{multicols}
\clearpage

% 马尔谷福音引言
\phantomsection
\addcontentsline{toc}{chapter}{马尔谷福音引言}
\input{introduction/38_Introduction_to_the_Mark}
\clearpage
% 马尔谷福音(谷)
\phantomsection
\addcontentsline{toc}{chapter}{马尔谷福音}
马尔谷福音

谷
\clearpage

% 路加福音引言
\phantomsection
\addcontentsline{toc}{chapter}{路加福音引言}
\input{introduction/39_Introduction_to_the_Luke}
\clearpage
% 路加福音(路)
\phantomsection
\addcontentsline{toc}{chapter}{路加福音}
% 分栏
\setlength\columnsep{0.6cm}
\begin{multicols}{2}

% 标题
\chapter*{路加福音}

% 右页眉
\rhead{}

% 正文

\end{multicols}
\clearpage

% 若望福音引言
\phantomsection
\addcontentsline{toc}{chapter}{若望福音引言}
\input{introduction/40_Introduction_to_the_John}
\clearpage
% 若望福音(若)
\phantomsection
\addcontentsline{toc}{chapter}{若望福音}
% 分栏
\setlength\columnsep{0.8cm}
\begin{multicols}{2}

% 标题
\chapter*{若望福音}

% 右页眉
\rhead{}

% 正文

\end{multicols}
\clearpage

% 宗徒大事录引言
\phantomsection
\addcontentsline{toc}{chapter}{宗徒大事录引言}
\input{introduction/41_Introduction_to_the_Acts}
\clearpage
% 宗徒大事录(宗)
\phantomsection
\addcontentsline{toc}{chapter}{宗徒大事录}
宗徒大事录

宗
\clearpage

% 保禄书信总论
\phantomsection
\addcontentsline{toc}{chapter}{保禄书信总论}
\input{introduction/42_Paul's_Epistles}
\clearpage

% 罗马书引言
\phantomsection
\addcontentsline{toc}{chapter}{罗马书引言}
\input{introduction/43_Introduction_to_the_Romans}
\clearpage
% 罗马书(罗)
\phantomsection
\addcontentsline{toc}{chapter}{罗马书}
罗马书

罗
\clearpage

% 格林多前后书引言
\phantomsection
\addcontentsline{toc}{chapter}{格林多前后书引言}
\input{introduction/44_Introduction_to_the_Corinthians}
\clearpage
% 格林多前书(格前)
\phantomsection
\addcontentsline{toc}{chapter}{格林多前书}
\chapter{格林多前书}
\clearpage
% 格林多后书(格后)
\phantomsection
\addcontentsline{toc}{chapter}{格林多后书}
\chapter{格林多后书}
\clearpage

% 迦拉达书引言
\phantomsection
\addcontentsline{toc}{chapter}{迦拉达书引言}
% 标题
\chapter*{迦拉达书引言}

% 右页眉
\rhead{迦拉达书(迦)引言}

\uline{迦拉达}人原是古代\uline{法国}南部\uline{高卢}人的一支,公元前三世纪先迁徙至\uline{小亚细亚}中部,以后逐渐扩展至\uline{小亚细亚}南部。公元前25年,\uline{奥古斯都}将他们的地域通划为\uline{迦拉达}皇帝省。\uline{保禄}在第一次传教行程中,同\uline{巴尔纳伯}已在\uline{迦拉达}南部创立了不少教会(\uwave{宗}13:14,14:24),在第二次传教时又到此处巡视(\uwave{宗}16:1-5);为此,我们认为本书即是写给此处的各教会:即\uline{丕息狄雅}的\uline{安提约基雅}、\uline{依科尼雍}、\uline{吕斯特辣}和\uline{德尔贝}等地的教会。

\uline{保禄}写这本书的动机,是因为他听说在\uline{迦拉达}各教会内,有些\uline{犹太}主义保守派人,到处散布邪说,攻击\uline{保禄},扬言\uline{保禄}既不属“十二宗徒集团”,当然不是真宗徒,因而他所传的福音,也不是\uline{基督}的真福音;并且声言:人为得救,必须遵守\uline{梅瑟}法律,并行割损礼。\uline{保禄}见教会处于这种重大的危机中,便写了这封书信,以驳斥这些邪说。

本书写作的动机既如上述,\uline{保禄}在口授这封信时,心情自然不免激昂愤慨,措辞不免有些锋利;但这不但不消减他对信友的慈爱,反而更彰显出他对信友的关怀,以及对\uline{基督}的满腔热爱。

至于本书写于何时何地的问题,虽没有决定性的答案,但从本书的内容与\uline{保禄}其他的书信比较来看,当在《格林多后书》之后及《罗马》书之前,即大约写于公元57年年底,地点当在\uline{格林多}或\uline{马其顿}。

本书除序言(1:1-5)和结论(6:11-18)外,可分为三段:第一段、\uline{保禄}极力证明自己的宗徒职权(1:6-2:21);第二段、力陈旧约法律为成义毫无作用,人为成义必须有赖对\uline{基督}的信德(3-4);第三段、略论人成义后所获得的地位,和几项针对信友实际生活的劝言(5:1-6:10)。
\clearpage
% 迦拉达书(迦)
\phantomsection
\addcontentsline{toc}{chapter}{迦拉达书}
% 标题
\chapter*{迦拉达书}

% 右页眉
\rhead{迦拉达书(迦)}

% 正文
\textbf{第一章\quad致候辞\quad}
\textsuperscript{1}
我\uline{保禄}宗徒——我蒙召为宗徒,并非由于人,也并非藉着人,而是由于\uline{耶稣}\uline{基督}和使他由死者中复活的天主父
\footnote{\uline{保禄}在致候辞内,已将本书的两个主题提出:(一)他作宗徒是直接由天主的召选;(二)\uline{耶稣}的死是人得救的唯一根源。他把这两个主题点出,反击\uline{犹太}保守派对他为宗徒的攻击,并对救恩所有的错误。}
——
\textsuperscript{2}
我和同我在一起的众弟兄,致书给\uline{迦拉达}众教会:
\textsuperscript{3}
愿恩宠与平安由天主我们的父及主\uline{耶稣}\uline{基督}赐与你们!
\textsuperscript{4}
这\uline{基督}按照天主我们父的旨意,为我们的罪恶舍弃了自己,为救我们脱离此邪恶的世代。
\textsuperscript{5}
愿光荣归于天主,至于无穷之世!阿门。

\begin{center}
	\textbf{\large{\songti 福音来自天主}}
\end{center}

\textbf{哀信友动摇之速\quad}
\textsuperscript{6}
我真奇怪,你们竟这样快离开了那以\uline{基督}的恩宠召叫你们的天主,而归向了另一福音;
\textsuperscript{7}
其实,并没有别的福音,只是有一些人扰乱你们,企图改变\uline{基督}的福音而已。
\textsuperscript{8}
但是,无论谁,即使是我们,或是从天上降下的一位天使,若给你们宣讲的福音,与我们给你们所宣讲的福音不同,当受诅咒。
\textsuperscript{9}
我们以前说过,如今我再说:谁若给你们宣讲福音与你们所接受的不同,当受诅咒。
\textsuperscript{10}
那么,我如今是讨人的喜爱,或是讨天主的喜爱呢?难道我是寻求人的欢心吗?如果我还求人的欢心,我就不是\uline{基督}的仆役。
\footnote{使\uline{保禄}深恶痛绝的,是他的教友“离开了”他所讲的福音,而“归向”即投向了\uline{犹太}保守派所讲的“福音”(守\uline{梅瑟}法律),因为使人得救的福音只有一个,决不能因人因时而有所变更。}

\textbf{福音与宗徒职位全由天主而来\quad}
\textsuperscript{11}
弟兄们,我告诉你们;我所宣讲的福音,并不是由人而来的,
\textsuperscript{12}
因为,我不是由人得来的,也不是由人学来的,而是由\uline{耶稣}\uline{基督}的启示得来的。
\textsuperscript{13}
你们一定听说过,我从前尚在\uline{犹太}教中的行动:我怎样激烈地迫害过天主的教会,竭力想把她消灭;
\textsuperscript{14}
我在\uline{犹太}教中比我本族许多同年的人更为急进,对我祖先的传授更富于热枕。
\textsuperscript{15}
但是,从母胎中已选拔我,以恩宠召叫我的天主,却决意
\textsuperscript{16}
将他的圣子启示给我,叫我在异民中传扬他。我当时没有与任何人商量,
\textsuperscript{17}
也没有上\uline{耶路撒冷}去见那些在我以前作宗徒的人,我立即去了\uline{阿剌伯},然后又回到了\uline{大马士革}。
\textsuperscript{18}
此后,过了三年,我才上\uline{耶路撒冷}去拜见\uline{刻法},在他那里逗留了十五天,
\textsuperscript{19}
除了主的兄弟\uline{雅各伯},我没有看见别的宗徒。
\textsuperscript{20}
我给你们写的都是真的,我在天主前作证,我决没有说谎。
\textsuperscript{21}
此后,我往\uline{叙利亚}和\uline{基里基雅}地域去了。
\textsuperscript{22}
那时,\uline{犹太}境内属于\uline{基督}的各教会,都没有见过我的面;
\textsuperscript{23}
只是听说过:“那曾经迫害我们的,如今却传扬他曾经想消灭的信仰了。”
\textsuperscript{24}
他们就为了我而光荣天主。
\footnote{\uline{保禄}所宣讲的福音不是由人传给他的,而是由\uline{耶稣}亲自启示的;他作宗徒也是由于天主的召选(见\uwave{宗}9:1-9)。18节特提出他拜见\uline{伯多禄}的事,表示他承认\uline{伯多禄}在宗徒中的优越地位。}

\textbf{第二章\quad其他宗徒赞同保禄\quad}
\textsuperscript{1}
过了十四年,我同\uline{巴尔纳伯}再上\uline{耶路撒冷}去,还带了\uline{弟铎}同去。
\textsuperscript{2}
我是受了启示而上去的;我在那里向他们陈述了我在异民中间所将的福音,和私下向那些有权威的人陈述过,免得我白白地奔跑,或者徒然奔走了。
\textsuperscript{3}
但是,即连跟我的\uline{弟铎},他虽是\uline{希腊}人,也没有被强迫领受割损,
\textsuperscript{4}
因为,有些潜入的假弟兄,曾要他受割损;这些人潜入了教会,是为窥探我们在\uline{基督}\uline{耶稣}内所享有的自由,好使我们再成为奴隶;
\textsuperscript{5}
可是对他们,我们连片刻时间也没有让步屈服,为使福音的真理在你们中保持不变。
\footnote{“过了十四年”(1节),即\uline{保禄}归化后第十四年(公元49或50年,见\uline{宗}15:2、3)。“再上\uline{耶路撒冷}去”,按1:18他已去过一次;且说“是受了启示而上去的”(2节),可知\uline{保禄}作事常随天主的指示。“那些有权威的”,即9节所称为柱石的三位大宗徒。由“免得我白白地奔跑……”一句,可见他对传福音多么谨慎,决不一意孤行;但对于不合福音真谛,迫使由外邦归化的信友受割损的事(\uwave{宗}15:1),他始终不肯让步。}
\textsuperscript{6}
至于那些所谓有权威的人——不论他们以前是何等人物,与我毫不相干;天主决不顾情面——那些有权威的人,也没有另外吩咐我什么;
\textsuperscript{7}
反而他们看出来,我是受了委托,向未受割损的人,宣传福音,就如\uline{伯多禄}被委派向受割损的人宣传福音一样;
\textsuperscript{8}
因为,那叫\uline{伯多禄}为受割损的人致力尽宗徒之职的,也叫我为外邦人致力尽宗徒之职。
\textsuperscript{9}
所以,他们一认清了所赋与我的恩宠,那称为柱石的\uline{雅各伯}、\uline{刻法}和\uline{若望},就与我和\uline{巴尔纳伯}握手,表示通力合作,叫我们往外邦人那里去,而他们却往受割损的人那里去。
\textsuperscript{10}
他们只要我们怀念穷人;对这一点我也曾尽力行了。
\footnote{这里所提的“穷人”是指在\uline{耶路撒冷}因受迫害而成了穷人的信徒(见\uline{格}前16:1;\uline{格}后8,9)。}

\textbf{安提约基雅的事件\quad}
\textsuperscript{11}
但是,当\uline{刻法}来到\uline{安提约基}\uline{雅}时,我当面反对了他,因为他有可责的地方。
\textsuperscript{12}
原来由\uline{雅各伯}那里来了一些人,在他们未到以前,他惯常同外邦人一起吃饭;可是他们一来到了,他因怕那些受割损的人,就退避了,自己躲开。
\textsuperscript{13}
其余的\uline{犹太}人也都跟他一起装假,以致连\uline{巴尔纳伯}也受了他们的牵引而装假。
\textsuperscript{14}
我一见他们的行为与福音的真理不合,就当着众人对\uline{刻法}说:“你是\uline{犹太}人,竟按照外邦人的方式,而不按照\uline{犹太}人的方式过活,你怎么敢强迫外邦人\uline{犹太}化呢?”
\footnote{1:8及本章2-9各节,充分表现\uline{保禄}决以\uline{伯多禄}为教会的元首,为宗徒之长;正因如此,他的一举一动更能影响信友,因此\uline{保禄}才指摘\uline{伯多禄}。由这样的责难,可见\uline{保禄}的勇敢,\uline{伯多禄}的谦逊。}
\textsuperscript{15}
我们生来是\uline{犹太}人,而不是出于外邦民族的罪人;
\textsuperscript{16}
可是我们知道:人成义不是由于遵行法律,而只是因着对\uline{耶稣}\uline{基督}的信仰,所以我们也信从了\uline{基督}\uline{耶稣},才能由于对\uline{基督}的信仰,而不由于遵行法律成义,因为由于遵守法律,任何人都不得成义。
\textsuperscript{17}
如果我们在\uline{基督}内求成义的人,仍如他们一样被视为罪人,那么,\uline{基督}岂不是成了支持罪恶的人了吗?绝对不是。
\textsuperscript{18}
如果我把我所拆毁的,再修建起来,我就证明我是个罪犯。
\textsuperscript{19}
其实,我已由于法律而死于法律了,为能生活于天主;我已同\uline{基督}被钉在十字架上了,
\textsuperscript{20}
所以,我生活已不是我生活,而是\uline{基督}在我内生活;我现今在肉身内生活,是生活在对天主子的信仰内;他爱了我,且为我舍弃了自己。
\textsuperscript{21}
我决不愿使天主的恩宠无效,因为,如果成义是赖着法律,那么,\uline{基督}就白白地死了。
\footnote{17节的意思是说:如果我们还以为必须守\uline{梅瑟}法律,才可在天主前成义,那么,我们因曾作证只赖\uline{耶稣}的恩宠才可成义得救,就成了罪人,如此,那废弃旧约法律的\uline{耶稣}就应为我们的罪负责。关于19节,参阅\uwave{罗}7注一。}

\begin{center}
	\textbf{\large{\songti 福音使人自由}}
\end{center}

\textbf{第三章\quad成义是由于信德\quad}
\textsuperscript{1}
无知的\uline{迦拉达}人啊!被钉在十字架上的\uline{耶稣}\uline{基督},已活现地摆在你们眼前,谁又迷惑了你们呢?
\textsuperscript{2}
我只愿向你们请教这一点:你们领受了圣神,是由于遵行法律呢?还是由于听信福音呢?
\textsuperscript{3}
你们竟这样无知吗?你们以圣神开始了,如今又愿以肉身结束吗?
\textsuperscript{4}
你们竟白白受了这么多的苦吗?果然是白白地吗?
\textsuperscript{5}
天主赐与你们圣神,并在你们中间施展了德能,是因为你们遵行法律呢?还是因为你们听信福音呢?
\textsuperscript{6}
经上这样记载说:“\uline{亚巴郎}信了天主,天主就以此算为他的正义。”
\textsuperscript{7}
为此你们该晓得:具有信德的人,才是\uline{亚巴郎}的子孙。
\textsuperscript{8}
圣经预见天主将使异民凭信德成义,就向\uline{亚巴郎}预报福音说:“万民都要因你获得祝福。”
\textsuperscript{9}
可见那些具有信德的人,与有信德的\uline{亚巴郎}同蒙祝福。
\footnote{\uline{保禄}在此段论人成义,不是因为遵守\uline{梅瑟}法律,而是因着信德。他先举出\uline{迦拉达}人成义的事实,是由于领受圣神,而不是有赖割损;然后举出\uline{亚巴郎}以信德而非以割损成义为例证。6,8两节所引,见\uline{创}15:6,12:3。}

\textbf{基督废弃了奴隶性的法律\quad}
\textsuperscript{10}
反之,凡是依恃遵行法律的,都应受咒骂,因为经上记载说:“凡不持守《律书》上所记载的一切,而依照遵行的,是可咒骂的。”
\textsuperscript{11}
所以很明显的,没有一个人能凭法律在天主前成义,因为经上说:“义人因信德而生活。”
\textsuperscript{12}
但是法律并非以信德为本,只说:“遵行法令的,必因此获得生命。”
\textsuperscript{13}
但\uline{基督}由法律的咒骂中赎出了我们,为我们成了可咒骂的,因为经上记载说:“凡被悬在木架上的,是可咒骂的。”
\textsuperscript{14}
这样天主使\uline{亚巴郎}所蒙受的祝福,在\uline{基督}\uline{耶稣}内普及于万民,并使我们能藉着信德领受所应许的圣神。
\footnote{\uline{保禄}进一步说明,那不愿以信仰\uline{基督},而只愿以守法律成义的人,使自己成了可咒骂的(参阅\uwave{申}27:26),因为人按自己的力量,不能全守\uline{梅瑟}的条文,所以为成义只应依仗\uline{基督}的恩宠。他为拯救人类,死在十字架上,使自己成了法律所“咒骂的”(参阅\uwave{申21:23}),为使万民因他的死而得到祝福。}
\textsuperscript{15}
弟兄们!就常规来说:连人的遗嘱,如果是正式成立的,谁也不得废除或增订。
\textsuperscript{16}
那么,恩许是向\uline{亚巴郎}和他的后裔所许诺的,并没有说“后裔们”,好像是向许多人说的,而是向一个人,即“你的后裔”,就是指\uline{基督}。
\textsuperscript{17}
我是说:天主先前所正式立定的誓约,决不能为四百三十年以后成立的法律所废除,以致使恩许失效。
\textsuperscript{18}
如果承受产业是由于法律,就已不是由于恩许;但天主是由于恩许把产业赐给了\uline{亚巴郎}。
\footnote{此段(15-18节)证明使人获得救恩的,不是由于法律,而是由于天主向\uline{亚巴郎}起誓所许的恩许;这恩许只赖\uline{基督}而实现了。}

\textsuperscript{19}
那么,为什么还有法律呢?它是为显露过犯而添设的,等他所恩许的后裔来到,它原是藉着天使,经过中人的手而立定的。
\textsuperscript{20}
可是如果出于单方,就不需要中人了,而天主是由单方赐与了恩许。
\textsuperscript{21}
那么,法律相反天主的恩许吗?绝对不是。如果所立定的法律能赐与人生命,正义就的确是出于法律了。
\textsuperscript{22}
但是圣经说过:一切人都被禁锢在罪恶权下,好使恩许藉着对\uline{基督}\uline{耶稣}的信仰,归于相信的人。
\textsuperscript{23}
在“信仰”尚未来到以前,我们都被禁锢在法律的监守之下,以期待“信仰”的出现。
\textsuperscript{24}
这样,法律就成了我们的启蒙师,领我们归于\uline{基督},好使我们由于信仰而成义。
\textsuperscript{25}
但是“信仰”一到,我们就不再处于启蒙师权下了。
\footnote{法律的目的是为维护恩许,是为预防\uline{以色列}人背弃天主,敬拜邪神,沾染外教人的恶习。法律好像启蒙师(护送儿童上学的奴隶),给\uline{以}民准备信仰\uline{默西亚}的道路。19节参阅\uwave{罗}7:7-25。23,25两节中的“信仰”指示新约制度。}
\textsuperscript{26}
其实你们众人都藉着对\uline{基督}\uline{耶稣}的信仰,成了天主的子女,
\textsuperscript{27}
因为你们凡是领了洗归于\uline{基督}的,就是穿上了\uline{基督}:
\textsuperscript{28}
不再分\uline{犹太}人或\uline{希腊}人,奴隶或自由人,男人或女人,因为你们众人在\uline{基督}\uline{耶稣}内已成了一个。
\textsuperscript{29}
如果你们属于\uline{基督},那么,你们就是\uline{亚巴郎}的后裔,就是按照恩许作承继的人。
\footnote{就像成人不再受启蒙师管辖,同样,\uline{犹太}人自从\uline{基督}来了以后,就不再受为启蒙师的法律所约束了。他们既不用遵守旧法律,外邦人更不用遵守,因为人在受洗后,都同样成了天主的子女,都有同样的地位,都穿上\uline{基督}与他密切结合,分享一切恩许。}

\textbf{第四章\quad宠爱使人为天主的子女\quad}
\textsuperscript{1}
再说:承继人几时还是孩童,虽然他是一切家业的主人,却与奴隶没有分别,
\textsuperscript{2}
仍属于监护人和代理人的权下,直到父亲预定的期限。
\textsuperscript{3}
同样,当我们以前还作孩童的时候,我们是隶属于今世的蒙学权下;
\textsuperscript{4}
但时期一满,天主就派遣了自己的儿子来,生于女人,生于法律之下,
\textsuperscript{5}
为把在法律之下的人赎出来,使我们获得义子的地位。
\textsuperscript{6}
为证实你们确实是天主的子女,天主派遣了自己儿子的圣神,到我们心内喊说:“阿爸,父啊!”
\textsuperscript{7}
所以你已不再是奴隶,而是儿子了;如果是儿子,赖天主的恩宠,也成了承继人。
\footnote{作者以未成年的孩童,比作旧约时代的人类,那时无论\uline{犹太}人或外邦人,都好像奴隶,“属于今世蒙学的权下”。所谓“蒙学”是指\uline{犹太}人的法律及教外人对道德观所有的原理,二者都不健全。到了\uline{默西亚}时代,\uline{耶稣}救赎人脱离了法律,提高了人的地位,使人成为天主的子女,人才可呼天主为“阿爸,父啊”(参见\uwave{罗}8:15;\uwave{谷}14:36)。}

\textbf{接受法律是再愿为奴\quad}
\textsuperscript{8}
当你们还不认识天主的时候,服事了一些本来不是神的神;
\textsuperscript{9}
但如今你们认识了天主,更好说为天主所认识;那么,你们怎么又再回到那无能无用的蒙学里去,情愿再作他们的奴隶呢?
\textsuperscript{10}
你们竟又谨守某日、某月、某时、某年!
\textsuperscript{11}
我真为你们担心,怕我白白地为你们辛苦了。
\footnote{\uline{迦拉达}人今若仍遵守\uline{犹太}人的法律,便是自愿放弃自由,再作奴隶。}

\textbf{劝迦拉达人不要受人愚弄\quad}
\textsuperscript{12}
弟兄们!我恳求你们要像我一样,因为我曾一度也像你们一样。你们一点也没有亏负过我。
\textsuperscript{13}
你们知道:当我初次给你们宣讲福音时,正当我身患重病,
\textsuperscript{14}
虽然我的病势为你们是个试探,你们却没有轻看我,也没有厌弃我,反接待我有如一位天主的天使,有如\uline{基督}\uline{耶稣}。
\textsuperscript{15}
那么,你们当日所庆幸的在哪里呢?我敢为你们作证:如若可能,你们那时也会把你们的眼睛挖出来给我。
\footnote{由13节确知,\uline{保禄}在\uline{迦}省传福音时曾患重病,是否与\uwave{格}后12:7所说相同,无法断定。}
\textsuperscript{16}
那么,只因我给你们说实话,就成了你们的仇人吗?
\textsuperscript{17}
那些人对你们表示关心,并不怀好意;他们只是愿意使你们与我隔绝,好叫你们也关心他们。
\textsuperscript{18}
受人关心固然是好的,但应怀好意,且该常常如此,并不单是我在你们中间的时候。
\textsuperscript{19}
我的孩子们!我愿为你们再受产痛,直到\uline{基督}在你们内形成为止。
\textsuperscript{20}
恨不得我现今就在你们跟前,改变我的声调,因为我对你们实在放心不下。
\footnote{\uline{保禄}以母亲自比:为使\uline{迦拉达}人成为完善的基督徒,愿再受产痛之苦来苦心栽培他们。}

\textbf{基督徒才是自由的子女\quad}
\textsuperscript{21}
你们愿意属于法律的,请告诉我:你们没有听见法律说什么吗?
\textsuperscript{22}
法律曾记载说:\uline{亚巴郎}有两个儿子:一个生于婢女,一个生于自由的妇人。
\textsuperscript{23}
那生于婢女的,是按常例而生的;但那生于自由妇人的,却是因恩许而生的。
\textsuperscript{24}
这都含有寓意:那两个妇人是代表两个盟约:一是出于\uline{西乃}山,生子为奴,那即是\uline{哈加尔}——
\textsuperscript{25}
\uline{西乃}山是在\uline{阿剌伯}——这\uline{哈加尔}相当于现在的\uline{耶路撒冷},因为\uline{耶路撒冷}与她的子女同为奴隶。
\textsuperscript{26}
然而那属于天上的\uline{耶路撒冷}却是自由的,她就是我们的母亲:
\textsuperscript{27}
诚如经上记载说:“不生育的石女,喜乐吧!未经产痛的女人,欢呼高唱吧!因为被弃者的子女比有夫者的子女还多。”
\footnote{“婢女”\uline{哈加尔}是旧约或\uline{西乃}山盟约的预像;“自由的妇人”\uline{撒辣}是新约或\uline{熙雍}山盟约的预像。由婢女所生的儿子当然是奴隶,不能承继\uline{亚巴郎}的家业;同样,属旧约或\uline{西乃}山下盟约的人也是奴隶,对\uline{亚巴郎}那神妙的家业无权承继;只有由自由妇人所生的儿子,是恩许的儿子,有权承受天主对\uline{亚巴郎}所许的恩许。27节参阅\uwave{依}54:1。}
\textsuperscript{28}
弟兄们!你们像\uline{依撒格}一样,是恩许的子女。
\textsuperscript{29}
但是,先前那按常例而生的怎样迫害了那按神恩而生的,如今还是这样。
\textsuperscript{30}
然而经上说了什么?“你将婢女和她的儿子赶走,因为婢女的儿子不能与自由妇人的儿子,一同承受家业。”
\textsuperscript{31}
所以,弟兄们,我们不是婢女的子女,而是自由妇人的子女。
\footnote{\uline{保禄}说:信友如\uline{依撒格}一样是恩许的子女(28,31两节),如果是恩许的子女,那么,当然是自由的了,不应再受\uline{梅瑟}法律的束缚。29,30两节见\uwave{创}21:9-12。}

\begin{center}
	\textbf{\large{\songti 基督徒的正当生活}}
\end{center}

\textbf{第五章\quad恩许的自由决不可放弃\quad}
\textsuperscript{1}
\uline{基督}解救了我们,是为使我们获得自由;所以你们要站稳,不可再让奴隶的轭束缚住你们。
\textsuperscript{2}
请注意,我\uline{保禄}告诉你们:若你们还愿意受割损,\uline{基督}对你们就没有什么益处。
\textsuperscript{3}
我再向任何自愿受割损的人声明:他有遵守全部法律的义务。
\textsuperscript{4}
你们这些靠法律寻求成义的人,是与\uline{基督}断绝了关系,由恩宠上跌了下来。
\footnote{若有信友仍以割损为得救的条件,他就丧失了从\uline{基督}得来的自由及恩宠,而又成了奴隶及罪犯。}
\textsuperscript{5}
至于我们,我们却是依赖圣神,由于信德,怀着能成义的希望,
\textsuperscript{6}
因为在\uline{基督}\uline{耶稣}内,割损或不割损都算不得什么,唯有以爱德行事的信德,才算什么。
\footnote{如果我们成义是因信赖\uline{耶稣},那么割损与不割损,便自然没有什么价值了(\uwave{格}前7:19)。关于“以爱德行事的信德”,参阅\uwave{格}前13:2;\uwave{雅}2:18,22。}
\textsuperscript{7}
以前你们跑得好!有谁拦阻了你们去追随真理呢?
\textsuperscript{8}
这种劝诱,决不是出自那召选你们的天主。
\textsuperscript{9}
少许的酵母就能使整个面团发酵。
\textsuperscript{10}
我在主内信任你们,认为你们不会有什么别的心思;但那扰乱你们的人,不论他是谁,必要承受惩罚。
\textsuperscript{11}
至于我,弟兄们,如果我仍宣讲割损的需要,那我为什么还受迫害?若是这样,十字架的绊脚石就早已除去了。
\textsuperscript{12}
巴不得那些扰乱你们的人,将自己割净了!
\footnote{11节参阅\uwave{格}前1:23。12节“割净自己”,即“阉割”的意思。}

\textbf{自由是爱的表现\quad}
\textsuperscript{13}
弟兄们,你们蒙召选,是为得到自由;但不要以这自由作为放纵肉欲的藉口,惟要以爱德彼此服事。
\textsuperscript{14}
因为全部法律总括在这句话内:“爱你的近人如你自己。”
\textsuperscript{15}
但如果你们彼此相咬相吞,你们要小心,免得同归于尽。
\textsuperscript{16}
我告诉你们:你们若随圣神的引导行事,就决不会去满足本性的私欲,
\textsuperscript{17}
因为本性的私欲相反圣神的引导,圣神的引导相反本性的私欲:二者互相敌对,致使你们不能行你们所愿意的事。
\textsuperscript{18}
但如果你们随圣神的引导,就不在法律权下。
\textsuperscript{19}
本性私欲的作为是显而易见的:即淫乱、不洁、放荡、
\textsuperscript{20}
崇拜偶像、施行邪法、仇恨、竞争、嫉妒、忿怒、争吵、不睦、分党、
\textsuperscript{21}
妒恨、【凶杀】、醉酒、宴乐,以及与这些相类似的事。我以前劝戒过你们,如今再说一次:做这种事的人,决不能承受天主的国。
\textsuperscript{22}
然而圣神的效果却是:仁爱、喜乐、平安、忍耐、良善、温和、忠信、
\textsuperscript{23}
柔和、节制:关于这样的事,并没有法律禁止。
\textsuperscript{24}
凡属于\uline{耶稣}\uline{基督}的人,已把肉身同邪情和私欲钉在十字架上了。
\textsuperscript{25}
如果我们因圣神生活,就应随从圣神的引导而行事。
\textsuperscript{26}
不要贪图虚荣,不要彼此挑拨,互相嫉妒。
\footnote{真正的自由不是顺从肉欲去放肆,而是随从圣神去行事;不是以自私,而是以爱德为生活的原则。凡已属\uline{基督}的人,早已把肉身及一切私欲偏情钉在十字架上了。为这样的人,已不需要什么法律。14节参阅\uwave{肋}19:18;\uwave{罗}13:8-10。按\uline{拉丁}通行本,22,23两节中还有“宽宏、仁慈、贞洁”三种美德。}

\textbf{第六章\quad应彼此担待\quad}
\textsuperscript{1}
弟兄们,如果见一个人陷于某种过犯,你们既是属神的人,就该以柔和的心神矫正他;但你们自己要小心,免得也陷于诱惑。
\textsuperscript{2}
你们应彼此协助背负重担,这样,你们就满全了\uline{基督}的法律。
\textsuperscript{3}
人本来不算什么,若自以为算什么,就是欺骗自己。
\textsuperscript{4}
各人只该考验自己的行为,这样,对自己也许有可夸耀之处,但不是对别人夸耀,
\textsuperscript{5}
因为各人要背负自己的重担。
\textsuperscript{6}
学习真道的,应让教师分享自己的一切财物。
\footnote{\uline{保禄}劝勉信友要各人留心考验自己的缺点和毛病,如此,就必担待别人,不至于骄矜自夸。2节“\uline{基督}的法律”,即是指爱德的命令(\uwave{若}13:34)。}

\textbf{有其因必有其果\quad}
\textsuperscript{7}
你们切不要错了,天主是嘲笑不得的:人种什么,就收什么。
\textsuperscript{8}
那随从肉情撒种的,必由肉情收获败坏;然而那随从圣神撒种的,必由圣神收获永生。
\textsuperscript{9}
为此,我们行善不要厌倦;如果不松懈,到了适当的时节,必可收获。
\textsuperscript{10}
所以,我们一有机会,就应向众人行善,尤其应向有同样信德的家人。
\footnote{此处把人的一生比作撒种与收获,把肉身和心神(即本性和超性的精神)比作田地,人在这田地内播种工作;他若仅随肉身的好恶去行,必趋于败坏(丧亡);但若他随圣神的指引行事,必获得永生。10节,众信徒因有同样的信仰,故同是“天主的家人”(\uwave{弗}2:19)。}

\begin{center}
	\textbf{\large{\songti 结\quad论}}
\end{center}

\textbf{十字架是保禄的夸耀\quad}
\textsuperscript{11}
你们看,我亲手给你们写的是多么大的字!
\textsuperscript{12}
那些逼迫你们受割损的人,是想以外表的礼节来图人称赞,免得因\uline{基督}的十字架遭受迫害;
\textsuperscript{13}
其实,他们虽然受了割损,却也不遵守法律;他们只是愿意你们受割损,为能因在你们的肉身上所行的礼仪而夸耀。
\textsuperscript{14}
至于我,我只以我们的主\uline{耶稣}\uline{基督}的十字架来夸耀,因为藉着\uline{基督},世界于我已被钉在十字架上了;我于世界也被钉在十字架上了。
\textsuperscript{15}
其实,割损或不割损都算不得什么,要紧的是新受造的人。
\textsuperscript{16}
凡以此为规律而行的,愿平安与怜悯降在他们身上,即降在天主的新\uline{以色列}身上!
\textsuperscript{17}
从今以后,我切愿没有人再烦扰我,因为在我身上,我带有\uline{耶稣}的烙印。

\textbf{祝福辞\quad}
\textsuperscript{18}
弟兄们!愿我们的主\uline{耶稣}\uline{基督}的恩宠,常与你们的心灵同在!阿门。
\footnote{由11节可知:以上是\uline{保禄}所口授的,以下才是\uline{保禄}亲笔所写的,以证明本书出于自己,确实无伪。他末后用特大的字来写这段的用意,是把本书的主旨再提示出来,要人注意:割损与不割损都算不了什么,要紧的是在基督内成“一个新受造物”(\uwave{格}后5:17),因为唯有新的受造物,才是天主的新\uline{以色列},才是蒙受天主恩许的子民(\uwave{罗}2:29,9:24-26)。17节“\uline{耶稣}的烙印”是指\uline{保禄}为\uline{耶稣}受刑后所留下的伤痕;他以此当作他为宗徒的印号和证明(\uwave{格}后11:23-25)。}
\clearpage

% 厄弗所书引言
\phantomsection
\addcontentsline{toc}{chapter}{厄弗所书引言}
\input{introduction/46_Introduction_to_the_Ephesians}
\clearpage
% 厄弗所书(弗)
\phantomsection
\addcontentsline{toc}{chapter}{厄弗所书}
\chapter{厄弗所书}
\clearpage

% 斐理伯书引言
\phantomsection
\addcontentsline{toc}{chapter}{斐理伯书引言}
\input{introduction/47_Introduction_to_the_Philippians}
\clearpage
% 斐理伯书(斐)
\phantomsection
\addcontentsline{toc}{chapter}{斐理伯书}
% 分栏
\setlength\columnsep{0.6cm}
\begin{multicols}{2}

% 标题
\chapter*{斐理伯书}

% 右页眉
\rhead{}

% 正文

\end{multicols}
\clearpage

% 哥罗森书引言
\phantomsection
\addcontentsline{toc}{chapter}{哥罗森书引言}
\input{introduction/48_Introduction_to_the_Colossians}
\clearpage
% 哥罗森书(哥)
\phantomsection
\addcontentsline{toc}{chapter}{哥罗森书}
哥罗森书(哥)
\clearpage

% 得撒洛尼前后书引言
\phantomsection
\addcontentsline{toc}{chapter}{得撒洛尼前后书引言}
\input{introduction/49_Introduction_to_the_Thessalonians}
\clearpage
% 得撒洛尼前书(得前)
\phantomsection
\addcontentsline{toc}{chapter}{得撒洛尼前书}
得撒洛尼前书

得前
\clearpage
% 得撒洛尼后书(得后)
\phantomsection
\addcontentsline{toc}{chapter}{得撒洛尼后书}
% 分栏
\setlength\columnsep{0.8cm}
\begin{multicols}{2}

% 标题
\chapter*{得撒洛尼后书}

% 右页眉
\rhead{}

% 正文

\end{multicols}
\clearpage

% 弟茂德前后书引言
\phantomsection
\addcontentsline{toc}{chapter}{弟茂德前后书引言}
\input{introduction/50_Introduction_to_the_Timothy}
\clearpage
% 弟茂德前书(弟前)
\phantomsection
\addcontentsline{toc}{chapter}{弟茂德前书}
弟茂德前书(弟前)
\clearpage
% 弟茂德后书(弟后)
\phantomsection
\addcontentsline{toc}{chapter}{弟茂德后书}
弟茂德后书(弟后)
\clearpage

% 弟铎书引言
\phantomsection
\addcontentsline{toc}{chapter}{弟铎书引言}
% 标题
\chapter*{弟铎书引言}

% 右页眉
\rhead{弟铎书(铎)引言}

\uline{弟铎}的事迹不见于\uwave{宗}只散见于\uline{保禄}各书信内(\uwave{迦}、\uwave{格}后、\uwave{铎}及\uwave{弟}后)。他生于外教家庭(\uwave{迦}2:3),大概是在\uline{安提约基}\uline{雅}为\uline{保禄}所归化,因而称为\uline{保禄}的“真子”(1:4)。公元49年,曾随\uline{保禄}前往\uline{耶}京开宗徒会议(\uwave{迦}2:1;\uwave{宗}15:2),以后在\uline{保禄}第三次传教行程中(53-58年),曾被委派办理了几项要务(见\uwave{格}后2:13,7:6,8:6、16,12:18)。65年随\uline{保禄}至\uline{克里特}岛,被祝圣为该地的主教(1:5);随后接获此信,前往\uline{尼苛颇里}(3:12),后又从该城被派往\uline{达耳玛提雅}(\uwave{弟}后4:10)。根据教会的口传,他后来又返回\uline{克}岛,在那里寿终正寝。

\uline{保禄}写本书的动机,由本书内容看来,与\uwave{弟}前完全一样,即教导他如何管理该岛的教会。本书写作的时间,应在\uwave{弟}前书写后不久,即在65年,因为本书与\uwave{弟}前内所有的劝言与指示,几乎完全相同;地点是在\uline{马其顿}。

本书可分为三段:第一段1:5-16:论选立圣职人员的规范和他们应有的品格;第二段2:1-3:7:论应如何对待各级人士,应如何训诲他们弃恶迁善;第三段3:8-11:论行善的必要,并应如何驳斥异端邪说。
\clearpage
% 弟铎书(铎)
\phantomsection
\addcontentsline{toc}{chapter}{弟铎书}
% 分栏
\setlength\columnsep{0.6cm}
\begin{multicols}{2}

% 标题
\chapter*{弟铎书}

% 右页眉
\rhead{弟铎书(铎)}

% 正文
\textbf{第一章\quad致候辞\quad}
\textsuperscript{1}
天主的仆人、作\uline{耶稣}\uline{基督}宗徒的\uline{保禄}——为引天主所选的人,去信从并认识合乎虔敬的真理,
\textsuperscript{2}
这虔敬是本于永生的希望,又是那不能说谎的天主,在久远的时代以前所预许的,
\textsuperscript{3}
他到了适当的时期,就籍着宣讲显示了他的圣道;我就是照我们救主天主的命令,受委托尽这宣讲的职务。——
\textsuperscript{4}
我\uline{保禄}致书给在共同信仰内作我真子的\uline{弟铎}:愿恩宠与平安,由天主父及我们的救主\uline{基督}\uline{耶稣}赐与你。
\renewcommand\thefootnote{\ding{\numexpr171+\value{footnote}}}
\footnote{宗徒的使命是引人信天主,并使人因而获得天主从古时就已许下的永生;为此宗徒们被派遣,就是为给人宣讲那“合乎虔敬的真理”,即“天主的圣道”,亦即福音。\uline{保禄}大概亲手给\uline{弟铎}付了洗,使他重生于\uline{基督},故称他是自己在信仰内的儿子,如\uline{弟茂德}一样(\uwave{弟}前1:2)。}

\textbf{长老应有的品格\quad}
\textsuperscript{5}
我留你在\uline{克里特},是要你整顿那些尚未完成的事,并照我所吩咐你的,在各城设立长老:
\textsuperscript{6}
长老应是无可指摘的,只做过一个妻子的丈夫,所有的子女都是信徒,又没有被控告为放荡不羁的,
\textsuperscript{7}
因为做监督的,既是天主的管家,就该是无可指摘的、不自负、不发怒、不嗜酒、不暴戾、不贪污;
\textsuperscript{8}
但该好客、乐善、慎重、公正、热心、有节,
\textsuperscript{9}
坚持那合乎教理的真道,好能以健全的道理劝戒并驳斥抗辩的人。
\footnote{\uline{弟铎}在\uline{克里特}岛应尽的任务,是继承宗徒的工作,使当地的教会发展,为此应在各城设立“长老”(参阅\uwave{宗}11注六)。身为长老的(参阅\uwave{弟}前3:2-4),不但不应有有辱于自己职位的毛病,而且还应修各样美德(\uwave{弟}前3:2-7),并“坚持那合乎教理的真道”,即全教会一脉相传的信仰,用这“健全的道理”(\uwave{弟}前1:10)驳斥那散布谬理的假学士(\uwave{弟}前4:1-3)。}

\textbf{应排斥异端\quad}
\textsuperscript{10}
实在有许多人尚不服从,好空谈,欺骗人,尤其是那些受过割损的人;
\textsuperscript{11}
应杜塞这些人的口,因为他们为了可耻的利润,竟教导那不应教导的事,破坏人的整个家庭。
\textsuperscript{12}
\uline{克里特}人中的一人,他们自己的一位先知曾这样说:“\uline{克里特}人常是些说谎者,是些可恶的野兽,贪口腹的懒汉。”
\textsuperscript{13}
这话说的很对。为此,你该严厉规劝他们,好叫他们在信德上健全无暇;
\textsuperscript{14}
不要听信\uline{犹太}人无稽的传说,和背弃真理之人的规定。
\footnote{\uline{克}岛曾有一位诗人(\uline{保禄}按\uline{希腊}人的习俗称他为“先知”),完全认出了自己同乡的缺点。14节“\uline{犹太}人无稽的传说”,见\uwave{弟}前1:3,4:7。}
\textsuperscript{15}
为洁净人一切都是洁净的,但为败坏的人和无信仰的人,没有一样是洁净的,就连他们的理性和良心都是污秽的。
\textsuperscript{16}
这样的人自称认识天主,但在行为上却否认天主;他们是可憎恶的,悖逆的,在一切善事上是无用的。
\footnote{“一切都是洁净的”(\uwave{罗}14:14、20),这话只是对心里洁净的人说的;至于那“败坏的人和无信仰的人”却不然,他们沉溺于罪恶,理智不辨别是非,良心不审断善恶,为他还有什么是洁净的呢(\uwave{玛}15:11;\uwave{弟}前4:3)。}

\textbf{第二章\quad信友应有的个别教训\quad}
\textsuperscript{1}
至于你,你所讲的,该合乎健全的道理;
\textsuperscript{2}
教训老人应节制、端庄、慎重,在信德、爱德和忍耐上,要正确健全。
\textsuperscript{3}
也要教训老妇在举止上要圣善,不毁谤人,不沉湎于酒,但教人行善,
\textsuperscript{4}
好能教导青年妇女爱丈夫,爱子女,
\textsuperscript{5}
慎重,贞节,勤理家务,良善,服从自己的丈夫,免得使人诋毁天主的圣道。
\footnote{参阅\uwave{弟}前5:1、2。此处应注意:\uline{保禄}将教导青年妇女的责任,托给年老的妇女。她们应树立良好的家风,好使年轻妇女有所遵循。因为良好家庭是广扬福音最有效的方法;但使福音真道遭受教外人士诟病的,也莫过于基督徒的不良家庭。}
\textsuperscript{6}
你也要教训青年人在一切事上要慎重。
\textsuperscript{7}
你该显示自己为行善的模范,在教导上应表示纯正庄重,
\textsuperscript{8}
要讲健全无可指摘的话,使反对的人感到羞愧,说不出我们什么不好来。
\footnote{参阅\uwave{弟}前4:12。}
\textsuperscript{9}
教训奴隶在一切事上要服从自己的主人,常叫他们喜悦,不要抗辩,
\textsuperscript{10}
不要窃取,惟要事事表示自己实在忠信,好使我们的救主天主的圣道,在一切事上获得光荣。
\footnote{关于奴隶制度,参见\uwave{格}前7注四,\uwave{弗}6:5-9;\uwave{弟}前6:1、2。奴隶的劳役与忠信,能发挥\uline{基督}的圣道,感化主人,这样使天主在人前获得光荣。}

\textbf{信友应如何在世上生活\quad}
\textsuperscript{11}
的确,天主救众人的恩宠已经出现,
\textsuperscript{12}
教导我们弃绝不虔敬的生活,和世俗的贪欲,有节地、公正地、虔敬地在今世生活,
\textsuperscript{13}
期待所希望的幸福,和我们伟大的天主及救主\uline{耶稣}\uline{基督}光荣的显现。
\textsuperscript{14}
他为我们舍弃了自己,是为救赎我们脱离一切罪恶,洗净我们,使我们能成为他的选民,叫我们热心行善。
\footnote{天主对众人的恩宠,在\uline{耶稣}降生和救赎的工程上,完全显示出来;这恩宠要求人脱离“世俗的贪欲”,而期待来世的真福和光荣。}
\textsuperscript{15}
你要宣讲这些事,以全权规劝和指摘,不要让任何人轻视你。

\textbf{第三章\quad信友应服从政权\quad}
\textsuperscript{1}
你要提醒人服从执政的官长,听从命令,准备行各种善事。
\footnote{\uline{保禄}叫\uline{弟铎}提醒信友服从政府,尽国民的责任,因为国家政权是来自天主(参阅\uwave{罗}13:1-7;\uwave{弟}前2:1-5;\uwave{伯}前2:13-17)。}
\textsuperscript{2}
不要辱骂,不要争吵,但要谦让,对众人表示极其温和,
\textsuperscript{3}
因为我们从前也是昏愚的,悖逆的,迷途的,受各种贪欲和逸乐所奴役,在邪恶和嫉妒中度日,自己是可憎恶的,又彼此仇恨。
\footnote{人类最大的昏愚是不认识天主。由于不认识天主才“彼此仇恨”,缺乏爱德。爱德原是信友生活的原动力,因为天主自己就是爱(\uline{若}一4:16)。}
\textsuperscript{4}
但当我们的救主天主的良善,和他对人的慈爱出现时,
\textsuperscript{5}
他救了我们,并不是由于我们本着义德所立的功劳,而是出于他的怜悯,籍着圣神所施行的重生和更新的洗礼,救了我们。
\textsuperscript{6}
这圣神是天主籍我们的救主\uline{耶稣}\uline{基督},丰富地倾注在我们身上的,
\textsuperscript{7}
好使我们因他的恩宠成义,本着希望成为永生的继承人。
\footnote{信友得以领洗重生,应归功于天主圣三:就是归于天父的慈爱怜悯;归于\uline{耶稣}\uline{基督},因他是我们的救主和中保;归于圣神,因他赐人超性的生命(\uwave{若}6:63,14:16)。信友既然成了天主的义子,也就成了天主产业的承继人(\uwave{迦}4:7)。要承继的产业就是永生。}

\textbf{责斥不务正道的人\quad}
\textsuperscript{8}
这话是确实的,我愿意你坚持这些事,好使那些已信奉天主的人,热心专务行善:这些都是美好而为人有益的事;
\textsuperscript{9}
至于那些愚昧的辩论、祖谱、争执和关于法律的争论,你务要躲避,因为这些都是无益的空谈。
\footnote{参阅\uwave{弟}前1:4,6。}
\textsuperscript{10}
对异端人,在谴责过一次两次以后,就该远离他。
\textsuperscript{11}
该知道:这样的人已背弃正道,犯罪作恶,自己给自己定了罪案。
\footnote{参阅\uwave{玛}18:17;\uwave{若}二10。}

\textbf{吩咐与嘱托\quad}
\textsuperscript{12}
当我打发\uline{阿尔特玛}或\uline{提希苛}到你那里以后,你赶快到\uline{尼苛颇里}来见我,因为我已决定在那里过冬。
\textsuperscript{13}
你打发法学士\uline{则纳}和\uline{阿颇罗}上路,要照顾周到,使他们什么也不缺少。
\textsuperscript{14}
我们的人也应当学着行善,为应付一切急需,免得成为不结果实的人。
\footnote{“行善”(14节),此处是指为帮助传教,不应吝惜财物而言。}

\textbf{问候与祝福\quad}
\textsuperscript{15}
同我在一起的弟兄都问候你;请问候那些在信德内爱我们的弟兄。愿恩宠与你们众人同在!
\end{multicols}
\clearpage

% 费肋孟书引言
\phantomsection
\addcontentsline{toc}{chapter}{费肋孟书引言}
% 标题
\chapter*{费肋孟书引言}

% 右页眉
\rhead{费肋孟书(费)引言}

\uline{费肋孟}原是\uline{哥罗森}城的富翁,他大概是在\uline{厄弗所}(\uwave{宗}19:10)直接为\uline{保禄}所归化的(19节)。他信教后,表现了非凡的信德与爱德,竟把自己的家献出,作为信友集会及举行圣祭之所(2,4-7节)。他既是富户人家,按当时的社会制度,也蓄养了许多奴隶,其中有一名叫\uline{敖乃息摩}的,尚未信教,作了一件对不起主人的事(大约偷了财物),因而畏罪逃亡,出走远方。

\uline{敖乃息摩}逃到了\uline{罗马},找到了正被囚的\uline{保禄}。\uline{保禄}为保护他,起初本想留下他服侍自己(13节),但后来因\uline{敖}氏已领洗入教(10,11两节),决意叫他跟\uline{提希苛}(\uline{保禄}致《哥罗森书》即由他带去),回到他的主人那里,因而写了这封保荐他的短信,求\uline{费肋孟}不但不要处罚他,而且还应以“弟兄”之谊接待他(16节)。这封短信,犹如在《厄弗所书引言》中所提过的,应是\uline{保禄}在63年于\uline{罗马}写成的。

这封优美的私人函件,篇幅极短,很可能全是\uline{保禄}亲笔所写。信内措词造句,委婉动人,务求达到目的;同时这封富于热情的短信也将\uline{保禄}内心的爱情,活显于纸上,实是不可多得的杰作。

这封私人函件,因为涉及了奴隶制度的社会问题,所以对教会以及社会发生了极大的影响。当基督教会出现于世时,经济与社会生活全系于奴隶。虽则如此,但当时的人却不以奴隶为人,而只视作货物。\uline{基督}的教义一传于世,开始了一种新的气象;所以这封短信实可称为“基督自由的宣言”(参阅\uwave{迦}3:27、28;\uwave{格}前7:20-22;\uwave{弗}6:5-9;\uwave{哥}3:11,4:1)。它虽不曾把奴隶制度立即废除,但由于基督教义的逐渐推进,把奴隶制度终归消灭。
\clearpage
% 费肋孟书(费)
\phantomsection
\addcontentsline{toc}{chapter}{费肋孟书}
% 分栏
\setlength\columnsep{0.8cm}
\begin{multicols}{2}

% 标题
\chapter*{费肋孟书}

% 右页眉
\rhead{}

% 正文

\end{multicols}
\clearpage

% 希伯来书引言
\phantomsection
\addcontentsline{toc}{chapter}{希伯来书引言}
\input{introduction/53_Introduction_to_the_Hebrews}
\clearpage
% 希伯来书(希)
\phantomsection
\addcontentsline{toc}{chapter}{希伯来书}
% 分栏
\setlength\columnsep{0.6cm}
\begin{multicols}{2}

% 标题
\chapter*{希伯来书}

% 右页眉
\rhead{}

% 正文

\end{multicols}
\clearpage

% 公函总论
\phantomsection
\addcontentsline{toc}{chapter}{公函总论}
\input{introduction/54_General_Theory_of_Official_Letter}
\clearpage

% 雅各伯书引言
\phantomsection
\addcontentsline{toc}{chapter}{雅各伯书引言}
\input{introduction/55_Introduction_to_the_James}
\clearpage
% 雅各伯书(雅)
\phantomsection
\addcontentsline{toc}{chapter}{雅各伯书}
雅各伯书(雅)
\clearpage

% 伯多禄前后书引言
\phantomsection
\addcontentsline{toc}{chapter}{伯多禄前后书引言}
% 标题
\chapter*{伯多禄前后书引言}

% 右页眉
\rhead{伯多禄前后书引言}

宗徒之长\uline{伯多禄}的小史史料来源有二:一是圣经,一是教会的圣传。\uline{伯多禄}在蒙召之前,名叫\uline{西满},\uline{伯多禄}(“磐石”之意,\uline{希伯来}文为\uline{刻法})是\uline{耶稣}给他改的名字(\uwave{若}1:24)。他与胞兄\uline{安德肋}出生于\uline{加里肋亚}湖北岸的\uline{贝特赛达}城,身为渔夫。在第一次捕鱼的奇迹后,\uline{耶稣}才召他为宗徒,为渔人的渔夫(\uwave{路}5:8-11)。\uline{伯多禄}在众宗徒中,连在\uline{耶稣}的三位爱徒中,常居首位(\uwave{玛}10:2,17:1,26:37;\uwave{谷}5:37);他领受了元首职权(\uwave{玛}16:13-19)和\uline{耶稣}特为他祈祷的预许(\uwave{路}22:31、32);在\uline{耶稣}复活后,隆重地接受了管理全教会的元首职权(\uwave{若}21:15-17);在\uline{耶稣}升天后,由《宗徒大事录》的前半部可知,他作了\uline{耶稣}在世的代表,始终执行了他的元首职权。公元约43、44年,按\uwave{宗}12:17的记载:他“往别的地方去了”,大概是去了\uline{罗马}。49年又出现于\uline{耶}京,主持宗徒会议。随后曾到过\uline{安提约基}\uline{雅}(\uwave{迦}2:11-14)。再后,除知道他由\uline{罗马}写了这两封书信外,《新约》经书再没有记载他的事迹。按\uline{欧色彼}和\uline{热罗尼莫}的记载,他以后定居于\uline{罗马},在\uline{尼禄}皇帝时,即公元67年,为主殉难而死,他的遗体葬于\uline{梵蒂冈}山岗。圣教会每年6月29日庆祝他殉难的节日。

\qquad伯多禄前书

\uwave{伯}前的作者确实是\uline{伯多禄}宗徒,因为在信首的致候辞内已明明写出;此外,古传说和书信内容也一致如此证明。若把本书内容和\uwave{宗}内\uline{伯多禄}的讲辞作一比较,彼此间也十分相合。

本书的收信人,是散居在\uline{小亚细亚}北部的信友。写本书的动机,是因为作者听到该处的信友,不断遭受教外与\uline{犹太}人的迫害,受着背弃信德的威胁(2:12,3:14-16,4:12-16),所以宗徒写了此信,为安慰他们的忧苦,坚固他们的信德,劝勉他们:困难无论如何重大,仍当善度真正信友的生活。

本书写于\uline{巴比伦}(5:13),即\uline{罗马}(以\uline{巴比伦}指\uline{罗马},见\uwave{默}17以及初世纪\uline{犹太}人的作品)。写作的时间,大约是在\uline{尼禄}教难之前,即公元63至64年间,其时\uline{保禄}适在\uline{西班牙}。

本书的\uline{希腊}文虽间有\uline{闪}族的语风,仍堪称典雅。全书的中心思想既是劝勉信友保持信德,效法\uline{基督}的德表,善度信友的生活,所以内容方面几乎全是有关道德的言论,对教义问题,只偶然涉及而已(1:3,3:18-22)。

本书除致候辞(1:1、2),序言(1:3-12)和结尾语外(5:12-14),可分为四段:第一段:泛论信友应该善度真正基督徒的生活(1:3-2:20);第二段:分论信友对各级人士,以及彼此间应尽的义务(2:21-3:12);第三段:劝勉信友要追随\uline{基督}的德表,忍受一切苦难与迫害(3:13-4:19);第四段:几项有关信友团体生活的特别劝言(5:1-11)。

\qquad伯多禄后书

由于\uwave{伯}后的语气与文笔和\uwave{伯}前有显著的不同,因而有不少人怀疑本书是出于\uline{伯多禄}之手;不过这种怀疑实属多余,因本书信信首,明言写信人是\uline{伯多禄}宗徒(1:1),并且在信内作者曾说自己瞻仰过\uline{耶稣}显圣容(1:17、18)。前后二书的语气与文笔之所以不同,圣\uline{热罗尼莫}认为是作者用了不同的代笔人,大部分学者皆以此说为是。

\uwave{伯}后的收信人与\uwave{伯}前同。作者写本书的动机,似乎是因接到了有关读者的一些消息,得知他们所处的环境较前更为恶劣,此时,除遭受政府方面的迫害外,在教会以内也发生了不少错误思想,因有些假教师潜入教会,扰乱信友(2:1-3、11)。所以作者写这信的目的,除安慰鼓励信友外,特别是为驳斥那些假教师的谎言谬论。

本书写作的时间,按1:14所记,应在作者逝世前不久,即约在公元66至67年间,其时\uline{保禄}大约再度被捕入狱。写作的地方仍是\uline{罗马}。

本书有不少地方与《犹达书》的内容相似,这极可能是\uline{伯多禄}参考过《犹达书》,他认为《犹达书》所写的,颇适合他的读者所处的环境,因而采用了一些语句。

本书除致候辞(1:1、2)和结尾语外(3:17、18),可分为两大段:第一段:劝勉信友注重修德行善的实际生活(1:3-21);第二段:特别驳斥假教师们的邪说谬论(2:1-3:16)。此外,读者读本书时,应注意\uline{伯多禄}在3:15、16对\uline{保禄}的书信所说的话。

\clearpage
% 伯多禄前书(伯前)
\phantomsection
\addcontentsline{toc}{chapter}{伯多禄前书}
% 标题
\chapter*{伯多禄前书}

% 右页眉
\rhead{伯多禄前书(伯前)}

% 正文
\textbf{第一章\quad致候辞\quad}
\textsuperscript{1}
\uline{耶稣}\uline{基督}的宗徒\uline{伯多禄}致书给散居在\uline{本都}、\uline{迦拉达}、\uline{卡帕多细雅}、\uline{亚细亚}和\uline{彼提尼雅}作旅客的选民:
\textsuperscript{2}
你们被召选,是照天主的预定;受圣神祝圣,是为服事\uline{耶稣}\uline{基督},和分沾他宝血洗净之恩。

愿恩宠和平安丰富地赐予你们!
\footnote{基督徒在世上好似“旅客”,因为他们的本乡是天堂(2:11;\uwave{格}后5:1、6;\uwave{斐}3:20)。“散居……蒙选者”是指散居在外教人中的信友。2节中一一提出天主圣三在救赎工程上每位的工作,以及每位对信友的关系。}

\textbf{序言\quad得沾救恩的福分\quad}
\textsuperscript{3}
愿我们的主\uline{耶稣}\uline{基督}的天父和父受赞美!他因自己的大仁慈,藉\uline{耶稣}\uline{基督}由死者中的复活,重生了我们,为获得那充满生命的希望,
\textsuperscript{4}
为获得那为你们已存留在天上的不坏、无瑕、不朽的产业,
\textsuperscript{5}
因为你们原是为天主的能力所保护,为使你们藉着信德,而获得那已准备好,在最后时期出现的救恩。
\textsuperscript{6}
为此,你们要欢跃,虽然如今你们暂时还该在各种试探中受苦,
\textsuperscript{7}
这是为使你们的信德,得以精炼,比经过火炼而仍易消失的黄金,更有价值,好在\uline{耶稣}\uline{基督}显现时,堪受称赞、光荣和尊敬。
\textsuperscript{8}
你们虽然没有见过他,却爱慕他;虽然你们如今仍看不见他,还是相信他;并且以不可言传,和充满光荣的喜乐而欢跃,
\textsuperscript{9}
因为你们已把握住信仰的效果:灵魂的救恩。
\textsuperscript{10}
关于这救恩,那些预言了你们要得恩宠的先知们,也曾经寻求过,考究过,
\textsuperscript{11}
就是考究那在他们内的\uline{基督}的圣神,预言那要临于\uline{基督}的苦难,和以后的光荣时,指的是什么时期,或怎样的光景。
\textsuperscript{12}
这一切给他们启示出来,并不是为他们自己,而是为给你们服务;这一切,如今藉着给你们宣传福音的人,依赖由天上派遣来的圣神,传报给你们;对于这一切奥迹,连众天使也都切望窥探。
\footnote{作者以赞颂天主圣三对信友所赐的救恩,而反映信友对这救恩所有的义务:信友既信天主给自己准备了永存不朽的产业(救恩),就应怀着喜乐的心情承受天主的试探,仰望来日的光荣。古先知所切望的救恩,连天使也愿知道的奥迹,都启示给信仰\uline{基督}的人了(\uwave{路}24:26、27;\uwave{玛}13:16、17;\uwave{格}前2:9-13;\uwave{哥}1:16)。12节论天使对救恩的奥迹所得的知识,见\uwave{弗}3:10。}

\begin{center}
	\textbf{\large{\songti 信友对天主的义务}}
\end{center}

\textbf{应度圣洁的生活\quad}
\textsuperscript{13}
为此,你们要束上腰,谨守心神,要清醒,要全心希望在\uline{耶稣}\uline{基督}显现时,给你们带来的恩宠;
\textsuperscript{14}
要做顺命的子女,不要符合你们昔日在无知中生活的欲望,
\textsuperscript{15}
但要像那召叫你们的圣者一样,在一切生活上是圣的,
\textsuperscript{16}
因为经上记载:“你们应是圣的,因为我是圣的。”
\textsuperscript{17}
你们既称呼那不看情面,而只按每人的作为行审判者为父,就该怀着敬畏,度过你们这旅居的时期。
\textsuperscript{18}
该知道:你们不是用能朽坏的金银等物,由你们祖传的虚妄生活中被赎出来的,
\textsuperscript{19}
而是用宝血,即无玷无瑕的羔羊\uline{基督}的宝血。
\textsuperscript{20}
他固然是在创世以前就被预定了的,但在最末的时期为了你们才出现,
\textsuperscript{21}
为使你们因着他,而相信那使他由死者中复活,并赐给他光荣的天主:这样你们的信德和望德,都同归于天主。
\footnote{信友既怀着永生的希望(3节),那么,一生的行动生活,便该持守圣洁:因为,一、召选他们为其子女的天主是圣的,他愿他所爱的子女也是圣洁的(\uwave{肋}11:44、45;\uwave{玛}5:48);二、天父不只是圣洁的父,而也是按公义行审判的父(\uwave{罗}2:11);三、信友自身的赎价,并不是金银财宝,而是\uline{基督}的宝血(\uwave{希}9:14;\uwave{默}5:6-13)。}

\textbf{赤诚相爱\quad}
\textsuperscript{22}
你们既因服从真理,而洁净了你们的心灵,获得了真实无伪的弟兄之爱,就该以赤诚的心,热切相爱,
\textsuperscript{23}
因为你们原是赖天主生活而永存的圣言,不是由于能坏的,而是由于不能坏的种子,得以重生。
\textsuperscript{24}
因为“凡有血肉的都似草,他的一切美丽都似草上的花:草枯萎了,花也就凋谢了;
\textsuperscript{25}
但上主的话却永远常存”:这话就是传报给你们的福音。
\footnote{24、25两节引自\uwave{依}40:6、8,证明信友由之而生的种子——天主的话——福音(见\uwave{雅}1:18),是永存不朽的。}

\textbf{第二章\quad应以基督为基石\quad}
\textsuperscript{1}
所以你们应放弃各种邪恶、各种欺诈、虚伪、嫉妒和各种诽谤,
\textsuperscript{2}
应如初生的婴儿贪求属灵性的纯奶,为使你们靠着它生长,以致得救;
\textsuperscript{3}
何况你们已尝到了“主是何等的甘饴”。
\textsuperscript{4}
你们接近了他,即接近了那为人所摈弃,但为天主所精选,所尊重的活石,
\textsuperscript{5}
你们也就成了活石,建成一座属神的殿宇,成为一班圣洁的司祭,以奉献因\uline{耶稣}\uline{基督}而中悦天主的属神的祭品。
\textsuperscript{6}
这就是经上所记载的:“看,我要在\uline{熙雍}安放一块精选的,宝贵的基石,凡信赖他的,决不会蒙羞。”
\textsuperscript{7}
所以为你们信赖的人,是一种荣幸;但为不信赖的人,是“匠人弃而不用的石头,反而成了屋角的基石”;
\textsuperscript{8}
并且是“一块绊脚石,和一块使人跌倒的磐石”。他们由于不相信天主的话,而绊倒了,这也是为他们预定了的。
\textsuperscript{9}
至于你们,你们却是特选的种族,王家的司祭,圣洁的国民,属于主的民族,为叫你们宣扬那由黑暗中召叫你们,进入他奇妙之光者的荣耀。
\textsuperscript{10}
你们从前不是天主的人民,如今却是天主的人民;从前没有蒙受爱怜,如今却蒙受了爱怜。
\footnote{信友过的圣洁生活,只有与\uline{基督}结合才能增进发展。原来信友之与\uline{基督},有如婴儿之与母亲(2、3两节),建筑物之与“基石”(4、5两节),因为信友不能脱离\uline{基督}而独存;何况信友因\uline{基督}才能成为“活石”,与\uline{基督}一起建成一座属于天主的圣殿(\uwave{格}前3:16、17;\uwave{弗}2:20-22),与\uline{基督}共同形成一司祭团(\uwave{默}1:1,5:10),将自己圣洁的生活,作为馨香的祭品,藉\uline{基督}奉献于天主(\uwave{罗}12:1;\uwave{斐}2:17)。3节引自\uwave{咏}34:9。6-8节,见\uwave{依}28:16,8:14;\uwave{咏}118:22。}

\textbf{应服从政权\quad}
\textsuperscript{11}
亲爱的!我劝你们作侨民和作旅客的,应戒绝与灵魂作战的肉欲;
\textsuperscript{12}
在外教人中要常保持良好的品行,好使那些诽谤你们为作恶者的人,因见到你们的善行,而在主眷顾的日子,归光荣于天主。
\textsuperscript{13}
你们要为主的缘故,服从人立的一切制度:或是服从帝王为最高的元首,
\textsuperscript{14}
或是服从帝王派遣来惩罚作恶者,奖赏行善者的总督,
\textsuperscript{15}
因为这原是天主的旨意:要你们行善,使那些愚蒙无知的人,闭口无言。
\textsuperscript{16}
你们要做自由的人,却不可做以自由为掩饰邪恶的人,但该做天主的仆人;
\textsuperscript{17}
要尊敬众人,友爱弟兄,敬畏天主,尊敬君王。
\footnote{信友应乐意服从一切合法的政府,因为一切权柄都是来自天主(\uwave{罗}13:1-7)。惟有在政府滥用政权时,人民才没有服从的义务(\uwave{宗}4:19、20,5:29)。信友如奉公守法(13、14两节),敬主爱人(17节),就可使那些侮蔑信友的人闭口无言。天主甚或有时以信友的善表感动(“眷顾”12节)他们,使他们归依圣教。}

\textbf{仆人对主人应有的态度\quad}
\textsuperscript{18}
你们做家仆的,要以完全敬畏的心服从主人,不但对良善和温柔的,就是对残暴的,也该如此。
\textsuperscript{19}
谁若明知是天主的旨意,而忍受不义的痛苦:这才是中悦天主的事。
\textsuperscript{20}
若你们因犯罪被打而受苦,那还有什么光荣?但若因行善而受苦,而坚心忍耐:这才是中悦天主的事。
\textsuperscript{21}
你们原是为此而蒙召的,因为\uline{基督}也为你们受了苦,给你们留下了榜样,叫你们追随他的足迹。
\textsuperscript{22}
“他没有犯过罪,他口中也从未出过谎言”;
\textsuperscript{23}
他受辱骂,却不还骂;他受虐待,却不报复,只将自己交给那照正义行审判的天主;
\textsuperscript{24}
他在自己的身上,亲自承担了我们的罪过,上了木架,为叫我们死于罪恶,而活于正义;“你们是因他的创伤而获得了痊愈”。
\textsuperscript{25}
你们从前有如迷途的亡羊,如今却被领回,归依你们的灵牧和监督。
\footnote{为奴的信友应如天主的仆人一样,服从主人。有关奴隶的劝言,见\uwave{费}、\uwave{弗}6:5-8;\uwave{哥}3:22-25等处。作者为鼓励为奴的信友,情愿为主忍受无理的虐待,特以\uline{基督}的忍耐作榜样。22-25节,见\uwave{依}53。}

\textbf{第三章\quad夫妇间应遵守的义务\quad}
\textsuperscript{1}
同样,你们做妻子的,应当服从自己的丈夫,好叫那些不信从天主话的,为了妻子无言的品行而受感化,
\textsuperscript{2}
因为他们看见了,你们怀有敬畏的贞洁品行。
\textsuperscript{3}
你们的装饰不应是外面的发型、金饰,或衣服的装束,
\textsuperscript{4}
而应是那藏于内心,基于不朽的温柔,和宁静心神的人格:这在天主前才是宝贵的。
\textsuperscript{5}
从前那些仰望天主的圣妇,正是这样装饰了自己,服从了自己的丈夫。
\textsuperscript{6}
就如\uline{撒辣}听从了\uline{亚巴郎},称他为“主”;你们如果行善,不害怕任何恐吓,你们就是她的女儿。

\textsuperscript{7}
同样,你们作丈夫的,应该凭着信仰的智慧与妻子同居,待她们有如较为脆弱的器皿,尊敬她们,有如与你们共享生命恩宠的继承人:这样你们的祈祷便不会受到阻碍。
\footnote{关于夫妇间彼此应遵守的义务,见\uwave{格}前7;\uwave{弗}5:22-33;\uwave{哥}3:18、19等处。做妻子的基本美德是服从;做丈夫的基本美德是爱怜。有了这两种基本美德,家庭的生活必能和谐幸福。6节\uline{撒辣}为信徒之母,见\uwave{迦}4:22-30;\uwave{罗}9:8、9。}

\textbf{应以爱德与人相处\quad}
\textsuperscript{8}
总之,你们都该同心合意,互表同情,友爱弟兄,慈悲为怀,谦逊温和;
\textsuperscript{9}
总不要以恶报恶,以骂还骂;但要祝福,因为你们原是为继承祝福而蒙召的。
\textsuperscript{10}
所以“凡愿意爱惜生命,和愿意享见幸福日子的,就应谨守口舌,不说坏话,克制嘴唇,不言欺诈;
\textsuperscript{11}
躲避邪恶,努力行善,寻求和平,全心追随,
\textsuperscript{12}
因为上主的双目垂顾正义的人,他的两耳俯听他们的哀声;但上主的威容敌视作恶的人。”
\footnote{9节见\uwave{玛}5:38-48;\uwave{路}6:28。10-12节引自\uwave{咏}34:13-17。}

\begin{center}
	\textbf{\large{\songti 苦难中应有的态度}}
\end{center}

\textbf{应安心受苦\quad}
\textsuperscript{13}
如果你们热心行善,谁能加害你们呢?
\textsuperscript{14}
但若你们为正义而受苦,才是有福的。你们不要害怕人们的恐吓,也不要心乱。
\textsuperscript{15}
你们但要在心内尊崇\uline{基督}为主;若有人询问你们心中所怀希望的理由,你们要时常准备答复,
\textsuperscript{16}
且要以温和、以敬畏之心答复,保持纯洁的良心,好使那些诬告你们在\uline{基督}内有良好品行的人,在他们诽谤你们的事上,感到羞愧。
\footnote{14、15两节见\uwave{玛}5:10;\uwave{依}8:12、13。}

\textbf{基督无辜受苦的模范\quad}
\textsuperscript{17}
若天主的旨意要你们因行善而受苦,自然比作恶而受苦更好,
\textsuperscript{18}
因为\uline{基督}也曾一次为罪而死,且是义人代替不义的人,为将我们领到天主面前;就肉身说,他固然被处死了;但就神魂说,他却复活了。
\footnote{见\uwave{希}9:26-28。所谓“神魂”是指\uline{耶稣}的灵魂,因为他的灵魂赖与她结合的天主性,有使他的肉身复活的能力。}
\textsuperscript{19}
他藉这神魂,曾去给那些在狱中的灵魂宣讲过;
\textsuperscript{20}
这些灵魂从前在\uline{诺厄}建造方舟的时日,天主耐心期待之时,原是不信的人;当时赖方舟经过水而得救的不多,只有八个生灵。
\footnote{\uline{耶稣}死后,他的灵魂给“在狱中的灵魂”,即给一切集中在“灵薄狱”的古善人的灵魂,报告了救恩的喜讯(4:6)。信经中“我信其降地狱”一句,即含有此意。作者在此另外提出被洪水淹没而在死前回头的人,为证明救恩的效力。关于洪水,见\uwave{创}6-9章。}
\textsuperscript{21}
这水所预表的圣洗,如今赖\uline{耶稣}\uline{基督}的复活拯救了你们,并不是涤除肉体的污秽,而是向天主要求一纯洁的良心。
\footnote{就如洪水浮起\uline{诺厄}的方舟,救了“八个生灵”(\uwave{创}7:7),这样圣洗圣事使信友得免于永远的沦亡。}
\textsuperscript{22}
至于\uline{耶稣}\uline{基督},他升了天,坐在天主的右边,
\footnote{见\uwave{弗}1:20-22。}
众天使、掌权者和异能者都屈伏在他权下。

\textbf{第四章\quad应度圣洁的生活\quad}
\textsuperscript{1}
\uline{基督}既然在肉身上受了苦难,你们就应该具备同样的见识,深信凡在肉身上受苦的,便与罪恶断绝了关系;
\textsuperscript{2}
今后不再顺从人性的情欲,而只随从天主的意愿,在肉身内度其余的时日。
\textsuperscript{3}
过去的时候,你们实行外教人的欲望,生活在放荡、情欲、酗酒、宴乐、狂饮和违法的偶像崇拜中,这已经够了!
\footnote{信友既因\uline{基督}的苦难圣死,藉圣洗已死于罪恶,活于天主,便常该纪念\uline{基督}的苦难(1节),怀有他“所怀有的心情”(\uwave{斐}2:5),远避罪恶(2、3两节)。}
\textsuperscript{4}
由于你们不再同他们狂奔于淫荡的洪流中,他们便引以为怪,遂诽谤你们;
\textsuperscript{5}
但他们要向那已准备审判生死者的主交账。
\textsuperscript{6}
也正是为此,给死者宣讲了这福音:他们虽然肉身方面如同人一样受了惩罚,可是神魂方面却同天主一起生活。
\footnote{“给死者宣讲了福音”一语,似乎应照3:18-20来解释(见3注五)。因为人死后,他永远祸福的结局已成定案,再无悔改的可能(\uwave{伯}后2:4;\uwave{玛}25:41;\uwave{路}16:25-31)。被洪水淹死的人,照人的看法是受了天主的严罚,但是天主却看见其中有许多因悔罪而得救的人。}

\textsuperscript{7}
万事的结局已临近了,所以你们应该慎重,应该醒寤祈祷。
\textsuperscript{8}
最重要的是:你们应该彼此热切相爱,因为爱德遮盖许多罪过;
\textsuperscript{9}
要彼此款待,而不出怨言。
\textsuperscript{10}
各人应依照自己所领受的神恩,彼此服事,善做天主各种恩宠的管理员。
\textsuperscript{11}
谁若讲道,就该按天主的话讲;谁若服事,就该本着天主所赐的德能服事,好叫天主在一切事上,因\uline{耶稣}\uline{基督}而受到光荣:愿光荣和权能归于他,至于无穷之世。阿门。
\footnote{8节见\uwave{雅}5:20。10、11两节:运用天主所赐的特恩,应依照天主的旨意,为光荣\uline{基督},切不可图谋私益和虚荣(\uwave{格}前12章)。}

\textbf{应乐于受苦\quad}
\textsuperscript{12}
亲爱的,你们不要因为在你们中,有试探你们的烈火而惊异,好像遭遇了一件新奇的事;
\textsuperscript{13}
反而要喜欢,因为分受了\uline{基督}的苦难,这样好使你们在他光荣显现的时候,也能欢喜踊跃。
\textsuperscript{14}
如果你们为了\uline{基督}的名字,受人辱骂,便是有福的,因为光荣的神,即天主的神,就安息在你们身上。
\textsuperscript{15}
惟愿你们中谁也不要因做凶手,或强盗,或坏人,或做煽乱的人而受苦。
\textsuperscript{16}
但若因为是基督徒而受苦,就不该以此为耻,反要为这名称光荣天主,
\footnote{宗徒劝告新奉教的人,蒙召入\uline{基督}所立的教会,在世上难免不遭受迫害;并且无辜为\uline{基督}而受人迫害,正是蒙拣选的表记,所以是可庆幸的事(\uwave{玛}5:11、12)。}
\textsuperscript{17}
因为时候已经到了,审判必从天主的家开始;如果先从我们开始,那些不信从天主福音者的结局,又将怎样呢?
\textsuperscript{18}
“如果义人还难以得救,那么恶人和罪人,要有什么结果呢?”
\textsuperscript{19}
故此,凡照天主旨意受苦的人,也要把自己的灵魂托付给忠信的造物主,专务行善。
\footnote{17节“天主的家”指圣教会(\uwave{弟}前3:15)。18节见\uwave{箴}11:31。}

\begin{center}
	\textbf{\large{\songti 各项劝言}}
\end{center}

\textbf{第五章\quad司牧与信友间的义务\quad}
\textsuperscript{1}
所以我这同为长老的,为\uline{基督}苦难作证的,以及同享那将要显示的光荣的人,劝勉你们中间的众长老:
\textsuperscript{2}
你们务要牧放天主托付给你们的羊群;尽监督之职,不是出于不得已,而是出于甘心,随天主的圣意;也不是出于贪卑鄙的利益,而是出于情愿;
\textsuperscript{3}
不是做托你们照管者的主宰,而是做群羊的模范:
\textsuperscript{4}
这样,当总司牧出现时,你们必领受那不朽的荣冠。
\footnote{4:17作者曾提及“天主的家”必要先受审判的话,那么那些管理各地教会的长老,既是管理天主的家人,因此同为长老的\uline{伯多禄}就劝告他们善尽己职,做群羊的模范,这样在审判时,他们才可领受光荣的花冠。\uline{耶稣}为“总司牧”见\uwave{希}13:20。}

\textsuperscript{5}
同样,你们青年人,应该服从长老;大家都该穿上谦卑作服装,彼此侍候,因为“天主拒绝骄傲人,却赏赐恩宠于谦逊人”。
\textsuperscript{6}
为此,你们该屈服在天主大能的手下,这样在适当的时候,他必举扬你们;
\textsuperscript{7}
将你们的一切挂虑都托给他,因为他必关照你们。
\footnote{“青年人”是指一切信友。5节见\uwave{雅}4:6;7节见\uwave{咏}55:23。}

\textbf{最后劝言\quad}
\textsuperscript{8}
你们要节制,要醒寤,因为,你们的仇敌魔鬼,如同咆哮的狮子巡游,寻找可吞食的人;
\textsuperscript{9}
应以坚固的信德抵抗他,也该知道:你们在世上的众弟兄,都遭受同样的苦痛。
\textsuperscript{10}
那赐万恩的天主,即在\uline{基督}内召叫你们进入他永远光荣的天主,在你们受少许苦痛之后,必要亲自使你们更为成全、坚定、强健、稳固。
\textsuperscript{11}
愿光荣与权能归于他,至于无穷之世。阿门。
\footnote{基督徒要以坚固的信德抵抗魔鬼(\uwave{雅}4:7),要彼此友爱和互相团结(\uwave{弟}后3:12;\uwave{希}10:24、25),且要对永远的光荣怀有热烈的希望(1:6、7),这样方能忍受一切暂时的苦难,而达于成全的境界。}

\textbf{问安与祝福\quad}
\textsuperscript{12}
我藉忠信的弟兄\uline{息耳瓦诺},给你们写了这封我认为简短的书信,为劝勉你们,并为证明这实在是天主的恩宠:
在这恩宠上你们应该站稳。
\textsuperscript{13}
与你们一同被选的\uline{巴比伦}教会问候你们;我儿\uline{马尔谷}也问候你们。
\textsuperscript{14}
你们要以爱的亲吻,彼此问候。愿平安与在\uline{基督}内的你们众人同在。
\footnote{最后三节可能是\uline{伯多禄}亲手所写,为说明此信是出于他,\uline{息耳瓦诺}(或\uline{息拉},原是\uline{保禄}的伴侣,见\uwave{宗}15:22、40,18:5等处),只是他的代笔人。\uline{巴比伦}暗指\uline{罗马}。关于\uline{马尔谷},见《马尔谷福音》引言。}
\clearpage
% 伯多禄后书(伯后)
\phantomsection
\addcontentsline{toc}{chapter}{伯多禄后书}
伯多禄后书

伯后
\clearpage

% 若望一书二书三书引言
\phantomsection
\addcontentsline{toc}{chapter}{若望一书二书三书引言}
% 标题
\chapter*{若望一书二书三书引言}

% 右页眉
\rhead{若望一书二书三书引言}

按历来的传统,名为\uline{若望}的三封书信是\uline{若望}宗徒兼圣史写的,并且由书信的内容,也可以证明这古传授的可靠。因为这三书信的语言、词汇、观念、语气和整个笔调,几乎与《若望福音》全同。作者的名字固然不见于三书内,但在后二书中,作者自称为“长老”,在审慎研究之后,敢断定此“长老”应是\uline{若望}宗徒。因为他在世寿数最长,在第1世纪末,宗徒中仅存的,也只有他一人。在初兴教会内尊称他为“长老”,他实当之而无愧。关于\uline{若望}的小传,已见于《若望福音》引言。兹仅就三书信的来历分别论述如下。

\qquad若望一书

本书虽无致候辞,未提收信人是谁,但由作者对收信人的亲切口吻看来(2:1、12-14、18、28,3:7,4:4),可知是\uline{若望}写给与自己有特殊关系的\uline{小亚细亚}的信友(见\uwave{默}2,3);从本书可知他们已是信德成熟的人(2:20、21、27),但因当时在那地方兴起一种异端,危害他们的信德和道德,因此\uline{若望}给他们写了本书,一来为攻斥旁门左道,二来为护卫信友免受毒害。

本书的\uline{希腊}文和文句十分平易浅近,但义理深邃,反复用“光”,“黑暗”,“生命”,“真理”,“爱”等概念,阐明了神学的深奥道理,如\uline{基督}的天主性(2:22、23,5:1),降生的事实(1:1-3,4:2-3),救赎的普遍性等(1:7,2:1、2)。

本书既与第四《福音》有密切的关系(1:1-4与\uwave{若}1:4、14,15:11,16:24等相比较),可说二者是同时著成的,即在第1世纪末,很可能都写于\uline{厄弗所}。

本书除序文(1:1-4)和结论(5:13-21)外,正文可分为三段:第一段:天主是光,所以信友应在光中行走(1:5-2:29);第二段:天主是父,信友都是他的子女,所以应是圣善的,并应彼此相爱(3:1-4:6);第三段:天主是爱,所以信友应在爱天主爱人之德上是成全的(4:7-5:12)。

\qquad若望二书

这是最短的一封信,仅有13节,是“长老”即\uline{若望}宗徒写给一位“蒙选的主母”。这名称很可能是暗指\uline{小亚细亚}的一教会。因作者有意前往该处视察,遂先以此信通知。此外,信中所论的即是应彼此相爱,应信\uline{耶稣}降生成人的道理。可说此书是《若望一书》的提要;既是如此,本书当写于《若望一书》之后,地点大概仍是\uline{厄弗所}。

\qquad若望三书

本书也是很短的一封信,仅有15节,为前书的同一“长老”写给一位名叫\uline{加约}的热心人。今不知此人究属于哪一教会。由本书可知\uline{若望}之所以写信给\uline{加约}一人,可能是因他所处的教会,有位名叫\uline{狄约勒斐}的教长,滥用神权,诽谤宗徒,不承认自己的地位,因此作者信中嘉许\uline{加约}之所为,并劝他继续资助传教士(1-7),严责\uline{狄约勒斐}而称赞\uline{德默特琉}(9-12)。本书大概亦是于第1世纪末写于\uline{厄弗所}的。
\clearpage
% 若望一书(若一)
\phantomsection
\addcontentsline{toc}{chapter}{若望一书}
% 分栏
\setlength\columnsep{0.6cm}
\begin{multicols}{2}

% 标题
\chapter*{若望一书}

% 右页眉
\rhead{若望一书(若一)}

\textbf{第一章\quad序言\quad}
\textsuperscript{1}
论到那从起初就有的生命的圣言,就是我们所见过,我们亲眼看见过,瞻仰过,以及我们亲手摸过的生命的圣言——
\textsuperscript{2}
这生命已显示出来,我们看见了,也为他作证,且把这原与父同在,且已显示给我们的永远的生命,传报给你们——
\textsuperscript{3}
我们将所见所闻的传报给你们,为使你们也同我们相通;原来我们是同父和他的子\uline{耶稣}\uline{基督}相通的。
\textsuperscript{4}
我们给你们写这些事是为叫我们的喜乐得以圆满。
\renewcommand\thefootnote{\ding{\numexpr171+\value{footnote}}}
\footnote{作者不依书信的普通格式,而写了这篇简短的序言,以说明本书的要旨大意。作者给降世的“生命的圣言”,即天主圣子\uline{耶稣}作证,为使读者信仰,好分享这永远的生命,而与天主并与基督徒彼此相通,此即圣\uline{保禄}所论\uline{基督}妙身的道理。作者还用了许多不同的语词(2:5、6、24-29,3:6、9、24,4:7、12、13,5:1、12、20),阐明这端道理。关于“生命的圣言”,见\uwave{若}1注1;11:25,14:6。}

\begin{center}
	\textbf{\large{\songti 天主是光}}
\end{center}

\textbf{人应在光中往来\quad}
\textsuperscript{5}
我们由他所听见,而传报给你们的,就是这个信息:天主是光,在他内没有一点黑暗。
\textsuperscript{6}
如果我们说我们与他相通,但仍在黑暗中行走,我们就是说谎,不履行真理。
\textsuperscript{7}
但如果我们在光中行走,如同他在光中一样,我们就彼此相通,他圣子\uline{耶稣}的血就会洗净我们的各种罪过。

\textsuperscript{8}
如果我们说我们没有罪过,就是欺骗自己,真理也不在我们内。
\textsuperscript{9}
但若我们明认我们的罪过,天主既是忠信正义的,必赦免我们的罪过,并洗净我们的各种不义。
\textsuperscript{10}
如果我们说我们没有犯过罪,我们就是拿他当说谎者,他的话就不在我们内。
\footnote{“光”象征圣善纯洁;“黑暗”象征邪恶罪过(\uwave{若}3:19-21,12:35、36)。人若与天主结合相通,应勉力度圣善纯洁的生活。虽然如此,但人性懦弱,仍难免不犯罪,因此人不但应自知有罪,而且应谦逊明认(\uwave{雅}5:16),如此赖\uline{耶稣}救赎的功劳,罪过必获得赦免;反之,若不承认自己有罪,则是罪上加罪,并且也以天主为“说谎者”,因为天主说明了,人都是罪人(\uwave{箴}20:9;\uwave{德}19:16;\uwave{玛}6:16等处)。}

\textbf{第二章\quad耶稣为人做了赎罪祭\quad}
\textsuperscript{1}
我的孩子们,我给你们写这些事,是为了叫你们不犯罪;但是,谁若犯了罪,我们在父那里有正义的\uline{耶稣}\uline{基督}作护慰者。
\textsuperscript{2}
他自己就是赎罪祭,赎我们的罪过,不但赎我们的,而且也赎全世界的罪过。
\footnote{本章的大意是接上章论人在光中生活的凭据:遵守命令,远离世俗,谨防异端。\uline{耶稣}为赎全世界的罪成了“赎罪祭”,见\uwave{罗}3:25;\uwave{若}11:52;\uwave{希}9:28。}

\textbf{遵守爱德的命令\quad}
\textsuperscript{3}
如果我们遵守他的命令,由此便知道我们认识他。
\textsuperscript{4}
那说“我认识他”,而不遵守他命令的,是撒谎的人,在他内没有真理。
\textsuperscript{5}
但是,谁若遵守他的话,天主的爱在他内才得以圆满;由此我们也知道,我们是在他内。
\textsuperscript{6}
那说自己住在他内的,就应当照那一位所行的去行。
\textsuperscript{7}
可爱的诸位,我给你们写的,不是一条新命令,而是你们从起初领受的旧命令:这旧命令就是你们所听的道理。
\textsuperscript{8}
另一方面说,我给你们写的也是一条新命令——就是在他和你们身上成为事实的——因为黑暗正在消逝,真光已在照耀。
\textsuperscript{9}
谁说自己在光中,而恼恨自己的弟兄,他至今仍是在黑暗中。
\textsuperscript{10}
凡爱自己弟兄的,就是存留在光中,对于他就没有任何绊脚石;
\textsuperscript{11}
但是恼恨自己弟兄的,就是在黑暗中,且在黑暗中行走,不知道自己往哪里去,因为黑暗弄瞎了他的眼睛。
\footnote{6节“那一位”,按\uline{若望}是指\uline{耶稣}。7节称爱人的命令为“旧命令”,因为是读者信教以来早已听过,而且遵守的命令(\uwave{若}二5);8节又称为“新命令”,因为不但是\uline{耶稣}重新所宣布的(\uwave{若}13:34),而且是因了另一个新的动机,就是爱人,要因他是\uline{耶稣}的弟兄(\uwave{玛}25:40;\uwave{若}15:12)。因这新命令,罪恶的黑暗逐渐消逝,福音的真光弥漫人世。}

\textbf{远离世俗\quad}
\textsuperscript{12}
孩子们,我给你们写说:因他的名字,你们的罪已获得赦免。
\textsuperscript{13}
父老们,我给你们写说:你们已认识了从起初就有的那一位。青年们,我给你们写说:你们已得胜了那恶者。
\textsuperscript{14}
小孩子们,我给你们写过:你们已认识了父。父老们,我给你们写过:你们已认识了从起初就有的那一位。青年们,我给你们写过:你们是强壮的,天主的话存留在你们内,你们也得胜了那恶者。

\textsuperscript{15}
你们不要爱世界,也不要爱世界上的事;谁若爱世界,天父的爱就不在他内。
\textsuperscript{16}
原来世界上的一切:肉身的贪欲,眼目的贪欲,以及人生的骄奢,都不是出于父,而是出于世界。
\footnote{为彻底根绝这三罪根,\uline{耶稣}提出了绝财、绝色、绝意的三种劝谕(\uwave{玛}19)。13、14两节内的“恶者”指魔鬼(3:8)。}
\textsuperscript{17}
这世界和它的贪欲都要过去;但那履行天主旨意的,却永远存在。

\textbf{提防假基督\quad}
\textsuperscript{18}
小孩子们,现在是最末的时期了!就如你们听说过假\uline{基督}要来,如今已经出了许多假\uline{基督},由此我们就知道现在是最末的时期了。
\textsuperscript{19}
他们是出于我们中的,但不是属于我们的,因为,如果是属于我们的,必存留在我们中;但这是为显示他们都不是属于我们。
\footnote{“最末的时期”,即指世纪的末期,此期由\uline{耶稣}升天开始直到他再来为止。“假\uline{基督}”是指\uwave{得}后2:3、4所述那世末要来的极力反抗\uline{基督}的魁首;“许多假\uline{基督}”是指作他前驱和预像的人,即一切迫害天国和传扬邪道的人(4:3;\uwave{若}二7;\uwave{玛}24:24)。}

\textsuperscript{20}
至于你们,你们由圣者领受了傅油,并且你们都晓得。
\footnote{“圣者”即指\uline{耶稣}。“傅油”或译作“傅润”,即指圣神,因傅油为领受圣神的记号(27节;\uwave{格}后1:21)。}
\textsuperscript{21}
我给你们写信,不是你们不明白真理,而是因为你们明白真理,并明白各种谎言不是出于真理。

\textsuperscript{22}
谁是撒谎的呢?岂不是那否认\uline{耶稣}为默西亚的吗?那否认父和子的,这人便是假\uline{基督}。
\textsuperscript{23}
凡否认子的,也否认父;那明认子的,也有父。
\footnote{参阅\uwave{若}5:23,15:25。}
\textsuperscript{24}
至于你们,应把从起初所听见的,存留在你们内;如果你们从起初所听见的,存留在你们内,你们必存留在子和父内。
\textsuperscript{25}
这就是他给我们所预许的恩惠:既永远的生命。

\textsuperscript{26}
这些就是我关于迷惑你们的人,给你们所写的。
\textsuperscript{27}
至于你们,你们由他所领受的傅油,常存在你们内,你们就不需要谁教训你们,而是有他的傅油教训你们一切。这傅油是真实的,决不虚假,所以这傅油怎样教训你们,你们就怎样存留在他内。
\footnote{“真理之神”教导信徒一切,见\uwave{若}14:26,16:13。}
\textsuperscript{28}
现在,孩子们,你们常存在他内罢!为的是当他显现时,我们可以放心大胆,在他来临时,不至于在他面前蒙羞。
\textsuperscript{29}
你们既然知道他是正义的,就该知道凡履行正义的,都是由他而生的。
\footnote{“履行正义”,即度圣善的生活。}

\begin{center}
	\textbf{\large{\songti 天主是父}}
\end{center}

\textbf{第三章\quad天父的子女应相似天父\quad}
\textsuperscript{1}
请看父赐给我们何等的爱情,使我们得称为天主的子女,而且我们也真是如此。世界所以不认识我们,是因为不认识父。
\textsuperscript{2}
可爱的诸位,现在我们是天主的子女,但我们将来如何,还没有显明;可是我们知道:一显明了,我们必要相似他,因为我们要看见他实在怎样。
\footnote{信友成为天主的子女,并不是虚名,而是实在享有为天主子女的福分,有分于天主的性体(\uwave{伯}后1:4)。如果人在今世已到了如此的地步,那么在来世所享的福分更是人难以想像的,他要面对面看见天主的本体,分享他的光荣(\uwave{格}前13:12;\uwave{罗}8:17;\uwave{格}后13:18)。}
\textsuperscript{3}
所以,凡对他怀着这希望的,必圣洁自己,就如那一位也是圣洁的一样。
\footnote{作天主子女的凭据有三:即应相似天主般圣洁(3:4-9),友爱兄弟(3:10-23),信\uline{耶稣}为天主子,为默西亚(4:1-6)。若无此三凭据,即是魔鬼的子女。}
\textsuperscript{4}
凡是犯罪的,也就是作违法的事,因为罪过就是违法。
\textsuperscript{5}
你们也知道,那一位曾显示出来,是为除免罪过,在他身上并没有罪过。
\textsuperscript{6}
凡存在他内的,就不犯罪过;凡犯罪过的,是没有看见过他,也没有认识过他。

\textsuperscript{7}
孩子们,万不要让人迷惑你们!那行正义的,就是正义的人,正如那一位是正义的一样
\textsuperscript{8}
那犯罪的,是属于魔鬼,因为魔鬼从起初就犯罪:天主子所以显现出来,是为消灭魔鬼的作为。
\textsuperscript{9}
凡由天主生的,就不犯罪过,因为天主的种子存留在他内,他不能犯罪,因为他是由天主生的。
\textsuperscript{10}
天主的子女和魔鬼的子女在这事上可以认出:就是凡不行正义的和不爱自己弟兄的,就不是出于天主。
\footnote{8节见\uwave{若}8:44,12:31。9节“天主的种子”,即指圣宠。此处不是说:信友已不能犯罪,而是说:当他与天主结合时,既与罪过势不两立,哪能再犯罪?(6节)人犯罪必是先离开了天主,拒绝了他的助佑。}

\textbf{爱人的命令\quad}
\textsuperscript{11}
原来你们从起初所听的训令就是:我们应彼此相爱;
\textsuperscript{12}
不可像那属于恶者和杀害自己兄弟的\uline{加音}。\uline{加音}究竟为什么杀了他?因为他自己的行为是邪恶的,而他兄弟的行为是正义的。

\textsuperscript{13}
弟兄们,如果世界恼恨你们,不必惊奇。
\textsuperscript{14}
我们知道,我们已出死入生了,因为我们爱弟兄们;那不爱的,就存在死亡内。
\textsuperscript{15}
凡恼恨自己弟兄的,便是杀人的;你们也知道:凡杀人的,便没有永远的生命存在他内。
\textsuperscript{16}
我们所以认识了爱,因为那一位为我们舍弃了自己的生命,我们也应当为弟兄们舍弃生命。
\textsuperscript{17}
谁若有今世的财物,看见自己的弟兄有急难,却对他关闭自己怜悯的心肠,天主的爱怎能存在他内?
\footnote{12节\uline{加音}的事,见\uwave{创}4:5、8。\uline{若望}举出不爱人的两个例证:杀人与见难不救。“杀人”在此不但指实在行凶,而且也指存恨人的心(见\uwave{玛}5:21、22)。}

\textsuperscript{18}
孩子们,我们爱,不可只用言语,也不可只用口舌,而要用行动和事实。
\textsuperscript{19}
在这一点上我们可以认出,我们是出于真理的,并且在他面前可以安心;
\textsuperscript{20}
纵然我们的心责备我们,我们还可以安心,因为天主比我们的心大,他原知道一切。
\footnote{由于爱天主而行爱人的善工,可知我们是出于天主,是存在他内的人;也确知我们的罪过已获得赦免,因为我们待人仁慈,天主也必待我们仁慈无限(\uwave{玛}6:14;\uwave{伯}前4:8)。}
\textsuperscript{21}
可爱的诸位,假使我们的心不责备我们,在天主前便可放心大胆;
\textsuperscript{22}
那么我们无论求什么,必由他获得,因为我们遵守了他的命令,行了他所喜悦的事。
\textsuperscript{23}
他的命令就是叫我们信他的子\uline{耶稣}\uline{基督}的名字,并按照他给我们所出的命令,彼此相爱。
\textsuperscript{24}
那遵守他命令的,就往在他内,天主也住在这人内。我们所以知道他住在我们内,是籍他赐给我们的圣神。
\footnote{22节见\uwave{若}15:7。24节“籍……圣神”,见\uwave{罗}8:9、15、16。}

\textbf{第四章\quad真理的神和欺诈的神\quad}
\textsuperscript{1}
可爱的诸位,不要凡神就信,但要考验那些神是否出于天主,因为有许多假先知来到了世界上。
\textsuperscript{2}
你们凭此可认出天主的神:凡明认\uline{耶稣}为\uline{默西亚},且在肉身内降世的神,便是出于天主;
\textsuperscript{3}
凡否认\uline{耶稣}的神,就不是出于天主,而是属于假\uline{基督}的;你们已听说过他要来,现今他已在世界上了。
\textsuperscript{4}
孩子们,你们出于天主,且已得胜了他们,因为那在你们内的,比那在世界上的更大。
\textsuperscript{5}
他们属于世界,因此讲论属于世界的事,而世界就听从他们。
\textsuperscript{6}
我们却是出于天主的;那认识天主的,必听从我们;那不出于天主的,便不听从我们:由此我们可以认出真理的神和欺诈的神来。
\footnote{“神”在此段中,指感发“先知”或“教师”的神能。这神能可能出于圣神,也可能出于魔鬼;因此信友应谨慎辨别这神能的真假(\uwave{格}前12:10;\uwave{得}前5:21)。辨别真假的标准,即是承认或否认\uline{耶稣}为\uline{默西亚},为降生成人的天主子。4节“那在世界上的”即指魔鬼(\uwave{若}12:30,14:30)。6节的“我们”指众宗徒和传教士(\uwave{路}10:16)。}

\begin{center}
	\textbf{\large{\songti 天主是爱}}
\end{center}

\textbf{以爱还爱\quad}
\textsuperscript{7}
可爱的诸位,我们应该彼此相爱,因为爱是出于天主;凡有爱的,都是生于天主,也认识天主;
\textsuperscript{8}
那不爱的,也不认识天主,因为天主是爱。
\textsuperscript{9}
天主对我们的爱在这事上已显出来:就是天主把自己的独生子,打发到世界上来,好使我们籍着他得到生命。
\textsuperscript{10}
爱就在于此:不是我们爱了天主,而是他爱了我们,且打发自己的儿子,为我们做赎罪祭。
\footnote{在本书末段(4:7-5:12),更积极讲明成为天主子女的信徒,应彼此相爱的原故,即因“天主是爱”。天主的性体是“爱”,是一切爱的根源。这是全部圣经中启示的最高峰。天主为表现他的爱,使圣子降生成人(\uwave{若}3:16),好使人分沾此爱。}

\textsuperscript{11}
可爱的诸位,既然天主这样爱了我们,我们也应该彼此相爱。
\textsuperscript{12}
从来没有人瞻仰过天主;如果我们彼此相爱,天主就存留在我们内,他的爱在我们内才是圆满的。
\textsuperscript{13}
我们所以知道我们存留在他内,他存留在我们内,就是由于他赐给了我们的圣神。
\footnote{天主是看不见的(\uwave{若}1:18),但人若彼此相爱,就可以知道自己与天主相通。因为除非天主存在人心内,人不能真心爱人,因为爱是圣神在信友心中所产生的效果(\uwave{迦}5:19)。}
\textsuperscript{14}
至于我们,我们却曾瞻仰过,并且作证:父打发了子来作世界的救主。
\textsuperscript{15}
谁若明认\uline{耶稣}是天主子,天主就存在他内,他也存在天主内。
\textsuperscript{16}
我们认识了,且相信了天主对我们所怀的爱。

天主是爱,那存留在爱内的,就存留在天主内,天主也存留在他内。
\textsuperscript{17}
我们内的爱得以圆满,即在于此:就是我们可在审判的日子放心大胆,因为那一位怎样,我们在这世界上也怎样。
\textsuperscript{18}
在爱内没有恐惧,反之,圆满的爱把恐惧驱逐于外,因为恐惧内含着惩罚;那惩罚的,在爱内还没有圆满。
\footnote{人彼此相爱,才知人的爱,是圆满的,是真实的(12节);有此爱的人,到审判之日,毫无所惧,因为爱使他与\uline{耶稣}相似(\uwave{若}15:12)。18节所说的恐惧,是指人怕惩罚而不敢犯罪,这是纯奴隶性的恐惧;像这样的人,心中只有自私,而没有爱。至于怕爱天主不足,或怕得罪天主的敬畏之情,是人应培养的超性德行。}

\textsuperscript{19}
我们应该爱,因为天主先爱了我们。
\textsuperscript{20}
假使有人说:我爱天主,但他却恼恨自己的弟兄,便是撒谎的;因为那不爱自己所看见的弟兄的,就不能爱自己所看不见的天主。
\textsuperscript{21}
我们从他蒙受了这命令:那爱天主的,也该爱自己的弟兄。
\footnote{见\uwave{玛}22:37-39;\uwave{若}15:9、12、17。}

\textbf{第五章\quad信德是得胜世界的力量和得永生的根基\quad}
\textsuperscript{1}
凡信\uline{耶稣}为\uline{默西亚}的,是由天主所生的;凡爱生他之父的,也必爱那由他所生。
\textsuperscript{2}
几时我们爱天主,又遵行他的诫命,由此知道我们真爱天主的子女。
\textsuperscript{3}
原来爱天主,就是遵行他的诫命,而他的诫命并不沉重,
\textsuperscript{4}
因为凡是由天主所生的,比得胜世界;得胜世界的胜利武器,就是我们的信德。
\textsuperscript{5}
谁是得胜世界的呢?不是那信\uline{耶稣}为天主子的人吗?
\footnote{信德是使人成为天主的子女,爱天主爱人的根基,又是得胜世界的武器(\uwave{若}16:33)。}

\textsuperscript{6}
这位就是经过水及血而来的\uline{耶稣}\uline{基督},他不但以水,而且也是水及血而来的;并且有圣神作证,因为圣神是真理。
\textsuperscript{7}
原来作证的有三个:
\textsuperscript{8}
就是圣神,水及血,而这三个是一致的。
\textsuperscript{9}
人的证据,我们既然接受,但天主的证据更大,因为天主的证据就是他为自己的子作证。
\textsuperscript{10}
那信天主子的,在自己内就怀有这证据;那不信天主的,就是以天主为撒谎者,因为他不信天主为自己的子所作的证。
\textsuperscript{11}
这证据就是天主将永远的生命赐给了我们,而这生命是在自己的子内。
\textsuperscript{12}
那有子的,就有生命;那没有天主子的,就没有生命。
\footnote{当时,有些异端人否认\uline{耶稣}是真天主子及救世者\uline{默西亚}(\uline{基督}),只承认\uline{基督}在受洗时,有神性的能力由天降下与他的人性结合,使他行奇迹异事;但在他受苦时又离去了,如此在十字架上死的,只是一个软弱的人,而他流的血不能救赎世人。\uline{若望}为攻斥这个异端,说明\uline{耶稣}的天主性与其人性结合,完成了救世的使命:他“经过水”,即在受洗时开始了救赎的工程(\uwave{玛}3:13-17;\uwave{若}1:32-34),他也“经过血”,即为救赎人类死了(\uwave{若}1:29,19:34-37)。现今为\uline{耶稣}作证更有力的,是他从天上打发来,保护圣教会,赐与信友永远生命的“真理的圣神”(\uwave{若}15:26,16:13-15)。——7节《拉丁通行本》增:“在天上作证的,有父、圣言和圣神,并且这三位是一致的。”这经文证明圣三的道理再明显没有了;但古教父从未引用,亦不见于古\uline{希腊}抄卷与古译本中,仅见于4世纪末的书籍中,后又窜入《拉丁通行本》内。}

\textbf{结论\quad信友的崇高地位\quad}
\textsuperscript{13}
我给你们这些信天主子名字的人,写了这些事,是为叫你们知道:你们已获有永远的生命。
\textsuperscript{14}
我们对天主所怀的依恃之心就是:如果我们按他的旨意求什么,他必俯听我们。
\footnote{见3:22。}
\textsuperscript{15}
我们既然知道:我们不拘向他祈求什么,他会俯听我们,我们也知道向他所祈求的,必要得到。
\textsuperscript{16}
谁若看见自己的弟兄犯了不至于死的罪,就应当祈求,天主必赏赐他生命:这是为那些犯不至于死的罪人而说的;然而有的罪却是至于死的罪,为这样的罪,我不说要人祈求。
\footnote{“至于死的罪”,即背教离弃生命的根源\uline{耶稣}的罪。\uline{若望}以为不必为这样的罪人祈祷,因为不是天主不赏他回头的神恩,而是罪人故意轻视赐神恩的\uline{耶稣}而固执于恶(\uwave{玛}12:31)。“不至于死的罪”,即人因软弱犯的罪,为这样的罪人,应求天主赏赐他悔改,恢复他超性的生命。}
\textsuperscript{17}
任何的不义都是罪过,但也有不至于死的罪过。

\textsuperscript{18}
我们知道:凡由天主生的,就不犯罪过;而且由天主生的那一位必保全他,那恶者不能侵犯他。
\footnote{见3:9。“由天主生的那一位”指\uline{耶稣}。“恶者”指魔鬼,见2:13、14。}
\textsuperscript{19}
我们知道我们属于天主,而全世界却屈服于恶者。
\textsuperscript{20}
我们也知道天主子来了,赐给了我们理智,叫我们认识那真实者;我们确实是在那真实者内,他即是真实的天主和永远的生命。
\footnote{“那真实者”是指天主圣父,也是指\uline{耶稣}(\uwave{若}14:6)。}
\textsuperscript{21}
孩子们,你们要谨慎,远避偶像!

\end{multicols}
\clearpage
% 若望二书(若二)
\phantomsection
\addcontentsline{toc}{chapter}{若望二书}
若望二书(若二)
\clearpage
% 若望三书(若三)
\phantomsection
\addcontentsline{toc}{chapter}{若望三书}
\chapter{若望三书}
\clearpage

% 犹达书引言
\phantomsection
\addcontentsline{toc}{chapter}{犹达书引言}
% 标题
\chapter*{犹达书引言}

% 右页眉
\rhead{犹达书(犹)引言}

本书作者的名字,在致候辞内即已标出:“\uline{雅各伯}的兄弟\uline{犹达}”。他是宗徒之一,在\uwave{玛}10:3及\uwave{谷}3:18又名为\uline{达陡},在\uwave{路}6:15和\uwave{宗}1:13又称为“\uline{雅各伯}的\uline{犹达}”。他自称为“\uline{雅各伯}的兄弟”,当然表示他和\uline{耶}京的主教\uline{雅各伯}宗徒有亲戚的关系。他们二人究有什么亲戚关系,很难确定,恐怕仅是堂兄弟或表亲而已,或者是同父异母的兄弟(\uwave{玛}27:56;\uwave{谷}15:40),很可能他和继\uline{雅各伯}作\uline{耶}京主教的\uline{西满}是同胞兄弟(\uwave{玛}13:55;\uwave{谷}6:3),为此被列于“主的兄弟”中。关于他的生平事迹,《新约》没有特殊的记载,口传关于他的记述也不多,且不可靠。他大概在\uline{叙利亚}及其附近传布了福音,最后为主殉道。圣教会每年于10月28日庆祝他殉道的节日。

本书是写给“在天主父内可爱的,为\uline{耶稣}\uline{基督}保存的蒙召者”:这样的称呼似乎是泛指一切信友;但从书信的内容来看,特别由作者屡次引用《旧约》和提及\uline{犹太}民间传说一点来看,可以推断收信人应是由\uline{犹太}教归化的信友,由此也可明白作者为什么特称自己为“\uline{雅各伯}的兄弟”的原因。

\uline{犹达}写此信的动机,是因为他听到这些信友已处在假学士和异端邪说的威胁下,所以写下此信,为保卫信友的信德,指明这些假学士及其害人的异端邪说(4,8,10,12,16,23节)。

本书的文笔简单,但生动有力,近似《旧约》中的先知文体。他喜用比喻,富于想象。作者为了容易表明真理,连\uline{犹太}民间所流行而载于伪经上的事迹言论也加以引用(9,14两节),正如\uline{保禄}也引用过外教作家和教外诗人的诗句一样。

本书写作的时期,大约是在64与65年间,即在次\uline{雅各伯}死后(62年),\uwave{伯}后写作之前。本书写作的地点,大概是在\uline{巴力斯坦},或者就是在\uline{耶路撒冷}。

本书仅有25节,可分为两段:在前段中(5-16节),说出《旧约》中的前例,恐吓假学士将要遭受天主的惩罚;在后段中(17-23节),劝告信友坚持信德,热心发挥爱德的力量,远避假学士的异端邪说。
\clearpage
% 犹达书(犹)
\phantomsection
\addcontentsline{toc}{chapter}{犹达书}
犹达书(犹)
\clearpage

% 若望默示录引言
\phantomsection
\addcontentsline{toc}{chapter}{若望默示录引言}
\input{introduction/59_Introduction_to_the_Revelation}
\clearpage
% 默示录(默)
\phantomsection
\addcontentsline{toc}{chapter}{默示录}
默示录(默)
\clearpage

\part{附\quad录}

% 附七 天主教基督教圣经目录比较表
\phantomsection
\addcontentsline{toc}{chapter}{附七\quad天主教基督教圣经目录比较表}
% 标题
\chapter*{附七\quad天主教基督教圣经目录比较表}

% 右页眉
\rhead{附七\quad天主教基督教圣经目录比较表}

\begin{center}
	\textbf{旧约}
\end{center}


{
	% 字体大小
	\scriptsize
	
	\begin{longtable}{|l|l|l|l|l|l|}
		
		% 表头
		\hline 
		\textbf{天主教译名} & \textbf{天主教简称} & \textbf{基督教译名} & \textbf{基督教简称} & \textbf{英文} & \textbf{英文简称} \\ 
		\hline 
		\endhead
		
		创世纪 & 创 & 创世记 & 创 & Genesis & Gn \\
		\hline
		出谷纪 & 出 & 出埃及记 & 出 & Exodus & Ex \\
		\hline
		肋未纪 & 肋 & 利未记 & 利 & Leviticus & Lev \\
		\hline
		户籍纪 & 户 & 民数记 & 民 & Numbers & Num \\
		\hline
		申命纪 & 申 & 申命记 & 申 & Deuteronomy & Dtn \\
		\hline
		诺苏厄书 & 苏 & 约书亚记 & 书 & Joshua & Josh \\
		\hline
		民长纪 & 民 & 士师记 & 士 & Judges & Judg \\
		\hline
		卢德传 & 卢 & 路得记 & 得 & Ruth & Ruth \\
		\hline
		撒慕尔纪上 & 撒上 & 撒母尔记上 & 撒上 & 1 Samuel & 1Sam \\
		\hline
		撒慕尔纪下 & 撒下 & 撒母尔记下 & 撒下 & 2 Samuel & 2Sam \\
		\hline
		列王纪上 & 列上 & 列王记上 & 列上 & 1 Kings & 1Kings \\
		\hline
		列王纪下 & 列下 & 列王记下 & 列下 & 2 Kings & 2Kings \\
		\hline
		编年纪上 & 编上 & 历代志上 & 历上 & 1 Chronicles & 1Chron \\
		\hline
		编年纪下 & 编下 & 历代志下 & 历下 & 2 Chronicles & 2Chron \\
		\hline
		厄斯德拉上 & 厄上 & 以斯拉书 & 拉 & Ezra & Ezra \\
		\hline
		厄斯德拉下 & 厄下 & 尼希米记 & 尼 & Nehemiah & Neh \\
		\hline
		多俾亚传 & 多 & × & × & Tobit & Tob \\
		\hline
		友弟德传 & 友 & × & × & Judith & Jud \\
		\hline
		艾斯德尔传 & 艾 & 以斯帖记 & 帖 & Esther & Esther \\
		\hline
		玛加伯上 & 加上 & × & × & 1 Maccabees & 1Mac \\
		\hline
		玛加伯下 & 加下 & × & × & 2 Maccabees & 2Mac \\
		\hline
		约伯传 & 约 & 约伯记 & 伯 & Job & Job \\
		\hline
		圣咏集 & 咏 & 诗篇 & 诗 & Psalms & Ps \\
		\hline
		箴言 & 箴 & 箴言 & 箴 & Proverbs & Prov \\
		\hline
		训道篇 & 训 & 传道书 & 传 & Ecclesiastes & Eccles \\
		\hline
		雅歌 & 歌 & 雅歌 & 歌 & Song of Solomon & Song \\
		\hline
		智慧篇 & 智 & × & × & Wisdom & Wis \\
		\hline
		德训篇 & 德 & × & × & Sirach/Ecclesiasticus & Sir \\
		\hline
		依撒意亚 & 依 & 以赛亚书 & 赛 & Isaiah & Is \\
		\hline
		耶肋米亚 & 耶 & 耶利米书 & 耶 & Jeremiah & Jer \\
		\hline
		哀歌 & 哀 & 耶利米哀歌 & 哀 & Lamentations & Lam \\
		\hline
		巴路克 & 巴 & × & × & Baruch & Bar \\
		\hline
		厄则克耳 & 则 & 以西结书 & 结 & Ezekiel & Ezk \\
		\hline
		达尼尔 & 达 & 但以理书 & 但 & Daniel & Dan \\
		\hline
		欧瑟亚 & 欧 & 何西阿书 & 何 & Hosea & Hos \\
		\hline
		岳厄尔 & 岳 & 约珥书 & 珥 & Joel & Jl \\
		\hline
		亚毛斯 & 亚 & 阿摩斯书 & 摩 & Amos & Am \\
		\hline
		亚北底亚 & 北 & 俄巴底亚书 & 俄 & Obadiah & Ob \\
		\hline
		约纳 & 纳 & 约拿书 & 拿 & Jonah & Jon \\
		\hline
		米该亚 & 米 & 弥迦书 & 弥 & Micah & Mic \\
		\hline
		纳鸿 & 鸿 & 那鸿 & 鸿 & Nahum & Nahum \\
		\hline
		哈巴谷 & 哈 & 哈巴谷书 & 哈 & Habakkuk & Hab \\
		\hline
		索福尼亚 & 索 & 西番亚书 & 番 & Zephaniah & Zep \\
		\hline
		哈盖 & 盖 & 哈该书 & 该 & Haggai & Hg \\
		\hline
		匝加利亚 & 匝 & 撒迦利亚书 & 亚 & Zechariah & Zec \\
		\hline
		玛拉基亚 & 拉 & 玛拉基书 & 玛 & Malachi & Mal \\
		\hline
		
	\end{longtable} 
	
}

\newpage

\begin{center}
	\textbf{新约}
\end{center}

{
	% 字体大小
	\scriptsize
	
	\begin{longtable}{|l|l|l|l|l|l|}
		
		% 表头
		\hline 
		\textbf{天主教译名} & \textbf{天主教简称} & \textbf{基督教译名} & \textbf{基督教简称} & \textbf{英文} & \textbf{英文简称} \\ 
		\hline 
		\endhead
		
		玛窦福音 & 玛 & 马太福音 & 太 & Matthew & Mt \\
		\hline
		马尔谷福音 & 谷 & 马可福音 & 可 & Mark & Mk \\
		\hline
		路加福音 & 路 & 路加福音 & 路 & Luke & Lk \\
		\hline
		若望福音 & 若 & 约翰福音 & 约 & John & Jn \\
		\hline
		宗徒大事录 & 宗 & 使徒行传 & 徒 & Acts & Acts \\
		\hline
		罗马书 & 罗 & 罗马人书 & 罗 & Romans & Rom \\
		\hline
		格林多前书 & 格前 & 哥林多前书 & 林前 & 1 Corinthians & 1Cor \\
		\hline
		格林多后书 & 格后 & 哥林多后书 & 林后 & 2 Corinthians & 2Cor \\
		\hline
		迦拉达书 & 迦 & 加拉太书 & 加 & Galatians & Gal \\
		\hline
		厄弗所书 & 弗 & 以弗所书 & 弗 & Ephesians & Eph \\
		\hline
		斐理伯书 & 斐 & 腓立比书 & 腓 & Philippians & Phil \\
		\hline
		哥罗森书 & 哥 & 歌罗西书 & 西 & Colossians & Col \\
		\hline
		得撒洛尼前书 & 得前 & 帖撒罗尼迦前书 & 帖前 & 1 Thessalonians & 1Thes \\
		\hline
		得撒洛尼后书 & 得后 & 帖撒罗尼迦后书 & 帖后 & 2 Thessalonians & 2Thes \\
		\hline
		弟茂德前书 & 弟前 & 提摩太前书 & 提前 & 1 Timothy & 1Tim \\
		\hline
		弟茂德后书 & 弟后 & 提摩太后书 & 提后 & 2 Timothy & 2Tim \\
		\hline
		弟铎书 & 铎 & 提多书 & 多 & Titus & Tt \\
		\hline
		费肋孟书 & 费 & 腓利门书 & 门 & Philemon & Phlm \\
		\hline
		希伯来书 & 希 & 希伯来书 & 来 & Hebrews & Heb \\
		\hline
		雅各伯书 & 雅 & 雅各书 & 雅 & James & Jm \\
		\hline
		伯多禄前书 & 伯前 & 彼得前书 & 彼前 & 1 Peter & 1P \\
		\hline
		伯多禄后书 & 伯后 & 彼得后书 & 彼后 & 2 Peter & 2P \\
		\hline
		若望一书 & 若一 & 约翰一书 & 约壹 & 1 John & 1Jn \\
		\hline
		若望二书 & 若二 & 约翰二书 & 约贰 & 2 John & 2Jn \\
		\hline
		若望三书 & 若三 & 约翰三书 & 约叁 & 3 John & 3Jn \\
		\hline
		犹达书 & 犹 & 犹大书 & 犹 & Jude & Jd \\
		\hline
		默示录 & 默 & 启示录 & 启 & Revelation & Rev \\
		\hline
		
	\end{longtable}
	
}
\clearpage

\end{document}
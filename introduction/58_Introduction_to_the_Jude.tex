% 标题
\chapter*{犹达书引言}

% 右页眉
\rhead{犹达书(犹)引言}

本书作者的名字,在致候辞内即已标出:“\uline{雅各伯}的兄弟\uline{犹达}”。他是宗徒之一,在\uwave{玛}10:3及\uwave{谷}3:18又名为\uline{达陡},在\uwave{路}6:15和\uwave{宗}1:13又称为“\uline{雅各伯}的\uline{犹达}”。他自称为“\uline{雅各伯}的兄弟”,当然表示他和\uline{耶}京的主教\uline{雅各伯}宗徒有亲戚的关系。他们二人究有什么亲戚关系,很难确定,恐怕仅是堂兄弟或表亲而已,或者是同父异母的兄弟(\uwave{玛}27:56;\uwave{谷}15:40),很可能他和继\uline{雅各伯}作\uline{耶}京主教的\uline{西满}是同胞兄弟(\uwave{玛}13:55;\uwave{谷}6:3),为此被列于“主的兄弟”中。关于他的生平事迹,《新约》没有特殊的记载,口传关于他的记述也不多,且不可靠。他大概在\uline{叙利亚}及其附近传布了福音,最后为主殉道。圣教会每年于10月28日庆祝他殉道的节日。

本书是写给“在天主父内可爱的,为\uline{耶稣}\uline{基督}保存的蒙召者”:这样的称呼似乎是泛指一切信友;但从书信的内容来看,特别由作者屡次引用《旧约》和提及\uline{犹太}民间传说一点来看,可以推断收信人应是由\uline{犹太}教归化的信友,由此也可明白作者为什么特称自己为“\uline{雅各伯}的兄弟”的原因。

\uline{犹达}写此信的动机,是因为他听到这些信友已处在假学士和异端邪说的威胁下,所以写下此信,为保卫信友的信德,指明这些假学士及其害人的异端邪说(4,8,10,12,16,23节)。

本书的文笔简单,但生动有力,近似《旧约》中的先知文体。他喜用比喻,富于想象。作者为了容易表明真理,连\uline{犹太}民间所流行而载于伪经上的事迹言论也加以引用(9,14两节),正如\uline{保禄}也引用过外教作家和教外诗人的诗句一样。

本书写作的时期,大约是在64与65年间,即在次\uline{雅各伯}死后(62年),\uwave{伯}后写作之前。本书写作的地点,大概是在\uline{巴力斯坦},或者就是在\uline{耶路撒冷}。

本书仅有25节,可分为两段:在前段中(5-16节),说出《旧约》中的前例,恐吓假学士将要遭受天主的惩罚;在后段中(17-23节),劝告信友坚持信德,热心发挥爱德的力量,远避假学士的异端邪说。
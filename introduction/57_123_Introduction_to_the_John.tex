% 标题
\chapter*{若望一书二书三书引言}

% 右页眉
\rhead{若望一书二书三书引言}

按历来的传统,名为\uline{若望}的三封书信是\uline{若望}宗徒兼圣史写的,并且由书信的内容,也可以证明这古传授的可靠。因为这三书信的语言、词汇、观念、语气和整个笔调,几乎与《若望福音》全同。作者的名字固然不见于三书内,但在后二书中,作者自称为“长老”,在审慎研究之后,敢断定此“长老”应是\uline{若望}宗徒。因为他在世寿数最长,在第1世纪末,宗徒中仅存的,也只有他一人。在初兴教会内尊称他为“长老”,他实当之而无愧。关于\uline{若望}的小传,已见于《若望福音》引言。兹仅就三书信的来历分别论述如下。

\qquad若望一书

本书虽无致候辞,未提收信人是谁,但由作者对收信人的亲切口吻看来(2:1、12-14、18、28,3:7,4:4),可知是\uline{若望}写给与自己有特殊关系的\uline{小亚细亚}的信友(见\uwave{默}2,3);从本书可知他们已是信德成熟的人(2:20、21、27),但因当时在那地方兴起一种异端,危害他们的信德和道德,因此\uline{若望}给他们写了本书,一来为攻斥旁门左道,二来为护卫信友免受毒害。

本书的\uline{希腊}文和文句十分平易浅近,但义理深邃,反复用“光”,“黑暗”,“生命”,“真理”,“爱”等概念,阐明了神学的深奥道理,如\uline{基督}的天主性(2:22、23,5:1),降生的事实(1:1-3,4:2-3),救赎的普遍性等(1:7,2:1、2)。

本书既与第四《福音》有密切的关系(1:1-4与\uwave{若}1:4、14,15:11,16:24等相比较),可说二者是同时著成的,即在第1世纪末,很可能都写于\uline{厄弗所}。

本书除序文(1:1-4)和结论(5:13-21)外,正文可分为三段:第一段:天主是光,所以信友应在光中行走(1:5-2:29);第二段:天主是父,信友都是他的子女,所以应是圣善的,并应彼此相爱(3:1-4:6);第三段:天主是爱,所以信友应在爱天主爱人之德上是成全的(4:7-5:12)。

\qquad若望二书

这是最短的一封信,仅有13节,是“长老”即\uline{若望}宗徒写给一位“蒙选的主母”。这名称很可能是暗指\uline{小亚细亚}的一教会。因作者有意前往该处视察,遂先以此信通知。此外,信中所论的即是应彼此相爱,应信\uline{耶稣}降生成人的道理。可说此书是《若望一书》的提要;既是如此,本书当写于《若望一书》之后,地点大概仍是\uline{厄弗所}。

\qquad若望三书

本书也是很短的一封信,仅有15节,为前书的同一“长老”写给一位名叫\uline{加约}的热心人。今不知此人究属于哪一教会。由本书可知\uline{若望}之所以写信给\uline{加约}一人,可能是因他所处的教会,有位名叫\uline{狄约勒斐}的教长,滥用神权,诽谤宗徒,不承认自己的地位,因此作者信中嘉许\uline{加约}之所为,并劝他继续资助传教士(1-7),严责\uline{狄约勒斐}而称赞\uline{德默特琉}(9-12)。本书大概亦是于第1世纪末写于\uline{厄弗所}的。
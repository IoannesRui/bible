% 标题
\chapter*{迦拉达书引言}

% 右页眉
\rhead{迦拉达书(迦)引言}

\uline{迦拉达}人原是古代\uline{法国}南部\uline{高卢}人的一支,公元前三世纪先迁徙至\uline{小亚细亚}中部,以后逐渐扩展至\uline{小亚细亚}南部。公元前25年,\uline{奥古斯都}将他们的地域通划为\uline{迦拉达}皇帝省。\uline{保禄}在第一次传教行程中,同\uline{巴尔纳伯}已在\uline{迦拉达}南部创立了不少教会(\uwave{宗}13:14,14:24),在第二次传教时又到此处巡视(\uwave{宗}16:1-5);为此,我们认为本书即是写给此处的各教会:即\uline{丕息狄雅}的\uline{安提约基雅}、\uline{依科尼雍}、\uline{吕斯特辣}和\uline{德尔贝}等地的教会。

\uline{保禄}写这本书的动机,是因为他听说在\uline{迦拉达}各教会内,有些\uline{犹太}主义保守派人,到处散布邪说,攻击\uline{保禄},扬言\uline{保禄}既不属“十二宗徒集团”,当然不是真宗徒,因而他所传的福音,也不是\uline{基督}的真福音;并且声言:人为得救,必须遵守\uline{梅瑟}法律,并行割损礼。\uline{保禄}见教会处于这种重大的危机中,便写了这封书信,以驳斥这些邪说。

本书写作的动机既如上述,\uline{保禄}在口授这封信时,心情自然不免激昂愤慨,措辞不免有些锋利;但这不但不消减他对信友的慈爱,反而更彰显出他对信友的关怀,以及对\uline{基督}的满腔热爱。

至于本书写于何时何地的问题,虽没有决定性的答案,但从本书的内容与\uline{保禄}其他的书信比较来看,当在《格林多后书》之后及《罗马》书之前,即大约写于公元57年年底,地点当在\uline{格林多}或\uline{马其顿}。

本书除序言(1:1-5)和结论(6:11-18)外,可分为三段:第一段、\uline{保禄}极力证明自己的宗徒职权(1:6-2:21);第二段、力陈旧约法律为成义毫无作用,人为成义必须有赖对\uline{基督}的信德(3-4);第三段、略论人成义后所获得的地位,和几项针对信友实际生活的劝言(5:1-6:10)。
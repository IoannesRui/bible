% 标题
\chapter*{弟铎书引言}

% 右页眉
\rhead{弟铎书(铎)引言}

\uline{弟铎}的事迹不见于\uwave{宗}只散见于\uline{保禄}各书信内(\uwave{迦}、\uwave{格}后、\uwave{铎}及\uwave{弟}后)。他生于外教家庭(\uwave{迦}2:3),大概是在\uline{安提约基}\uline{雅}为\uline{保禄}所归化,因而称为\uline{保禄}的“真子”(1:4)。公元49年,曾随\uline{保禄}前往\uline{耶}京开宗徒会议(\uwave{迦}2:1;\uwave{宗}15:2),以后在\uline{保禄}第三次传教行程中(53-58年),曾被委派办理了几项要务(见\uwave{格}后2:13,7:6,8:6、16,12:18)。65年随\uline{保禄}至\uline{克里特}岛,被祝圣为该地的主教(1:5);随后接获此信,前往\uline{尼苛颇里}(3:12),后又从该城被派往\uline{达耳玛提雅}(\uwave{弟}后4:10)。根据教会的口传,他后来又返回\uline{克}岛,在那里寿终正寝。

\uline{保禄}写本书的动机,由本书内容看来,与\uwave{弟}前完全一样,即教导他如何管理该岛的教会。本书写作的时间,应在\uwave{弟}前书写后不久,即在65年,因为本书与\uwave{弟}前内所有的劝言与指示,几乎完全相同;地点是在\uline{马其顿}。

本书可分为三段:第一段1:5-16:论选立圣职人员的规范和他们应有的品格;第二段2:1-3:7:论应如何对待各级人士,应如何训诲他们弃恶迁善;第三段3:8-11:论行善的必要,并应如何驳斥异端邪说。
% 标题
\chapter*{出谷纪引言}

% 右页眉
\rhead{出谷纪(出)引言}

“五书”的第二部紧接《创世纪》,继续记述\uline{以色列}子孙的事迹。我国译本名为《出埃及记》,或《出谷纪》。后一译名似乎更合适本书的深意,因为天主将\uline{以色列}子民由\uline{埃及}救出的事实,实是他要把全人类由罪恶的深渊中,救出来的预象和初步实现。

本书上编(1-18章)记述\uline{以色列}(雅各伯)的子孙,在\uline{埃及}国所受的压迫。但天主决未忘却他向\uline{亚巴郎}和其他圣祖所许的诺言,所以选拔了\uline{梅瑟}为民族的救星,叫他领导自己的同胞出离\uline{埃及},以伟大的奇迹,救他们脱离奴隶的生活,领他们来到\uline{西乃}旷野,一路上使他们经验了天主大能的呵护,令他们坚心信赖天主的照顾。

中编(19-24章)和下编(25-40章)可称为全旧约的中心,记载天主在\uline{西乃}山上,将自己启示给\uline{以色列}子民,给他们颁布了十诫和法律,藉\uline{梅瑟}与他们立了盟约,使他们成为特选的民族,成为神权政体的国民。从那时起,天主自己作他们的君王和领袖,住在百姓中的帐幕内;并任命\uline{肋未}的子孙,代百姓在会幕内服役,行祭献天主的大礼。

本书虽可说是\uline{以色列}子民,建国立宪的一部极有关系的史书,但因为没有提及\uline{埃及}王朝或君王的名号,故此本书的史事,究竟发生在何时,无法确定。据一般经学家的推究,大约是在纪元前十三世纪中叶。

本书就神学观点来说,具有极崇高的价值,给人类启示了天主的超然存在,和他的唯一性以及至圣性;同时也显示了他对自己百姓的慈爱和照顾。至于他向人类要求的,是对他应怀有赤诚的信赖,以及知恩报爱等美德。由于本书所记载的史事,大都含有预象的意义,为此为新约教会的生活和礼仪,也有其特殊的价值。
% 标题
\chapter*{序}

% 右页眉
\rhead{序}

中文《新旧约》合订本,几经艰辛,终于出版面试了。现在谨以孝爱的心肠,与兴奋的心情,将此书献给我国的天主教会。我们由衷地感谢万善万美之源在天之父,虔恭地恳求他,降福我们的每一位神形恩友。我们明知,如果没有他们精神方面的支持与物质方面的援助,这部\uline{中}译圣经合订本,很难于今日完成。我们更恳求天父,以圣神的恩泽,充溢每位读者的心灵,为光荣他,和他的圣子,我们的救主\uline{耶稣}\uline{基督}。最后,我们以至诚的心,深望每位读者,效法天主之母,童贞圣母\uline{玛利亚},常将天主子\uline{耶稣}\uline{基督}的奥迹,默存于心,反复思念(\uwave{路}2:19、51)。

我们坦诚地说过应说的话以后,兹将有关出版这部合订本的原则与方法,向诸位读者报告如下:

本学会同仁,对本学会以前的译文,作了一次彻底的修订;特别对于《旧约》部分,觉得有重新翻译的必要。这次的修订,一如以往翻译一样,仍旧依据原文,即\uline{希伯来}、\uline{阿剌美}和\uline{希腊}文,间或有时依据古译本,将艰涩的经文稍加修改,很少采用近代学者的臆测。为了促进合一运动,我们在《新约》的修订工作上,曾参考了最近基督教五个圣经公会联合印行的《圣经》希腊版本。这版本是1966年由\uline{艾朗德}(Aland)\uline{白赖克}(Black)\uline{墨则格}(Metzger)\uline{魏革伦}(Wikgren)四位极负盛名的学者所细心校勘的。不过,在选择异文方面,有时我们未能尽表赞同或采纳。然而,众所周知,在校勘方面,要想使所有学者的意见,完全趋于一致,几乎是不可能的事;同时,艰深疑难的问题,也实在可说是无法解决;但为了遵守有助于合一运动的措施,我们仍尊重而参考了他们的意见。

在翻译《多俾亚传》和《友弟德传》时,我们采用了《西乃抄卷》,因而与\uline{拉丁}通行本和其他近代译本,略有出入,因为这些译本,大半是根据《亚历山大抄卷》或《梵蒂冈抄卷》而译成的。

《德训篇》一书,我们虽译自\uline{希腊}文,但同时也参考了最近在古\uline{开罗}的旧书库中,以及在\uline{死海}近旁的\uline{古木兰},和1965年在\uline{玛撒达}所发现的本书\uline{希伯来}文残卷。为了更进一步了解本书\uline{希腊}译者的原意,我们自始至终,将\uline{希腊}本,与古\uline{拉丁}译本和《叙利亚培西托》译本,加以比较和对正。至于\uline{拉丁}通行本所特有的辞句,我们仍旧全部保留,以小字体排印;我们如此作,是因为这些经文往往与\uline{希伯来}文残卷和\uline{叙利亚}译本极相吻合;尤其因为在\uline{拉丁}教会内,许多世纪以来,在礼仪上采用了这些译文。至于本书的章节,全依\uline{拉丁}通行本排列。

至论《圣咏集》的翻译,虽也译自\uline{希伯来}原文,但我们却作了一个新的尝试,就是放弃了过去无韵文的翻译,而试作有韵文的翻译。采取这种翻译的理由,是为了在礼仪上便于诵读。

为了切合实际,我们尽量缩短了每卷的引言和注释。对于那些欲求圣经高深知识的读者,我们仍建议,请他们参考本学会以前出版的《圣经》各卷注释。对于那些工作繁重与事务缠身,而没有充分时间的读者,我们认为本书的引言、注释以及各种附录和图表,已够他们了解《圣经》的主旨。如果读者能进一步,如圣教会所期望的,以祈祷和默祷的方式阅读本书,则更受益匪浅。

现在再向读者特别介绍本书后面的三种附录,这类附录在\uline{中}文《圣经》方面尚属少见。

一、“历代大司祭一览表”:读者可从这表内,正确地明晰一切有关宗教事务的事实,发生于何时何地,发生在哪一位大司祭的任期中。

二、“圣经与世界大事年表”:在这表内,我们以编年体,将选民的历史,以及与选民有关的其他民族的历史,和我们\uline{中华}民族的历史,加以比较,列表阐明。我们认为这一附录,可使我国同胞,更方便而有效地传诵救恩史。

三、“圣经教义索引”:这是为了符合\uline{梵蒂冈}第二届大公会议的期望,而特别编制的,目的是使信友迅速而无误地,直接从圣经中吸取天主启示的各端要理,走上愈显天父的光荣,生活于基督内的康庄大道。我们希望藉此索引,得到抛砖引玉的效果,就是说,希望每位读者藉此索引,能将救恩史的大纲,加以质的扩充和量的增补。

至于地理图表,以及其他有关圣经的插图,都是参照专家和学者们最新的探讨与研究的心得,而予以重新编制的。本书的题字“圣经”、“旧约”、“新约”、六字,是临自《大秦景教流行中国碑》。

最后,我们愿用一首庄严而又古雅的赞颂辞,向我全国教胞表达我们的心愿,并结束这篇序文:愿光荣藉着天主圣子,在天主圣神内,并在圣教会内,永远归于天主圣父!

% 空三行
\myspace{3}

\rightline{思高圣经学会谨识于香港\qquad}
\rightline{1968年8月15日\qquad\qquad}
\rightline{圣母升天瞻礼 \qquad\qquad}
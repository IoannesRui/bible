% 标题
\chapter*{新约全书导论}

% 右页眉
\rhead{新约全书导论}

“新约全书”是\uline{耶稣}死后,由其宗徒弟子,在天主圣神的默感与引导之下,所写成的经典汇集。此汇集由第二世纪起即称为《新约书》,或简称《新约》。称之为“约”,因为其中所讲论的,是天主与人类所立的盟约;称之为“新”,以别于“旧约”。“旧约”是天主与\uline{以}民在\uline{西乃}山上所立的圣约,而“新约”是\uline{基督}以自己的圣血与圣死,在天主与人间,所建立的救恩圣约(参阅\uwave{玛}26:28;\uwave{谷}14:24等处)。

“新约全书”,按圣教会古老的传授,共计二十七卷。

《历史书》五卷:《玛窦福音》、《马尔谷福音》、《路加福音》、《若望福音》和《宗徒大事录》。

《训诲书》二十一卷:圣\uline{保禄}书信十四封:《罗马书》、《格林多》前后二书、《迦拉达书》、《厄弗所书》、《斐理伯书》、《哥罗森书》、《得撒洛尼》前后二书、《弟茂德》前后二书、《弟铎书》、《费肋孟书》和《希伯来书》;公函七封:《雅各伯书》、《伯多禄》前后二书、《若望》一、二、三书并《犹达书》。

《先知书》一卷:《若望默示录》。

《新约全书》,除《玛窦福音》的原文为\uline{阿剌美}文外,都是用\uline{希腊}文写成的。这些似乎有些奇怪,因为按当时\uline{耶稣}在世时,和宗徒最初讲道时所用的语言,本来都是\uline{阿剌美}语,并且全部《新约》作者,除圣\uline{路加}外,又都是\uline{犹太}人;那么为什么不用本国文字编写呢?其理由是因为只有《玛窦福音》是写给\uline{巴力斯坦}的\uline{犹太}人,而其余的书都是写给说\uline{希腊}话的基督徒,其中很少有通晓\uline{阿剌美}语的;更何况《新约》又是向天下万民所公布的;因此以当时\uline{罗马}帝国内所通行的\uline{希腊}语编写,是很自然的事。

《新约全书》(或《新经》),就宗教方面来说,远远超过《旧约全书》(或《古经》),因为天主在旧约时代只是“多次并以多种方式,籍着先知对祖先说过话”;然而在新约时代却是“籍着子对我们说了话”(\uwave{希}1:1)。如此,旧约的启示在新约内才得以圆满;旧约的预许在新约内才得以实现。所以吾人除非认识《新约》,决不能完全明了《旧约》;为此,可说《新约全书》实是世界上最重要和最宝贵的作品。
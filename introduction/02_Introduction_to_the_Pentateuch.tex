\chapter{梅瑟五书引论}

《旧约全书》前五卷,通称“梅瑟五书”,或简称“五书”。因为在这五卷书内,包含着《旧约》中最重要的一部分,即\uline{梅瑟}给\uline{以色列}人所宣布的法律,为此圣经上多次称《五书》为“法律”。\uline{希伯来}人称之为“托辣”。

《梅瑟五书》虽然在纪元前已有如此的分划:即《创世纪》、《出谷纪》、《肋未纪》、《户籍纪》、《申命纪》,但自古以来这五卷书常视为一部,且是一部世界文学上的杰作。

如果说全部圣经的主题是阐述人类的救赎史,那么“五书”即是记述这救赎史的开端。作者从天地和人类的创造开始,说到人类因违背天主的命令,而失掉原有的幸福,再扼要地叙述各民族的太古史,继而只着重于\uline{以色列}民族的起源,及其成为天主选民的历史。这历史的中心即在于天主与选民在\uline{西乃}山上所结的盟约。天主很早即对\uline{以色列}人的先祖再三地许诺,要以特殊照顾和非凡的奇事,准备\uline{以}民的心灵,使他们对于天主养成坚定不移的信仰,以后好籍着\uline{梅瑟}选立他们为自己的国民,颁下当遵行的法律,在世上建立起神权政体的神国。以后又在旷野四十年之久,以种种试探考验了他们的忠诚和信心;最后引他们到了\uline{约但}河东岸,在那里又籍着\uline{梅瑟}劝告他们,重述以前教导过他们的一切,准备他们进占已预许给他们祖先的福地。所以从历史方面来看,“五书”有其统一的目标,实是一部上下一贯的著述。

按古来一致的传授,“五书”的作者是\uline{梅瑟}。称他为作者,并不是说全书每字每句都出于他一个人的手笔,而是说他曾搜集了不少当时所能找得到的史料、文献和法律。且在他死后,有许多历史或法律部分是后人增补的,因为“五书”原是\uline{以色列}人宗教、政治、社会生活的法典,所以常有随时增添一些解释的必要,为使\uline{梅瑟}法律能随历史的演变,而适应时代的环境。

从以上所述,可知“五书”为\uline{以色列}人具有多么重要的关系。如果我们对“五书”没有认识,便不能明了\uline{以色列}子民的历史,因为他们生活在一个神权政体的制度之下,他们的存亡盛衰,全系于他们是否忠实履行天主籍\uline{梅瑟}所颁布的法律。在《旧约》其他经书内,常不断指出法律的这种重要关系,并依法律为原则,来批判一切历史的得失。但这法律的最终目的,诚如圣\uline{保禄}所说:“法律的终向本来是基督”(\uwave{罗}10:4)。为此,法律为\uline{以色列}人,好像是“归于基督的启蒙师”(\uwave{迦}3:24)。换句话说,法律应领导\uline{以色列}人,认识并信仰将要来临的默西亚。
% TODO \uline{耶稣}\uline{基督}----下划线太长,导致两个\uline{}像一个。
当默西亚\uline{耶稣}\uline{基督}一降生,法律的使命就算完结,\uline{耶稣}所宣讲的“爱的诫命”,满全了整个法律(\uwave{罗}13:10)。虽然如此,“五书”为《新约》的教会,仍未失其重要性,因为本书含有永生天主的启示,以及教会信仰的基础。
\chapter{新旧约全书总论}

《新旧约全书》,是数十卷经书的总集。这些经书的特点,在于它们的写成有超乎自然之处,因为这些经书都是在天主圣神默感下写成,赐予天主的之民——教会——的礼物。

圣教会自古以来,一致主张这部总集包括《旧约》46卷,《新约》27卷,共计73卷。但大多数的\uline{基督教}派,由于只相信以\uline{希伯来}文写成的书才为圣经,因此现今只有\uline{希腊}原文的《巴路克》、《多俾亚传》、《友弟德传》、《玛加伯》上下、《智慧篇》和《德训篇》7卷,未著录在他们的圣经书目内。而天主教会自古即以\uline{希腊}文七十贤士本为圣经,因而对上述7卷也一律认为是圣经。

这些经书称为“约”,因为其中心思想,是天主与人类所立的盟约。天主与\uline{以色列}民族在\uline{西乃}山上所立的盟约,称为“旧约”;\uline{耶稣}以自己的圣血和圣死为全人类所立的永远盟约,称为“新约”。这些经书又称为“圣经”,是为表示这些书所具有的独特地位和神圣权威。书中所记述的一切,是吾人信仰及道德的大经,又为吾人立身经世的大道。

《旧约》经书的原文,除几卷和几小段外,大都以\uline{希伯来}文写成。后来侨居\uline{北非}受了\uline{希腊}文化影响的\uline{犹太}人,因多不谙\uline{希伯来}文,\uline{犹太}人遂在公元前三至二世纪,将《旧约》各书译为\uline{希腊}文,即今所称的“七十贤士译本”。以后\uline{希腊}语文也成了\uline{罗马}帝国的通用语言,宗徒们在各地宣讲福音,为了方便起见,即时常利用这部\uline{希腊}文圣经,为此这部\uline{希腊}文圣经(包括46卷)自初即为教会所尊重,并具有极大的权威。

《新约》各书,全部是以\uline{希腊}文写成,只有《玛窦福音》,原文虽为\uline{阿辣美}文,但很早即已失传;今所留传的,只有\uline{希腊}文本。

《旧约全书》的写成,凡经一千余年(约由公元前1300至100年),而逐渐汇为一集。《新约全书》是公元初世纪宗徒时代的作品。

《新旧约全书》,通常分为三大类:即历史书、先知书和智慧书(或训诲书),这是很广泛的分类。至于作者,《旧约》大多出于先知及其他贤哲的手笔,《新约》是宗徒和宗徒弟子的写作。但因全部圣经都是“因天主的默感写成的”(\uwave{第}后3:16),经内的话是“由天主所派遣的圣人,在圣神推动之下说出来的”(\uwave{伯}后1:21),为此我们不得不承认圣经的首要作者是天主。所谓“默感”,即是说:圣经的作者与编者(人),在天主的灵性感动之下,写下天主愿向他的子民(旧约与新约的教会)所要说的话,记下天主要他们记述的史事。有时天主也曾向他们透露某些重要的事迹,或直接向他们说话;这样,作者不仅获得了“默感”,同时也获得了“启示”。既然天主是圣经的首要作者,那么圣经上所记载的即是天主的话,即是天主的“圣言”。既然是天主的话,那么圣经上所载的一切,句句都“真实无误”。就是说:圣经作者在天主默感下所愿表示出来的意义,是不会错误的。但为了解作者所要表达的本意,必须先注意经书中每部书的文体和体裁:是散文或是诗体?是历史或是传奇?是寓言或是训诲?因为每种文体有其独特的意义。同时还应注意作者或编者的时代背景,因为时代不同,论事的观点也各有异。比如古代民族,尤其\uline{以色列}人对历史的观点,和今日的史学家的观点,有绝大的不同。尤其圣经的作者或编者,是本着宗教观点来编述历史的过程。他们看历史时,常着眼于天主为历史的推定者和支配者;人民的盛衰兴亡,常系之于他们是否遵守天主的法律。

另一个极重要的问题,是圣经与科学。圣经的作者决无意以教授自然科学(如宇宙学、天文学、生物学、人类学等)为写作的目的。圣经作者的目的,是在于启迪人类“获得拯救的智慧”(\uwave{第}后3:15);为此他们无意研究自然界的进化和人体的构造。其用意只在说明自然界和人类与天主的关系,教导世人,天地万物都来自天主,一切都因天主的照顾而生存,最后又归于天主。

还应当注意的是:为适当地研究圣经和解释经意,人人必须先有信仰,并甘心接受圣教会的指导,因为天主把圣经委托给教会保管,因此只有教会才有解释圣经的特权。

圣经中所记载的都是些最重要的真理,教父多称圣经为天主给流徒的世人寄来的家书或天书。在这部天书内,天主先将自己启示给世人并告知世人,天地万物的来历和目的,告诉我们天主原先怎样给人类备下了幸福,现今的痛苦、患难和死亡又怎样来的;天主在漫长的历史过程中,怎样逐步实现了他救赎人类的大计划。《旧约》所载,即是天主为\uline{以色列}民族所行的大事,所定的法律,所发的劝言和警告。这一切都是他为完成救赎人类工程的准备,甚至\uline{以}民的被选,也是为准备万民获得救恩。《新约》则是全人类得救的大喜讯,记载人类唯一的救主\uline{耶稣} \uline{基督}救赎人类的大事;因为\uline{耶稣}是天主第二位圣子,只有他才能把天主性体的真理启示给人类。他降生成人,籍着自己的人性,完成了天主慈爱的计划,使人与天主重归于好,且提高了人类的地位,使人分享天主性的永福。

由此看来,\uline{耶稣} \uline{基督}实为《旧新》二约的中心,是“法律”《旧约》的终向(\uwave{罗}10:4),是“新约的中保”(\uwave{希}8:6)。《旧新》二约各经卷的最后目的,就是叫人准备期待“我们伟大的天主及救主耶稣基督的光荣显现”(\uwave{铎}2:13)。

圣经为人类得救既有如此重要的关系,因此圣经对圣教会,对于一切基督信徒,对全人类,的确是举世无双的无价宝书。圣\uline{保禄}论《旧约》说:“凡所写的,都是为教训我们写的”(\uwave{罗}15:4),“为教训,为督责,为矫正,为教导人学正义,都是有益的”(\uwave{第后3:16})。这些话对《新约》来说,更为恰当;而且可说,如果没有圣经,我们无论对天主,或是对人,不会有一个正确的认识,因为只凭理性的自然神学是不够的,决不能打动人心,唯有研读圣经才能触及我们灵魂的深处,使我们听得见“生活天主”的话,领略天主威爱兼有的声音,洞见他全能的伟大化工,明白他怎样生养保存万物,怎样以他至高无上的主权宰治一切,裁判一切;又怎样以他慈父心肠,导引迷途的荡子,回归父家。无怪乎教宗\uline{良}十三称圣经为“神学的灵魂”。

诚然,一个怀有信德的教友,在恭读默思圣经时,应觉得是与天主会晤,是在静听天主的劝导,是在听他在天之父的慈音。当他心有所得,情有所动时,自然就向天主说话,这即是祈祷。无怪乎圣教会自古即以圣经为赞美、祈祷、默想最好的宝书。信友如能日日如此读经,与天主互诉衷曲,在日常的生活上或工作上,必能时时对越天主,承行他的圣意,臻于圣化一切的至境。为此教会不断劝勉信友多读圣经,尤其这次大公会议,对圣经研究与圣经诵读特予强调。愿我信友善体慈母教会的劝告,勉力天天去阅读这部天赐宝书。
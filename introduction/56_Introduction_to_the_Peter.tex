% 标题
\chapter*{伯多禄前后书引言}

% 右页眉
\rhead{伯多禄前后书引言}

宗徒之长\uline{伯多禄}的小史史料来源有二:一是圣经,一是教会的圣传。\uline{伯多禄}在蒙召之前,名叫\uline{西满},\uline{伯多禄}(“磐石”之意,\uline{希伯来}文为\uline{刻法})是\uline{耶稣}给他改的名字(\uwave{若}1:24)。他与胞兄\uline{安德肋}出生于\uline{加里肋亚}湖北岸的\uline{贝特赛达}城,身为渔夫。在第一次捕鱼的奇迹后,\uline{耶稣}才召他为宗徒,为渔人的渔夫(\uwave{路}5:8-11)。\uline{伯多禄}在众宗徒中,连在\uline{耶稣}的三位爱徒中,常居首位(\uwave{玛}10:2,17:1,26:37;\uwave{谷}5:37);他领受了元首职权(\uwave{玛}16:13-19)和\uline{耶稣}特为他祈祷的预许(\uwave{路}22:31、32);在\uline{耶稣}复活后,隆重地接受了管理全教会的元首职权(\uwave{若}21:15-17);在\uline{耶稣}升天后,由《宗徒大事录》的前半部可知,他作了\uline{耶稣}在世的代表,始终执行了他的元首职权。公元约43、44年,按\uwave{宗}12:17的记载:他“往别的地方去了”,大概是去了\uline{罗马}。49年又出现于\uline{耶}京,主持宗徒会议。随后曾到过\uline{安提约基}\uline{雅}(\uwave{迦}2:11-14)。再后,除知道他由\uline{罗马}写了这两封书信外,《新约》经书再没有记载他的事迹。按\uline{欧色彼}和\uline{热罗尼莫}的记载,他以后定居于\uline{罗马},在\uline{尼禄}皇帝时,即公元67年,为主殉难而死,他的遗体葬于\uline{梵蒂冈}山岗。圣教会每年6月29日庆祝他殉难的节日。

\qquad伯多禄前书

\uwave{伯}前的作者确实是\uline{伯多禄}宗徒,因为在信首的致候辞内已明明写出;此外,古传说和书信内容也一致如此证明。若把本书内容和\uwave{宗}内\uline{伯多禄}的讲辞作一比较,彼此间也十分相合。

本书的收信人,是散居在\uline{小亚细亚}北部的信友。写本书的动机,是因为作者听到该处的信友,不断遭受教外与\uline{犹太}人的迫害,受着背弃信德的威胁(2:12,3:14-16,4:12-16),所以宗徒写了此信,为安慰他们的忧苦,坚固他们的信德,劝勉他们:困难无论如何重大,仍当善度真正信友的生活。

本书写于\uline{巴比伦}(5:13),即\uline{罗马}(以\uline{巴比伦}指\uline{罗马},见\uwave{默}17以及初世纪\uline{犹太}人的作品)。写作的时间,大约是在\uline{尼禄}教难之前,即公元63至64年间,其时\uline{保禄}适在\uline{西班牙}。

本书的\uline{希腊}文虽间有\uline{闪}族的语风,仍堪称典雅。全书的中心思想既是劝勉信友保持信德,效法\uline{基督}的德表,善度信友的生活,所以内容方面几乎全是有关道德的言论,对教义问题,只偶然涉及而已(1:3,3:18-22)。

本书除致候辞(1:1、2),序言(1:3-12)和结尾语外(5:12-14),可分为四段:第一段:泛论信友应该善度真正基督徒的生活(1:3-2:20);第二段:分论信友对各级人士,以及彼此间应尽的义务(2:21-3:12);第三段:劝勉信友要追随\uline{基督}的德表,忍受一切苦难与迫害(3:13-4:19);第四段:几项有关信友团体生活的特别劝言(5:1-11)。

\qquad伯多禄后书

由于\uwave{伯}后的语气与文笔和\uwave{伯}前有显著的不同,因而有不少人怀疑本书是出于\uline{伯多禄}之手;不过这种怀疑实属多余,因本书信信首,明言写信人是\uline{伯多禄}宗徒(1:1),并且在信内作者曾说自己瞻仰过\uline{耶稣}显圣容(1:17、18)。前后二书的语气与文笔之所以不同,圣\uline{热罗尼莫}认为是作者用了不同的代笔人,大部分学者皆以此说为是。

\uwave{伯}后的收信人与\uwave{伯}前同。作者写本书的动机,似乎是因接到了有关读者的一些消息,得知他们所处的环境较前更为恶劣,此时,除遭受政府方面的迫害外,在教会以内也发生了不少错误思想,因有些假教师潜入教会,扰乱信友(2:1-3、11)。所以作者写这信的目的,除安慰鼓励信友外,特别是为驳斥那些假教师的谎言谬论。

本书写作的时间,按1:14所记,应在作者逝世前不久,即约在公元66至67年间,其时\uline{保禄}大约再度被捕入狱。写作的地方仍是\uline{罗马}。

本书有不少地方与《犹达书》的内容相似,这极可能是\uline{伯多禄}参考过《犹达书》,他认为《犹达书》所写的,颇适合他的读者所处的环境,因而采用了一些语句。

本书除致候辞(1:1、2)和结尾语外(3:17、18),可分为两大段:第一段:劝勉信友注重修德行善的实际生活(1:3-21);第二段:特别驳斥假教师们的邪说谬论(2:1-3:16)。此外,读者读本书时,应注意\uline{伯多禄}在3:15、16对\uline{保禄}的书信所说的话。

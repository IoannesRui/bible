% 标题
\chapter*{玛窦福音引言}

% 右页眉
\rhead{玛窦福音(玛)引言}

第一部《福音》的作者是圣\uline{玛窦}宗徒。\uline{玛窦}又名\uline{肋未},是\uline{阿耳斐}的儿子(\uwave{谷}2:14)。他在\uline{耶稣}召叫之前,曾在\uline{葛法翁}作过税吏。他一被召,即刻舍弃一切,跟随了\uline{耶稣}(\uwave{玛}9:9;\uwave{谷}2:13、14;\uwave{路}5:27、28)。\uline{耶稣}升天后,他先在\uline{巴力斯坦}一带,给自己的同胞宣讲福音多年,然后动身往外方传教去了。最后死在何处何时,史无确证。圣教会从古以来,即认他为一位为主殉道的宗徒,每年九月二十一日庆祝他的瞻礼。

据最古的传授,圣教会始终认为圣\uline{玛窦}是第一部《福音》的作者;这也可由《福音》书内的暗示得到证明:例如\uline{马尔谷}与\uline{路加}记载十二位宗徒名单时,只记了\uline{玛窦}的名字,然而在第一部《福音》内,于“\uline{玛窦}”名字前却加上了受人歧视的“税吏”头衔,可知原作者对自己的职位,毫不避讳。

《玛窦福音》的原著为\uline{阿剌美}文,因为是为\uline{巴力斯坦}的\uline{犹太}人写的,这是自古以来圣教会一致公认的事。此书后来不知由何人译为\uline{希腊}文。本《福音》因为是写给归化的\uline{犹太}人,因此特别力证\uline{耶稣}\uline{基督}即是天主所预许及先知所预言的“默西亚”。虽然大多数\uline{犹太}人否认\uline{耶稣}为默西亚,并把他置于死地:然而他却由死者中光荣复活,并建立了自己的教会作为天国在世上的开端,继续他救世的使命。由于这个特殊的目的,\uline{玛窦}比其他三位圣史,更强调先知们的预言在\uline{耶稣}身上全应验了。

本书的著作地点,大概是\uline{耶路撒冷}。至于著作时期,原文可说是写于其他《福音》之前,大约著于公元50年左右;现行的\uline{希腊}译本,大概是成于《马尔谷》和《路加》两福音之后,约在公元70年左右。

本书记述\uline{耶稣}言行,并未全按编年的次第,而是出于作者的匠心独运。他把\uline{耶稣}公开传教的整个生活分作五段,每段先记事,后记言。此五段即是:(一)3-7;(二)8-10;(三)11-13:53;(四)13:54-18;(五)19-25。

本《福音》因是四《福音》中材料最丰富的一部,在结构上又是最有系统的一部,为此本《福音》在教会内应用最广,引用最多。
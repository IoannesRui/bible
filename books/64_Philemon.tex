% 标题
\chapter*{费肋孟书}

% 右页眉
\rhead{费肋孟书(费)}

% 正文
\textbf{致候辞\quad}
\textsuperscript{1}
\uline{基督}\uline{耶稣}的被囚者\uline{保禄},和\uline{弟茂德}弟兄,致书给我们可爱的合作者\uline{费肋孟},
\textsuperscript{2}
并给姊妹\uline{阿丕雅},和我们的战友\uline{阿尔希颇},以及在你家中的教会。
\textsuperscript{3}
愿恩宠与平安,由天主我们的父及主\uline{基督}\uline{耶稣}赐与你们!
\renewcommand\thefootnote{\ding{\numexpr171+\value{footnote}}}
\footnote{\uline{保禄}常将信友的生活与传教的事业比作战斗(\uwave{罗}7:23;\uwave{格}后10:4;\uwave{弟}前1:18;\uwave{弟}后2:3),因此也将从事宗徒事业的人称为战友。2节“在你家中的教会”,参阅本书引言。}

\textbf{谢恩\quad}
\textsuperscript{4}
在我的祈祷中纪念你时,我常感谢我的天主,
\textsuperscript{5}
因为,听说你对主\uline{耶稣},和对众圣徒所表现的爱德与信德。
\textsuperscript{6}
我祈求天主,为使你因信德而怀有的慷慨发生功效,使你认清我们所能行的一切善事,都是为\uline{基督}而行的。
\textsuperscript{7}
弟兄,我由于你的爱德,确实获得了极大的喜乐和安慰,因为藉着你,圣徒们的心都舒畅了。
\footnote{\uline{保禄}在祈祷中,时常为\uline{费}氏感谢天主,是因了他所有的爱德与信德:他的信德表现在行爱人的善工上,他的爱德却因信德而发。7节“圣徒们”是指当地的信徒们。}

\textbf{为敖乃息摩求情\quad}
\textsuperscript{8}
为此,我虽然在\uline{基督}内,能放心大胆地命你去作这件该作的事,
\textsuperscript{9}
可是,我这年老的\uline{保禄},如今且为\uline{基督}\uline{耶稣}作囚犯的,宁愿因着爱德求你,
\textsuperscript{10}
就是为我在锁链中所生的儿子\uline{敖乃息摩}来求你。
\textsuperscript{11}
他曾一度为你是无用的,可是,如今为你为我都有用了;
\textsuperscript{12}
我现今把他给你打发回去,【你收下】他,他是我的心肝。
\textsuperscript{13}
我本来愿意将他留在我这里,叫他替你服侍我这为福音而被囚的人,
\textsuperscript{14}
可是没有你的同意,我什么也不愿意做,好叫你所行的善不是出于勉强,而是出于甘心。
\textsuperscript{15}
也许他暂时离开了你,是为叫你永远收下他,
\textsuperscript{16}
不再当一个奴隶,而是超过奴隶,作可爱的弟兄:他为我特别可爱,但为你不拘是论肉身方面,或是论主方面,更加可爱。
\footnote{\uline{保禄}为感动\uline{费}氏,叫他宽免并善待潜逃的奴隶\uline{敖乃息摩},便在信里加上自己年老及为\uline{耶稣}坐监等情况(9节)。他称\uline{敖}氏为自己在锁链中所生之子(10节),是说自己在狱中归化了他,给他付了洗,使他重生于\uline{基督};更称他是自己的“心肝”(12节),是希望\uline{费}氏不再将\uline{敖}氏当奴隶看待,却当“可爱的弟兄”看待,因为人一领洗,就成了天主的子女,彼此平等,都是弟兄,在天主前已没有主奴之分。从这里可以清楚看出:\uline{保禄}愿用\uline{基督}的博爱主义逐渐废除奴隶制度。16节“论肉身方面,或是论主方面”,是说:\uline{敖}氏无论从本性方面(因为他是\uline{费}家的奴隶),或超性方面(因为他在\uline{基督}内成了\uline{费}氏的弟兄),都应受主人的爱护。}
\textsuperscript{17}
所以,若你以我为同志,就收留他当作收留我罢!
\textsuperscript{18}
他若亏负了你或欠下你什么,就算在我的账上罢!
\textsuperscript{19}
我\uline{保禄}亲手签字:“我必要偿还。”至于你,你所欠于我的,竟是你本身;这我就不必对你说了!
\footnote{“你所欠于我的”,是说:你因了我的讲劝得了超性的生命,这神恩远远超过\uline{敖}氏所欠于你的账。}
\textsuperscript{20}
弟兄!望你使我在主内得此恩惠,并在\uline{基督}内使我的心舒畅!
\textsuperscript{21}
我自信你必听从,才给你写了这信;我知道,就是超过我所说的,你也必作。
\textsuperscript{22}
同时,也请你给我准备一个住处,因为我希望因你们的祈祷,主必要把我赐与你们。
\footnote{\uline{保禄}希望因信友们的祈祷,自己能够出狱,再往东方去,与他们相见,促膝谈心。}

\textbf{问安与祝福\quad}
\textsuperscript{23}
为\uline{基督}\uline{耶稣}与我一同被囚的\uline{厄帕夫辣}、
\textsuperscript{24}
我的合作者\uline{马尔谷}、\uline{阿黎斯塔苛}、\uline{德玛斯}、\uline{路加}都问候你。
\textsuperscript{25}
愿主\uline{耶稣}\uline{基督}的恩宠,与你们的心灵同在!阿门。

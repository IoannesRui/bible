% 标题
\chapter*{弟铎书}

% 右页眉
\rhead{弟铎书(铎)}

% 正文
\textbf{第一章\quad致候辞\quad}
\textsuperscript{1}
天主的仆人、作\uline{耶稣}\uline{基督}宗徒的\uline{保禄}——为引天主所选的人,去信从并认识合乎虔敬的真理,
\textsuperscript{2}
这虔敬是本于永生的希望,又是那不能说谎的天主,在久远的时代以前所预许的,
\textsuperscript{3}
他到了适当的时期,就籍着宣讲显示了他的圣道;我就是照我们救主天主的命令,受委托尽这宣讲的职务。——
\textsuperscript{4}
我\uline{保禄}致书给在共同信仰内作我真子的\uline{弟铎}:愿恩宠与平安,由天主父及我们的救主\uline{基督}\uline{耶稣}赐与你。
\renewcommand\thefootnote{\ding{\numexpr171+\value{footnote}}}
\footnote{宗徒的使命是引人信天主,并使人因而获得天主从古时就已许下的永生;为此宗徒们被派遣,就是为给人宣讲那“合乎虔敬的真理”,即“天主的圣道”,亦即福音。\uline{保禄}大概亲手给\uline{弟铎}付了洗,使他重生于\uline{基督},故称他是自己在信仰内的儿子,如\uline{弟茂德}一样(\uwave{弟}前1:2)。}

\textbf{长老应有的品格\quad}
\textsuperscript{5}
我留你在\uline{克里特},是要你整顿那些尚未完成的事,并照我所吩咐你的,在各城设立长老:
\textsuperscript{6}
长老应是无可指摘的,只做过一个妻子的丈夫,所有的子女都是信徒,又没有被控告为放荡不羁的,
\textsuperscript{7}
因为做监督的,既是天主的管家,就该是无可指摘的、不自负、不发怒、不嗜酒、不暴戾、不贪污;
\textsuperscript{8}
但该好客、乐善、慎重、公正、热心、有节,
\textsuperscript{9}
坚持那合乎教理的真道,好能以健全的道理劝戒并驳斥抗辩的人。
\footnote{\uline{弟铎}在\uline{克里特}岛应尽的任务,是继承宗徒的工作,使当地的教会发展,为此应在各城设立“长老”(参阅\uwave{宗}11注六)。身为长老的(参阅\uwave{弟}前3:2-4),不但不应有有辱于自己职位的毛病,而且还应修各样美德(\uwave{弟}前3:2-7),并“坚持那合乎教理的真道”,即全教会一脉相传的信仰,用这“健全的道理”(\uwave{弟}前1:10)驳斥那散布谬理的假学士(\uwave{弟}前4:1-3)。}

\textbf{应排斥异端\quad}
\textsuperscript{10}
实在有许多人尚不服从,好空谈,欺骗人,尤其是那些受过割损的人;
\textsuperscript{11}
应杜塞这些人的口,因为他们为了可耻的利润,竟教导那不应教导的事,破坏人的整个家庭。
\textsuperscript{12}
\uline{克里特}人中的一人,他们自己的一位先知曾这样说:“\uline{克里特}人常是些说谎者,是些可恶的野兽,贪口腹的懒汉。”
\textsuperscript{13}
这话说的很对。为此,你该严厉规劝他们,好叫他们在信德上健全无暇;
\textsuperscript{14}
不要听信\uline{犹太}人无稽的传说,和背弃真理之人的规定。
\footnote{\uline{克}岛曾有一位诗人(\uline{保禄}按\uline{希腊}人的习俗称他为“先知”),完全认出了自己同乡的缺点。14节“\uline{犹太}人无稽的传说”,见\uwave{弟}前1:3,4:7。}
\textsuperscript{15}
为洁净人一切都是洁净的,但为败坏的人和无信仰的人,没有一样是洁净的,就连他们的理性和良心都是污秽的。
\textsuperscript{16}
这样的人自称认识天主,但在行为上却否认天主;他们是可憎恶的,悖逆的,在一切善事上是无用的。
\footnote{“一切都是洁净的”(\uwave{罗}14:14、20),这话只是对心里洁净的人说的;至于那“败坏的人和无信仰的人”却不然,他们沉溺于罪恶,理智不辨别是非,良心不审断善恶,为他还有什么是洁净的呢(\uwave{玛}15:11;\uwave{弟}前4:3)。}

\textbf{第二章\quad信友应有的个别教训\quad}
\textsuperscript{1}
至于你,你所讲的,该合乎健全的道理;
\textsuperscript{2}
教训老人应节制、端庄、慎重,在信德、爱德和忍耐上,要正确健全。
\textsuperscript{3}
也要教训老妇在举止上要圣善,不毁谤人,不沉湎于酒,但教人行善,
\textsuperscript{4}
好能教导青年妇女爱丈夫,爱子女,
\textsuperscript{5}
慎重,贞节,勤理家务,良善,服从自己的丈夫,免得使人诋毁天主的圣道。
\footnote{参阅\uwave{弟}前5:1、2。此处应注意:\uline{保禄}将教导青年妇女的责任,托给年老的妇女。她们应树立良好的家风,好使年轻妇女有所遵循。因为良好家庭是广扬福音最有效的方法;但使福音真道遭受教外人士诟病的,也莫过于基督徒的不良家庭。}
\textsuperscript{6}
你也要教训青年人在一切事上要慎重。
\textsuperscript{7}
你该显示自己为行善的模范,在教导上应表示纯正庄重,
\textsuperscript{8}
要讲健全无可指摘的话,使反对的人感到羞愧,说不出我们什么不好来。
\footnote{参阅\uwave{弟}前4:12。}
\textsuperscript{9}
教训奴隶在一切事上要服从自己的主人,常叫他们喜悦,不要抗辩,
\textsuperscript{10}
不要窃取,惟要事事表示自己实在忠信,好使我们的救主天主的圣道,在一切事上获得光荣。
\footnote{关于奴隶制度,参见\uwave{格}前7注四,\uwave{弗}6:5-9;\uwave{弟}前6:1、2。奴隶的劳役与忠信,能发挥\uline{基督}的圣道,感化主人,这样使天主在人前获得光荣。}

\textbf{信友应如何在世上生活\quad}
\textsuperscript{11}
的确,天主救众人的恩宠已经出现,
\textsuperscript{12}
教导我们弃绝不虔敬的生活,和世俗的贪欲,有节地、公正地、虔敬地在今世生活,
\textsuperscript{13}
期待所希望的幸福,和我们伟大的天主及救主\uline{耶稣}\uline{基督}光荣的显现。
\textsuperscript{14}
他为我们舍弃了自己,是为救赎我们脱离一切罪恶,洗净我们,使我们能成为他的选民,叫我们热心行善。
\footnote{天主对众人的恩宠,在\uline{耶稣}降生和救赎的工程上,完全显示出来;这恩宠要求人脱离“世俗的贪欲”,而期待来世的真福和光荣。}
\textsuperscript{15}
你要宣讲这些事,以全权规劝和指摘,不要让任何人轻视你。

\textbf{第三章\quad信友应服从政权\quad}
\textsuperscript{1}
你要提醒人服从执政的官长,听从命令,准备行各种善事。
\footnote{\uline{保禄}叫\uline{弟铎}提醒信友服从政府,尽国民的责任,因为国家政权是来自天主(参阅\uwave{罗}13:1-7;\uwave{弟}前2:1-5;\uwave{伯}前2:13-17)。}
\textsuperscript{2}
不要辱骂,不要争吵,但要谦让,对众人表示极其温和,
\textsuperscript{3}
因为我们从前也是昏愚的,悖逆的,迷途的,受各种贪欲和逸乐所奴役,在邪恶和嫉妒中度日,自己是可憎恶的,又彼此仇恨。
\footnote{人类最大的昏愚是不认识天主。由于不认识天主才“彼此仇恨”,缺乏爱德。爱德原是信友生活的原动力,因为天主自己就是爱(\uline{若}一4:16)。}
\textsuperscript{4}
但当我们的救主天主的良善,和他对人的慈爱出现时,
\textsuperscript{5}
他救了我们,并不是由于我们本着义德所立的功劳,而是出于他的怜悯,籍着圣神所施行的重生和更新的洗礼,救了我们。
\textsuperscript{6}
这圣神是天主籍我们的救主\uline{耶稣}\uline{基督},丰富地倾注在我们身上的,
\textsuperscript{7}
好使我们因他的恩宠成义,本着希望成为永生的继承人。
\footnote{信友得以领洗重生,应归功于天主圣三:就是归于天父的慈爱怜悯;归于\uline{耶稣}\uline{基督},因他是我们的救主和中保;归于圣神,因他赐人超性的生命(\uwave{若}6:63,14:16)。信友既然成了天主的义子,也就成了天主产业的承继人(\uwave{迦}4:7)。要承继的产业就是永生。}

\textbf{责斥不务正道的人\quad}
\textsuperscript{8}
这话是确实的,我愿意你坚持这些事,好使那些已信奉天主的人,热心专务行善:这些都是美好而为人有益的事;
\textsuperscript{9}
至于那些愚昧的辩论、祖谱、争执和关于法律的争论,你务要躲避,因为这些都是无益的空谈。
\footnote{参阅\uwave{弟}前1:4,6。}
\textsuperscript{10}
对异端人,在谴责过一次两次以后,就该远离他。
\textsuperscript{11}
该知道:这样的人已背弃正道,犯罪作恶,自己给自己定了罪案。
\footnote{参阅\uwave{玛}18:17;\uwave{若}二10。}

\textbf{吩咐与嘱托\quad}
\textsuperscript{12}
当我打发\uline{阿尔特玛}或\uline{提希苛}到你那里以后,你赶快到\uline{尼苛颇里}来见我,因为我已决定在那里过冬。
\textsuperscript{13}
你打发法学士\uline{则纳}和\uline{阿颇罗}上路,要照顾周到,使他们什么也不缺少。
\textsuperscript{14}
我们的人也应当学着行善,为应付一切急需,免得成为不结果实的人。
\footnote{“行善”(14节),此处是指为帮助传教,不应吝惜财物而言。}

\textbf{问候与祝福\quad}
\textsuperscript{15}
同我在一起的弟兄都问候你;请问候那些在信德内爱我们的弟兄。愿恩宠与你们众人同在!

\chapter{出谷纪}

\begin{center}
	\textbf{上编 }
	\textbf{以色列人出离埃及(1-18)}
\end{center}

\textbf{第一章 }
\textbf{以色列人在埃及兴旺 }
\textsuperscript{1}
\uline{以色列}的儿子们,各带了家眷,同\uline{雅各伯}来到\uline{埃及};他们的名字记载如下:
\textsuperscript{2}
\uline{勒乌本}、\uline{西默盎}、\uline{肋未}和\uline{犹大},
\textsuperscript{3}
\uline{依撒}\uline{加尔},\uline{则步隆}和\uline{本雅明},
\textsuperscript{4}
\uline{丹}和\uline{纳斐塔里},\uline{加得}和\uline{阿协尔}。
\textsuperscript{5}
他们全是\uline{雅各伯}所生的,一共七十人;\uline{若瑟}那时已经在\uline{埃及}。
\textsuperscript{6}
\uline{若瑟}和他的众兄弟,以及这一代人死了以后,
\textsuperscript{7}
\uline{以色列}的子孙生育繁殖,数目增多,极其强盛,布满了那地。
\renewcommand\thefootnote{\ding{\numexpr171+\value{footnote}}}
\footnote{\uline{雅各伯}的儿子们(\uwave{创}46:8-27)在此没有按照长幼,而是按照生母编列,先记主母,后记婢女的儿子。}

\textbf{受压迫 }
\textsuperscript{8}
有位不认识\uline{若瑟}的新王兴起,统治了\uline{埃及}。
\textsuperscript{9}
他对自己的人民说:“看,\uline{以色列}子民,比我们又多又强。
\textsuperscript{10}
来,我们要用智谋对付他们,免得他们繁盛起来,一遇战争,就去与我们的敌人联合,攻击我们,然后离开此地。”
\textsuperscript{11}
于是派定督工管制他们,以苦役压迫他们,叫他们给法郎建筑\uline{丕通}和\uline{辣默}\uline{色斯}两座贮货城。
\textsuperscript{12}
但是越压迫他们,他们越增多,也越繁殖,以至\uline{埃及}人都怕\uline{以色列}子民。
\textsuperscript{13}
于是\uline{埃及}人更严厉地强迫\uline{以色列}子民做苦工,
\textsuperscript{14}
强迫他们作和泥做砖的苦工,田间的一切劳工,以及种种苦工,使他们的生活十分痛苦。
\footnote{“新王”(8节)大概是\uline{辣默色斯}二世(公元前1292-1225)。\uline{丕通}即今之\uline{特耳玛斯}\uline{顾大};以王名得名的\uline{辣默色斯}城,可能即今之\uline{塔尼斯}地方。由14节可以推知法郎对待\uline{以色列}人,有如对待战俘。}

\textbf{残害男婴 }
\textsuperscript{15}
\uline{埃及}王又吩咐为\uline{希伯来}女人接生的收生婆,一个名叫\uline{史斐辣},一个名叫\uline{普亚}的说“
\textsuperscript{16}
你们为\uline{希伯来}女人接生时,要看着她们的临盆!若是男孩,就杀死;若是女孩,就让她活着。”
\textsuperscript{17}
但是收生婆敬畏天主,没有照\uline{埃及}王的吩咐去作,保留了男孩的性命。
\textsuperscript{18}
\uline{埃及}王将收生婆召来,问她们说:“你们为什么这样做,竟叫男孩子活着呢?”
\textsuperscript{19}
收生婆回答法郎说:“\uline{希伯来}女人与\uline{埃及}女人不同,她们富有生机,收生婆还没有来,她们已生产了。”
\textsuperscript{20}
天主遂恩待了收生婆。\uline{以色列}子民更加增多起来,更加强盛。
\textsuperscript{21}
因为收生婆敬畏天主,天主就使她们家门兴旺。
\textsuperscript{22}
法郎于是训令他的全体人民说:“凡\uline{希伯来}人所生的男孩,你们应把他丢在\uline{尼罗}河里;凡是女孩,留她活着!”
\footnote{为\uline{以色列}妇女接生的,大概是\uline{埃及}女人。由于她们也敬畏上主,不但未下毒手,反而保护了\uline{希伯来}人初生的男婴,所以天主也祝福了她们。参阅\uwave{申}25:9;\uwave{撒}下7:11;\uwave{列}上2:24。}
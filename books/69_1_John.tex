% 分栏
\setlength\columnsep{0.6cm}
\begin{multicols}{2}

% 标题
\chapter*{若望一书}

% 右页眉
\rhead{若望一书(若一)}

\textbf{第一章\quad序言\quad}
\textsuperscript{1}
论到那从起初就有的生命的圣言,就是我们所见过,我们亲眼看见过,瞻仰过,以及我们亲手摸过的生命的圣言——
\textsuperscript{2}
这生命已显示出来,我们看见了,也为他作证,且把这原与父同在,且已显示给我们的永远的生命,传报给你们——
\textsuperscript{3}
我们将所见所闻的传报给你们,为使你们也同我们相通;原来我们是同父和他的子\uline{耶稣}\uline{基督}相通的。
\textsuperscript{4}
我们给你们写这些事是为叫我们的喜乐得以圆满。
\renewcommand\thefootnote{\ding{\numexpr171+\value{footnote}}}
\footnote{作者不依书信的普通格式,而写了这篇简短的序言,以说明本书的要旨大意。作者给降世的“生命的圣言”,即天主圣子\uline{耶稣}作证,为使读者信仰,好分享这永远的生命,而与天主并与基督徒彼此相通,此即圣\uline{保禄}所论\uline{基督}妙身的道理。作者还用了许多不同的语词(2:5、6、24-29,3:6、9、24,4:7、12、13,5:1、12、20),阐明这端道理。关于“生命的圣言”,见\uwave{若}1注1;11:25,14:6。}

\begin{center}
	\textbf{\large{\songti 天主是光}}
\end{center}

\textbf{人应在光中往来\quad}
\textsuperscript{5}
我们由他所听见,而传报给你们的,就是这个信息:天主是光,在他内没有一点黑暗。
\textsuperscript{6}
如果我们说我们与他相通,但仍在黑暗中行走,我们就是说谎,不履行真理。
\textsuperscript{7}
但如果我们在光中行走,如同他在光中一样,我们就彼此相通,他圣子\uline{耶稣}的血就会洗净我们的各种罪过。

\textsuperscript{8}
如果我们说我们没有罪过,就是欺骗自己,真理也不在我们内。
\textsuperscript{9}
但若我们明认我们的罪过,天主既是忠信正义的,必赦免我们的罪过,并洗净我们的各种不义。
\textsuperscript{10}
如果我们说我们没有犯过罪,我们就是拿他当说谎者,他的话就不在我们内。
\footnote{“光”象征圣善纯洁;“黑暗”象征邪恶罪过(\uwave{若}3:19-21,12:35、36)。人若与天主结合相通,应勉力度圣善纯洁的生活。虽然如此,但人性懦弱,仍难免不犯罪,因此人不但应自知有罪,而且应谦逊明认(\uwave{雅}5:16),如此赖\uline{耶稣}救赎的功劳,罪过必获得赦免;反之,若不承认自己有罪,则是罪上加罪,并且也以天主为“说谎者”,因为天主说明了,人都是罪人(\uwave{箴}20:9;\uwave{德}19:16;\uwave{玛}6:16等处)。}

\textbf{第二章\quad耶稣为人做了赎罪祭\quad}
\textsuperscript{1}
我的孩子们,我给你们写这些事,是为了叫你们不犯罪;但是,谁若犯了罪,我们在父那里有正义的\uline{耶稣}\uline{基督}作护慰者。
\textsuperscript{2}
他自己就是赎罪祭,赎我们的罪过,不但赎我们的,而且也赎全世界的罪过。
\footnote{本章的大意是接上章论人在光中生活的凭据:遵守命令,远离世俗,谨防异端。\uline{耶稣}为赎全世界的罪成了“赎罪祭”,见\uwave{罗}3:25;\uwave{若}11:52;\uwave{希}9:28。}

\textbf{遵守爱德的命令\quad}
\textsuperscript{3}
如果我们遵守他的命令,由此便知道我们认识他。
\textsuperscript{4}
那说“我认识他”,而不遵守他命令的,是撒谎的人,在他内没有真理。
\textsuperscript{5}
但是,谁若遵守他的话,天主的爱在他内才得以圆满;由此我们也知道,我们是在他内。
\textsuperscript{6}
那说自己住在他内的,就应当照那一位所行的去行。
\textsuperscript{7}
可爱的诸位,我给你们写的,不是一条新命令,而是你们从起初领受的旧命令:这旧命令就是你们所听的道理。
\textsuperscript{8}
另一方面说,我给你们写的也是一条新命令——就是在他和你们身上成为事实的——因为黑暗正在消逝,真光已在照耀。
\textsuperscript{9}
谁说自己在光中,而恼恨自己的弟兄,他至今仍是在黑暗中。
\textsuperscript{10}
凡爱自己弟兄的,就是存留在光中,对于他就没有任何绊脚石;
\textsuperscript{11}
但是恼恨自己弟兄的,就是在黑暗中,且在黑暗中行走,不知道自己往哪里去,因为黑暗弄瞎了他的眼睛。
\footnote{6节}

\textbf{远离世俗\quad}
\textsuperscript{12}
孩子们,我给你们写说:因他的名字,你们的罪已获得赦免。
\textsuperscript{13}
父老们,我给你们写说:你们已认识了从起初就有的那一位。青年们,我给你们写说:你们已得胜了那恶者。
\textsuperscript{14}
小孩子们,我给你们写过:你们已认识了父。父老们,我给你们写过:你们已认识了从起初就有的那一位。青年们,我给你们写过:你们是强壮的,天主的话存留在你们内,你们也得胜了那恶者。

\textsuperscript{15}
你们不要爱世界,也不要爱世界上的事;谁若爱世界,天父的爱就不在他内。
\textsuperscript{16}
原来世界上的一切:肉身的贪欲,眼目的贪欲,以及人生的骄奢,都不是出于父,而是出于世界。
\footnote{}
\textsuperscript{17}
这世界和它的贪欲都要过去;但那履行天主旨意的,却永远存在。

\textbf{提防假基督\quad}
\textsuperscript{18}
小孩子们,现在是最末的时期了!就如你们听说过假\uline{基督}要来,如今已经出了许多假\uline{基督},由此我们就知道现在是最末的时期了。
\textsuperscript{19}
他们是出于我们中的,但不是属于我们的,因为,如果是属于我们的,必存留在我们中;但这是为显示他们都不是属于我们。
\footnote{}

\textsuperscript{20}
至于你们,你们由圣者领受了傅油,并且你们都晓得。
\footnote{}
\textsuperscript{21}
我给你们写信,不是你们不明白真理,而是因为你们明白真理,并明白各种谎言不是出于真理。

\textsuperscript{22}
谁是撒谎的呢?岂不是那否认\uline{耶稣}为默西亚的吗?那否认父和子的,这人便是假\uline{基督}。
\textsuperscript{23}
凡否认子的,也否认父;那明认子的,也有父。
\footnote{}
\textsuperscript{24}
至于你们,应把从起初所听见的,存留在你们内;如果你们从起初所听见的,存留在你们内,你们必存留在子和父内。
\textsuperscript{25}
这就是他给我们所预许的恩惠:既永远的生命。

\textsuperscript{26}
这些就是我关于迷惑你们的人,给你们所写的。
\textsuperscript{27}
至于你们,你们由他所领受的傅油,常存在你们内,你们就不需要谁教训你们,而是有他的傅油教训你们一切。这傅油是真实的,决不虚假,所以这傅油怎样教训你们,你们就怎样存留在他内。
\footnote{}
\textsuperscript{28}
现在,孩子们,你们常存在他内罢!为的是当他显现时,我们可以放心大胆,在他来临时,不至于在他面前蒙羞。
\textsuperscript{29}
你们既然知道他是正义的,就该知道凡履行正义的,都是由他而生的。
\footnote{}

\begin{center}
	\textbf{\large{\songti 天主是父}}
\end{center}

\textbf{第三章\quad天父的子女应相似天父\quad}
\textsuperscript{1}
请看父赐给我们何等的爱情,使我们得称为天主的子女,而且我们也真是如此。世界所以不认识我们,是因为不认识父。
\textsuperscript{2}
可爱的诸位,现在我们是天主的子女,但我们将来如何,还没有显明;可是我们知道:一显明了,我们必要相似他,因为我们要看见他实在怎样。
\footnote{}
\textsuperscript{3}
所以,凡对他怀着这希望的,必圣洁自己,就如那一位也是圣洁的一样。
\footnote{}
\textsuperscript{4}
凡是犯罪的,也就是作违法的事,因为罪过就是违法。
\textsuperscript{5}
你们也知道,那一位曾显示出来,是为除免罪过,在他身上并没有罪过。
\textsuperscript{6}
凡存在他内的,就不犯罪过;凡犯罪过的,是没有看见过他,也没有认识过他。

\textsuperscript{7}
孩子们,万不要让人迷惑你们!那行正义的,就是正义的人,正如那一位是正义的一样
\textsuperscript{8}
那犯罪的,是属于魔鬼,因为魔鬼从起初就犯罪:天主子所以显现出来,是为消灭魔鬼的作为。
\textsuperscript{9}
凡由天主生的,就不犯罪过,因为天主的种子存留在他内,他不能犯罪,因为他是由天主生的。
\textsuperscript{10}
天主的子女和魔鬼的子女在这事上可以认出:就是凡不行正义的和不爱自己弟兄的,就不是出于天主。
\footnote{}

\textbf{爱人的命令\quad}
\textsuperscript{11}
原来你们从起初所听的训令就是:我们应彼此相爱;
\textsuperscript{12}
不可像那属于恶者和杀害自己兄弟的\uline{加音}。\uline{加音}究竟为什么杀了他?因为他自己的行为是邪恶的,而他兄弟的行为是正义的。

\textsuperscript{13}
弟兄们,如果世界恼恨你们,不必惊奇。
\textsuperscript{14}
我们知道,我们已出死入生了,因为我们爱弟兄们;那不爱的,就存在死亡内。
\textsuperscript{15}
凡恼恨自己弟兄的,便是杀人的;你们也知道:凡杀人的,便没有永远的生命存在他内。
\textsuperscript{16}
我们所以认识了爱,因为那一位为我们舍弃了自己的生命,我们也应当为弟兄们舍弃生命。
\textsuperscript{17}
谁若有今世的财物,看见自己的弟兄有急难,却对他关闭自己怜悯的心肠,天主的爱怎能存在他内?
\footnote{}

\textsuperscript{18}
孩子们,我们爱,不可只用言语,也不可只用口舌,而要用行动和事实。
\textsuperscript{19}
在这一点上我们可以认出,我们是出于真理的,并且在他面前可以安心;
\textsuperscript{20}
纵然我们的心责备我们,我们还可以安心,因为天主比我们的心大,他原知道一切。
\footnote{}
\textsuperscript{21}
可爱的诸位,假使我们的心不责备我们,在天主前便可放心大胆;
\textsuperscript{22}
那么我们无论求什么,必由他获得,因为我们遵守了他的命令,行了他所喜悦的事。
\textsuperscript{23}
他的命令就是叫我们信他的子\uline{耶稣}\uline{基督}的名字,并按照他给我们所出的命令,彼此相爱。
\textsuperscript{24}
那遵守他命令的,就往在他内,天主也住在这人内。我们所以知道他住在我们内,是籍他赐给我们的圣神。
\footnote{}

\textbf{第四章\quad真理的神和欺诈的神\quad}
\textsuperscript{1}
可爱的诸位,不要凡神就信,但要考验那些神是否出于天主,因为有许多假先知来到了世界上。
\textsuperscript{2}
你们凭此可认出天主的神:凡明认\uline{耶稣}为\uline{默西亚},且在肉身内降世的神,便是出于天主;
\textsuperscript{3}
凡否认\uline{耶稣}的神,就不是出于天主,而是属于假\uline{基督}的;你们已听说过他要来,现今他已在世界上了。
\textsuperscript{4}
孩子们,你们出于天主,且已得胜了他们,因为那在你们内的,比那在世界上的更大。
\textsuperscript{5}
他们属于世界,因此讲论属于世界的事,而世界就听从他们。
\textsuperscript{6}
我们却是出于天主的;那认识天主的,必听从我们;那不出于天主的,便不听从我们:由此我们可以认出真理的神和欺诈的神来。
\footnote{}

\begin{center}
	\textbf{\large{\songti 天主是爱}}
\end{center}

\textbf{以爱还爱\quad}
\textsuperscript{7}
可爱的诸位,我们应该彼此相爱,因为爱是出于天主;凡有爱的,都是生于天主,也认识天主;
\textsuperscript{8}
那不爱的,也不认识天主,因为天主是爱。
\textsuperscript{9}
天主对我们的爱在这事上已显出来:就是天主把自己的独生子,打发到世界上来,好使我们籍着他得到生命。
\textsuperscript{10}
爱就在于此:不是我们爱了天主,而是他爱了我们,且打发自己的儿子,为我们做赎罪祭。
\footnote{}

\textsuperscript{11}
可爱的诸位,既然天主这样爱了我们,我们也应该彼此相爱。
\textsuperscript{12}
从来没有人瞻仰过天主;如果我们彼此相爱,天主就存留在我们内,他的爱在我们内才是圆满的。
\textsuperscript{13}
我们所以知道我们存留在他内,他存留在我们内,就是由于他赐给了我们的圣神。
\footnote{}
\textsuperscript{14}
至于我们,我们却曾瞻仰过,并且作证:父打发了子来作世界的救主。
\textsuperscript{15}
谁若明认\uline{耶稣}是天主子,天主就存在他内,他也存在天主内。
\textsuperscript{16}
我们认识了,且相信了天主对我们所怀的爱。

天主是爱,那存留在爱内的,就存留在天主内,天主也存留在他内。
\textsuperscript{17}
我们内的爱得以圆满,即在于此:就是我们可在审判的日子放心大胆,因为那一位怎样,我们在这世界上也怎样。
\textsuperscript{18}
在爱内没有恐惧,反之,圆满的爱把恐惧驱逐于外,因为恐惧内含着惩罚;那惩罚的,在爱内还没有圆满。
\footnote{}

\textsuperscript{19}
我们应该爱,因为天主先爱了我们。
\textsuperscript{20}
假使有人说:我爱天主,但他却恼恨自己的弟兄,便是撒谎的;因为那不爱自己所看见的弟兄的,就不能爱自己所看不见的天主。
\textsuperscript{21}
我们从他蒙受了这命令:那爱天主的,也该爱自己的弟兄。
\footnote{}

\textbf{第五章\quad信德是得胜世界的力量和得永生的根基\quad}
\textsuperscript{1}
凡信\uline{耶稣}为\uline{默西亚}的,是由天主所生的;凡爱生他之父的,也必爱那由他所生。
\textsuperscript{2}
几时我们爱天主,又遵行他的诫命,由此知道我们真爱天主的子女。
\textsuperscript{3}
原来爱天主,就是遵行他的诫命,而他的诫命并不沉重,
\textsuperscript{4}
因为凡是由天主所生的,比得胜世界;得胜世界的胜利武器,就是我们的信德。
\textsuperscript{5}
谁是得胜世界的呢?不是那信\uline{耶稣}为天主子的人吗?
\footnote{}

\textsuperscript{6}
这位就是经过水及血而来的\uline{耶稣}\uline{基督},他不但以水,而且也是水及血而来的;并且有圣神作证,因为圣神是真理。
\textsuperscript{7}
原来作证的有三个:
\textsuperscript{8}
就是圣神,水及血,而这三个是一致的。
\textsuperscript{9}
人的证据,我们既然接受,但天主的证据更大,因为天主的证据就是他为自己的子作证。
\textsuperscript{10}
那信天主子的,在自己内就怀有这证据;那不信天主的,就是以天主为撒谎者,因为他不信天主为自己的子所作的证。
\textsuperscript{11}
这证据就是天主将永远的生命赐给了我们,而这生命是在自己的子内。
\textsuperscript{12}
那有子的,就有生命;那没有天主子的,就没有生命。
\footnote{}

\textbf{结论\quad信友的崇高地位\quad}
\textsuperscript{13}
我给你们这些信天主子名字的人,写了这些事,是为叫你们知道:你们已获有永远的生命。
\textsuperscript{14}
我们对天主所怀的依恃之心就是:如果我们按他的旨意求什么,他必俯听我们。
\footnote{}
\textsuperscript{15}
我们既然知道:我们不拘向他祈求什么,他会俯听我们,我们也知道向他所祈求的,必要得到。
\textsuperscript{16}
谁若看见自己的弟兄犯了不至于死的罪,就应当祈求,天主必赏赐他生命:这是为那些犯不至于死的罪人而说的;然而有的罪却是至于死的罪,为这样的罪,我不说要人祈求。
\footnote{}
\textsuperscript{17}
任何的不义都是罪过,但也有不至于死的罪过。

\textsuperscript{18}
我们知道:凡由天主生的,就不犯罪过;而且由天主生的那一位必保全他,那恶者不能侵犯他。
\footnote{}
\textsuperscript{19}
我们知道我们属于天主,而全世界却屈服于恶者。
\textsuperscript{20}
我们也知道天主子来了,赐给了我们理智,叫我们认识那真实者;我们确实是在那真实者内,他即是真实的天主和永远的生命。
\footnote{}
\textsuperscript{21}
孩子们,你们要谨慎,远避偶像!

\end{multicols}
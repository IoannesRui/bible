% 分栏
\setlength\columnsep{0.8cm}
\begin{multicols}{2}

% 标题
\chapter*{犹达书}

% 右页眉
\rhead{犹达书(犹)}

\textbf{致候辞 }
\textsuperscript{1}
\uline{耶稣}\uline{基督}的仆人,\uline{雅各伯}的兄弟\uline{犹达},致书给在天主父内蒙爱,为\uline{耶稣}\uline{基督}而保存的蒙召者。
\textsuperscript{2}
愿仁慈、平安、爱情丰富地赐予你们。
\footnote{在《伯后引言》中已提及\uwave{伯}后引用本书的地方很多,因为二位宗徒所驳斥的是同样的异端,因为以下的注释只注意本书特有者。}

\textbf{写信的动机 }
\textsuperscript{3}
亲爱的,我早已切望给你们写信,讨论我们共享的救恩;但现在不得不给你们写信,劝勉你们应奋斗,维护从前一次而永远传与圣徒的信仰。
\footnote{“一次而永远传……的信仰”说明教会信德的道理是由宗徒传下来而永不变更的(\uwave{格}前11:2;\uwave{弟}前6:20)。}
\textsuperscript{4}
因为有些早已被注定要受这审判的人,潜入你们中间;他们是邪恶的人,竟把我们天主的恩宠,变为放纵情欲的机会,并否认我们独一的主宰和主\uline{耶稣}\uline{基督}。

\textbf{引史事为戒 }
\textsuperscript{5}
虽然你们一次而永远知道这一切,但我仍愿提醒你们:主固然由\uline{埃及}地救出了百姓,随后却把那些不信的人消灭了;
\textsuperscript{6}
至于那些没有保持自己尊位,而离弃自己居所的天使,主也用永远的锁链,把他们拘留在幽暗中,以等候那伟大日子的审判;
\textsuperscript{7}
同样,\uline{索多玛}和\uline{哈摩辣}及其附近的城市,因为也和他们一样恣意行淫,随从逆性的肉欲,至今受着永火的刑罚,作为鉴戒。

\textbf{异端人的素描 }
\textsuperscript{8}
可是这些作梦的人照样玷污肉身,拒绝主权者,亵渎众尊荣者。
\textsuperscript{9}
当总领天使\uline{弥额尔},为了\uline{梅瑟}的尸体和魔鬼激烈争辩时,尚且不敢以侮辱的言词下判决,而只说:“愿主叱责你!”
\footnote{5节见\uwave{户}14:1-35。6-8节见\uwave{伯}后2:4-6、10。9节大概是引用\uline{犹太}人的口传,或者是一本名为《梅瑟升天录》的伪经。}
\textsuperscript{10}
这些人却不然,凡他们所不明白的事就亵渎,而他们按本性所体验的事,却像无理性的畜牲一样,就在这些事上败坏自己。
\textsuperscript{11}
这些人是有祸的!因为他们走了\uline{加音}的路,为图利而自陷于\uline{巴郎}的错误,并因\uline{科辣黑}一样的叛逆,而自取灭亡。
\textsuperscript{12}
这些人是你们爱宴上的污点,他们同人宴乐,毫无廉耻,只顾自肥;他们像无水的浮云,随风飘荡;又像晚秋不结果实,死了又死,该连根拔出来的树木;
\textsuperscript{13}
像海里的怒涛,四下飞溅他们无耻的白沫;又像出轨的行星;为他们所存留的,乃是直到永远的黑暗的幽冥。
\footnote{加音的罪是未听从天主的劝告(\uwave{创}4:7)。“\uline{巴郎}的错误”,见\uwave{伯}后2注三。\uline{科辣黑}的叛逆,见\uwave{户}16:1-35。12节“爱宴”,见\uwave{格}前11:17-23。}

\textsuperscript{14}
针对这些人,\uline{亚当}后第七代圣组\uline{哈诺客}也曾预言说:“看,主带着他千万的圣者降来,
\textsuperscript{15}
要审判众人,指证一切恶人所行的一切恶事,和邪僻的人所说的一切亵渎他的言语。”
\textsuperscript{16}
这些人好出怨言,不满命运;按照自己的私欲行事,他们的口好说大话,为了利益而奉承他人。
\footnote{关于\uline{哈诺客}见\uwave{创}5:21-24。\uline{哈诺客}的“预言”,见于当时\uline{犹太}人熟悉的一本名叫《哈诺客书》的伪经。}

\textbf{劝信友提防放荡的人 }
\textsuperscript{17}
但是你们,亲爱的,你们要记得我们的主\uline{耶稣}\uline{基督}的宗徒所预言过的话,
\textsuperscript{18}
他们曾向你们说过:“到末期,必有一些好嘲弄人的人,按照他们个人邪恶的私欲行事。”
\textsuperscript{19}
这就是那些好分党分派,属于血肉,没有圣神的人。

\textbf{信友应如何持身待人 }
\textsuperscript{20}
可是你们,亲爱的,你们要把自己建筑在你们至圣的信德上,在圣神内祈祷;
\textsuperscript{21}
这样保存你们自己常在天主的爱内,期望赖我们的主\uline{耶稣}\uline{基督}的仁慈,入于永生。
\textsuperscript{22}
对那些怀疑不信的人,你们要说服;
\textsuperscript{23}
对另一些人,你们要拯救,把他们从火里拉出来;但对另一些人,你们固然要怜悯,可是应存戒惧的心,甚至连他们肉身所玷污了的内衣,也要憎恶。
\footnote{作者劝告信友对迷于异端的人应有爱德,设法拯救他们;但应留神,不要被他们所诱惑,而与他们同流合污。23节末所说,是借用《旧约》对癞病人的法律(\uwave{肋}13:47),说明淫乱的罪就如人灵魂所患的癞病:人如不提防,易受传染。}

\textbf{结语 }
\textbf{致候辞 }
\textsuperscript{24}
愿那能保护你们不失足,并能叫你们无暇地,在欢跃中立在他光荣面前的,
\textsuperscript{25}
惟一的天主,我们的救主,籍我们的主\uline{耶稣}\uline{基督},获享光荣、尊威、主权和能力,于万世之前,现在,至于无穷之世。阿们。

\end{multicols}
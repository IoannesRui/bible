\chapter{创世纪(创)}

\begin{center}
	\textbf{前编 太古史(1-11)}
\end{center}

\textbf{第一章 }
\textbf{天地万物的创造 }
\textsuperscript{1}
在起初天主创造了天地。
\textsuperscript{2}  
大地还是混沌空虚,深渊上还是一团黑暗,天主的神在水面上运行。
\renewcommand\thefootnote{\ding{\numexpr171+\value{footnote}}}
\footnote{"在起初……"一语,暗示创造万物之时,除天主外,一无所有。“天地”二字此处有宇宙万物之意。作者用诗人的想象力描写天主好似一个工程师,在六天以内创造了万物,到第七天休息。首先所创造的是混沌的无生之物,后将这混沌之物分成天、地、海三大部分,然后以日月、星辰、草木、飞禽、走兽等来点缀天地海洋。最后天主照自己的肖像造了人。作者从创造混沌之物说起,到创造人,表示人是万物之灵,应效法造物主工作和守安息日。此开宗明义第一章是远古时代的文学杰作,是一篇宗教的重要文告,并不是自然科学的论著。按古代各民族对开天辟地,人类诞生的传说,没有可与《创世纪》第一章相比拟的。“天主的神”指施生命之神力,但若通观新旧二约的全部启示,此处也指赐生命的“天主圣神”。}
\textsuperscript{3}
天主说:“有光!”就有了光。
\textsuperscript{4}
天主见光好,就将光与黑暗分开。
\textsuperscript{5}
天主称光为“昼”,称黑暗为“夜”。过了晚上,过了早晨,这是第一天。

\textsuperscript{6}
天主说:“在水与水之间要有穹苍,将水分开!”事就这样成了。
\textsuperscript{7}
天主造了穹苍,分开了穹苍以下的水和穹苍以上的水。
\textsuperscript{8}
天主称穹苍为“天”,天主看了认为好。过了晚上,过了早晨,这是第二天。

\textsuperscript{9}
天主说:“天下的水应聚在一处,使旱地出现!”事就这样成了。
\textsuperscript{10}
天主称旱地为“陆地”,称水汇合处为“海洋”。天主看了认为好。
\textsuperscript{11}
天主说:“在陆地上,土地要生出青草、结种子的蔬菜和结果子的果树,各按照在它内的种子的种类!”事就这样成了。
\textsuperscript{12}
土地就生出了青草、结种子的蔬菜,各按其类,和结果子的树木,各按照在它内的种子的种类。天主看了认为好。
\textsuperscript{13}
过了晚上,过了早晨,这是第三天。

\textsuperscript{14}
天主说:“在天空中要有光体,以分别昼夜,作为规定时节和年月日的记号。
\textsuperscript{15}
要在天空中放光,照耀大地!”事就这样成了。
\textsuperscript{16}
天主于是造了两个大光体:较大的控制白天,较小的控制黑夜,并造了星宿。
\textsuperscript{17}
天主将星宿摆列在天空,照耀大地,
\textsuperscript{18}
控制昼夜,分别明与暗。天主看了认为好。
\textsuperscript{19}
过了晚上,过了早晨,这是第四天。

\textsuperscript{20}
天主说:“水中要繁生蠕动的生物,地面上、天空中要有鸟飞翔!”事就这样成了。
\textsuperscript{21}
天主于是造了大鱼和所有在水中孳生的蠕动生物,各按其类,以及各种飞鸟,各按其类。天主看了认为好。
\textsuperscript{22}
遂祝福它们说:“你们要孳生繁殖,充满海洋;飞鸟也要在地上繁殖!”
\textsuperscript{23}
过了晚上,过了早晨,这是第五天。

\textsuperscript{24}
天主说:“地上要生出生物,各按其类;走兽、爬虫和地上的各种生物,各按其类!”事就这样成了。
\textsuperscript{25}
天主于是造了地上的生物,各按其类;各种走兽,各按其类;以及地上所有的爬虫,各按其类。天主看了认为好。
\textsuperscript{26}
天主说:“让我们照我们的肖像,按我们的模样造人,叫他管理海中的鱼、天空的飞鸟、牲畜、各种野兽、在地上爬行的各种爬虫。”
\textsuperscript{27}
天主于是照自己的肖像造了人,就是照天主的肖像造了人:造了一男一女。
\textsuperscript{28}
天主祝福他们说:“你们要生育繁殖,充满大地,治理大地,管理海中的鱼、天空的飞鸟、各种在地上爬行的生物!”
\footnote{“人”按原文有红土或黄土的意思,是说人是属于土的造物。“我们”(26节)按古\uline{犹太}经师的解释,是指天主和天使,好似天主同天使商量;但有些学者主张为“威严复数”或“议决复数”。教父和神学家多以为此复数暗示天主圣三的奥理。此说若照启示的演进说是对的。人相似天主是按灵魂说的,相似天主有理智、意志和记忆。论人的肉身,当天主造\uline{亚当}时,已预见作\uline{亚当}第二的\uline{基督}(\uwave{罗}5:14)。“造了一男一女”,指婚姻一夫一妻制和不可分离性(\uwave{玛}19:1-6;\uwave{拉}2:15、16)。天主祝福原祖生育繁殖的话,说明婚姻的首要目的是生养教育子女(8:17;\uwave{咏}127:3、4)。}
\textsuperscript{29}
天主又说:“看,全地面上结种子的各种蔬菜,在果内含有种子的各种果树,我都给你们作食物;
\textsuperscript{30}
至于地上的各种野兽,天空中的各种飞鸟,在地上爬行有生魂的各种动物,我把一切青草给它们作食物。”事就这样成了。
\textsuperscript{31}
天主看了他造的一切,认为样样都很好。过了晚上,过了早晨,这是第六天。
\footnote{天主造了原祖,也赐给了他们和他们传生的人类的食物,并将普世交给他们统治。所造的万物样样都好,是说万物都合天主的旨意,都为他所喜爱。参阅\uwave{咏}19:1-6,104,145,148,150。}

\textbf{第二章 }
\textbf{安息日 }
\textsuperscript{1}
这样,天地和天地间的一切点缀都完成了。
\textsuperscript{2}
到第七天天主造物的工程已完成,就在第七天休息,停止了所作的一切工程。
\textsuperscript{3}
天主祝福了第七天,定为圣日,因为这一天,天主停止了他所行的一切创造工作。
\footnote{1-3节属前章,劝人守安息日为圣日。守安息日的原因与目的,见\uwave{出}23:12;\uwave{申}5:12-15。}

\textbf{人与乐园 }
\textsuperscript{4}
这是创造天地的来历:在上主天主创造天地时,
\textsuperscript{5}
地上还没有灌木,田间也没有生出蔬菜,因为上主天主还没有使雨降在地上,也没有人耕种土地,
\textsuperscript{6}
有从地下涌出的水浸润所有地面。
\textsuperscript{7}
上主天主用地上的灰土形成了人,在他鼻孔内吹了一口生气,人就成了一个有灵的生物。
\textsuperscript{8}
上主天主在\uline{伊甸}东部种植了一个乐园,就将他形成的人安置在里面。
\textsuperscript{9}
上主天主使地面生出各种好看好吃的果树,生命树和知善恶树在乐园中央。
\footnote{2:4-3:24为创造天地万物的另一记载。原来在这记载中只用了天主(雅威)的名词,但将这个记载与上章的记载编在一起,补入了“天主”的名词。这记载的中心为人:天主对人,人对天主的态度。关于人的来历和本性,作者用简略的话,教训人一端论宗教和文化的最高深的道理:人肉身的形成,好像其他的动物,是由尘土造成的,但对于灵魂却有极大的区别,它是直接由天主所造。\uline{伊甸}乐园位于何处,人不得而知。乐园是天主考验人的地方。“生命树”所象征的是天主愿意赐给人的“不死”之恩。“知善恶的树”,是试探人的工具。“知善恶”的意思,大概是说:人一犯天主的禁令,就知道所失去的超性恩宠——真善,是多么美善,所犯的罪恶——真恶,是如何凶恶。}
\textsuperscript{10}
有一条河由\uline{伊甸}流出灌溉乐园,由那里分为四支:
\textsuperscript{11}
第一支名叫\uline{丕雄},环流产金的\uline{哈威拉}全境;
\textsuperscript{12}
那地方的金子很好,那里还产珍珠和玛瑙;
\textsuperscript{13}
第二支河名叫\uline{基红},环流\uline{雇士}全境;
\textsuperscript{14}
第三支河名叫\uline{底格里斯},流入\uline{亚述}东部;第四支河即\uline{幼发拉的}。
\textsuperscript{15}
上主天主将人安置在\uline{伊甸}的乐园内,叫他耕种,看守乐园。
\footnote{说明人犯罪以前,天主已叫人应该工作。}
\textsuperscript{16}
上主天主给人下令说:“乐园中各树上的果子,你都可吃,
\textsuperscript{17}
只有知善恶树上的果子你不可吃,因为那一天你吃了,必定要死。”

\textbf{造女人立婚姻 }
\textsuperscript{18}
上主天主说:“人单独不好,我要给他造个与他相称的助手。”
\textsuperscript{19}
上主天主用尘土造了各种野兽和天空中的各种飞鸟,都引到人面前,看他怎样起名;凡人给生物起的名字,就成了那生物的名字。
\textsuperscript{20}
人遂给各种畜牲、天空中的各种飞鸟和各种野兽起了名字;但他没有找着一个与自己相称的助手。
\textsuperscript{21}
上主天主遂使人熟睡,当他睡着了,就取出了他的一根肋骨,再用肉补满原处。
\textsuperscript{22}
然后上主天主用那由人取来的肋骨,形成了一个女人,引她到人前,
\textsuperscript{23}
人遂说:“这才真是我的骨中之骨,肉中之肉,她应称为“女人”,因为是由男人取出的。”
\textsuperscript{24}
为此人应离开自己的父母,依附自己的妻子,二人成为一体。
\textsuperscript{25}
当时,男女二人都赤身露体,并不害羞。
\footnote{本段的要义有二:一、人给动物命名,是表示人受有统治一切造物的权柄;二、从\uline{亚当}的肉身形成了第一个女人,是指女人同他有一样的人性,像\uline{亚当}一样是照天主的肖像受造的。夫妇结为一体,表示婚姻的结合是天主制定的,人不能拆散(\uwave{玛}19:5、6)。赤身不害羞,是说原祖未犯罪前纯洁无罪的状态,还未体验到罪过的恶果。}

\textbf{第三章 }
\textbf{原祖违命 }
\textsuperscript{1}
在上主天主所造的一切野兽中,蛇是最狡猾的。蛇对女人说:“天主真说了,你们不可吃乐园中任何树上的果子吗?”
\textsuperscript{2}
女人对蛇说:“乐园中树上的果子,我们都可吃;
\textsuperscript{3}
只有乐园中央那颗树上的果子,天主说过,你们不可以吃,也不可摸,免得死亡。”
\textsuperscript{4}
蛇对女人说:“你们决不会死!
\textsuperscript{5}
因为天主知道,你们那天吃了这果子,你们的眼就会开了,将如同天主一样知道善恶。”
\textsuperscript{6}
女人看那棵果树实在好吃好看,令人羡慕,且能增加智慧,遂摘下一个果子吃了,又给了她的男人一个,他也吃了。
\textsuperscript{7}
于是二人的眼立即开了,发觉自己赤身露体,遂用无花果树叶,编了个裙子围身。
\textsuperscript{8}
当\uline{亚当}和他的妻子听见了上主天主趁晚凉在乐园中散步的声音,就躲藏在乐园的树林中,怕见上主天主的面。
\footnote{本章记的蛇就是魔鬼。他籍蛇形诱惑了\uline{厄娃}(\uwave{智}2:23、24;\uwave{若}8:44;\uwave{默}12:9,20:2)。原祖所犯是骄傲背命的罪。“发觉自己赤身”,是指失去天主的宠爱和原始的纯洁。}
\textsuperscript{9}
上主天主呼唤\uline{亚当}对他说:“你在哪里?”
\textsuperscript{10}
他答说:“我在乐园中听到了你的声音,就害怕起来,因为我赤身露体,遂躲藏了。”
\textsuperscript{11}
天主说:“谁告诉了你,赤身露体?莫非你吃了我禁止你吃的果子?”
\textsuperscript{12}
\uline{亚当}说:“是你给我作伴的那个女人给了我那树上的果子,我才吃了。”
\textsuperscript{13}
上主天主遂对女人说:“你为什么作了这事?”女人答说:“是蛇哄骗了我,我才吃了。”
\footnote{天主询问时,没有询问魔鬼,只询问了亚当\uline{亚当} \uline{厄娃};但惩罚时却按罪过的原因和轻重:先是魔鬼,后是\uline{厄娃},最后是\uline{亚当}(\uwave{第}前2:13-15)。}

\textbf{处罚与预许 }
\textsuperscript{14}
上主天主对蛇说:“因你做了这事,你在一切畜牲和野兽中,是可咒骂的;你要用肚子爬行,毕生日日吃土。
\textsuperscript{15}
我要把仇恨放在你和女人,你的后裔和她的后裔之间,她的后裔要踏碎你的头颅,你要伤害他的脚跟。”
\textsuperscript{16}
后对女人说:“我要增加你怀孕的苦楚,在痛苦中生子;你要依恋你的丈夫,也要受他的管辖。”
\textsuperscript{17}
后对\uline{亚当}说:“因为你听了你妻子的话,吃了我禁止你吃的果子,为了你的缘故,地成了可咒骂的;你一生日日劳苦才能得到吃食。
\textsuperscript{18}
地要给你生出荆棘和蒺藜,你要吃田间的蔬菜;
\textsuperscript{19}
你必须汗流满面,才有饭吃,直到你归于土中,因为你是由土来的;你既是灰土,你还要归于灰土。”
\footnote{天主的仁慈即超过了他的公义,故此他在义怒中给人类预许了人终要得胜魔鬼的诺言;因这许诺,3:15称为“原始福音”。大义是:踏碎蛇头是得胜魔鬼的象征;“女人的后裔”虽然也指犯罪败坏的人类,但在此特别指拯救人类的新元首\uline{基督}(\uwave{哥}1:15-18),只有他打败了魔鬼;故此圣\uline{保禄}称他为“新亚当”(\uwave{罗}5:12-15)。魔鬼同\uline{厄娃}的对白,与天使同\uline{玛利亚}的对白恰恰相反:一是诱惑的对白,一是商讨救赎的对白;因此教父由第2世纪起即称\uline{玛利亚}为“新\uline{厄娃}”。又因她与\uline{基督}的密切结合,她也踏碎了魔鬼的头颅。圣母始胎无玷的道理,由此处已露曙光(\uwave{路}1:26-38;\uwave{默}12)。}

\textbf{被逐出乐园 }
\textsuperscript{20}
\uline{亚当}给自己的妻子起名叫\uline{厄娃},因为她是众生的母亲。
\textsuperscript{21}
上主天主为\uline{亚当}和他的妻子做了件皮衣,给他们穿上;
\textsuperscript{22}
然后上主天主说:“看,人已相似我们中的一个,知道了善恶;如今不要让他伸手再摘取生命树上的果子,吃了活到永远。”
\textsuperscript{23}
上主天主遂把他赶出\uline{伊甸}乐园,叫他耕种他所由出的土地。
\textsuperscript{24}
天主将\uline{亚当}逐出了以后,就在\uline{伊甸}乐园的东面,派了“革鲁宾”和刀光四射的火剑,防守到生命树去的路。
\footnote{“革鲁宾”按\uline{巴比伦}语有“保护者”之意(\uwave{出}25:18-22;\uwave{则}1:11)。}

\textbf{第四章 }
\textbf{杀弟之罪 }
\textsuperscript{1}
\uline{亚当}认识了自己的妻子\uline{厄娃},\uline{厄娃}怀了孕,生了\uline{加音}说:“我赖上主获得了一个人。”
\textsuperscript{2}
以后她生了\uline{加音}的弟弟\uline{亚伯尔};\uline{亚伯尔}牧羊,\uline{加音}耕田。
\textsuperscript{3}
有一天,\uline{加音}把田地的出产作祭品献给天主;
\textsuperscript{4}
同时\uline{亚伯尔}献上自己羊群中最肥美而又是首生的羊;上主惠顾了\uline{亚伯尔}和他的祭品,
\footnote{“认识妻子”是表示夫妻结合的委婉语。\uline{厄娃}生子后说的话意义深奥,说明天主特籍为人母者广传肖似天主的人类(\uwave{加}下7:26-29)。由本章证明献祭从人类起初即有了;献祭的真正价值是在于人的敬心诚意(\uwave{希}11:4)。}
\textsuperscript{5}
% TODO




\begin{center}
	\textbf{后编 圣组史(12-50)}
\end{center}
\chapter{创世纪(创)}

\begin{center}
	\textbf{前编 }
	\textbf{太古史(1-11)}
\end{center}

\textbf{第一章 }
\textbf{天地万物的创造 }
\textsuperscript{1}
在起初天主创造了天地。
\textsuperscript{2}  
大地还是混沌空虚,深渊上还是一团黑暗,天主的神在水面上运行。
\renewcommand\thefootnote{\ding{\numexpr171+\value{footnote}}}
\footnote{“在起初……”一语,暗示创造万物之时,除天主外,一无所有。“天地”二字此处有宇宙万物之意。作者用诗人的想象力描写天主好似一个工程师,在六天以内创造了万物,到第七天休息。首先所创造的是混沌的无生之物,后将这混沌之物分成天、地、海三大部分,然后以日月、星辰、草木、飞禽、走兽等来点缀天地海洋。最后天主照自己的肖像造了人。作者从创造混沌之物说起,到创造人,表示人是万物之灵,应效法造物主工作和守安息日。此开宗明义第一章是远古时代的文学杰作,是一篇宗教的重要文告,并不是自然科学的论著。按古代各民族对开天辟地,人类诞生的传说,没有可与《创世纪》第一章相比拟的。“天主的神”指施生命之神力,但若通观新旧二约的全部启示,此处也指赐生命的“天主圣神”。}
\textsuperscript{3}
天主说:“有光!”就有了光。
\textsuperscript{4}
天主见光好,就将光与黑暗分开。
\textsuperscript{5}
天主称光为“昼”,称黑暗为“夜”。过了晚上,过了早晨,这是第一天。

\textsuperscript{6}
天主说:“在水与水之间要有穹苍,将水分开!”事就这样成了。
\textsuperscript{7}
天主造了穹苍,分开了穹苍以下的水和穹苍以上的水。
\textsuperscript{8}
天主称穹苍为“天”,天主看了认为好。过了晚上,过了早晨,这是第二天。

\textsuperscript{9}
天主说:“天下的水应聚在一处,使旱地出现!”事就这样成了。
\textsuperscript{10}
天主称旱地为“陆地”,称水汇合处为“海洋”。天主看了认为好。
\textsuperscript{11}
天主说:“在陆地上,土地要生出青草、结种子的蔬菜和结果子的果树,各按照在它内的种子的种类!”事就这样成了。
\textsuperscript{12}
土地就生出了青草、结种子的蔬菜,各按其类,和结果子的树木,各按照在它内的种子的种类。天主看了认为好。
\textsuperscript{13}
过了晚上,过了早晨,这是第三天。

\textsuperscript{14}
天主说:“在天空中要有光体,以分别昼夜,作为规定时节和年月日的记号。
\textsuperscript{15}
要在天空中放光,照耀大地!”事就这样成了。
\textsuperscript{16}
天主于是造了两个大光体:较大的控制白天,较小的控制黑夜,并造了星宿。
\textsuperscript{17}
天主将星宿摆列在天空,照耀大地,
\textsuperscript{18}
控制昼夜,分别明与暗。天主看了认为好。
\textsuperscript{19}
过了晚上,过了早晨,这是第四天。

\textsuperscript{20}
天主说:“水中要繁生蠕动的生物,地面上、天空中要有鸟飞翔!”事就这样成了。
\textsuperscript{21}
天主于是造了大鱼和所有在水中孳生的蠕动生物,各按其类,以及各种飞鸟,各按其类。天主看了认为好。
\textsuperscript{22}
遂祝福它们说:“你们要孳生繁殖,充满海洋;飞鸟也要在地上繁殖!”
\textsuperscript{23}
过了晚上,过了早晨,这是第五天。

\textsuperscript{24}
天主说:“地上要生出生物,各按其类;走兽、爬虫和地上的各种生物,各按其类!”事就这样成了。
\textsuperscript{25}
天主于是造了地上的生物,各按其类;各种走兽,各按其类;以及地上所有的爬虫,各按其类。天主看了认为好。
\textsuperscript{26}
天主说:“让我们照我们的肖像,按我们的模样造人,叫他管理海中的鱼、天空的飞鸟、牲畜、各种野兽、在地上爬行的各种爬虫。”
\textsuperscript{27}
天主于是照自己的肖像造了人,就是照天主的肖像造了人:造了一男一女。
\textsuperscript{28}
天主祝福他们说:“你们要生育繁殖,充满大地,治理大地,管理海中的鱼、天空的飞鸟、各种在地上爬行的生物!”
\footnote{“人”按原文有红土或黄土的意思,是说人是属于土的造物。“我们”(26节)按古\uline{犹太}经师的解释,是指天主和天使,好似天主同天使商量;但有些学者主张为“威严复数”或“议决复数”。教父和神学家多以为此复数暗示天主圣三的奥理。此说若照启示的演进说是对的。人相似天主是按灵魂说的,相似天主有理智、意志和记忆。论人的肉身,当天主造\uline{亚当}时,已预见作\uline{亚当}第二的\uline{基督}(\uwave{罗}5:14)。“造了一男一女”,指婚姻一夫一妻制和不可分离性(\uwave{玛}19:1-6;\uwave{拉}2:15、16)。天主祝福原祖生育繁殖的话,说明婚姻的首要目的是生养教育子女(8:17;\uwave{咏}127:3、4)。}
\textsuperscript{29}
天主又说:“看,全地面上结种子的各种蔬菜,在果内含有种子的各种果树,我都给你们作食物;
\textsuperscript{30}
至于地上的各种野兽,天空中的各种飞鸟,在地上爬行有生魂的各种动物,我把一切青草给它们作食物。”事就这样成了。
\textsuperscript{31}
天主看了他造的一切,认为样样都很好。过了晚上,过了早晨,这是第六天。
\footnote{天主造了原祖,也赐给了他们和他们传生的人类的食物,并将普世交给他们统治。所造的万物样样都好,是说万物都合天主的旨意,都为他所喜爱。参阅\uwave{咏}19:1-6,104,145,148,150。}

\textbf{第二章 }
\textbf{安息日 }
\textsuperscript{1}
这样,天地和天地间的一切点缀都完成了。
\textsuperscript{2}
到第七天天主造物的工程已完成,就在第七天休息,停止了所作的一切工程。
\textsuperscript{3}
天主祝福了第七天,定为圣日,因为这一天,天主停止了他所行的一切创造工作。
\footnote{1-3节属前章,劝人守安息日为圣日。守安息日的原因与目的,见\uwave{出}23:12;\uwave{申}5:12-15。}

\textbf{人与乐园 }
\textsuperscript{4}
这是创造天地的来历:在上主天主创造天地时,
\textsuperscript{5}
地上还没有灌木,田间也没有生出蔬菜,因为上主天主还没有使雨降在地上,也没有人耕种土地,
\textsuperscript{6}
有从地下涌出的水浸润所有地面。
\textsuperscript{7}
上主天主用地上的灰土形成了人,在他鼻孔内吹了一口生气,人就成了一个有灵的生物。
\textsuperscript{8}
上主天主在\uline{伊甸}东部种植了一个乐园,就将他形成的人安置在里面。
\textsuperscript{9}
上主天主使地面生出各种好看好吃的果树,生命树和知善恶树在乐园中央。
\footnote{2:4-3:24为创造天地万物的另一记载。原来在这记载中只用了天主(雅威)的名词,但将这个记载与上章的记载编在一起,补入了“天主”的名词。这记载的中心为人:天主对人,人对天主的态度。关于人的来历和本性,作者用简略的话,教训人一端论宗教和文化的最高深的道理:人肉身的形成,好像其他的动物,是由尘土造成的,但对于灵魂却有极大的区别,它是直接由天主所造。\uline{伊甸}乐园位于何处,人不得而知。乐园是天主考验人的地方。“生命树”所象征的是天主愿意赐给人的“不死”之恩。“知善恶的树”,是试探人的工具。“知善恶”的意思,大概是说:人一犯天主的禁令,就知道所失去的超性恩宠——真善,是多么美善,所犯的罪恶——真恶,是如何凶恶。}
\textsuperscript{10}
有一条河由\uline{伊甸}流出灌溉乐园,由那里分为四支:
\textsuperscript{11}
第一支名叫\uline{丕雄},环流产金的\uline{哈威拉}全境;
\textsuperscript{12}
那地方的金子很好,那里还产珍珠和玛瑙;
\textsuperscript{13}
第二支河名叫\uline{基红},环流\uline{雇士}全境;
\textsuperscript{14}
第三支河名叫\uline{底格里斯},流入\uline{亚述}东部;第四支河即\uline{幼发拉的}。
\textsuperscript{15}
上主天主将人安置在\uline{伊甸}的乐园内,叫他耕种,看守乐园。
\footnote{说明人犯罪以前,天主已叫人应该工作。}
\textsuperscript{16}
上主天主给人下令说:“乐园中各树上的果子,你都可吃,
\textsuperscript{17}
只有知善恶树上的果子你不可吃,因为那一天你吃了,必定要死。”

\textbf{造女人立婚姻 }
\textsuperscript{18}
上主天主说:“人单独不好,我要给他造个与他相称的助手。”
\textsuperscript{19}
上主天主用尘土造了各种野兽和天空中的各种飞鸟,都引到人面前,看他怎样起名;凡人给生物起的名字,就成了那生物的名字。
\textsuperscript{20}
人遂给各种畜牲、天空中的各种飞鸟和各种野兽起了名字;但他没有找着一个与自己相称的助手。
\textsuperscript{21}
上主天主遂使人熟睡,当他睡着了,就取出了他的一根肋骨,再用肉补满原处。
\textsuperscript{22}
然后上主天主用那由人取来的肋骨,形成了一个女人,引她到人前,
\textsuperscript{23}
人遂说:“这才真是我的骨中之骨,肉中之肉,她应称为“女人”,因为是由男人取出的。”
\textsuperscript{24}
为此人应离开自己的父母,依附自己的妻子,二人成为一体。
\textsuperscript{25}
当时,男女二人都赤身露体,并不害羞。
\footnote{本段的要义有二:一、人给动物命名,是表示人受有统治一切造物的权柄;二、从\uline{亚当}的肉身形成了第一个女人,是指女人同他有一样的人性,像\uline{亚当}一样是照天主的肖像受造的。夫妇结为一体,表示婚姻的结合是天主制定的,人不能拆散(\uwave{玛}19:5、6)。赤身不害羞,是说原祖未犯罪前纯洁无罪的状态,还未体验到罪过的恶果。}

\textbf{第三章 }
\textbf{原祖违命 }
\textsuperscript{1}
在上主天主所造的一切野兽中,蛇是最狡猾的。蛇对女人说:“天主真说了,你们不可吃乐园中任何树上的果子吗?”
\textsuperscript{2}
女人对蛇说:“乐园中树上的果子,我们都可吃;
\textsuperscript{3}
只有乐园中央那颗树上的果子,天主说过,你们不可以吃,也不可摸,免得死亡。”
\textsuperscript{4}
蛇对女人说:“你们决不会死!
\textsuperscript{5}
因为天主知道,你们那天吃了这果子,你们的眼就会开了,将如同天主一样知道善恶。”
\textsuperscript{6}
女人看那棵果树实在好吃好看,令人羡慕,且能增加智慧,遂摘下一个果子吃了,又给了她的男人一个,他也吃了。
\textsuperscript{7}
于是二人的眼立即开了,发觉自己赤身露体,遂用无花果树叶,编了个裙子围身。
\textsuperscript{8}
当\uline{亚当}和他的妻子听见了上主天主趁晚凉在乐园中散步的声音,就躲藏在乐园的树林中,怕见上主天主的面。
\footnote{本章记的蛇就是魔鬼。他籍蛇形诱惑了\uline{厄娃}(\uwave{智}2:23、24;\uwave{若}8:44;\uwave{默}12:9,20:2)。原祖所犯是骄傲背命的罪。“发觉自己赤身”,是指失去天主的宠爱和原始的纯洁。}
\textsuperscript{9}
上主天主呼唤\uline{亚当}对他说:“你在哪里?”
\textsuperscript{10}
他答说:“我在乐园中听到了你的声音,就害怕起来,因为我赤身露体,遂躲藏了。”
\textsuperscript{11}
天主说:“谁告诉了你,赤身露体?莫非你吃了我禁止你吃的果子?”
\textsuperscript{12}
\uline{亚当}说:“是你给我作伴的那个女人给了我那树上的果子,我才吃了。”
\textsuperscript{13}
上主天主遂对女人说:“你为什么作了这事?”女人答说:“是蛇哄骗了我,我才吃了。”
\footnote{天主询问时,没有询问魔鬼,只询问了亚当\uline{亚当} \uline{厄娃};但惩罚时却按罪过的原因和轻重:先是魔鬼,后是\uline{厄娃},最后是\uline{亚当}(\uwave{第}前2:13-15)。}

\textbf{处罚与预许 }
\textsuperscript{14}
上主天主对蛇说:“因你做了这事,你在一切畜牲和野兽中,是可咒骂的;你要用肚子爬行,毕生日日吃土。
\textsuperscript{15}
我要把仇恨放在你和女人,你的后裔和她的后裔之间,她的后裔要踏碎你的头颅,你要伤害他的脚跟。”
\textsuperscript{16}
后对女人说:“我要增加你怀孕的苦楚,在痛苦中生子;你要依恋你的丈夫,也要受他的管辖。”
\textsuperscript{17}
后对\uline{亚当}说:“因为你听了你妻子的话,吃了我禁止你吃的果子,为了你的缘故,地成了可咒骂的;你一生日日劳苦才能得到吃食。
\textsuperscript{18}
地要给你生出荆棘和蒺藜,你要吃田间的蔬菜;
\textsuperscript{19}
你必须汗流满面,才有饭吃,直到你归于土中,因为你是由土来的;你既是灰土,你还要归于灰土。”
\footnote{天主的仁慈即超过了他的公义,故此他在义怒中给人类预许了人终要得胜魔鬼的诺言;因这许诺,3:15称为“原始福音”。大义是:踏碎蛇头是得胜魔鬼的象征;“女人的后裔”虽然也指犯罪败坏的人类,但在此特别指拯救人类的新元首\uline{基督}(\uwave{哥}1:15-18),只有他打败了魔鬼;故此圣\uline{保禄}称他为“新亚当”(\uwave{罗}5:12-15)。魔鬼同\uline{厄娃}的对白,与天使同\uline{玛利亚}的对白恰恰相反:一是诱惑的对白,一是商讨救赎的对白;因此教父由第2世纪起即称\uline{玛利亚}为“新\uline{厄娃}”。又因她与\uline{基督}的密切结合,她也踏碎了魔鬼的头颅。圣母始胎无玷的道理,由此处已露曙光(\uwave{路}1:26-38;\uwave{默}12)。}

\textbf{被逐出乐园 }
\textsuperscript{20}
\uline{亚当}给自己的妻子起名叫\uline{厄娃},因为她是众生的母亲。
\textsuperscript{21}
上主天主为\uline{亚当}和他的妻子做了件皮衣,给他们穿上;
\textsuperscript{22}
然后上主天主说:“看,人已相似我们中的一个,知道了善恶;如今不要让他伸手再摘取生命树上的果子,吃了活到永远。”
\textsuperscript{23}
上主天主遂把他赶出\uline{伊甸}乐园,叫他耕种他所由出的土地。
\textsuperscript{24}
天主将\uline{亚当}逐出了以后,就在\uline{伊甸}乐园的东面,派了“革鲁宾”和刀光四射的火剑,防守到生命树去的路。
\footnote{“革鲁宾”按\uline{巴比伦}语有“保护者”之意(\uwave{出}25:18-22;\uwave{则}1:11)。}

\textbf{第四章 }
\textbf{杀弟之罪 }
\textsuperscript{1}
\uline{亚当}认识了自己的妻子\uline{厄娃},\uline{厄娃}怀了孕,生了\uline{加音}说:“我赖上主获得了一个人。”
\textsuperscript{2}
以后她生了\uline{加音}的弟弟\uline{亚伯尔};\uline{亚伯尔}牧羊,\uline{加音}耕田。
\textsuperscript{3}
有一天,\uline{加音}把田地的出产作祭品献给天主;
\textsuperscript{4}
同时\uline{亚伯尔}献上自己羊群中最肥美而又是首生的羊;上主惠顾了\uline{亚伯尔}和他的祭品,
\footnote{“认识妻子”是表示夫妻结合的委婉语。\uline{厄娃}生子后说的话意义深奥,说明天主特籍为人母者广传肖似天主的人类(\uwave{加}下7:26-29)。由本章证明献祭从人类起初即有了;献祭的真正价值是在于人的敬心诚意(\uwave{希}11:4)。}
\textsuperscript{5}
却没有惠顾\uline{加音}和他的祭品;因此\uline{加音}大怒,垂头丧气。
\textsuperscript{6}
上主对\uline{加音}说:“你为什么发怒?为什么垂头丧气?
\textsuperscript{7}
你若做得好,岂不也可仰起头来?你若做得不好,罪恶就伏在你门前,企图对付你,但你应制服它。”

\textsuperscript{8}
事后\uline{加音}对他弟弟\uline{亚伯尔}说:“我们到田间去!”当他们在田间的时候,\uline{加音}就袭击了弟弟\uline{亚伯尔},将他杀死。
\textsuperscript{9}
上主对\uline{加音}说:“你弟弟\uline{亚伯尔}在哪里?”他答说:“我不知道,难道我是看守我弟弟的人?”
\textsuperscript{10}
上主说:“你作了什么事?听!你弟弟的血由地上向我喊冤。
\textsuperscript{11}
你现在是地上所咒骂的人,地张开口由你手中接收了你弟弟的血,
\textsuperscript{12}
从此你即使耕种,地也不会给你出产;你在地上要成个流离失所的人。”
\textsuperscript{13}
\uline{加音}对上主说:“我的罪罚太重,无法承担。
\textsuperscript{14}
看你今天将我由这地面上驱逐,我该躲避你的面,在地上成了个流离失所的人;那么凡遇见我的,必要杀我。”
\textsuperscript{15}
上主对他说:“决不这样,凡杀\uline{加音}的人,一定要受七倍的罚。”上主遂给\uline{加音}一个记号,以免遇见他的人击杀他。
\textsuperscript{16}
\uline{加音}就离开上主的面,住在\uline{伊甸}东方的\uline{诺得}地方。
\footnote{杀兄弟的暴行为原祖犯罪的恶果。“凡遇见我的”一句,假定除\uline{加音}、\uline{亚伯尔}、\uline{舍特}三人外,\uline{亚当}还生了别的一些儿女。圣经只记载此三人,因为他们的命运为叙述救赎史已够了。初民为遵从天主叫人传生人类的命令,不能不兄妹结合。但日后人类增多了,兄妹的结合为宗教与礼法所禁止。}

\textbf{加音的后代 }
\textsuperscript{17}
\uline{加音}认识了自己的妻子,她怀了孕,生了\uline{哈诺客}。\uline{加音}建筑了一座城,即以他儿子的名字,给这城起名叫“\uline{哈诺客}”。
\textsuperscript{18}
\uline{哈诺客}生了\uline{依辣得};\uline{依辣得}生了\uline{默胡雅耳};\uline{默胡雅耳}生了\uline{默突沙耳};\uline{默突沙耳}生了\uline{拉默客}。
\textsuperscript{19}
\uline{拉默客}娶了两个妻子:一个名叫\uline{阿达},一个名叫\uline{漆拉}。
\textsuperscript{20}
\uline{阿达}生了\uline{雅巴耳},他是住在帐幕内畜牧者的始祖。
\textsuperscript{21}
他的弟弟名叫\uline{犹巴耳},他是所有弹琴吹箫者的始祖。
\textsuperscript{22}
同时\uline{漆拉}也生了\uline{突巴耳}\uline{加音},他是制造各种铜铁器具的匠人。\uline{突巴耳}\uline{加音}有个姊妹名叫\uline{纳阿玛}。
\textsuperscript{23}
\uline{拉默客}对自己的妻子说:“\uline{阿拉}和\uline{漆拉}倾听我的声音,\uline{拉默客}的妻子,静聆我的言语:因我受伤,杀了一成年;因我受损,杀了一青年;
\textsuperscript{24}
杀\uline{加音}的受罚是七倍,杀\uline{拉默客}的是七十七倍。”
\footnote{此段略记\uline{加音}的后代子孙,和他们的发明以及文化的初步演进。由此可知在洪水之前文明已达到了相当的程度。\uline{巴比伦}史家亦主此说。\uline{拉默客}是违犯一夫一妻制的第一人,违犯了婚姻一夫一妻的理想。}

\textbf{舍特的子孙 }
\textsuperscript{25}
\uline{亚当}又认识了自己的妻子,她生了个儿子,给他起名叫\uline{舍特}说:“天主又赐给了我一个儿子,代替\uline{加音}杀了的\uline{亚伯尔}。”
\textsuperscript{26}
\uline{舍特}也生了一个儿子,给他起名叫\uline{厄诺士}。那时人才开始呼求上主的名。
\footnote{从\uline{亚当}到\uline{厄诺士},人在祈祷和祭献时一定呼求天主助佑。但由\uline{厄诺士}开始举行公众崇拜天主的敬礼。}

\textbf{第五章 }
\textbf{洪水前亚当的后代 }
\textsuperscript{1}
以下是\uline{亚当}后裔的族谱:当天主造人的时候,是按天主的肖像造的,
\textsuperscript{2}
造了一男一女,且在造他们的那一天,祝福了他们,称他们为“人”。
\textsuperscript{3}
\uline{亚当}一百三十岁时,生了一个儿子,也像自己的模样和肖像,给他起名叫\uline{舍特}。
\footnote{\uline{亚当}生了相似自己的子女,因为他是照天主的肖像受造的,他生儿养女,即是传生天生至尊贵的肖像于万世万代的人类。}
\textsuperscript{4}
\uline{亚当}生\uline{舍特}后,还活了八百年,生了其他的儿女。
\textsuperscript{5}
\uline{亚当}共活了九百三十岁死了。
\textsuperscript{6}
\uline{舍特}一百零五岁时,生了\uline{厄诺士}。
\textsuperscript{7}
\uline{舍特}生\uline{厄诺士}后,还活了八百零七年,生了其他的儿女。
\footnote{本章所记为洪水前的十位祖宗,他们的长寿若与其他民族传说的古人比较,所记的年龄还不算太大。虽然如此,有关十位祖宗的年龄,是不易解决的难题。——圣经上说的年是指十二个月的年,月指廿九或三十日的月。——上古人类是否能享如此的长寿,考古人类学至今尚未有一圆满的答案。教父和神学家提出了两个理由来解释原始人的长寿原因:一、自然环境的优良条件:即在人犯原罪之后,仍未丧失天主在造人时所赋的优良人性;二、长寿的主因是天主上智的措施,使人能迅速繁殖,并使人将天主的原始启示传于后代子孙。}
\textsuperscript{8}
\uline{舍特}共活了九百一十二岁死了。
\textsuperscript{9}
\uline{厄诺士}九十岁时生了\uline{刻南}。
\textsuperscript{10}
\uline{厄诺士}生\uline{刻南}后,还活了八百一十五年,生了其他的儿女。
\textsuperscript{11}
\uline{厄诺士}共活了九百零五岁死了。
\textsuperscript{12}
\uline{刻南}七十岁时,生了\uline{玛拉肋耳}。
\textsuperscript{13}
\uline{刻南}生\uline{玛拉肋耳}后,还活了八百四十年,生了其他的儿女。
\textsuperscript{14}
\uline{刻南}共活了九百一十岁死了。
\textsuperscript{15}
\uline{玛拉肋耳}六十五岁时,生了\uline{耶勒得}。
\textsuperscript{16}
\uline{玛拉肋耳}生\uline{耶勒得}后,还活了八百三十年,生了其他的儿女。
\textsuperscript{17}
\uline{玛拉肋耳}共活了八百九十五岁死了。
\textsuperscript{18}
\uline{耶勒得}一百六十二岁时,生了\uline{哈诺客}。
\textsuperscript{19}
\uline{耶勒得}生\uline{哈诺客}后。还活了八百年,生了其他的儿女。
\textsuperscript{20}
\uline{耶勒得}共活了九百六十二岁死了。
\textsuperscript{21}
\uline{哈诺客}六十五岁时,生了\uline{默突舍拉}。
\textsuperscript{22}
\uline{哈诺客}常与天主往来。\uline{哈诺客}生\uline{默突舍拉}后,还活了三百年,生了其他的儿女。
\textsuperscript{23}
\uline{哈诺客}共活了三百六十五岁。
\textsuperscript{24}
\uline{哈诺客}时与天主往来,然后就不见了,因为天主将他提去。
\footnote{论\uline{哈诺客}的事,见\uwave{德}44:16;\uwave{犹}14、15;\uwave{希}11:5。}
\textsuperscript{25}
\uline{默突舍拉}一百八十七岁时,生了\uline{拉默客}。
\textsuperscript{26}
\uline{默突舍拉}生了\uline{拉默客}后,还活了七百八十二年,生了其他的儿女。
\textsuperscript{27}
\uline{默突舍拉}共活了九百六十九岁死了。
\textsuperscript{28}
\uline{拉默客}一百八十二岁时,生了一个儿子,
\textsuperscript{29}
给他起名叫\uline{诺厄}说:“这孩子要使我们在上主诅咒的地上,在我们做的工作和劳苦上,获得欣慰!”
\textsuperscript{30}
\uline{拉默客}生\uline{诺厄}后,还活了五百九十五年,生了其他的儿女。
\textsuperscript{31}
\uline{拉默客}共活了七百七十七岁死了。
\textsuperscript{32}
\uline{诺厄}五百岁时,生了\uline{闪}、\uline{含}和\uline{耶斐特}。
\footnote{\uline{拉默客}对\uline{诺厄}的祝福是一预言。此预言在8:22,9:8-17实现了。}

\textbf{第六章 }
\textbf{人类的败坏 }
\textsuperscript{1}
当人在地上开始繁殖,生养女儿时,
\textsuperscript{2}
天主的儿子见人的女儿美丽,就随意选取,作为妻子。
\textsuperscript{3}
上主于是说:“因为人即属于血肉,我的神不能常在他内;他的寿数只可到一百二十岁。”
\textsuperscript{4}
当天主的儿子与人的儿女结合生子时,在地上已有一些巨人,(以后也有),他们就是古代的英雄,著名的人物。
\footnote{洪水之罚是人类的败坏所引起的,这败坏的近因是因天主的儿子们娶了人的女儿们。所谓天主的儿子即恭敬天主的\uline{舍特}的子孙;人的女儿即指\uline{加音}不恭敬天主的子女。“我的神”此处是指天主赋于人的生活之力(2:7)。“血肉”即指易于沉湎于肉身之乐的人性。“巨人”的来历无法考定。巨人的事迹,多见于《旧约》中(\uwave{户}13:33;\uwave{申}3:11;\uwave{撒}上17;\uwave{巴}3:26-28等处)。此处作者并非说巨人是由天主的儿子和人的女儿生的,而只是说当天主的儿子和人的女儿结合时,地上已有巨人。这些巨人相似那些强悍善战,不认识智慧之道的巨人(\uwave{巴}3:26、27)。}

\textbf{上主决意消灭世界 }
\textsuperscript{5}
上主见人在地上的罪恶重大,人心天天所思念的无非是邪恶;
\textsuperscript{6}
上主遂后悔在地上造了人,心中很是悲痛。
\textsuperscript{7}
上主于是说:“我要将我所造的人,连人带野兽、爬虫和天空的飞鸟,都由地面上消灭,因为我后悔造了他们。”
\textsuperscript{8}
惟有\uline{诺厄}在上主眼中蒙受恩爱。
\footnote{有关洪水的记载(6-8章),大概来自两种有关洪水的记述。近东古代史家编纂史书,多只穿插古文件,而对文件中互异之处,多不加修改。有关洪水的传说,古代民族大都有所记载。本书所记就结构和体裁而言,与\uwave{叔默尔}和\uwave{巴比伦}的洪水神话有很多类似之处,但根本的区别很大,因本书中决无多神的不经之论;且本书所记是在教训世人几端道德和宗教的高深道理,如天主的正义、仁慈、召选、救恩和盟约的道理(9:1-17)。《旧约》的作者多以\uwave{诺厄}和洪水的事为天主施恩和惩罚的预象(\uwave{依}54:7-10;\uwave{德}44:17-19;\uwave{智}10:4)。《新约》多以洪水的事为公审判(\uwave{玛}24:37-39),或圣洗圣事的预象(\uwave{伯}前3:18-22)。}

\textbf{诺厄建造方舟 }
\textsuperscript{9}
以下是\uline{诺厄}的小史:\uline{诺厄}是他同时代唯一正义齐全的人,常同天主往来。
\textsuperscript{10}
他生了三个儿子:就是\uline{闪}、\uline{含}、和\uline{耶斐特}。
\textsuperscript{11}
大地已在天主面前败坏,到处充满了强暴。
\textsuperscript{12}
天主见大地已败坏,因为凡有血肉的人,品行在地上全败坏了,
\textsuperscript{13}
天主遂对\uline{诺厄}说:“我已决定要结果一切有血肉的人,因为他们使大地充满了强暴,我要将他们由大地上消灭。
\textsuperscript{14}
你要用柏木造一只方舟,舟内建造一些舱房,内外都涂上沥青。
\textsuperscript{15}
你要这样建造:方舟要有三百肘长,五十肘宽,三十肘高。
\textsuperscript{16}
方舟上层四面做上窗户,高一肘;门要安在侧面;方舟要分为上中下三层。
\textsuperscript{17}
看我要使洪水在地上泛滥,消灭天下一切有生气的血肉;凡地上所有的都要灭亡。
\textsuperscript{18}
但我要与你立约,你以及你的儿子、妻子和儿媳,要与你一同进入方舟。
\textsuperscript{19}
你要由一切有血肉的生物中,各带一对,即一公一母,进入方舟,与你一同生活;
\textsuperscript{20}
各种飞鸟、各种牲畜、地上所有的各种爬虫,皆取一对同你进去,得以保存生命。
\textsuperscript{21}
此外,你还应带上各种吃用的食物,贮存起来,作你和他们的食物。”
\textsuperscript{22}
\uline{诺厄}全照办了;天主怎样吩咐了他,他就怎样做了。

\textbf{第七章 }
\textbf{洪水灭世 }
\textsuperscript{1}
上主对\uline{诺厄}说:“你和你全家进入方舟,因为在这一世代,我看只有你在我面前正义。
\textsuperscript{2}
由一切洁净牲畜中,各取公母七对;由那些不洁净的牲畜中,各取公母一对;
\textsuperscript{3}
由天空的飞鸟中,也各取公母七对;好在全地面上传种。
\textsuperscript{4}
因为还有七天,我要在地上降雨四十天四十夜,消灭我在地面上所造的一切生物。”
\textsuperscript{5}
\uline{诺厄}全照上主吩咐他的做了。
\footnote{带进方舟的牲畜,自然洁净的多于不洁净的,因为洁净的可为食用,又可为祭献天主之用。见8:20-22;\uwave{肋}11。}
\textsuperscript{6}
当洪水在地上泛滥时,\uline{诺厄}已六百岁。
\textsuperscript{7}
\uline{诺厄}和他的儿子,他的妻子和他的儿媳,同他进了方舟,为躲避洪水。
\textsuperscript{8}
洁净的牲畜和不洁净的牲畜,飞鸟和各种在地上爬行的动物,
\textsuperscript{9}
一对一对地同\uline{诺厄}进了方舟;都是一公一母,照天主对他所吩咐的。
\textsuperscript{10}
七天一过,洪水就在地上泛滥。
\textsuperscript{11}
\uline{诺厄}六百岁那一年,二月十七日那天,所有深渊的泉水都冒出,天上的水闸都开放了;
\textsuperscript{12}
大雨在地上下了四十天四十夜。
\textsuperscript{13}
正在这一天,\uline{诺厄}和他的儿子\uline{闪}、\uline{含}、\uline{耶斐特},他的妻子和他的三个儿媳,一同进了方舟。
\textsuperscript{14}
他们八口和所有的野兽、各种牲畜、各种在地上爬行的爬虫、各种飞禽,
\textsuperscript{15}
一切有生气有血肉的,都一对一对地同\uline{诺厄}进了方舟。
\textsuperscript{16}
凡有血肉的,都是一公一母地进了方舟,如天主对\uline{诺厄}所吩咐的。随后上主关了门。

\textsuperscript{17}
洪水在地上泛滥了四十天;水不断增涨,浮起了方舟,方舟遂由地面上升起。
\textsuperscript{18}
洪水汹涌,在地上猛涨,方舟飘浮在水面上。
\textsuperscript{19}
洪水在地上一再猛涨,天下所有的高山都没了顶;
\textsuperscript{20}
洪水高出淹没的群山十有五肘。
\textsuperscript{21}
凡地上行动而有血肉的生物:飞禽、牲畜、野兽、在地上爬行的爬虫,以及所有的人全灭亡了;
\textsuperscript{22}
凡在旱地上以鼻呼吸的生灵都死了。
\textsuperscript{23}
这样,天主消灭了在地面上的一切生物,由人以至于牲畜、爬虫以及天空中的飞鸟,这一切都由地上消灭了,只剩下\uline{诺厄}和同他在方舟内的人物。
\textsuperscript{24}
洪水在地上泛滥了一百五十天。
\footnote{据古\uwave{希伯来}人的宇宙观:天的上边(1:7;\uwave{咏}104:3-13,148:4),地的下面都为水所包围;地下的水也叫深渊(\uwave{依}51:10;\uwave{咏}36:6;\uwave{亚}7:4)。关于洪水泛滥的日期,17节为四十天,24节为一百五十天,大概由两种不同的文献而来。}

\textbf{第八章 }
\textbf{洪水退落 }
\textsuperscript{1}
天主想起了\uline{诺厄}和同他在方舟内的一切野兽和牲畜,遂使风吹过大地,水渐渐退落;
\textsuperscript{2}
深渊的泉源和天上的水闸已关闭,雨也由天上停止降落,
\textsuperscript{3}
于是水逐渐由地上退去;过了一百五十天,水就低落了。
\textsuperscript{4}
七月十七日,方舟停在\uline{阿辣}\uline{辣特}山上。
\textsuperscript{5}
洪水继续减退,直到十月;十月一日,许多山顶都露出来。
\textsuperscript{6}
过了四十天,\uline{诺厄}开了在方舟上做的窗户,
\textsuperscript{7}
放了一只乌鸦;乌鸦飞去又飞回,直到地上的水都干了。
\textsuperscript{8}
\uline{诺厄}等待了七天,又放出了一只鸽子,看看水是否已由地面退尽。
\textsuperscript{9}
但是,因为全地面上还有水,鸽子找不着落脚的地方,遂飞回方舟;\uline{诺厄}伸手将它接入方舟内。
\textsuperscript{10}
再等了七天,他由方舟中又放出一只鸽子,
\textsuperscript{11}
傍晚时,那只鸽子飞回他那里,看,嘴里衔着一根绿的橄榄树枝;\uline{诺厄}于是知道,水已由地上退去。
\textsuperscript{12}
\uline{诺厄}又等了七天再放出一只鸽子;这只鸽子没有回来。
\footnote{通观有关洪水的记载(6:5-7,7:1-8:12),似乎全世界都为洪水所淹没,人类除\uline{诺厄}一家外全都消灭了。但若注意近东古代史家的渲染夸大的作风,“全地”、“天下”或类似的词句,仅指作者所知道的地方(\uwave{创}41:54、57;\uwave{宗}2:5等)。由此可知洪水的泛滥仅是局部的,而未遍及于全世界。淹死的人也只是作者所知道的人民,而不是全人类。细察本书作者的目的,只是记载启示的历史,或天主在世建立神国的历史,所以与启示或与\uline{以色列}人无关系的历史与人物一概不提。}

\textbf{诺厄出方舟 }
\textsuperscript{13}
\uline{诺厄}六百零一岁,正月初一,地上的水都干了,\uline{诺厄}就撤开方舟的顶观望,看见地面已干。
\textsuperscript{14}
二月二十七日,大地全干了。
\textsuperscript{15}
天主于是吩咐\uline{诺厄}说:
\textsuperscript{16}
“你和你的妻子、儿子及儿媳,同你由方舟出来;
\textsuperscript{17}
所有同你在方舟内的有血肉的生物:飞禽、牲畜和各种地上的爬虫,你都带出来,叫他们在地上滋生,在地上生育繁殖。”
\textsuperscript{18}
\uline{诺厄}遂同他的儿子、妻子及儿媳出来;
\textsuperscript{19}
所有的爬虫、飞禽和地上所有的动物,各依其类出了方舟。
\textsuperscript{20}
\uline{诺厄}给上主筑了一座祭坛,拿各种洁净的牲畜和洁净的飞禽,献在祭坛上,作为全燔祭。
\textsuperscript{21}
上主闻到了馨香,心里说:“我再不为人的缘故咒骂大地,因为人心的思念从小就邪恶;我也不再照我所作的打击一切生物了,
\textsuperscript{22}
只愿大地存在之日,稼穑寒暑,冬夏昼夜,循环不息。”
\footnote{由洪水之罚,作者教训人几端重要的道理:罪恶使大地回到了原始的混沌状态;罪恶连累了无灵的受造之物(6:13;\uwave{罗}8:19-22);方舟为圣教会的预象(\uwave{伯}前3:20、21);祭献的举行使人再蒙受天主的祝福;生命之可贵;天主同\uline{诺厄}所立的盟约也及于天地万物。}

\textbf{第九章 }
\textbf{人类复兴 }
\textsuperscript{1}
天主祝福\uline{诺厄}和他的儿子们说:“你们要滋养繁殖,充满大地。
\textsuperscript{2}
地上的各种野兽,天空的各种飞鸟,地上的各种爬虫和水中的各种游鱼,都要对你们表示惊恐畏惧:这一切都已交在你们手中。
\textsuperscript{3}
凡有生命的动物,都可作你们的食物;我将这一切赐给你们,有如以前赐给你们蔬菜一样;
\textsuperscript{4}
凡有生命,带血的肉,你们不可吃;
\textsuperscript{5}
并且,我要追讨害你们生命的血债:向一切野兽追讨,向人,向为弟兄的人,追讨人命。
\textsuperscript{6}
凡流人血的,他的血也要为人所流,因为人是造天主的肖像造的。
\textsuperscript{7}
你们要生育繁殖,在地上滋生繁衍。”
\footnote{天主祝福他们传生人类的话,像祝福原祖一样(1:28-30)。起初天主似乎禁止人吃肉,只准吃蔬菜果品(1:29),现今都准许了,但不准吃带血的肉。按古人的思想,血是生命之所在,是生活的动力。这生命直接来自天主(\uwave{申}12:16、23,15:23;\uwave{肋}3:17,7:26,17:10-14;\uwave{宗}15:29),为此禁止吃血。5、6两节为日后报复法的根据(\uwave{户}35:19;\uwave{出}21:23-25;\uwave{申}19:18-21)。}

\textbf{天主与诺厄立约 }
\textsuperscript{8}
天主对\uline{诺厄}和他的儿子们说:
\textsuperscript{9}
“看,我现在与你们和你们未来的后裔立约,
\textsuperscript{10}
并与同你们在一起的一切生物:飞鸟、牲畜和一切地上野兽,即凡由方舟出来的一切地上生物立约。
\textsuperscript{11}
我与你们立约:凡有血肉的,以后决不再受洪水湮灭,再没有洪水来毁灭大地。”
\textsuperscript{12}
天主说:“这是我在我与你们以及同你们在一起的一切生物之间,立约的永远标记:
\textsuperscript{13}
我把虹霓放在云间,作我与大地之间立约的标记。
\textsuperscript{14}
几时我兴云遮盖大地,云中要出现虹霓,
\textsuperscript{15}
那时我便想起我与你们以及各种属血肉的生物之间所立的盟约:这样水就不会再成为洪水,毁灭一切血肉的生物。
\textsuperscript{16}
几时虹霓在云间出现,我一看见,就想起在天主与地上各种属血肉的生物之间所立的永远盟约。”
\textsuperscript{17}
天主对\uline{诺厄}说:“这就是我在我与地上一切有血肉的生物之间,所立的盟约的标记。”
\footnote{此段所说的盟约原是天主无条件赐恩的许诺;天主以此诺言保证自然界从今以后不再受洪水之害。虹霓的自然现象在洪水前虽已有,但从今以后当作天主诺言的保证。}

\textbf{诺厄的诅咒与祝福 }
\textsuperscript{18}
\uline{诺厄}的儿子由方舟出来的,有\uline{闪}、\uline{含}、和\uline{耶斐特}。含是\uline{客纳罕}的父亲。
\textsuperscript{19}
这三人是\uline{诺厄}的儿子;人类就由这三人分布天下。
\textsuperscript{20}
\uline{诺厄}原是农夫,遂开始种植葡萄园。
\textsuperscript{21}
一天他喝酒喝醉了,就在自己的帐幕内脱去了衣服。
\textsuperscript{22}
\uline{客纳罕}的父亲\uline{含}看见了父亲赤身露体,遂去告诉外面的两个兄弟。
\textsuperscript{23}
\uline{闪}和\uline{耶斐特}二人于是拿了件外衣,搭在肩上,倒退着走进去,盖上父亲的裸体。他们的脸背着,没有看见父亲的裸体。
\textsuperscript{24}
\uline{诺厄}醒了后,知道了小儿对他作的事,
\textsuperscript{25}
就说:“\uline{客纳罕}是可咒骂的,给兄弟当最下贱的奴隶。”
\footnote{\uline{诺厄}诅咒了\uline{客纳罕}而未诅咒\uline{含},一、因天主早祝福了\uline{含}(1节);二、因\uline{客纳罕}的后裔日后放荡无耻。\uline{闪}特受祝福,因他是\uline{亚巴郎}的祖宗,因\uline{亚巴郎}的后裔(基督),万民将获得祝福。\uline{闪}受的祝福也及于\uline{耶斐特},到新约时代也及于\uline{含}和普世万民(\uwave{宗}2:5、9-11;\uwave{路}3:6)。}
\textsuperscript{26}
又说:“上主,闪的天主,应受赞美,\uline{客纳罕}应作他的奴隶。
\textsuperscript{27}
愿天主扩展\uline{耶斐特},使他住在\uline{闪}的帐幕内;\uline{客纳罕}应作他的奴隶。”
\textsuperscript{28}
洪水以后,\uline{诺厄}又活了三百五十年。
\textsuperscript{29}
\uline{诺厄}共活了九百五十岁死了。

\textbf{第十章 }
\textbf{诺厄三子的后裔 }
\textsuperscript{1}
以下是\uline{诺厄}的儿子\uline{闪}、\uline{含}、和\uline{耶斐特}的后裔。洪水以后,他们都生了子孙。

\textsuperscript{2}
\uline{耶斐特}的之孙:\uline{哥默尔}、\uline{玛哥格}、\uline{玛待}、\uline{雅汪}、\uline{突巴耳}、\uline{默舍客}和\uline{提辣斯}。
\textsuperscript{3}
\uline{哥默尔}的子孙:\uline{阿市}\uline{革纳次}、\uline{黎法特}和\uline{托加尔玛}。
\textsuperscript{4}
\uline{雅汪}的子孙:\uline{厄里沙}、\uline{塔尔史士}、\uline{基延}和\uline{多丹}。
\textsuperscript{5}
那些分布于岛上的民族,就是出于这些人:以上这些人按疆域、语言、宗教和国籍,都属\uline{耶斐特}的子孙。

\textsuperscript{6}
\uline{含}的子孙:\uline{雇士}、\uline{米兹辣殷}、\uline{普特}、和\uline{客纳罕}。
\textsuperscript{7}
\uline{雇士}的子孙:\uline{色巴}、\uline{哈威拉}、\uline{撒贝拉}、\uline{辣阿玛}和\uline{撒贝特加}。\uline{辣阿玛}的之孙:\uline{舍巴}和\uline{德丹}。
\footnote{古地理学家多以本章民族的名单为一种极宝贵的文献。各民族的分布是:北有\uline{耶斐特}的后裔,南有\uline{含}的后裔,在二者之间为\uline{闪}的后裔。有许多名字至今已不可考。}
\textsuperscript{8}
\uline{雇士}生\uline{尼默洛得},他是世上第一个强人。
\textsuperscript{9}
他在上主面前是个有本事的猎人,为此有句俗话说:“如在上主面前,有本领的猎人\uline{尼默洛得}。”
\textsuperscript{10}
他开始建国于\uline{巴比伦}、\uline{厄勒客}和\uline{阿加得},都在\uline{史纳尔}地域。
\textsuperscript{11}
他由那地方去了\uline{亚述},建设了\uline{尼尼微}、\uline{勒曷波特}城、\uline{加拉}
\textsuperscript{12}
和在\uline{尼尼微}与\uline{加拉}之间的\uline{勒森}(\uline{尼尼微}即是那大城)。
\footnote{\uline{尼默洛得}(实体书上是:尼默洛特,有误?)的故事,是上古巨人的故事之一,参见6章注一。}
\textsuperscript{13}
\uline{米兹辣殷}生\uline{路丁}人、\uline{阿纳明}人、\uline{肋哈宾}人、\uline{纳斐突歆}人、
\textsuperscript{14}
\uline{帕特洛斯}人、\uline{加斯路}人和\uline{加非}\uline{托尔}人。\uline{培肋}\uline{舍特}人即出自此族。
\textsuperscript{15}
\uline{客纳罕}生长子\uline{漆冬},以后生\uline{赫特}、
\textsuperscript{16}
\uline{耶步斯}人、\uline{阿摩黎}人、\uline{基尔加}\uline{士}人、
\textsuperscript{17}
\uline{希威}人、\uline{阿尔克}人、\uline{息尼}人、
\textsuperscript{18}
\uline{阿尔瓦得}人、\uline{责玛}\uline{黎}人和\uline{哈玛}\uline{特}人;以后,\uline{客纳罕}的宗教分散了,
\textsuperscript{19}
一致\uline{客纳罕}人的边疆,自\uline{漆冬}经过\uline{革辣尔}直到\uline{迦萨},又经过\uline{索多玛}、\uline{哈摩辣}、\uline{阿德玛}和\uline{责波殷},直到\uline{肋沙}。
\textsuperscript{20}
以上这些人按疆域、语言、宗族和国籍,都属\uline{含}的子孙。

\textsuperscript{21}
\uline{耶斐特}的长兄,即\uline{厄贝尔}所有子孙的祖先\uline{闪},也生了儿子。
\textsuperscript{22}
\uline{闪}的子孙:\uline{厄蓝}、\uline{亚述}、\uline{阿帕革}\uline{沙得}、\uline{路得}和\uline{阿兰}。
\textsuperscript{23}
\uline{阿兰}的之孙:\uline{伍兹}、\uline{胡耳}、\uline{革特尔}和\uline{玛士}。
\textsuperscript{24}
\uline{阿帕革}\uline{沙得}生\uline{舍拉};\uline{舍拉}生\uline{厄贝尔}。
\textsuperscript{25}
\uline{厄贝尔}生了两个儿子:一个名叫\uline{培肋格},因为在他的时代世界分裂了;他的兄弟名叫\uline{约刻堂}。
\textsuperscript{26}
\uline{约刻堂}生\uline{阿耳}\uline{摩达得}、\uline{舍肋夫}、\uline{哈匝玛委特}、\uline{耶辣}、
\textsuperscript{27}
\uline{哈多兰}、\uline{乌匝耳}、\uline{狄刻拉}、
\textsuperscript{28}
\uline{敖巴耳}、\uline{阿彼玛耳}、\uline{舍巴}、
\textsuperscript{29}
\uline{敖非尔}、\uline{哈威拉}和\uline{约巴布}:以上都是\uline{约刻堂}的子孙。
\textsuperscript{30}
他们居住的地域,从\uline{默沙}经过\uline{色法尔}直到东面的山地:
\textsuperscript{31}
以上这些人按疆域、语言、宗族和国籍,都属\uline{闪}的子孙:
\textsuperscript{32}
以上这些人按他们的出身和国籍,都是\uline{诺厄}子孙的家族;洪水以后,地上的民族都是由他们分出来的。
\footnote{本章的主旨有二:一、指出\uline{希伯来}人所知道的主要民族和他们的关系;二、说明\uline{以色列}人在他们中的地位。此外值得注意的是:作者将民族的分布情况说明之后,将范围逐渐缩小,一直缩到\uline{以色列}人的历史。作者列出\uline{诺厄}的后代之后,只说\uline{闪}的嫡系,再后只说\uline{特辣黑}的一支。选民的始祖\uline{亚巴郎}即由此支而来。}

\textbf{第十一章 }
\textbf{巴贝耳塔 }
\textsuperscript{1}
当时全世界只有一种语言和一样的话。
\textsuperscript{2}
当人们由东方迁移的时候,在\uline{史纳尔}地方找到了一块平原,就在那里住下了。
\textsuperscript{3}
他们彼此说:“来,我们这砖,用火烧透。”他们遂拿砖当石,拿沥青代灰泥。
\textsuperscript{4}
然后彼此说:“来,让我们建造一城一塔,塔顶摩天,好给我们作纪念,免得我们在全地面上分散了!”
\textsuperscript{5}
上主遂下来,要看看世人所造的城和塔。
\textsuperscript{6}
上主说:“看,他们都是一个民族,都说一样的语言。他们如今就开始做这事;以后他们所想做的,就没有不成功的了。
\textsuperscript{7}
来,我们下去,混乱他们的语言,使他们彼此语言不通。”
\textsuperscript{8}
于是上主将他们分散到全地面,他们遂停止建造那城。
\textsuperscript{9}
为此人称那地为“巴贝耳”,因为上主在那里混乱了全地的语言,且从那里将他们分山到全地面。
\footnote{原祖由于骄傲违背了天主的命令,洪水以后的人类也犯了同样的罪,因此也受了天主的惩罚。再说,人由于骄傲,彼此不和,必分离四散;分离即久,语言必渐分歧。人类的合一,只有赖\uline{基督}的神国方可实现(\uwave{宗}2:5-21;\uwave{默}7:9、10;\uwave{若}11:52)。古人所建的城和塔。即含有骄傲的意思(4-7节),因此日后的先知多以此城象征世上的恶势力(\uwave{哈}1:11;\uwave{依}11:11;\uwave{达}1,2)。}

\textbf{闪族的家谱 }
\textsuperscript{10}
以下是\uline{闪}的后裔:洪水后两年,\uline{闪}正一百岁,生了\uline{阿帕革}\uline{沙得};
\textsuperscript{11}
生\uline{阿帕革}\uline{沙得}后,\uline{闪}还活了五百年,也生了其他的儿女。
\textsuperscript{12}
\uline{阿帕革}\uline{沙得}三十五岁时,生了\uline{舍拉};
\textsuperscript{13}
生\uline{舍拉}后,\uline{阿帕革}\uline{沙得}还活了四百零三年,也生了其他的儿女。
\textsuperscript{14}
\uline{舍拉}三十岁时,生了\uline{厄贝尔};
\textsuperscript{15}
生\uline{厄贝尔}后,\uline{舍拉}还活了四百零三年,也生了其他的儿女。
\textsuperscript{16}
\uline{厄贝尔}三十四岁时,生了\uline{培肋格};
\textsuperscript{17}
生\uline{培肋格}后,\uline{厄贝尔}还活了四百三十年,也生了其他的儿女。
\textsuperscript{18}
\uline{培肋格}三十岁时,生了\uline{勒伍};
\textsuperscript{19}
生\uline{勒伍}后,\uline{培肋格}还活了二百零九年,也生了其他的儿女。
\textsuperscript{20}
\uline{勒伍}三十二岁时,生了\uline{色鲁格};
\textsuperscript{21}
生\uline{色鲁格}后,\uline{勒伍}还活了二百零七年,也生了其他的儿女。
\textsuperscript{22}
\uline{色鲁格}三十岁时,生了\uline{纳曷尔};
\textsuperscript{23}
生\uline{纳曷尔}后,\uline{色鲁格}还活了二百零七年,也生了其他的儿女。
\textsuperscript{24}
\uline{纳曷尔}活到二十九岁时,生了\uline{特辣黑};
\textsuperscript{25}
生\uline{特辣黑}后,\uline{纳曷尔}还活了一百一十九年,也生了其他的儿女。
\textsuperscript{26}
\uline{特辣黑}七十岁时,生了\uline{亚巴郎}、\uline{纳曷尔}、\uline{哈朗}。
\footnote{\uline{闪}族的十大祖先,相似洪水前的十大祖先(5章)。这个族谱,决不是全人类的整个历史,而只是启示历史的系统而已。}

\textbf{特辣黑的后裔 }
\textsuperscript{27}
以下是\uline{特辣黑}的后裔:\uline{特辣黑}生了\uline{亚巴郎}、\uline{纳曷尔}、\uline{哈朗};\uline{哈朗}生了\uline{罗特}。
\textsuperscript{28}
\uline{哈朗}在他的出生地,\uline{加色丁}人的\uline{乌尔},死在他父亲\uline{特辣黑}面前。
\textsuperscript{29}
\uline{亚巴郎}和\uline{纳曷尔}都娶了妻子,\uline{亚巴郎}的妻子名叫\uline{撒辣依};\uline{纳曷尔}的妻子名叫\uline{米耳加},她是\uline{哈朗}的女儿;\uline{哈朗}是\uline{米耳加}和\uline{依色加}的父亲。
\textsuperscript{30}
\uline{撒辣依}不生育,没有子女。
\textsuperscript{31}
\uline{特辣黑}带了自己的儿子\uline{亚巴郎}和孙子,即\uline{哈朗}的儿子\uline{罗特},并儿媳,即\uline{亚巴郎}的妻子\uline{撒辣依},一同由\uline{加色丁}的\uline{乌尔}出发,往\uline{客纳罕}地去;他们到了\uline{哈兰},就在那里住下了。
\textsuperscript{32}
\uline{特辣黑}死于\uline{哈兰},享寿二百零五岁。
\footnote{按作者的意思,真宗教的信仰只保存在\uline{闪}族的伟大后裔\uline{亚巴郎}那一支内,他是选民的始祖,众信友之父(\uwave{世}32:13;\uwave{依}51:1、2;\uwave{罗}4:11)。}

\begin{center}
	\textbf{后编 }
	\textbf{圣组史(12-50)}
\end{center}

\textbf{第十二章 }
\textbf{亚巴郎蒙召 }
\textsuperscript{1}
上主对\uline{亚巴郎}说:“离开你的故乡、你的家族和父家,往我指给你的地方去。
\textsuperscript{2}
我要使你成为一个大民族,我必祝福你,使你成名,成为一个福源。
\textsuperscript{3}
我要祝福那祝福你的人,咒骂那咒骂你的人;地上万民都要因你获得祝福。”
\textsuperscript{4}
\uline{亚巴郎}遂照上主的吩咐起了身,\uline{罗特}也同他一起走了。\uline{亚巴郎}离开\uline{哈兰}时,已七十五岁。
\textsuperscript{5}
他带了妻子\uline{撒辣依}、他兄弟的儿子\uline{罗特}和他在\uline{哈兰}积蓄的财物,获得的仆婢,一同往\uline{客纳罕}地去,终于到了\uline{客纳罕}地。
\textsuperscript{6}
\uline{亚巴郎}经过那地,直到了\uline{舍根}地\uline{摩勒}橡树区;当时\uline{客纳罕}人尚住在那地方。
\textsuperscript{7}
上主显现给\uline{亚巴郎}说:“我要将这地方赐给你的后裔。”\uline{亚巴郎}就在那里给显现于他的上主,筑了一座祭坛。
\textsuperscript{8}
从那里又迁移到\uline{贝特耳}东面山区,在那里搭了帐幕,西有\uline{贝特耳},东有\uline{哈依};他在那里又为上主筑了一座祭坛,呼求上主的名。
\textsuperscript{9}
以后\uline{亚巴郎}渐渐移往\uline{乃革布}区。
\footnote{\uline{亚巴郎}因坚信天主的话,遵命起身,遂蒙了天主祝福,使他的子孙成为一大民族,而且因了他子孙中所出生的\uline{默西亚},为了那些效法他信德的人,\uline{亚巴郎}成了他们幸福的泉源,因为他保存了真宗教和对\uline{默西亚}的希望,因而万民籍\uline{默西亚}得到了救赎,即如圣\uline{保禄}所说:“\uline{亚巴郎}的祝福在\uline{基督}\uline{耶稣}内普及于万民”(\uwave{迦}3:14;\uwave{希}11:8-12)。}

\textbf{亚巴郎去埃及 }
\textsuperscript{10}
其时那地方起了饥荒,\uline{亚巴郎}遂下到埃及,寄居在那里,因为那地方饥荒十分严重。
\textsuperscript{11}
当他要进\uline{埃及}时,对妻子\uline{撒辣依}说:“我知道你是个貌美的女人;
\textsuperscript{12}
\uline{埃及}人见了你,必要说:这是他的妻子;他们定要杀我,让你活着。
\textsuperscript{13}
所以请你说:你是我的妹妹,这样我因了你而必获优待,赖你的情面,保全我的生命。”
\textsuperscript{14}
果然,当\uline{亚巴郎}一到了\uline{埃及},\uline{埃及}人就注意了这女人实在美丽。
\textsuperscript{15}
法郎的朝臣也看见了她,就在法郎前赞她美丽;这女人就被带入法郎的宫中。
\textsuperscript{16}
\uline{亚巴郎}因了她果然蒙了优待,得了些牛羊、公驴、仆婢、母驴和骆驼。
\textsuperscript{17}
但是,上主为了\uline{亚巴郎}的妻子\uline{撒辣依}的事,降下大难打击了法郎和他全家。
\footnote{\uline{亚巴郎}为保护自己的生命财产的所言所行,虽不可取法,但由此可知,圣经中不拘善恶都记载下来,连圣祖的毛病罪过都一一记叙;而且圣经不但对启示的道理有所进展,而且对道德的概念也有所演进。}
\textsuperscript{18}
法郎遂叫\uline{亚巴郎}来说:“你对我作的是什么事?为什么你没有告诉我,她是你的妻子?
\textsuperscript{19}
为什么你说:她是我的妹妹,以致我娶了她做我的妻子?现在,你的妻子在这里,你带她去吧!”
\textsuperscript{20}
法郎于是吩咐人送走了\uline{亚巴郎}和他的妻子以及他所有的一切。

\textbf{第十三章 }
\textbf{亚巴郎回贝特耳 }
\textsuperscript{1}
\uline{亚巴郎}带了妻子和他所有的一切,与\uline{罗特}一同由\uline{埃及}上来,往\uline{乃革布}去。
\textsuperscript{2}
\uline{亚巴郎}有许多牲畜和金银。
\textsuperscript{3}
他由\uline{乃革布}逐渐往\uline{贝特耳}移动帐幕,到了先前他在\uline{贝特耳}与\uline{哈依}之间,支搭帐幕的地方,
\textsuperscript{4}
亦即他先前筑了祭坛,呼求上主之名的地方。

\textbf{亚巴郎与罗特分离 }
\textsuperscript{5}
与\uline{亚巴郎}同行的\uline{罗特},也有羊群、牛群和帐幕,
\textsuperscript{6}
那地方容不下他们住在一起,因为他们的产业太多,无法住在一起。
\textsuperscript{7}
牧放\uline{亚巴郎}牲畜的人与牧放\uline{罗特}牲畜的人,时常发士口角,——当时\uline{客纳罕}人和\uline{培黎齐}人尚住在那里。
\textsuperscript{8}
\uline{亚巴郎}遂对\uline{罗特}说:“在我与你,我的牧人与你的牧人之间,请不要发士口角,因为我们是至亲。
\textsuperscript{9}
所有的地方不都是在你面前吗?请你与我分开。你若往左,我就往右;你若往右,我就往左。”
\textsuperscript{10}
\uline{罗特}举目看见\uline{约但}河整个平原,直到\uline{左哈尔}一带全有水灌溉,——这是在上主消灭\uline{索多玛}和\uline{哈摩辣}以前的事,——有如上主的乐园,有如\uline{埃及}地。
\textsuperscript{11}
\uline{罗特}选了\uline{约但}河的整个平原,遂向东方迁移;这样,他们就彼此分开了:
\textsuperscript{12}
\uline{亚巴郎}住在\uline{客纳罕}地;\uline{罗特}住在平原的城市中,渐渐移动帐幕,直到\uline{索多玛}。
\textsuperscript{13}
\uline{索多玛}人在上主面前罪大恶极。

\textsuperscript{14}
\uline{罗特}与\uline{亚巴郎}分离以后,上主对\uline{亚巴郎}说:“请你举起眼来,由你所在的地方,向东西南北观看;
\textsuperscript{15}
凡你看见的地方,我都要永远赐给你和你的后裔。
\textsuperscript{16}
我要使你的后裔有如地上的灰尘;如果人能数清地上的灰尘,也能数清你的后裔。
\textsuperscript{17}
你起来,纵横走遍这地,因为我要将这地赐给你。”
\textsuperscript{18}
于是\uline{亚巴郎}移动了帐幕,来到\uline{赫贝龙}的\uline{玛默勒}橡树区居住,在那里给上主筑了一座祭坛。
\footnote{论及\uline{亚巴郎}的宽宏大量,金口圣\uline{若望}说:“长辈对晚辈,老者对少年的\uline{罗特},叔父对侄子,言谈有如兄弟,如平辈,让侄子任意选择。”本章给予的教训是:\uline{罗特}喜爱优裕的生活,没有躲避恶劣的环境,而遭受了重罚;\uline{亚巴郎}的慷慨却受到了厚报。当知本章为18:20、21,19:4-19的前奏。}

\textbf{第十四章 }
\textbf{亚巴郎突袭获胜 }
\textsuperscript{1}
那时\uline{史纳尔}王\uline{阿默}\uline{辣斐耳},\uline{厄拉}\uline{撒尔}王\uline{阿黎约客},\uline{厄蓝}王\uline{革多尔}\uline{老默尔},\uline{哥因}王\uline{提达耳},
\textsuperscript{2}
兴兵攻击\uline{索多玛}王\uline{贝辣},\uline{哈摩辣}王\uline{彼尔沙},\uline{阿德玛}王\uline{史纳布},\uline{责波殷}王\uline{舍默}\uline{贝尔}及\uline{贝拉}即\uline{左哈尔}王。
\textsuperscript{3}
那些王子会合于\uline{息丁}山谷,即今日的\uline{盐海}。
\textsuperscript{4}
他们十二年之久隶属于\uline{革多尔}\uline{老默尔},在十三年上就背叛了。
\textsuperscript{5}
在十四年上,\uline{革多尔}\uline{老默尔}率领与他联盟的君王前来,在\uline{阿市}\uline{塔特}\uline{卡尔}\uline{纳殷}击败了\uline{勒法因},在\uline{哈木}击败了\uline{组斤},在\uline{克黎}\uline{雅塔殷}平原击败了\uline{厄明},
\textsuperscript{6}
在\uline{曷黎}人的\uline{色依尔}山击败了\uline{曷黎}人,一直杀到靠近旷野的\uline{厄耳帕兰};
\textsuperscript{7}
然后回军转到\uline{恩米市}\uline{帕特},即\uline{卡德士},征服了\uline{阿玛肋克}人的全部领土,也征服了住在\uline{哈匝宗}\uline{塔玛尔}的\uline{阿摩黎}人。
\textsuperscript{8}
\uline{索多玛}王\uline{哈摩辣}王,\uline{阿德玛}王,\uline{责波殷}王和\uline{贝拉}即\uline{左哈尔}王,于是出来,在\uline{息丁}山谷列阵,
\textsuperscript{9}
与\uline{厄蓝}王\uline{革多尔}\uline{老默尔},\uline{哥因}王\uline{提达耳},\uline{史纳尔}王\uline{阿默}\uline{辣斐耳}和\uline{厄拉}\uline{撒尔}王\uline{阿黎约客}交战:四个王子敌对五个王子。
\textsuperscript{10}
\uline{息丁}山谷遍地都是沥青坑;\uline{索多玛}王和\uline{哈摩辣}王逃跑时都跌在坑里;其余的人都逃到山里去了。
\textsuperscript{11}
那四个王子劫走了\uline{索多玛}和\uline{哈摩辣}所有的财物和一切食粮,
\textsuperscript{12}
连\uline{亚巴郎}兄弟的儿子\uline{罗特}和他的财物也带走了,因为那时他正住在\uline{索多玛}。
\textsuperscript{13}
有个逃出的人跑来,将这事告诉了\uline{希伯来}人\uline{亚巴郎},他那时住在\uline{阿摩黎}人\uline{玛默勒}的橡树区;这\uline{阿摩黎}人原是\uline{亚巴郎}的盟友\uline{厄市}\uline{苛耳}和\uline{阿乃尔}的兄弟。
\textsuperscript{14}
\uline{亚巴郎}一听说他的亲人被人掳去,遂率领家中的步兵三百一十八人,直追至\uline{丹};
\textsuperscript{15}
夜间又和自己的仆人分队袭击,将他们击败,直追至\uline{大马}\uline{士革}以北的\uline{曷巴},
\textsuperscript{16}
夺回了所有的财物,连他的亲属\uline{罗特}和他的财物,以及妇女和人民都夺回来了。
\footnote{本段所记一定是根据了一段很古的文献。从前的史学家多以为\uline{阿默}\uline{辣斐耳}是古\uline{巴比伦}著名的帝王\uline{哈慕辣彼};但近来的史家已不主张此说。史家都以为本章所述的史事,很适合公元前十九世纪近东各国的情况。\uwave{创}的作者把此段插入本章,愿意显示\uline{亚巴郎}在\uline{客纳罕}已蒙受了天主的祝福,成为一位有力量的酋长;愿说明他对\uline{罗特}的宽宏大量,尤其愿说明他要成为众信友之父,同未来真宗教的中心\uline{耶路撒冷}所有的关系。为了这些原故,他受了\uline{耶路撒冷}王兼司祭的祝福。}

\textbf{默基瑟德祝福亚巴郎 }
\textsuperscript{17}
\uline{亚巴郎}击败\uline{革多尔}\uline{老默尔}和与他联盟的王子回来时,\uline{索多玛}王出来,到沙委山谷,即“君王山谷”迎接他。
\textsuperscript{18}
\uline{撒冷}王\uline{默基}\uline{瑟德}也带了饼酒来,他是至高者天主的司祭,
\footnote{\uline{撒冷}即\uline{耶路撒冷}(\uwave{咏}76:3,110:4)。\uline{默基}\uline{瑟德}虽不属于\uline{亚巴郎}的家族,但和他有同样的信仰。身兼君王和司祭的\uline{默基}\uline{瑟德}是为君王为司祭的\uline{默西亚}的预象,因他所献的饼酒预兆了新约的圣经祭献(\uwave{希}7:1-28)。}
\textsuperscript{19}
祝福他说:“愿\uline{亚巴郎}蒙受天地的主宰,至高者天主的祝福!
\textsuperscript{20}
愿将你的敌人交于你手中的至高者的天主受赞美!”\uline{亚巴郎}遂将所得的,拿出十分之一,给了\uline{默基}\uline{瑟德}。
\textsuperscript{21}
\uline{索多玛}王对\uline{亚巴郎}说:“请你将人交给我,财物你都拿去罢!”
\textsuperscript{22}
\uline{亚巴郎}却对\uline{索多玛}王说:“我向上天、至高者天主、天地的主宰举手起誓:
\textsuperscript{23}
凡属于你的,连一根线,一根鞋带,我也不拿,免得你说:我使\uline{亚巴郎}发了财。
\textsuperscript{24}
除了仆从吃用了的以外,我什么也不要;至于与我同行的人\uline{阿乃尔}、\uline{厄市}\uline{苛耳}和\uline{玛默勒}所应得的一分,应让他们拿去。”

\textbf{第十五章 }
\textbf{上主与亚巴郎立约 }
\textsuperscript{1}
这些事以后,有上主的话在神视中对\uline{亚巴郎}说:“\uline{亚巴郎},你不要怕,我是你的盾牌;你得的报酬比很丰厚!”
\textsuperscript{2}
\uline{亚巴郎}说:“我主上主!你能给我什么?我一直没有儿子;继承我家业的是\uline{大马}\uline{士革}人\uline{厄里}\uline{厄则尔}。”
\textsuperscript{3}
\uline{亚巴郎}又说:“你既没有赐给我后裔,那么只有一个家仆来作我的承继人。”
\textsuperscript{4}
有上主的话答复他说:“这人决不会是你的承继人,而是你亲生的要做你的承继人。”
\textsuperscript{5}
上主遂领他到外面说:“请你仰观苍天,数点星辰,你能够数清吗?”继而对他说:“你的后裔也将这样。”
\textsuperscript{6}
\uline{亚巴郎}相信了上主,上主就以此算为他的正义。
\textsuperscript{7}
上主又对他说:“我是上主,我从\uline{加色丁}人的\uline{乌尔}领你出来,是为将这地赐给你作为产业。”
\textsuperscript{8}
\uline{亚巴郎}说:“我主上主!我如何知道我要占有此地为产业?”
\textsuperscript{9}
上主对他说:“你给我拿来一只三岁的母牛,一只三岁的母山羊,一只三岁的公绵羊,一只斑鸠和一只雏鸽。”
\textsuperscript{10}
\uline{亚巴郎}便把这一切拿了来,每样从中剖开,将一半与另一半相对排列,只有飞鸟没有剖开。
\textsuperscript{11}
有鸷鸟落在兽尸上,\uline{亚巴郎}就把它们赶走。
\footnote{天主早已应许\uline{亚巴郎}子孙有如繁星,要将\uline{客纳罕}地赐给他们(13:14-17);但他至今无子,又到处漂泊,天主遂又重复所许;他纵已年老,妻子又不生育,还是坚信了天主的话,因此天主使他成义(\uwave{罗}4;\uwave{迦}3:6-9;\uwave{希}11:11-12;\uwave{雅}2:23)。又为嘉许他的信德,就同他立了约。这约不是双方的结盟,而是天主单方面的恩赐。天主立约所规定的献祭,似乎是古时立约的仪式(\uwave{耶}34:18)。}
\textsuperscript{12}
太阳快要西落时,\uline{亚巴郎}昏沉地睡去,忽觉阴森万分,遂害怕起来。
\textsuperscript{13}
上主对\uline{亚巴郎}说:“你当知道,你的后裔必要寄居在异邦,受人奴役虐待四百年之久。
\textsuperscript{14}
但是,我要亲自惩罚他们所要服事的民族;如此你的后裔必要带着丰富的财物由那里出来。
\textsuperscript{15}
至于你,你要享受高寿,以后平安回到你列祖那里,被人埋葬。
\textsuperscript{16}
到了第四代,他们必要回到这里,因为\uline{阿摩黎}人的罪恶至今尚未满贯。”
\footnote{此预言叫\uline{亚巴郎}明白,向他的子孙预许的不只是幸福,而也有苦难。“四百年”\uwave{出}12:41作“四百三十年”。这种差异是本书作者喜用成数的原故。}
\textsuperscript{17}
当日落天黑的时候,看,有冒烟的火炉和燃着的火炬,由那些肉块间经过。
\footnote{冒烟冒火是天主亲临的表示(\uwave{出}3:2;13:21),保证天主既许必践。}
\textsuperscript{18}
在这一天,上主与\uline{亚巴郎}立约说:“我要赐给你后裔的这土地,是从\uline{埃及}河直到\uline{幼发拉的}河,
\textsuperscript{19}
就是\uline{刻尼}人、\uline{刻纳次}人、\uline{卡德}\uline{摩尼}人、
\textsuperscript{20}
\uline{赫特}人、\uline{培黎齐}人、\uline{勒法因}人、
\textsuperscript{21}
\uline{阿摩黎}人、\uline{客纳罕}人、\uline{基尔加}\uline{士}人和\uline{耶步斯}人的土地。”

\textbf{第十六章 }
\textbf{依市玛耳诞生 }
\textsuperscript{1}
\uline{亚巴郎}的妻子\uline{撒辣依},没有给他生孩子,她有个\uline{埃及}婢女,名叫\uline{哈加尔}。
\textsuperscript{2}
\uline{撒辣依}就对\uline{亚巴郎}说:“请看,上主即使我不能生育,你可去亲近我的婢女,或许我能由她得到孩子。”\uline{亚巴郎}就听了\uline{撒辣依}的话。
\footnote{\uline{撒辣依}把婢女给\uline{亚巴郎}作妾生子一事,原是近东古俗(30:1-6,9-31),参见《哈慕辣彼法典》。}
\textsuperscript{3}
\uline{亚巴郎}住在\uline{客纳罕}地十年后,\uline{亚巴郎}的妻子\uline{撒辣依}将自己的\uline{埃及}婢女\uline{哈加尔},给了丈夫\uline{亚巴郎}做妾。
\textsuperscript{4}
\uline{亚巴郎}自从同\uline{哈加尔}亲近,\uline{哈加尔}就怀了孕;她见自己怀了孕;就看不起自己的主母。
\textsuperscript{5}
\uline{撒辣依}对\uline{亚巴郎}说:“我受羞辱是你的过错。我将我的婢女放在你怀里,她一见自己怀了孕,便看不起我。愿上主在我与你之间来判断!”
\textsuperscript{6}
\uline{亚巴郎}对\uline{撒辣依}说:“你的婢女是在你手中;你看怎么好,就怎样待她吧!”于是\uline{撒辣依}就虐待她,她便由\uline{撒辣依}面前逃跑了。
\textsuperscript{7}
上主的使者在旷野的水泉旁,即在往\uline{叔尔}道上的水泉旁,遇见了她,
\textsuperscript{8}
对她说:“\uline{撒辣依}的婢女\uline{哈加尔}!你从哪里来,要往哪里去?”她答说:“我由我主母\uline{撒辣依}那里逃出来的。”
\textsuperscript{9}
上主的使者对她说:“你要回到你主母那里,屈服在她手下。”
\textsuperscript{10}
上主的使者又对她说:“我要使你的后裔繁衍,多得不可胜数。”
\textsuperscript{11}
上主的使者再对她说:“看,你已怀孕,要生个儿子;要给他起名叫\uline{依市玛耳},因为上主俯听了你的哭诉。
\footnote{“天主的使者”(初见本章内)是指天主自己(16:13;22:11;\uwave{出}3:2;\uwave{民}2:1),人即然不能见到天主,更不能接近,因为人见了天主必死无疑,因此天主为就和人的软弱,籍天使的形象显现。古人的这种思想表示承认天主的超越性。教父多以为“上主的天使”,是天主的圣言,即要降生成人的生子。}
\textsuperscript{12}
他将来为人,像头野驴;他要反对众人,众人也要反对他;他要冲着自己的众兄弟支搭帐幕。”
\textsuperscript{13}
\uline{哈加尔}遂给那对她说话的上主起名叫“你是看顾人的天主”,因为她说:“我不是也看见了那看顾人的天主吗?”
\footnote{“野驴”的比喻直指\uline{依市玛耳},但也暗示他强悍的后裔\uline{阿剌伯}人。}
\textsuperscript{14}
为此她给那井起名叫\uline{拉海}\uline{洛依}井。这井是在\uline{卡德士}与\uline{贝勒得}之间。
\textsuperscript{15}
\uline{哈加尔}给\uline{亚巴郎}生了一个儿子,\uline{亚巴郎}给\uline{哈加尔}所生的儿子,起名叫\uline{依市玛耳}。
\textsuperscript{16}
\uline{哈加尔}给\uline{亚巴郎}生\uline{依市玛耳}时,\uline{亚巴郎}已八十六岁。

\textbf{第十七章 }
\textbf{立割损礼 }
\textsuperscript{1}
\uline{亚巴郎}九十九岁时,上主显视给他,对他说:“我是全能的天主,你当在我面前行走,作个成全的人。
\textsuperscript{2}
我要与你立约,使你极其繁盛。”
\textsuperscript{3}
\uline{亚巴郎}遂俯伏在地;天主又对他说:
\textsuperscript{4}
“看,是我与你立约:你要成为万民之父;
\textsuperscript{5}
以后,你不再叫做\uline{亚巴郎},要叫做\uline{亚巴辣罕},因为我已立定你为万民之父,
\textsuperscript{6}
使你极其繁衍,成为一大民族,君王要由你而出。
\textsuperscript{7}
我要在我与你和你历代后裔之间,订立我的约,当作永久的约,就是我要做你和你后裔的天主。
\textsuperscript{8}
我必将你现今侨居之地,即\uline{客纳罕}全地,赐给你和你的后裔做永久的产业;我要作他们的天主。”
\textsuperscript{9}
天主又对\uline{亚巴郎}说:“你和你的后裔,世世代代应遵守我的约。
\footnote{天主更改\uline{亚巴郎}和他妻子的名字(5和15节),表示喊他们名字的,要记得他们是万民真福之源,万民都要因着\uline{亚巴郎}获得祝福(12:1-3;\uwave{依}51:2)从本章6节\uline{亚巴郎}一名应作“亚贝辣罕(亚巴辣罕)”,但因\uline{亚巴郎}一名已习用,故不改。}
\textsuperscript{10}
这就是你们应遵守的,在我与你们以及你的后裔之间所立的约:你们中所有的男子都应受割损。
\textsuperscript{11}
你们都应割去肉体上的包皮,作为我与你们之间的盟约的标记。
\textsuperscript{12}
你们中世世代代所有的男子,在生后八日都应受割损;连家中生的,或是用钱买来而不属你种族的外方人,都应受割损。
\textsuperscript{13}
凡在你家中生的,和你用钱买来的奴仆,都该受割损。这样,我的约刻在你们肉体上作为永久的约。
\textsuperscript{14}
凡未割去包皮,未受割损的男子,应由民间铲除;因他违犯了我的约。”
\footnote{割损礼在\uline{亚巴郎}之前已为许多民族所通行;当时此礼只表示男子已成人,可以结婚。但天主给\uline{亚巴郎}和他的后代子孙所制定的,是男孩生后第八日应行割损。此割损礼为\uline{以色列}人具有一种宗教的深意:即籍此礼\uline{以色列}的男子加入天主的选民,从此他们有遵守法律的义务(\uwave{出}12:44;\uwave{肋}12:3;\uwave{苏}5:2-8)。本章前段所提的盟约,并非真正的盟约,只是天主赐恩的好意。连割损也不是盟约,而是天主定的章程,作为盟约的标记。\uwave{以色列}人多以为受了割损礼,即成了\uline{亚巴郎}之子孙,比受祝福。因此\uline{梅瑟}和先知常强调心灵的割损(\uwave{申}10:12-22;\uwave{肋}26:41;\uwave{耶}4:4,9:24-26;\uwave{则}44:7)。新约以割损为圣洗的预象(\uwave{斐}3:3;\uwave{哥}2:11;\uwave{伯}前3:21)。}

\textbf{应许生子 }
\textsuperscript{15}
天主又对\uline{亚巴郎}说:“你的妻子\uline{撒辣依},你要再叫她\uline{撒辣依},而要叫她\uline{撒辣}。
\textsuperscript{16}
我必要祝福她,使她也给你生个儿子。我要祝福她,使她成为一大民族,人民的君王要由她而生。”
\textsuperscript{17}
\uline{亚巴郎}遂俯伏在地笑起来,心想:“百岁的人还能生子吗?\uline{撒辣}已九十岁,还能生子?”
\textsuperscript{18}
\uline{亚巴郎}对天主说:“只望\uline{依市玛耳}在你面前生存就够了!”
\textsuperscript{19}
天主说:“你的妻子\uline{撒辣}确要给你生个儿子,你要给他起名叫\uline{依撒格};我要与他和他的后裔,订立我的约当作永久的约。
\textsuperscript{20}
至于\uline{依市玛耳},我也听从你;我要祝福他,使他繁衍,极其昌盛。他要生十二个族长,我要使他成为一大民族。
\textsuperscript{21}
但是我的约,我要与明年此时\uline{撒辣}给你生的\uline{依撒格}订立。”
\textsuperscript{22}
天主同\uline{亚巴郎}说完话,就离开他上升去了。
\footnote{从此圣经少记\uline{依市玛耳}而多叙\uline{依撒格}的事,更证明本书的体例以选民为正统,不涉及选民以外的事。见8章注一。}

\textbf{亚巴郎全家男子受割损 }
\textsuperscript{23}
当天,\uline{亚巴郎}就照天主所吩咐的,召集他的儿子\uline{依市玛耳}以及凡家中生的,和用钱买来的奴仆,即自己家中的一切男子,割去了他们肉体上的包皮。
\textsuperscript{24}
\uline{亚巴郎}受割损时,已九十九岁;
\textsuperscript{25}
他的儿子,\uline{依市玛耳}受割损时,是十三岁。
\textsuperscript{26}
\uline{亚巴郎}和他的儿子\uline{依市玛耳}在同日上受了割损。
\textsuperscript{27}
他家中所有的男人,不论是家中生的,或是由外方人那里用钱买来的奴仆,都与他一同受了割损。

\textbf{第十八章 }
\textbf{天主显现给亚巴郎 }
\textsuperscript{1}
天正热的时候,\uline{亚巴郎}坐在帐幕门口,上主在\uline{玛默勒}橡树林那里,给他显现出来。
\textsuperscript{2}
他举目一望,见有三人站在对面。他一见就由帐幕门口跑去迎接他们,俯伏在地,
\textsuperscript{3}
说:“我主如果我蒙你垂爱,请你不要由你仆人这里走过去,
\textsuperscript{4}
我叫人拿点水来,洗洗你们的脚,然后在树下休息休息。
\textsuperscript{5}
你们既然路过你仆人这里,等我拿点饼来,吃点点心,然后再走。”他们答说:“就照你所说的做吧!”
\textsuperscript{6}
\uline{亚巴郎}赶快进入帐幕,到\uline{撒辣}前说:“你快拿三斗细面,和一和,作些饼。”
\textsuperscript{7}
遂又跑到牛群中,选了一头又嫩又肥的牛犊,交给仆人,要他赶快煮好。
\textsuperscript{8}
\uline{亚巴郎}遂拿了凝乳和牛奶,及预备好了的牛犊,摆在他们面前;他们吃的时候,自己在树下侍候。
\footnote{本段所述似乎是\uline{亚巴郎}见的一种神视。三位旅客中有一位是天主(17节),另两位为天使(19:1)。教父多以这三位为天主圣三的预像。由天主预许\uline{撒辣}生子的事,证明天主的全能和他任意的召选。}
\textsuperscript{9}
他们对他说:“你的妻子\uline{撒辣}在哪里?”他答说:“在帐幕里。”
\textsuperscript{10}
其中一位说:“明年此时我比回到你这里,那时,你的妻子\uline{撒辣}要有一个儿子。”\uline{撒辣}其时正在那人背后的帐幕门口窃听。
\textsuperscript{11}
\uline{亚巴郎}和\uline{撒辣}都已年老,年纪很大,而且\uline{撒辣}的月经早已停止。
\textsuperscript{12}
\uline{撒辣}遂心里窃笑说:“现在,我已衰老,同我年老的丈夫,还有这喜事吗?”
\textsuperscript{13}
上主对\uline{亚巴郎}说:“\uline{撒辣}为什么笑?且说:像我这样老,真的还能生育?
\textsuperscript{14}
为上主岂有难事?明年这时,我必要回到你这里,那时\uline{撒辣}必有一个儿子。”
\textsuperscript{15}
\uline{撒辣}害怕了,否认说:“我没有笑。”但是那位说:“不,你实在笑了。”
\footnote{论述\uline{撒辣}老年生子的事,亦见于\uwave{希}11:11;\uwave{罗}4:19-21。}

\textbf{亚巴郎求情 }
\textsuperscript{16}
后来那三人由那里起身,望着\uline{索多玛}前行;\uline{亚巴郎}送他们,也一同前行。
\textsuperscript{17}
上主说:“我要作的事,岂能瞒着\uline{亚巴郎}?
\textsuperscript{18}
因为他要成为一强大而又兴盛的民族,地上所有的民族,都要因他蒙受祝福;
\textsuperscript{19}
何况我拣选了他,是要他训令自己的子孙和未来的家族,保持上主的正道,实行公义正道,好使上主能实现他对\uline{亚巴郎}所许的事。”
\textsuperscript{20}
上主于是说:“控告\uline{索多玛}和\uline{哈摩辣}的声音实在很大,他们的罪恶实在深重!
\textsuperscript{21}
我要下去看看,愿意知道:是否他们所行的全如达到我前的呼声一样。”
\textsuperscript{22}
三人中有二人转身向\uline{索多玛}走去:\uline{亚巴郎}却仍立在上主面前。
\footnote{因\uline{亚巴郎}能“立在上主面前”,故被人称为“天主的朋友”。由此可知他在天主前转求的能力是多么大。}
\textsuperscript{23}
\uline{亚巴郎}近前来说:“你真要将义人同恶人一起消灭吗?
\textsuperscript{24}
假如城中有五十个义人,你还要消灭吗?不为其中的那五十个义人,赦免那地方吗?
\textsuperscript{25}
你决不能如此行事,将义人同恶人一并诛灭;将义人如恶人一样看待,你决不能!审判全地的主,岂能不行公义?”
\textsuperscript{26}
上主答说:“假如我在\uline{索多玛}城中找出了五十个义人,为了他们我要赦免整个地方。”
\textsuperscript{27}
\uline{亚巴郎}接着说:
\textsuperscript{28}
“我虽只是尘埃灰土,胆敢再对我主说:假如五十个义人中少了五个怎样?你就为了少五个而毁灭全城吗?”他答说:“假如我在那里找到四十五个,我不毁灭。”
\textsuperscript{29}
\uline{亚巴郎}再向他进言说:“假如在那里找到四十个怎样?”他答说:“为了这四十个我也不做这事。”
\textsuperscript{30}
\uline{亚巴郎}说:“求我主且勿动怒,容我再进一言:假如在那里找到三十个怎样?”他答说:“假如在那里我找到三十个,我也不做这事。”
\textsuperscript{31}
\uline{亚巴郎}说:“我再放胆对我主进一言:假如在那里找到二十个怎样?”他答说:“为了这二十个,我也不毁灭。”
\textsuperscript{32}
\uline{亚巴郎}说:“求我主且勿动怒,容我最后一次进言:假如在那里找到十个怎么?”他答说:“为了这十个我也不毁灭。”
\textsuperscript{33}
上主向\uline{亚巴郎}说完话就走了;亚巴郎也回家去了。
\footnote{由天主同\uline{亚巴郎}之间的对话,证明圣者的祈祷有多大的效能;此外尚有\uline{梅瑟}(\uwave{出}17:11;\uwave{户}21:7)、\uline{厄里亚}(\uwave{列}上18:36)、\uline{亚毛斯}(7:1)、\uline{耶肋米亚}(14:19,37:3,42:2;\uwave{加}下15:12-16)。圣教会依赖圣人的转达即基于此,见\uwave{雅}5:16-18;\uwave{宗}7:60。}

\textbf{第十九章 }
\textbf{索多玛的邪行 }
\textsuperscript{1}
黄昏时,那两位使者来到了\uline{索多玛},\uline{罗特}正坐在\uline{索多玛}城门口,他一看见他们,就起身前去迎接,俯伏在地,
\textsuperscript{2}
说:“请二位先生下到你们仆人家中住一夜,洗洗你们的脚;明天早起,再赶你们的路”。他们答说:“不,我们愿在街上过宿。”
\textsuperscript{3}
但因\uline{罗特}极力邀请,他们才转身跟着进了他的家。\uline{罗特}为他们备办了宴席,烤了无酵饼;他们就吃了。
\textsuperscript{4}
他们尚未就寝,合城的人,即\uline{索多玛}男人,不论年轻年老的,没有例外,全都来围住他的家,
\textsuperscript{5}
向\uline{罗特}喊说:“今晚来到你这里的那两个男人在哪里?给我们领出来,叫我们好认识他们。”
\textsuperscript{6}
\uline{罗特}就出来,随手关上门,到门口见他们,
\textsuperscript{7}
说:“我的弟兄们!请你们切不可作恶。
\textsuperscript{8}
看,我有两个女儿,尚未认识过男人,容我领出她们来,任凭你们对待她们;只是这两个男人,既然来到舍下,请你们不要对他们行事。”
\textsuperscript{9}
他们反说:“滚开!”继而说:“来这里的这个外方人,居然做起判官来!现在我们待你比他们还要厉害。”他们遂用力向\uline{罗特}冲去,一齐向前要打破那门。
\textsuperscript{10}
那两个人却伸出手来,将\uline{罗特}拉进屋内,关上了门;
\textsuperscript{11}
又使那些在屋门口的男人,无论大小都迷了眼,找不着门口。
\textsuperscript{12}
那两个人对\uline{罗特}说:“你这里还有什么人?带你的女婿、儿女,以及城中你所有的人,离开这地方,
\textsuperscript{13}
因为我们要毁灭这地方;由于在上主面前控告他们的声音实在大,所以上主派遣我们来毁灭这地方。”
\textsuperscript{14}
\uline{罗特}遂出去,告诉要娶他女儿的两个女婿说:“起来,离开这地方,因为上主要毁灭这城!”但他的两个女婿却以为他在开玩笑。
\footnote{此段述\uline{罗特}优待旅客胜过保护女儿贞洁之事。他住在恶劣的环境中,勉作义人实在难得(\uwave{伯}后2:6-9)。\uline{索多玛}人所犯的是逆性男色之罪(\uwave{肋}18:22;20:13、23;\uwave{罗}1:26、27)。圣经常把严罚五城之事,提出来警告后人(\uwave{申}29:22;\uwave{依}1:9,13:19;\uwave{耶}49:18,50:40;\uwave{亚}4:11;\uwave{智}10:6、7;\uwave{玛}10:15,11:23、24;\uwave{路}17:28;\uwave{伯}后2:6;\uwave{犹}7)。}

\textbf{索多玛受罚 }
\textsuperscript{15}
天一亮,两位使者就催促\uline{罗特}说:“起来,快领你的妻子和你这里的两个女儿逃走,免得你因这城的罪恶同遭灭亡。”
\textsuperscript{16}
\uline{罗特}仍迟延不走;但因为上主怜恤他,那两个人就拉着他的手,他妻子的和他两个女儿的手领出城外。
\textsuperscript{17}
二人领他们到城外,其中一个说:“快快逃命,不要往后看,也不要在平原任何地方站住;该逃往山中,免得同遭灭亡。”
\textsuperscript{18}
\uline{罗特}对他们说:“啊!不,我主!
\textsuperscript{19}
你仆人即蒙你垂爱,又蒙你大显仁慈,得以保全我的性命;但是现在我不能逃往山中,怕遇见灾祸死去。
\textsuperscript{20}
看这座城很近,容我逃往那里,那只是一座小城;请容我逃往那里;那不只是一座小城吗?在那里我可保全性命。”
\textsuperscript{21}
其中一个对他说:“好吧!连在这事上我也顾全你的脸面,我必不消灭你提及的这座城。
\textsuperscript{22}
你赶快逃往那里,因为你不到那里,我不能行事。”为此那城称为\uline{左哈尔}。
\textsuperscript{23}
\uline{罗特}到了\uline{左哈尔}时,太阳已升出地面;
\textsuperscript{24}
上主遂使硫磺和火,从天上上主那里,降于\uline{索多玛}和\uline{哈摩辣},
\textsuperscript{25}
毁灭了这几座城市和整个平原,以及城中所有的居民和地上的草木。
\textsuperscript{26}
\uline{罗特}的妻子因回头观看,立即变为盐柱。
\footnote{除\uline{索多玛}及\uline{哈摩辣}两城外,同时被毁者尚有\uline{阿得玛}和\uline{责波殷}(14:2;\uwave{申}29:22;\uwave{欧}11:8);\uline{左哈尔}因\uline{罗特}逃入得免。}

\textsuperscript{27}
\uline{亚巴郎}清晨起来,到了以前他立在上主面前的地方,
\textsuperscript{28}
向\uline{索多玛}和\uline{哈摩辣},以及整个平原一带眺望,看见那地烟火上腾,有如烧窑一般。
\textsuperscript{29}
当天主毁灭平原诸城,消灭\uline{罗特}所住的城市时,想起了\uline{亚巴郎},由灭亡中救了\uline{罗特}。

\textbf{摩阿布和阿孟二族的由来 }
\textsuperscript{30}
\uline{罗特}因为怕住在\uline{左哈尔},便与他的两个女儿离开\uline{左哈尔},上了山住在那里;他和两个女儿同住在一个山洞里。
\textsuperscript{31}
长女对幼女说:“我们的父亲已经老了,地上又没有男人依照世界上的礼俗来与我们亲近。
\textsuperscript{32}
来让我们用酒将父亲灌醉,与他同睡:这样我们可由父亲传生后代。”
\textsuperscript{33}
那天夜里,她们就用酒将父亲灌醉;长女遂进去用酒将父亲灌醉;长女遂进去与父亲同睡;女儿几时卧下,几时起来,他都不知道。
\textsuperscript{34}
第二天,长女对幼女说:“看,昨夜我与父亲同睡了,今夜我们再用酒灌醉他,你好进去与他同睡,由我们的父亲传生后代。”
\textsuperscript{35}
那天夜里,她们又用酒将父亲灌醉,幼女遂进去与他同睡;女儿几时卧下,几时起来,他都不知道。
\textsuperscript{36}
这样,\uline{罗特}的两个女儿都由父亲怀了孕。
\textsuperscript{37}
长女生了一个儿子,给他起名叫\uline{摩阿布},是现今\uline{摩阿布}人的始祖;
\textsuperscript{38}
幼女也生了一个儿子,给他起名叫\uline{本阿米},是现今\uline{阿孟}子民的始祖。
\footnote{作者为避免\uline{摩阿布}和\uline{阿孟}人因\uline{罗特}同\uline{亚巴郎}的亲属关系,而自夸为\uline{以色列}人的亲属,遂插入此段乱伦的事,说明他们的祖宗既违背了伦常(\uwave{肋}18:6-18),与\uline{亚巴郎}的亲戚关系,毫无荣誉可言。}

\textbf{第二十章 }
\textbf{亚巴郎迁往革辣尔 }
\textsuperscript{1}
\uline{亚巴郎}从那里迁往\uline{乃革布}地,定居在\uline{卡德士}和\uline{叔尔}之间。当他住在\uline{革辣尔}时,
\textsuperscript{2}
\uline{亚巴郎}一论到他的妻子\uline{撒辣}就说:“她是我的妹妹。”\uline{革辣尔}王\uline{阿彼默}\uline{肋客}于是派人来娶了\uline{撒辣}去。
\textsuperscript{3}
但是夜间,天主在梦中来对\uline{阿彼默}\uline{肋客}说:“为了你娶来的那个女人你该死,因为她原是有夫之妇。”
\textsuperscript{4}
\uline{阿彼默}\uline{肋客}尚未接近她,于是说:“我主!连正义的人你也杀害吗?
\textsuperscript{5}
那男人不是对我说过‘她是我的妹妹’吗?连她自己也说‘他是我的哥哥’。我做了这事,是出于心正手洁呀!”
\textsuperscript{6}
天主在梦中对他说:“我也知道,你是出于心正做了这事,所以我阻止了你犯罪得罪我,也没有让你接触她。
\textsuperscript{7}
现在你应将女人还给那人,因为他是一位先知,他要为你转求,你才可生存;倘若你不归还,你该知道:你以及凡属于你的,必死无疑。”
\footnote{本章所记与12:10-20相似,但不能证明这两种记载是一回事。古时既有抢妻杀夫的恶习,\uline{亚巴郎}遂同\uline{撒辣}有此约定,好保全性命(13节)。天主籍梦警告或指示世人,常见于本书,如15:12,28:12,31:24,37:5,40,41。称\uline{亚巴郎}为先知(7节),说明他是天主与人间的中保,也说明他同天主的密切关系。}

\textsuperscript{8}
\uline{阿彼默}\uline{肋客}很早就起来召集了众臣仆,将全部实情告诉给他们听;这些人都很害怕。
\textsuperscript{9}
然后\uline{阿彼默}\uline{肋客}叫了\uline{亚巴郎}来,对他说:“你对我们作的是什么事?我在什么事上得罪了你,竟给我和我的王国招来了这样大的罪过?你对我作了不应该作的事。”
\textsuperscript{10}
\uline{阿彼默}\uline{肋客}继而对\uline{亚巴郎}说:“你作这事,究有什么意思?”
\textsuperscript{11}
\uline{亚巴郎}答说:“我以为在这地方一定没有人敬畏天主,人会为了我妻子的缘故杀害我。
\textsuperscript{12}
何况她实在是我的妹妹,虽不是我母亲的女儿,却是我父亲的女儿;后来做了我的妻子。
\textsuperscript{13}
当天主叫我离开父家,在外漂流的时候,我对她说:我们无论到什么地方,你要说我是你的哥哥,这就是你待我的大恩。”
\footnote{同父异母的兄妹结婚的事,\uline{亚巴郎}时代尚许可,但以后却为法律所禁止(\uwave{肋}18:9-11;\uwave{申}27:22;\uwave{则}22:11)。}

\textsuperscript{14}
\uline{阿彼默}\uline{肋客}把些牛羊奴婢,送给了\uline{亚巴郎},也将他的妻子\uline{撒辣}归还了他;
\textsuperscript{15}
然后对他说:“看,我的土地尽在你面前,你愿住在哪里,就住在哪里。”
\textsuperscript{16}
继而对\uline{撒辣}说:“看,我给了你哥哥一千银子,作为你在阖家人前的遮羞钱;这样,各方面无可指摘。”
\textsuperscript{17}
\uline{亚巴郎}恳求了天主,天主就医好了\uline{阿彼默}\uline{肋客},他的妻子和他的众婢女,使她们都能生育,
\textsuperscript{18}
因为上主为了\uline{亚巴郎}妻子\uline{撒辣}的事,关闭了\uline{阿彼默}\uline{肋客}家中所有妇女的子宫。
\footnote{王给\uline{撒辣}的巨款,是为赔偿她名誉的损失。}

\textbf{第二十一章 }
\textbf{依撒格诞生 }
\textsuperscript{1}
上主照所许的,眷顾了\uline{撒辣};上主对\uline{撒辣}实践了他所说的话。
\textsuperscript{2}
\uline{撒辣}怀了孕,在天主所许的时期,给年老的\uline{亚巴郎}生了一个儿子。
\textsuperscript{3}
\uline{亚巴郎}为\uline{撒辣}给他所生的儿子,起名叫\uline{依撒格}。
\textsuperscript{4}
生后第八天,\uline{亚巴郎}照天主所吩咐的给自己的儿子\uline{依撒格}行了割损。
\textsuperscript{5}
他儿子\uline{依撒格}诞生时,\uline{亚巴郎}正一百岁;
\textsuperscript{6}
为此\uline{撒辣}说:“天主使我笑,凡听见的也要与我一同笑。”
\textsuperscript{7}
又说:“谁能告诉\uline{亚巴郎},\uline{撒辣}要哺养儿子呢?可是我在他老年,却给他生了个儿子。”
\textsuperscript{8}
孩子渐渐长大,断了乳;在\uline{依撒格}断乳的那天,\uline{亚巴郎}摆了盛宴。
\footnote{\uline{依撒格}的诞生是天主所预许的(17:19;18:10-14),\uline{亚巴郎}即遵照天主的命令,第八天给他行了割损(17:12)。}

\textsuperscript{9}
\uline{撒辣}见那\uline{埃及}女人\uline{哈加尔}给\uline{亚巴郎}生的儿子,戏笑自己的儿子\uline{依撒格},
\textsuperscript{10}
就对\uline{亚巴郎}说:“你该赶走这婢女和她的儿子,因为这婢女的儿子不能和我的儿子\uline{依撒格}一同承受家业。”
\textsuperscript{11}
\uline{亚巴郎}为了他这儿子的事很是苦恼。
\textsuperscript{12}
但天主对\uline{亚巴郎}说:“你不必为这孩子和你的婢女苦恼;凡\uline{撒辣}对你说的,你都可听从,因为只有籍\uline{依撒格}你的名才能传后。
\textsuperscript{13}
至于这婢女的儿子,我也要使他成为一大民族,因为他也是你的孩子。”
\footnote{天主所拣选的\uline{亚巴郎}的继承人是\uline{依撒格},而不是\uline{依市玛耳}。对此召选的深意,参见\uwave{迦}4:22-31;\uwave{罗}9:7;\uwave{希}11:17-19。}
\textsuperscript{14}
第二天,\uline{亚巴郎}清早起来,拿了饼和一皮囊水,交给\uline{哈加尔},放在她肩上,又将孩子交给她,打发她走了。她一路前行,在\uline{贝尔}\uline{舍巴}旷野迷失了路。
\textsuperscript{15}
那时皮囊里的水已用尽,她把孩子丢在一堆小树下,
\textsuperscript{16}
自己走开,在约离一箭之远的对面坐下,自言自语说:“我不忍见这孩子夭折。”就坐在对面放声大哭。
\textsuperscript{17}
天主听见孩子啼哭,天主的使者由天上呼唤\uline{哈加尔}说:“\uline{哈加尔}!你有什么事?不要害怕,因为天主已听见孩子在那里的哭声。
\textsuperscript{18}
起来,去扶起孩子来,用手搀着他,因为我要使他成为一大民族。”
\textsuperscript{19}
天主于是开了她的眼,她看见了一口水井,遂去将皮囊灌满了水,给孩子喝。
\textsuperscript{20}
天主与孩子同在,他渐渐长大,住在旷野,成了个弓手。
\textsuperscript{21}
他住在\uline{帕兰}旷野;他母亲由\uline{埃及}地给他娶了一个妻子。
\footnote{\uline{依市玛耳}因是“天主的朋友”\uline{亚巴郎}的儿子,也蒙天主祝福,成了一强大民族(阿剌伯)的始祖。}

\textbf{阿彼默肋客与亚巴郎立约 }
\textsuperscript{22}
那时\uline{阿彼默}\uline{肋客}同他的司令\uline{非苛耳}对\uline{亚巴郎}说:“在你所作的一切事上,天主常与你同在。
\textsuperscript{23}
现在你在这里要指着天主对我起誓:总不要亏待我和我的子子孙孙。我怎样厚待了你,你也要怎样厚待我和你寄居的地方。”
\textsuperscript{24}
\uline{亚巴郎}回答说:“我愿起誓。”
\textsuperscript{25}
\uline{亚巴郎}为了\uline{阿彼默}\uline{肋客}的仆人霸占了一口水井,就责斥\uline{阿彼默}\uline{肋客}。
\textsuperscript{26}
\uline{阿彼默}\uline{肋客}回答说:“我不知道谁作了这事,你也没有通知我,我到今天才听说。”
\textsuperscript{27}
\uline{亚巴郎}遂把牛羊交给\uline{阿彼默}\uline{肋客},二人互相立了约。
\textsuperscript{28}
\uline{亚巴郎}又将羊群中七只母羔羊,另放在一边。
\textsuperscript{29}
\uline{阿彼默}\uline{肋客}问\uline{亚巴郎}说:“将这七只母羔羊另放在一边,请问有什么意思?”
\textsuperscript{30}
他答说:“你要由我手中接受这七只母羔羊,作我挖了这口水井的凭据。”
\textsuperscript{31}
因为他们二人在那里起了誓,为此那地方叫做\uline{贝尔}\uline{舍巴}。
\footnote{\uline{贝尔}\uline{舍巴}即“誓约井”之意。}

\textsuperscript{32}
他们在\uline{贝尔}\uline{舍巴}立了约,\uline{阿彼默}\uline{肋客}同他的司令\uline{非苛耳}起身回\uline{培肋}\uline{舍特}地去了。
\textsuperscript{33}
\uline{亚巴郎}在\uline{贝尔}\uline{舍巴}栽了一株怪柳,在那里呼求了上主,永恒天主的名。
\textsuperscript{34}
\uline{亚巴郎}在\uline{培肋}\uline{舍特}人地内住了许久。

\textbf{第二十二章 }
\textbf{亚巴郎从命献子 }
\textsuperscript{1}
这些事以后,天主试探\uline{亚巴郎}说:“\uline{亚巴郎}!”他答说:“我在这里。”
\textsuperscript{2}
天主说:“带你心爱的独生子\uline{依撒格}往\uline{摩黎雅}地方去,在我所要指给你的一座山上,将他献为全燔祭。”
\textsuperscript{3}
\uline{亚巴郎}次日清早起来,备好驴,带了两个仆人和自己的儿子\uline{依撒格},劈好为全燔祭用的木柴,就起身往天主指给他的地方去了。
\textsuperscript{4}
第三天,\uline{亚巴郎}举目远远看见了那个地方,
\textsuperscript{5}
就对仆人说:“你们同驴在这里等候,我和孩子要到那边去朝拜,以后就回到你们这里来。”
\textsuperscript{6}
\uline{亚巴郎}将为全燔祭用的木柴,放在儿子\uline{依撒格}的肩上,自己手中拿着刀和火,两人一同前行。
\textsuperscript{7}
路上\uline{依撒格}对父亲\uline{亚巴郎}说:“阿爸!”他答说:“我儿,我在这里。”\uline{依撒格}说:“看,这里由火有柴,但是哪里有作全燔祭的羔羊?”
\textsuperscript{8}
\uline{亚巴郎}答说:“我儿!天主自会照料作全燔祭的羔羊。”于是二人再继续一同前行。
\textsuperscript{9}
当他们到了天主指给他的地方,\uline{亚巴郎}便在那里筑了一座祭坛,摆好木柴,将儿子\uline{依撒格}捆好,放在祭坛上的木柴上。
\textsuperscript{10}
\uline{亚巴郎}正伸手举刀要宰献自己的儿子时,
\textsuperscript{11}
上主的使者从天上对他喊说:“\uline{亚巴郎}!\uline{亚巴郎}!”他答说:“我在这里。”
\textsuperscript{12}
使者说:“不可在这孩子身上下手,不要伤害他!我现在知道你实在敬畏天主,因为你为了我竟连你的独生子也不顾惜。”
\textsuperscript{13}
\uline{亚巴郎}举目一望,见有一只公绵羊,两角缠在灌木中,遂前去取了那只公绵羊,代替自己的儿子,献为全燔祭。
\textsuperscript{14}
\uline{亚巴郎}给那地方起名叫:“上主自会照料”。直到今日人还说:“在山上,上主自会照料。”

\textsuperscript{15}
上主的使者由天上又呼喊\uline{亚巴郎}说:
\textsuperscript{16}
“我指自己起誓,——上主的断语,——因为你作了这事,没有顾惜你的独生子,
\textsuperscript{17}
我必多多祝福你,使你的后裔繁多,如天上的星辰,如海边的沙粒。你的后裔必占领他们仇敌的城门;
\textsuperscript{18}
地上万民要因你的后裔蒙受祝福,因为你听从了我的话。”
\textsuperscript{19}
\uline{亚巴郎}回到自己仆人那里,一同起身回了\uline{贝尔}\uline{舍巴},遂住在\uline{贝尔}\uline{舍巴}。
\footnote{这篇感动人心的记录,说明:一、天主试探\uline{亚巴郎},为彰显他的德行;二、斥责当时\uline{客纳罕}人所献人祭之非;三、\uline{亚巴郎}遵命祭杀爱子,和\uline{依撒格}甘心顺命,是天主爱世人交出独子\uline{耶稣},和\uline{耶稣}甘心从命救赎人类的预象(\uwave{希}11:17-19)。\uline{摩黎雅}山按\uwave{编}下3:1的记载,也许就是\uline{撒罗满}在其上修建圣殿的山。14节的意思是谓:天主看到人的急需必加以照顾。}

\textbf{纳曷尔的后代 }
\textsuperscript{20}
这些事以后,有人告诉\uline{亚巴郎}说:\uline{米耳加}也给你的兄弟\uline{纳曷尔}生了儿子:
\textsuperscript{21}
长子\uline{伍兹},他的弟弟\uline{布次}、\uline{阿兰}的父亲\uline{刻慕耳}、
\textsuperscript{22}
\uline{革色得}、\uline{哈左}、\uline{丕耳达士}、\uline{依德拉夫}和\uline{贝突耳};——
\textsuperscript{23}
\uline{贝突耳}生了\uline{黎贝加},——这八人都是\uline{米耳加}给\uline{亚巴郎}的兄弟\uline{纳曷尔}生的儿子。
\textsuperscript{24}
此外他的妾,名叫\uline{勒乌玛}的,给他生了\uline{特巴黑}、\uline{加罕}、\uline{塔哈士}和\uline{玛阿加}。
\footnote{此段记\uline{亚巴郎}兄弟\uline{纳曷尔}的家谱,因为\uline{依撒格}的妻子\uline{黎贝加}出生于他家(24章)。}

\textbf{第二十三章 }
\textbf{撒辣逝世埋葬 }
\textsuperscript{1}
\uline{撒辣}一生的寿数是一百二十七岁。
\textsuperscript{2}
\uline{撒辣}死在\uline{客纳罕}地的\uline{克黎雅特阿尔巴},即\uline{赫贝龙}。\uline{亚巴郎}来举哀哭\uline{撒辣};
\textsuperscript{3}
然后从死者面前起来,对\uline{赫特}人说道:
\textsuperscript{4}
“我在你们中是个外乡侨民,请你们在这里卖给我一块坟地,我好将我的死者移去埋葬。”
\footnote{\uline{赫特}人自公元前二十世纪到十二世纪在\uline{小亚细亚}一带建立了一个大帝国。十四世纪时,帝国的势力达到了\uline{叙利亚}和\uline{客纳罕}。}
\textsuperscript{5}
\uline{赫特}人答复\uline{亚巴郎}说:
\textsuperscript{6}
“先生,请听:你在我们中是天主的宠臣,你可在我们最好的坟地埋葬你的死者,我们没有人会拒绝你,在他的坟地内埋葬你的死者。”
\textsuperscript{7}
\uline{亚巴郎}遂起来,向当地人民\uline{赫特}人下拜,
\textsuperscript{8}
然后对他们说:“如果你们实在愿意我将死者移去埋葬,请你们答应我,为我请求\uline{祚哈尔}的儿子\uline{厄斐龙},
\textsuperscript{9}
要他卖给我他那块田地尽头所有的\uline{玛革}\uline{培拉}山洞;要他按实价在你们面前卖给我,作为私有坟地。”
\textsuperscript{10}
当时,\uline{厄斐龙}也坐在\uline{赫特}人中间。这\uline{赫特}人\uline{厄斐龙}遂在聚于城门口的\uline{赫特}人面前,高声答复\uline{亚巴郎}说:
\textsuperscript{11}
“先生,不要这样。请听我说:我送给你这块田,连其中的山洞,也送给你。我愿在我同族的人民眼前送给你,埋葬你的死者。”
\textsuperscript{12}
\uline{亚巴郎}又在当地人民面前下拜,
\textsuperscript{13}
然后对当地人民高声向\uline{厄斐龙}说:“假如你乐意,请你听我说:我愿给你地价,你收下后,我才在那里埋葬我的死者。”
\textsuperscript{14}
\uline{厄斐龙}答复\uline{亚巴郎}说:
\textsuperscript{15}
“先生,请听我说:一块值四百‘协刻耳’银子的地,在你和我之间,算得什么!你尽管去埋葬你的死者吧!”
\textsuperscript{16}
\uline{亚巴郎}明白了\uline{厄斐龙}的意思,便照他在\uline{赫特}人前大声提出的价值,按流行的市价称了四百‘协刻耳’银子给他。
\textsuperscript{17}
这样,\uline{厄斐龙}在\uline{玛革}\uline{培拉}面对\uline{玛默勒}的那块田地,连田地带其中的山洞,以及在田地四周所有的树木,
\textsuperscript{18}
当着聚在城门口的\uline{赫特}人面前,全移交给\uline{亚巴郎}作产业。
\textsuperscript{19}
事后,\uline{亚巴郎}遂将自己的妻子\uline{撒辣}葬在\uline{客纳罕}地,即葬在那块面对\uline{玛默勒},【即\uline{赫贝龙}】,\uline{玛革}\uline{培拉}田地的山洞内。
\textsuperscript{20}
这样,这块田地和其中的山洞,由\uline{赫特}人移交给\uline{亚巴郎}作为私有坟地。
\footnote{作者详述\uline{亚巴郎}买地一事,有两个原因:一、由于买地\uline{亚巴郎}被承认为\uline{客纳罕}的公民,因此他和自己的后代子孙,除有天主的应许之外,在法律上也获得了公民权。二、对后事的关怀;不但\uline{撒辣}和\uline{亚巴郎}都葬于此,而且\uline{依撒格}、\uline{黎贝加}、\uline{肋阿}和\uline{雅各伯}也都葬于此(25:9、10,35:27-29,49:31-33,50:13)。}

\textbf{第二十四章 }
\textbf{亚巴郎为子娶妻 }
\textsuperscript{1}
\uline{亚巴郎}年纪已老,上主在一切事上常祝福他。
\textsuperscript{2}
\uline{亚巴郎}对管理他所有家产的老仆人说:“请你将手放在我的胯下,
\textsuperscript{3}
要你指着上主、天地的天主起誓:你决不要为我的儿子,由我现住的\uline{客纳罕}人中,娶一个女子为妻;
\textsuperscript{4}
却要到我的故乡,我的亲族中去,为我的儿子\uline{依撒格}娶妻。”
\textsuperscript{5}
仆人对他说:“假使那女子不愿跟我到此地来,我能否带你的儿子回到你的本乡?”
\textsuperscript{6}
\uline{亚巴郎}答复他说:“你切不可带我的儿子回到那里去。
\textsuperscript{7}
那引我出离父家和我生身地,同我谈过话,对我起誓说‘我必将这地赐给你后裔’的上主,上天的天主,必派遣自己的使者作你的前导,领你由那里给我儿子娶个妻子。
\textsuperscript{8}
设若那女子不愿跟你来,你对我起的誓,就与你无涉;无论如何,你不能带我的儿子回到那里去。”
\textsuperscript{9}
仆人遂将手放在主人\uline{亚巴郎}的胯下,为这事向他起了誓。
\textsuperscript{10}
仆人就由他主人的骆驼中,牵了十匹骆驼,带着主人的各样宝物,起身往\uline{美索不}\uline{达米亚}的\uline{纳曷尔}城去了。
\textsuperscript{11}
傍晚,女人们出来打水的时候,他叫骆驼卧在城外的水井旁,
\textsuperscript{12}
然后说:“上主、我主人\uline{亚巴郎}的天主!求你对我主人\uline{亚巴郎}施行仁慈,今日使我幸运。
\textsuperscript{13}
看我站在水泉旁,此时城中的女子正出来打水。
\textsuperscript{14}
我对那个少女说:请你放下水罐,让我喝点水。如果她答说:请喝!并且我还要打水给你的骆驼喝,她即是你为你的仆人\uline{依撒格}预定的少女;由此我知道,你对我主人施行了仁慈。”
\textsuperscript{15}
话还没有说完,\uline{黎贝加}就肩着水罐出来了。她是\uline{亚巴郎}的兄弟\uline{纳曷尔}的妻子\uline{米耳加}的儿子\uline{贝突耳}的女儿。
\textsuperscript{16}
这少女容貌很美,是个还没有人认识的处女。她下到水泉,灌满了水罐,就上去了。
\textsuperscript{17}
仆人就跑上前去迎着她说:“请让我喝点你水罐里的水,好吗!”
\textsuperscript{18}
她回答说:“先生!请喝!”她急忙将水罐放低,托在手上让他喝。
\textsuperscript{19}
他喝足了以后,少女说:“我再为你的骆驼打水,叫它们也喝足。”
\textsuperscript{20}
遂急忙将罐里的水倒在槽里,再跑到那井里去打水,打水给他的骆驼喝。
\textsuperscript{21}
仆人在旁静静地注视她,急愿知道,是否上主已使他此行成功。
\textsuperscript{22}
骆驼喝完了水以后,老人就拿出一个半‘协刻耳’重的金鼻环,和一对重十‘协刻耳’的金手镯,给她戴上,
\textsuperscript{23}
然后说:“请你告诉我你是谁的女儿?你父亲家里,有没有地方可让我们过宿?”
\textsuperscript{24}
她回答说:“我是\uline{米耳加}给\uline{纳曷尔}所生之子\uline{贝突耳}的女儿。”
\textsuperscript{25}
她又继续说:“我们家里有很多草料和饲粮,而且还有地方可供过宿。”
\textsuperscript{26}
老人就俯身朝拜了上主,说:
\textsuperscript{27}
“上主,我主人\uline{亚巴郎}的天主应受赞美!因为他不断以仁慈和忠信善待了我的主人。上主也一路引我来到了我主人的老家。”
\\
\indent
\textsuperscript{28}
那少女跑回去,将这一切事告诉了她母亲家中的人。
\textsuperscript{29-30}
\uline{黎贝加}有个哥哥名叫\uline{拉班},他一看见他妹妹鼻上的金环,和手腕上的金镯,听见他妹妹\uline{黎贝加}说:“那人如此如此对我说。”\uline{拉班}就跑去见那在郊外水泉旁的人,迨他来到那人那里,见他仍站在靠近水泉的骆驼旁,
\textsuperscript{31}
就对他说:“你这受上主祝福的人,请来!我已预备好了房屋和喂骆驼的地方;你为什么还站在郊外?”
\textsuperscript{32}
\uline{拉班}便将那人领进家去,卸了骆驼,喂上草料和饲粮;又拿水给他和与他同来的人洗脚,
\textsuperscript{33}
然后在他面前摆上饭,但仆人却说:“在我未说明我的来意之前,我不吃饭。”\uline{拉班}说:“你说吧!”
\textsuperscript{34}
他说:“我是\uline{亚巴郎}的仆人,
\textsuperscript{35}
上主厚厚地祝福了我的主人,使他十分富有,赐了他羊群、牛群、金银、仆婢、骆驼和驴子。
\textsuperscript{36}
我主人的妻子\uline{撒辣},在老年给我主人生了一个儿子,主人遂将所有的财产给了他。
\textsuperscript{37}
我主人叫我起誓说:你决不要给我的儿子,由我现居地的\uline{客纳罕}人中,娶一个女子为妻。
\textsuperscript{38}
你该到我的父家和我的同族中,为我儿子娶妻。
\textsuperscript{39}
我对我主人说:假使女儿不愿跟我来怎么办?
\textsuperscript{40}
他回答我说:我一向在上主面前行走,他必派遣自己的使者与你同行,使你此行必成功,能由我的同族,我的父家,为我儿子娶妻。
\textsuperscript{41}
只要你去了我同族那里,你就履行了对我起的誓;若是他们不给你,你对我起的誓,就与你无涉。
\textsuperscript{42}
今天我到了水泉那里就说:上主,我主人\uline{亚巴郎}的天主!惟愿你使我此行成功。
\textsuperscript{43}
看我现在站在泉旁,我对那个出来打水的少女说:请你让我喝点你水罐里的水吧!
\textsuperscript{44}
如果她对我说:请喝,并且我还要打水给你的骆驼喝,她即是上主为我主人的儿子预定的妻子。
\textsuperscript{45}
我心里尚未说完这话,看,\uline{黎贝加}肩着水罐来了,下到水泉那里打水,我就对她说:请给我一点水喝!
\textsuperscript{46}
她急忙从肩上放下水罐说:你喝,并且我还要打水给你的骆驼喝。我喝了,同时她也给了骆驼水喝。
\textsuperscript{47}
我于是问她说:你是谁的女儿?她答说:我是\uline{米耳加}给\uline{纳曷尔}生的儿子\uline{贝突耳}的女儿。我就将鼻环戴在她鼻上,将手镯带在她手腕上。
\textsuperscript{48}
然后我俯身朝拜了上主,赞颂了上主、我主人\uline{亚巴郎}的天主,因为他引我走了正路,为我主人的儿子娶了我主人兄弟的孙女。
\textsuperscript{49}
现在,如果你们愿意以仁慈和忠信善待我主人,请告诉我;如果不肯,也请告诉我;我好决定行事。”
\\
\indent
\textsuperscript{50}
\uline{拉班}和\uline{贝突耳}答说:“这件事既是出于上主,我们不能对你说好说坏。
\textsuperscript{51}
看\uline{黎贝加}在你面前,你可带她去做你主人儿子的妻子,如上主所说的。”
\textsuperscript{52}
\uline{亚巴郎}的仆人一听见他们说出这话,就俯伏在地朝拜了上主;
\textsuperscript{53}
然后拿出金银的珍饰和衣服来,送给了\uline{黎贝加},又送给了她的哥哥和她的母亲一些宝贵礼品。
\textsuperscript{54}
这以后,他和同他前来的人才吃喝,并住了一宿。清早起来,他就说:“请让我回到我主人那里去!”
\textsuperscript{55}
\uline{黎贝加}的哥哥和母亲说:“让少女同我们再住上几天或十天,然后走吧!”
\textsuperscript{56}
他回答他们说:“你们不要挽留我,上主既使我此行成功,请你们让我走,回到我主人那里去。”
\textsuperscript{57}
他们说:“我们可叫少女来,问问她的意思。”
\textsuperscript{58}
他们就将\uline{黎贝加}叫来问她说:“你愿意跟这人去吗?”她答说:“愿意。”
\textsuperscript{59}
于是他们打发自己的姊妹\uline{黎贝加}和她的乳母,同\uline{亚巴郎}的仆人和与他同来的人一起走了。
\textsuperscript{60}
他们祝福\uline{黎贝加}说:“我们的妹妹,愿你子孙无数!愿你的后裔,占领仇敌的城门!”
\textsuperscript{61}
\uline{黎贝加}便和自己的婢女们起来,上了骆驼,跟那人去了。仆人便带着\uline{黎贝加}起了程。
\footnote{本章叙事的美妙,可媲美于\uline{若瑟}(37-50章)和\uline{卢德}的故事。此外本章还记述了一些古代的风俗,以及\uline{亚巴郎}和老仆的宗教热诚。此老仆大概即是\uline{厄里}\uline{厄则尔}(15:2)。将手放在胯下是\uline{闪}族人起誓的一种仪式(47:29)。至于\uline{黎贝加}的哥哥\uline{拉班}几乎全权处理了妹妹的婚事,也是当时长子当家主事的风俗,至今有些\uline{阿剌伯}人还是如此。}
\\
\indent
\textbf{依撒格完婚 }
\textsuperscript{62}
那时\uline{依撒格}刚来到\uline{拉海}\uline{洛依}井旁附近,他原住在\uline{乃革布}地方。
\textsuperscript{63}
傍晚时,\uline{依撒格}出来在田间来回沉思,举目一望,看见了一对骆驼。
\textsuperscript{64}
\uline{黎贝加}举目看见了\uline{依撒格},便由骆驼上下来,
\textsuperscript{65}
问仆人说:“田间前来迎接我们的那人是谁?”仆人答说:“是我的主人。”\uline{黎贝加}遂拿面纱蒙在脸上。
\textsuperscript{66}
仆人就将自己所作的一切事,告诉了\uline{依撒格}。
\textsuperscript{67}
\uline{依撒格}便领\uline{黎贝加}进入自己母亲\uline{撒辣}的帐幕,娶了她为妻,很是爱他。\uline{依撒格}自从母亲死后,这才有了安慰。
\\
\indent
\textbf{第二十五章 }
\textbf{亚巴郎的晚年 }
\textsuperscript{1}
\uline{亚巴郎}又续娶了一个妻子,名叫\uline{刻突辣}。
\textsuperscript{2}
\uline{刻突辣}给他生了\uline{齐默郎}、\uline{约刻商}、\uline{默丹}、\uline{米德杨}、\uline{依市}\uline{巴克}和\uline{叔哈}。
\textsuperscript{3}
\uline{约刻商}生了\uline{舍巴}和\uline{德丹};\uline{德丹}的子孙是\uline{阿叔陵}人、\uline{肋突兴}人和\uline{肋乌明}人。
\textsuperscript{4}
\uline{米德杨}的子孙是\uline{厄法}、\uline{厄斐尔}、\uline{哈诺客}、\uline{阿彼达}和\uline{厄耳达阿}:以上都是\uline{刻突辣}的子孙。
\textsuperscript{5}
\uline{亚巴郎}将自己所有的财产都给了\uline{依撒格};
\textsuperscript{6}
至于妾所生的儿子,\uline{亚巴郎}给了他们一些礼物,在自己活着时,就叫他们离开自己的儿子\uline{依撒格},打发他们向东去,住在东方。
\footnote{\uline{亚巴郎}的妾\uline{刻突辣}所生的儿子,虽是\uline{亚巴郎}的儿子,却不是承嗣的儿子。}

\textbf{亚巴郎逝世 }
\textsuperscript{7}
\uline{亚巴郎}一生的岁月共是一百七十五岁。
\textsuperscript{8}
\uline{亚巴郎}寿高年老,已享天年,遂断气而死,归到他亲族那里去了。
\footnote{“归到他亲族那里去了”一语,常见于圣经中,意思是被天主召集到他死去的祖先那里,降到阴间与祖先相聚。此语暗示人灵不死不灭的道理。}
\textsuperscript{9}
他的儿子\uline{依撒格}和\uline{依市玛耳},将他葬在面对\uline{玛默勒},\uline{赫特}人\uline{祚哈尔}之子\uline{厄斐龙}的田间\uline{玛革}\uline{培拉}山洞内。
\textsuperscript{10}
这块田地是\uline{亚巴郎}由\uline{赫特}人那里买来的;\uline{亚巴郎}和妻子\uline{撒辣}都葬在那里。
\textsuperscript{11}
\uline{亚巴郎}死后,天主祝福了他的儿子\uline{依撒格}。\uline{依撒格}定居在\uline{拉海}\uline{洛依}井附近。
\footnote{“\uline{拉海}\uline{洛依}”,见16:13、14。}

\textbf{依市玛耳的后代 }
\textsuperscript{12}
以下是\uline{撒辣}的婢女,\uline{埃及}人\uline{哈加尔}给\uline{亚巴郎}生的\uline{依市玛耳}的后裔。
\textsuperscript{13}
\uline{依市玛耳}的子孙名单,依照出生的次第记载如下:\uline{依市玛耳}的长子是\uline{乃巴约特},其次是\uline{刻达尔}、\uline{阿德贝耳}、\uline{米贝散}、
\textsuperscript{14}
\uline{米市玛}、\uline{杜玛}、\uline{玛萨}、
\textsuperscript{15}
\uline{哈达得}、\uline{特玛}、\uline{耶突尔}、\uline{纳菲士}和\uline{刻德玛}:
\textsuperscript{16}
以上都是\uline{依市玛耳}的儿子,这些人的名字也是他们的村庄和营地的名字,又是十二家族的族长。
\textsuperscript{17}
\uline{依市玛耳}一生的岁月是一百三十七岁;然后断气而死,归到他亲族那里去了。
\textsuperscript{18}
他的子孙住在从\uline{哈威拉}直到\uline{埃及}东面的\uline{叔尔},往\uline{亚述}道上,对着自己的众兄弟支搭帐幕。
\footnote{\uline{依市玛耳}的子孙大概即是日后的\uline{阿剌伯}各民族。}

\textbf{依撒格的后代 }
\textsuperscript{19}
以下是\uline{亚巴郎}的儿子\uline{依撒格}的历史:\uline{亚巴郎}生\uline{依撒格}。
\textsuperscript{20}
\uline{依撒格}四十岁时娶了\uline{帕丹}\uline{阿兰}地\uline{阿兰}人\uline{贝突耳}的女儿,\uline{阿兰}人\uline{拉班}的妹妹\uline{黎贝加}为妻。
\textsuperscript{21}
\uline{依撒格}因为自己的妻子不生育,便为她恳求上主;上主俯允了他的祈求,他的妻子\uline{黎贝加}遂怀了孕,
\textsuperscript{22}
双胎在她腹内互相冲突,于是她说:“若是这样,我可怎么办?”遂去求问上主。
\textsuperscript{23}
上主答复她说:“你一胎怀了两个国家,你腹中所生的要分为两个民族:一个民族强于另一个民族,年长的要服事年幼的。”
\footnote{天主说的话不是指二子,而是指二子后代子孙的命运。}
\textsuperscript{24}
到了生产的时候,她腹内果然是一对双生;
\textsuperscript{25}
首先产出的发红,浑身是毛,如披毛裘,给他起名\uline{厄撒乌}。
\textsuperscript{26}
他的弟弟随后出生,一手握着\uline{厄撒乌}的脚跟,为此给他起名叫\uline{雅各伯}。他们诞生时,\uline{依撒格}已是六十岁的人。

\textbf{厄撒乌出卖长子名分 }
\textsuperscript{27}
两个孩童渐渐长大,\uline{厄撒乌}成了个好打猎的人,喜居户外;\uline{雅各伯}却为人恬静,深居幕内。
\textsuperscript{28}
\uline{依撒格}爱\uline{厄撒乌},因为他爱吃野味;\uline{黎贝加}却爱\uline{雅各伯}。
\textsuperscript{29}
有一天,\uline{雅各伯}正煮好豆羹,\uline{厄撒乌}由田间回来,饥饿疲乏,
\textsuperscript{30}
便对\uline{雅各伯}说:“请将这红红的东西给我点吃,因为我实在又饿又乏。”——因此他的名字又叫“\uline{厄东}”。
\textsuperscript{31}
\uline{雅各伯}回答说:“你要将你长子的名分先卖给我。”
\textsuperscript{32}
\uline{厄撒乌}说:“我快要死了,这长子的名分为我还有什么益处?”
\textsuperscript{33}
\uline{雅各伯}接着说:“你得立刻对我起誓。”\uline{厄撒乌}遂对他起了誓,将自己长子的名分卖给了\uline{雅各伯}。
\textsuperscript{34}
\uline{雅各伯}遂将饼和扁豆羹给了\uline{厄撒乌};他吃了喝了,起身走了。——\uline{厄撒乌}竟如此轻视了长子的名分。
\footnote{关于天主召选\uline{雅各伯}而让\uline{厄撒乌}丧失长子的名分,参阅\uwave{拉}1:2-5;\uwave{罗}9:10-13。关于长子的特权,见\uwave{申}21:17。}
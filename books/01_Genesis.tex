\chapter{创世纪(创)}

\begin{center}
	\textbf{前编 }
	\textbf{太古史(1-11)}
\end{center}

\textbf{第一章 }
\textbf{天地万物的创造 }
\textsuperscript{1}
在起初天主创造了天地。
\textsuperscript{2}  
大地还是混沌空虚,深渊上还是一团黑暗,天主的神在水面上运行。
\renewcommand\thefootnote{\ding{\numexpr171+\value{footnote}}}
\footnote{"在起初……"一语,暗示创造万物之时,除天主外,一无所有。“天地”二字此处有宇宙万物之意。作者用诗人的想象力描写天主好似一个工程师,在六天以内创造了万物,到第七天休息。首先所创造的是混沌的无生之物,后将这混沌之物分成天、地、海三大部分,然后以日月、星辰、草木、飞禽、走兽等来点缀天地海洋。最后天主照自己的肖像造了人。作者从创造混沌之物说起,到创造人,表示人是万物之灵,应效法造物主工作和守安息日。此开宗明义第一章是远古时代的文学杰作,是一篇宗教的重要文告,并不是自然科学的论著。按古代各民族对开天辟地,人类诞生的传说,没有可与《创世纪》第一章相比拟的。“天主的神”指施生命之神力,但若通观新旧二约的全部启示,此处也指赐生命的“天主圣神”。}
\textsuperscript{3}
天主说:“有光!”就有了光。
\textsuperscript{4}
天主见光好,就将光与黑暗分开。
\textsuperscript{5}
天主称光为“昼”,称黑暗为“夜”。过了晚上,过了早晨,这是第一天。

\textsuperscript{6}
天主说:“在水与水之间要有穹苍,将水分开!”事就这样成了。
\textsuperscript{7}
天主造了穹苍,分开了穹苍以下的水和穹苍以上的水。
\textsuperscript{8}
天主称穹苍为“天”,天主看了认为好。过了晚上,过了早晨,这是第二天。

\textsuperscript{9}
天主说:“天下的水应聚在一处,使旱地出现!”事就这样成了。
\textsuperscript{10}
天主称旱地为“陆地”,称水汇合处为“海洋”。天主看了认为好。
\textsuperscript{11}
天主说:“在陆地上,土地要生出青草、结种子的蔬菜和结果子的果树,各按照在它内的种子的种类!”事就这样成了。
\textsuperscript{12}
土地就生出了青草、结种子的蔬菜,各按其类,和结果子的树木,各按照在它内的种子的种类。天主看了认为好。
\textsuperscript{13}
过了晚上,过了早晨,这是第三天。

\textsuperscript{14}
天主说:“在天空中要有光体,以分别昼夜,作为规定时节和年月日的记号。
\textsuperscript{15}
要在天空中放光,照耀大地!”事就这样成了。
\textsuperscript{16}
天主于是造了两个大光体:较大的控制白天,较小的控制黑夜,并造了星宿。
\textsuperscript{17}
天主将星宿摆列在天空,照耀大地,
\textsuperscript{18}
控制昼夜,分别明与暗。天主看了认为好。
\textsuperscript{19}
过了晚上,过了早晨,这是第四天。

\textsuperscript{20}
天主说:“水中要繁生蠕动的生物,地面上、天空中要有鸟飞翔!”事就这样成了。
\textsuperscript{21}
天主于是造了大鱼和所有在水中孳生的蠕动生物,各按其类,以及各种飞鸟,各按其类。天主看了认为好。
\textsuperscript{22}
遂祝福它们说:“你们要孳生繁殖,充满海洋;飞鸟也要在地上繁殖!”
\textsuperscript{23}
过了晚上,过了早晨,这是第五天。

\textsuperscript{24}
天主说:“地上要生出生物,各按其类;走兽、爬虫和地上的各种生物,各按其类!”事就这样成了。
\textsuperscript{25}
天主于是造了地上的生物,各按其类;各种走兽,各按其类;以及地上所有的爬虫,各按其类。天主看了认为好。
\textsuperscript{26}
天主说:“让我们照我们的肖像,按我们的模样造人,叫他管理海中的鱼、天空的飞鸟、牲畜、各种野兽、在地上爬行的各种爬虫。”
\textsuperscript{27}
天主于是照自己的肖像造了人,就是照天主的肖像造了人:造了一男一女。
\textsuperscript{28}
天主祝福他们说:“你们要生育繁殖,充满大地,治理大地,管理海中的鱼、天空的飞鸟、各种在地上爬行的生物!”
\footnote{“人”按原文有红土或黄土的意思,是说人是属于土的造物。“我们”(26节)按古\uline{犹太}经师的解释,是指天主和天使,好似天主同天使商量;但有些学者主张为“威严复数”或“议决复数”。教父和神学家多以为此复数暗示天主圣三的奥理。此说若照启示的演进说是对的。人相似天主是按灵魂说的,相似天主有理智、意志和记忆。论人的肉身,当天主造\uline{亚当}时,已预见作\uline{亚当}第二的\uline{基督}(\uwave{罗}5:14)。“造了一男一女”,指婚姻一夫一妻制和不可分离性(\uwave{玛}19:1-6;\uwave{拉}2:15、16)。天主祝福原祖生育繁殖的话,说明婚姻的首要目的是生养教育子女(8:17;\uwave{咏}127:3、4)。}
\textsuperscript{29}
天主又说:“看,全地面上结种子的各种蔬菜,在果内含有种子的各种果树,我都给你们作食物;
\textsuperscript{30}
至于地上的各种野兽,天空中的各种飞鸟,在地上爬行有生魂的各种动物,我把一切青草给它们作食物。”事就这样成了。
\textsuperscript{31}
天主看了他造的一切,认为样样都很好。过了晚上,过了早晨,这是第六天。
\footnote{天主造了原祖,也赐给了他们和他们传生的人类的食物,并将普世交给他们统治。所造的万物样样都好,是说万物都合天主的旨意,都为他所喜爱。参阅\uwave{咏}19:1-6,104,145,148,150。}

\textbf{第二章 }
\textbf{安息日 }
\textsuperscript{1}
这样,天地和天地间的一切点缀都完成了。
\textsuperscript{2}
到第七天天主造物的工程已完成,就在第七天休息,停止了所作的一切工程。
\textsuperscript{3}
天主祝福了第七天,定为圣日,因为这一天,天主停止了他所行的一切创造工作。
\footnote{1-3节属前章,劝人守安息日为圣日。守安息日的原因与目的,见\uwave{出}23:12;\uwave{申}5:12-15。}

\textbf{人与乐园 }
\textsuperscript{4}
这是创造天地的来历:在上主天主创造天地时,
\textsuperscript{5}
地上还没有灌木,田间也没有生出蔬菜,因为上主天主还没有使雨降在地上,也没有人耕种土地,
\textsuperscript{6}
有从地下涌出的水浸润所有地面。
\textsuperscript{7}
上主天主用地上的灰土形成了人,在他鼻孔内吹了一口生气,人就成了一个有灵的生物。
\textsuperscript{8}
上主天主在\uline{伊甸}东部种植了一个乐园,就将他形成的人安置在里面。
\textsuperscript{9}
上主天主使地面生出各种好看好吃的果树,生命树和知善恶树在乐园中央。
\footnote{2:4-3:24为创造天地万物的另一记载。原来在这记载中只用了天主(雅威)的名词,但将这个记载与上章的记载编在一起,补入了“天主”的名词。这记载的中心为人:天主对人,人对天主的态度。关于人的来历和本性,作者用简略的话,教训人一端论宗教和文化的最高深的道理:人肉身的形成,好像其他的动物,是由尘土造成的,但对于灵魂却有极大的区别,它是直接由天主所造。\uline{伊甸}乐园位于何处,人不得而知。乐园是天主考验人的地方。“生命树”所象征的是天主愿意赐给人的“不死”之恩。“知善恶的树”,是试探人的工具。“知善恶”的意思,大概是说:人一犯天主的禁令,就知道所失去的超性恩宠——真善,是多么美善,所犯的罪恶——真恶,是如何凶恶。}
\textsuperscript{10}
有一条河由\uline{伊甸}流出灌溉乐园,由那里分为四支:
\textsuperscript{11}
第一支名叫\uline{丕雄},环流产金的\uline{哈威拉}全境;
\textsuperscript{12}
那地方的金子很好,那里还产珍珠和玛瑙;
\textsuperscript{13}
第二支河名叫\uline{基红},环流\uline{雇士}全境;
\textsuperscript{14}
第三支河名叫\uline{底格里斯},流入\uline{亚述}东部;第四支河即\uline{幼发拉的}。
\textsuperscript{15}
上主天主将人安置在\uline{伊甸}的乐园内,叫他耕种,看守乐园。
\footnote{说明人犯罪以前,天主已叫人应该工作。}
\textsuperscript{16}
上主天主给人下令说:“乐园中各树上的果子,你都可吃,
\textsuperscript{17}
只有知善恶树上的果子你不可吃,因为那一天你吃了,必定要死。”

\textbf{造女人立婚姻 }
\textsuperscript{18}
上主天主说:“人单独不好,我要给他造个与他相称的助手。”
\textsuperscript{19}
上主天主用尘土造了各种野兽和天空中的各种飞鸟,都引到人面前,看他怎样起名;凡人给生物起的名字,就成了那生物的名字。
\textsuperscript{20}
人遂给各种畜牲、天空中的各种飞鸟和各种野兽起了名字;但他没有找着一个与自己相称的助手。
\textsuperscript{21}
上主天主遂使人熟睡,当他睡着了,就取出了他的一根肋骨,再用肉补满原处。
\textsuperscript{22}
然后上主天主用那由人取来的肋骨,形成了一个女人,引她到人前,
\textsuperscript{23}
人遂说:“这才真是我的骨中之骨,肉中之肉,她应称为“女人”,因为是由男人取出的。”
\textsuperscript{24}
为此人应离开自己的父母,依附自己的妻子,二人成为一体。
\textsuperscript{25}
当时,男女二人都赤身露体,并不害羞。
\footnote{本段的要义有二:一、人给动物命名,是表示人受有统治一切造物的权柄;二、从\uline{亚当}的肉身形成了第一个女人,是指女人同他有一样的人性,像\uline{亚当}一样是照天主的肖像受造的。夫妇结为一体,表示婚姻的结合是天主制定的,人不能拆散(\uwave{玛}19:5、6)。赤身不害羞,是说原祖未犯罪前纯洁无罪的状态,还未体验到罪过的恶果。}

\textbf{第三章 }
\textbf{原祖违命 }
\textsuperscript{1}
在上主天主所造的一切野兽中,蛇是最狡猾的。蛇对女人说:“天主真说了,你们不可吃乐园中任何树上的果子吗?”
\textsuperscript{2}
女人对蛇说:“乐园中树上的果子,我们都可吃;
\textsuperscript{3}
只有乐园中央那颗树上的果子,天主说过,你们不可以吃,也不可摸,免得死亡。”
\textsuperscript{4}
蛇对女人说:“你们决不会死!
\textsuperscript{5}
因为天主知道,你们那天吃了这果子,你们的眼就会开了,将如同天主一样知道善恶。”
\textsuperscript{6}
女人看那棵果树实在好吃好看,令人羡慕,且能增加智慧,遂摘下一个果子吃了,又给了她的男人一个,他也吃了。
\textsuperscript{7}
于是二人的眼立即开了,发觉自己赤身露体,遂用无花果树叶,编了个裙子围身。
\textsuperscript{8}
当\uline{亚当}和他的妻子听见了上主天主趁晚凉在乐园中散步的声音,就躲藏在乐园的树林中,怕见上主天主的面。
\footnote{本章记的蛇就是魔鬼。他籍蛇形诱惑了\uline{厄娃}(\uwave{智}2:23、24;\uwave{若}8:44;\uwave{默}12:9,20:2)。原祖所犯是骄傲背命的罪。“发觉自己赤身”,是指失去天主的宠爱和原始的纯洁。}
\textsuperscript{9}
上主天主呼唤\uline{亚当}对他说:“你在哪里?”
\textsuperscript{10}
他答说:“我在乐园中听到了你的声音,就害怕起来,因为我赤身露体,遂躲藏了。”
\textsuperscript{11}
天主说:“谁告诉了你,赤身露体?莫非你吃了我禁止你吃的果子?”
\textsuperscript{12}
\uline{亚当}说:“是你给我作伴的那个女人给了我那树上的果子,我才吃了。”
\textsuperscript{13}
上主天主遂对女人说:“你为什么作了这事?”女人答说:“是蛇哄骗了我,我才吃了。”
\footnote{天主询问时,没有询问魔鬼,只询问了亚当\uline{亚当} \uline{厄娃};但惩罚时却按罪过的原因和轻重:先是魔鬼,后是\uline{厄娃},最后是\uline{亚当}(\uwave{第}前2:13-15)。}

\textbf{处罚与预许 }
\textsuperscript{14}
上主天主对蛇说:“因你做了这事,你在一切畜牲和野兽中,是可咒骂的;你要用肚子爬行,毕生日日吃土。
\textsuperscript{15}
我要把仇恨放在你和女人,你的后裔和她的后裔之间,她的后裔要踏碎你的头颅,你要伤害他的脚跟。”
\textsuperscript{16}
后对女人说:“我要增加你怀孕的苦楚,在痛苦中生子;你要依恋你的丈夫,也要受他的管辖。”
\textsuperscript{17}
后对\uline{亚当}说:“因为你听了你妻子的话,吃了我禁止你吃的果子,为了你的缘故,地成了可咒骂的;你一生日日劳苦才能得到吃食。
\textsuperscript{18}
地要给你生出荆棘和蒺藜,你要吃田间的蔬菜;
\textsuperscript{19}
你必须汗流满面,才有饭吃,直到你归于土中,因为你是由土来的;你既是灰土,你还要归于灰土。”
\footnote{天主的仁慈即超过了他的公义,故此他在义怒中给人类预许了人终要得胜魔鬼的诺言;因这许诺,3:15称为“原始福音”。大义是:踏碎蛇头是得胜魔鬼的象征;“女人的后裔”虽然也指犯罪败坏的人类,但在此特别指拯救人类的新元首\uline{基督}(\uwave{哥}1:15-18),只有他打败了魔鬼;故此圣\uline{保禄}称他为“新亚当”(\uwave{罗}5:12-15)。魔鬼同\uline{厄娃}的对白,与天使同\uline{玛利亚}的对白恰恰相反:一是诱惑的对白,一是商讨救赎的对白;因此教父由第2世纪起即称\uline{玛利亚}为“新\uline{厄娃}”。又因她与\uline{基督}的密切结合,她也踏碎了魔鬼的头颅。圣母始胎无玷的道理,由此处已露曙光(\uwave{路}1:26-38;\uwave{默}12)。}

\textbf{被逐出乐园 }
\textsuperscript{20}
\uline{亚当}给自己的妻子起名叫\uline{厄娃},因为她是众生的母亲。
\textsuperscript{21}
上主天主为\uline{亚当}和他的妻子做了件皮衣,给他们穿上;
\textsuperscript{22}
然后上主天主说:“看,人已相似我们中的一个,知道了善恶;如今不要让他伸手再摘取生命树上的果子,吃了活到永远。”
\textsuperscript{23}
上主天主遂把他赶出\uline{伊甸}乐园,叫他耕种他所由出的土地。
\textsuperscript{24}
天主将\uline{亚当}逐出了以后,就在\uline{伊甸}乐园的东面,派了“革鲁宾”和刀光四射的火剑,防守到生命树去的路。
\footnote{“革鲁宾”按\uline{巴比伦}语有“保护者”之意(\uwave{出}25:18-22;\uwave{则}1:11)。}

\textbf{第四章 }
\textbf{杀弟之罪 }
\textsuperscript{1}
\uline{亚当}认识了自己的妻子\uline{厄娃},\uline{厄娃}怀了孕,生了\uline{加音}说:“我赖上主获得了一个人。”
\textsuperscript{2}
以后她生了\uline{加音}的弟弟\uline{亚伯尔};\uline{亚伯尔}牧羊,\uline{加音}耕田。
\textsuperscript{3}
有一天,\uline{加音}把田地的出产作祭品献给天主;
\textsuperscript{4}
同时\uline{亚伯尔}献上自己羊群中最肥美而又是首生的羊;上主惠顾了\uline{亚伯尔}和他的祭品,
\footnote{“认识妻子”是表示夫妻结合的委婉语。\uline{厄娃}生子后说的话意义深奥,说明天主特籍为人母者广传肖似天主的人类(\uwave{加}下7:26-29)。由本章证明献祭从人类起初即有了;献祭的真正价值是在于人的敬心诚意(\uwave{希}11:4)。}
\textsuperscript{5}
却没有惠顾\uline{加音}和他的祭品;因此\uline{加音}大怒,垂头丧气。
\textsuperscript{6}
上主对\uline{加音}说:“你为什么发怒?为什么垂头丧气?
\textsuperscript{7}
你若做得好,岂不也可仰起头来?你若做得不好,罪恶就伏在你门前,企图对付你,但你应制服它。”

\textsuperscript{8}
事后\uline{加音}对他弟弟\uline{亚伯尔}说:“我们到田间去!”当他们在田间的时候,\uline{加音}就袭击了弟弟\uline{亚伯尔},将他杀死。
\textsuperscript{9}
上主对\uline{加音}说:“你弟弟\uline{亚伯尔}在哪里?”他答说:“我不知道,难道我是看守我弟弟的人?”
\textsuperscript{10}
上主说:“你作了什么事?听!你弟弟的血由地上向我喊冤。
\textsuperscript{11}
你现在是地上所咒骂的人,地张开口由你手中接收了你弟弟的血,
\textsuperscript{12}
从此你即使耕种,地也不会给你出产;你在地上要成个流离失所的人。”
\textsuperscript{13}
\uline{加音}对上主说:“我的罪罚太重,无法承担。
\textsuperscript{14}
看你今天将我由这地面上驱逐,我该躲避你的面,在地上成了个流离失所的人;那么凡遇见我的,必要杀我。”
\textsuperscript{15}
上主对他说:“决不这样,凡杀\uline{加音}的人,一定要受七倍的罚。”上主遂给\uline{加音}一个记号,以免遇见他的人击杀他。
\textsuperscript{16}
\uline{加音}就离开上主的面,住在\uline{伊甸}东方的\uline{诺得}地方。
\footnote{杀兄弟的暴行为原祖犯罪的恶果。“凡遇见我的”一句,假定除\uline{加音}、\uline{亚伯尔}、\uline{舍特}三人外,\uline{亚当}还生了别的一些儿女。圣经只记载此三人,因为他们的命运为叙述救赎史已够了。初民为遵从天主叫人传生人类的命令,不能不兄妹结合。但日后人类增多了,兄妹的结合为宗教与礼法所禁止。}

\textbf{加音的后代 }
\textsuperscript{17}
\uline{加音}认识了自己的妻子,她怀了孕,生了\uline{哈诺客}。\uline{加音}建筑了一座城,即以他儿子的名字,给这城起名叫“\uline{哈诺客}”。
\textsuperscript{18}
\uline{哈诺客}生了\uline{依辣得};\uline{依辣得}生了\uline{默胡雅耳};\uline{默胡雅耳}生了\uline{默突沙耳};\uline{默突沙耳}生了\uline{拉默客}。
\textsuperscript{19}
\uline{拉默客}娶了两个妻子:一个名叫\uline{阿达},一个名叫\uline{漆拉}。
\textsuperscript{20}
\uline{阿达}生了\uline{雅巴耳},他是住在帐幕内畜牧者的始祖。
\textsuperscript{21}
他的弟弟名叫\uline{犹巴耳},他是所有弹琴吹箫者的始祖。
\textsuperscript{22}
% TODO \uline{突巴耳}\uline{加音}----下划线太长,导致两个\uline{}像一个。
同时\uline{漆拉}也生了\uline{突巴耳}\uline{加音},他是制造各种铜铁器具的匠人。\uline{突巴耳}\uline{加音}有个姊妹名叫\uline{纳阿玛}。
\textsuperscript{23}
\uline{拉默客}对自己的妻子说:“\uline{阿拉}和\uline{漆拉}倾听我的声音,\uline{拉默客}的妻子,静聆我的言语:因我受伤,杀了一成年;因我受损,杀了一青年;
\textsuperscript{24}
杀\uline{加音}的受罚是七倍,杀\uline{拉默客}的是七十七倍。”
\footnote{此段略记\uline{加音}的后代子孙,和他们的发明以及文化的初步演进。由此可知在洪水之前文明已达到了相当的程度。\uline{巴比伦}史家亦主此说。\uline{拉默客}是违犯一夫一妻制的第一人,违犯了婚姻一夫一妻的理想。}

\textbf{舍特的子孙 }
\textsuperscript{25}
\uline{亚当}又认识了自己的妻子,她生了个儿子,给他起名叫\uline{舍特}说:“天主又赐给了我一个儿子,代替\uline{加音}杀了的\uline{亚伯尔}。”
\textsuperscript{26}
\uline{舍特}也生了一个儿子,给他起名叫\uline{厄诺士}。那时人才开始呼求上主的名。
\footnote{从\uline{亚当}到\uline{厄诺士},人在祈祷和祭献时一定呼求天主助佑。但由\uline{厄诺士}开始举行公众崇拜天主的敬礼。}

\textbf{第五章 }
\textbf{洪水前亚当的后代 }
\textsuperscript{1}
以下是\uline{亚当}后裔的族谱:当天主造人的时候,是按天主的肖像造的,
\textsuperscript{2}
造了一男一女,且在造他们的那一天,祝福了他们,称他们为“人”。
\textsuperscript{3}
\uline{亚当}一百三十岁时,生了一个儿子,也像自己的模样和肖像,给他起名叫\uline{舍特}。
\footnote{\uline{亚当}生了相似自己的子女,因为他是照天主的肖像受造的,他生儿养女,即是传生天生至尊贵的肖像于万世万代的人类。}
\textsuperscript{4}
\uline{亚当}生\uline{舍特}后,还活了八百年,生了其他的儿女。
\textsuperscript{5}
\uline{亚当}共活了九百三十岁死了。
\textsuperscript{6}
\uline{舍特}一百零五岁时,生了\uline{厄诺士}。
\textsuperscript{7}
\uline{舍特}生\uline{厄诺士}后,还活了八百零七年,生了其他的儿女。
\footnote{本章所记为洪水前的十位祖宗,他们的长寿若与其他民族传说的古人比较,所记的年龄还不算太大。虽然如此,有关十位祖宗的年龄,是不易解决的难题。——圣经上说的年是指十二个月的年,月指廿九或三十日的月。——上古人类是否能享如此的长寿,考古人类学至今尚未有一圆满的答案。教父和神学家提出了两个理由来解释原始人的长寿原因:一、自然环境的优良条件:即在人犯原罪之后,仍未丧失天主在造人时所赋的优良人性;二、长寿的主因是天主上智的措施,使人能迅速繁殖,并使人将天主的原始启示传于后代子孙。}
\textsuperscript{8}
\uline{舍特}共活了九百一十二岁死了。
\textsuperscript{9}
\uline{厄诺士}九十岁时生了\uline{刻南}。
\textsuperscript{10}
\uline{厄诺士}生\uline{刻南}后,还活了八百一十五年,生了其他的儿女。
\textsuperscript{11}
\uline{厄诺士}共活了九百零五岁死了。
\textsuperscript{12}
\uline{刻南}七十岁时,生了\uline{玛拉肋耳}。
\textsuperscript{13}
\uline{刻南}生\uline{玛拉肋耳}后,还活了八百四十年,生了其他的儿女。
\textsuperscript{14}
\uline{刻南}共活了九百一十岁死了。
\textsuperscript{15}
\uline{玛拉肋耳}六十五岁时,生了\uline{耶勒得}。
\textsuperscript{16}
\uline{玛拉肋耳}生\uline{耶勒得}后,还活了八百三十年,生了其他的儿女。
\textsuperscript{17}
\uline{玛拉肋耳}共活了八百九十五岁死了。
\textsuperscript{18}
\uline{耶勒得}一百六十二岁时,生了\uline{哈诺客}。
\textsuperscript{19}
\uline{耶勒得}生\uline{哈诺客}后。还活了八百年,生了其他的儿女。
\textsuperscript{20}
\uline{耶勒得}共活了九百六十二岁死了。
\textsuperscript{21}
\uline{哈诺客}六十五岁时,生了\uline{默突舍拉}。
\textsuperscript{22}
\uline{哈诺客}常与天主往来。\uline{哈诺客}生\uline{默突舍拉}后,还活了三百年,生了其他的儿女。
\textsuperscript{23}
\uline{哈诺客}共活了三百六十五岁。
\textsuperscript{24}
\uline{哈诺客}时与天主往来,然后就不见了,因为天主将他提去。
\footnote{论\uline{哈诺客}的事,见\uwave{德}44:16;\uwave{犹}14、15;\uwave{希}11:5。}
\textsuperscript{25}
\uline{默突舍拉}一百八十七岁时,生了\uline{拉默客}。
\textsuperscript{26}
\uline{默突舍拉}生了\uline{拉默客}后,还活了七百八十二年,生了其他的儿女。
\textsuperscript{27}
\uline{默突舍拉}共活了九百六十九岁死了。
\textsuperscript{28}
\uline{拉默客}一百八十二岁时,生了一个儿子,
\textsuperscript{29}
给他起名叫\uline{诺厄}说:“这孩子要使我们在上主诅咒的地上,在我们做的工作和劳苦上,获得欣慰!”
\textsuperscript{30}
\uline{拉默客}生\uline{诺厄}后,还活了五百九十五年,生了其他的儿女。
\textsuperscript{31}
\uline{拉默客}共活了七百七十七岁死了。
\textsuperscript{32}
\uline{诺厄}五百岁时,生了\uline{闪}、\uline{含}和\uline{耶斐特}。
\footnote{\uline{拉默客}对\uline{诺厄}的祝福是一预言。此预言在8:22,9:8-17实现了。}

\textbf{第六章 }
\textbf{人类的败坏 }
\textsuperscript{1}
当人在地上开始繁殖,生养女儿时,
\textsuperscript{2}
天主的儿子见人的女儿美丽,就随意选取,作为妻子。
\textsuperscript{3}
上主于是说:“因为人即属于血肉,我的神不能常在他内;他的寿数只可到一百二十岁。”
\textsuperscript{4}
当天主的儿子与人的儿女结合生子时,在地上已有一些巨人,(以后也有),他们就是古代的英雄,著名的人物。
\footnote{洪水之罚是人类的败坏所引起的,这败坏的近因是因天主的儿子们娶了人的女儿们。所谓天主的儿子即恭敬天主的\uline{舍特}的子孙;人的女儿即指\uline{加音}不恭敬天主的子女。“我的神”此处是指天主赋于人的生活之力(2:7)。“血肉”即指易于沉湎于肉身之乐的人性。“巨人”的来历无法考定。巨人的事迹,多见于《旧约》中(\uwave{户}13:33;\uwave{申}3:11;\uwave{撒}上17;\uwave{巴}3:26-28等处)。此处作者并非说巨人是由天主的儿子和人的女儿生的,而只是说当天主的儿子和人的女儿结合时,地上已有巨人。这些巨人相似那些强悍善战,不认识智慧之道的巨人(\uwave{巴}3:26、27)。}

\textbf{上主决意消灭世界 }
\textsuperscript{5}
上主见人在地上的罪恶重大,人心天天所思念的无非是邪恶;
\textsuperscript{6}
上主遂后悔在地上造了人,心中很是悲痛。
\textsuperscript{7}
上主于是说:“我要将我所造的人,连人带野兽、爬虫和天空的飞鸟,都由地面上消灭,因为我后悔造了他们。”
\textsuperscript{8}
惟有\uline{诺厄}在上主眼中蒙受恩爱。
\footnote{有关洪水的记载(6-8章),大概来自两种有关洪水的记述。近东古代史家编纂史书,多只穿插古文件,而对文件中互异之处,多不加修改。有关洪水的传说,古代民族大都有所记载。本书所记就结构和体裁而言,与\uwave{叔默尔}和\uwave{巴比伦}的洪水神话有很多类似之处,但根本的区别很大,因本书中决无多神的不经之论;且本书所记是在教训世人几端道德和宗教的高深道理,如天主的正义、仁慈、召选、救恩和盟约的道理(9:1-17)。《旧约》的作者多以\uwave{诺厄}和洪水的事为天主施恩和惩罚的预象(\uwave{依}54:7-10;\uwave{德}44:17-19;\uwave{智}10:4)。《新约》多以洪水的事为公审判(\uwave{玛}24:37-39),或圣洗圣事的预象(\uwave{伯}前3:18-22)。}

\textbf{诺厄建造方舟 }
\textsuperscript{9}
以下是\uline{诺厄}的小史:\uline{诺厄}是他同时代唯一正义齐全的人,常同天主往来。
\textsuperscript{10}
他生了三个儿子:就是\uline{闪}、\uline{含}、和\uline{耶斐特}。
\textsuperscript{11}
大地已在天主面前败坏,到处充满了强暴。
\textsuperscript{12}
天主见大地已败坏,因为凡有血肉的人,品行在地上全败坏了,
\textsuperscript{13}
天主遂对\uline{诺厄}说:“我已决定要结果一切有血肉的人,因为他们使大地充满了强暴,我要将他们由大地上消灭。
\textsuperscript{14}
你要用柏木造一只方舟,舟内建造一些舱房,内外都涂上沥青。
\textsuperscript{15}
你要这样建造:方舟要有三百肘长,五十肘宽,三十肘高。
\textsuperscript{16}
方舟上层四面做上窗户,高一肘;门要安在侧面;方舟要分为上中下三层。
\textsuperscript{17}
看我要使洪水在地上泛滥,消灭天下一切有生气的血肉;凡地上所有的都要灭亡。
\textsuperscript{18}
但我要与你立约,你以及你的儿子、妻子和儿媳,要与你一同进入方舟。
\textsuperscript{19}
你要由一切有血肉的生物中,各带一对,即一公一母,进入方舟,与你一同生活;
\textsuperscript{20}
各种飞鸟、各种牲畜、地上所有的各种爬虫,皆取一对同你进去,得以保存生命。
\textsuperscript{21}
此外,你还应带上各种吃用的食物,贮存起来,作你和他们的食物。”
\textsuperscript{22}
\uline{诺厄}全照办了;天主怎样吩咐了他,他就怎样做了。

\textbf{第七章 }
\textbf{洪水灭世 }
\textsuperscript{1}
上主对\uline{诺厄}说:“你和你全家进入方舟,因为在这一世代,我看只有你在我面前正义。
\textsuperscript{2}
由一切洁净牲畜中,各取公母七对;由那些不洁净的牲畜中,各取公母一对;
\textsuperscript{3}
由天空的飞鸟中,也各取公母七对;好在全地面上传种。
\textsuperscript{4}
因为还有七天,我要在地上降雨四十天四十夜,消灭我在地面上所造的一切生物。”
\textsuperscript{5}
\uline{诺厄}全照上主吩咐他的做了。
\footnote{带进方舟的牲畜,自然洁净的多于不洁净的,因为洁净的可为食用,又可为祭献天主之用。见8:20-22;\uwave{肋}11。}
\textsuperscript{6}
当洪水在地上泛滥时,\uline{诺厄}已六百岁。
\textsuperscript{7}
\uline{诺厄}和他的儿子,他的妻子和他的儿媳,同他进了方舟,为躲避洪水。
\textsuperscript{8}
洁净的牲畜和不洁净的牲畜,飞鸟和各种在地上爬行的动物,
\textsuperscript{9}
一对一对地同\uline{诺厄}进了方舟;都是一公一母,照天主对他所吩咐的。
\textsuperscript{10}
七天一过,洪水就在地上泛滥。
\textsuperscript{11}
\uline{诺厄}六百岁那一年,二月十七日那天,所有深渊的泉水都冒出,天上的水闸都开放了;
\textsuperscript{12}
大雨在地上下了四十天四十夜。
\textsuperscript{13}
正在这一天,\uline{诺厄}和他的儿子\uline{闪}、\uline{含}、\uline{耶斐特},他的妻子和他的三个儿媳,一同进了方舟。
\textsuperscript{14}
他们八口和所有的野兽、各种牲畜、各种在地上爬行的爬虫、各种飞禽,
\textsuperscript{15}
一切有生气有血肉的,都一对一对地同\uline{诺厄}进了方舟。
\textsuperscript{16}
凡有血肉的,都是一公一母地进了方舟,如天主对\uline{诺厄}所吩咐的。随后上主关了门。

\textsuperscript{17}
洪水在地上泛滥了四十天;水不断增涨,浮起了方舟,方舟遂由地面上升起。
\textsuperscript{18}
洪水汹涌,在地上猛涨,方舟飘浮在水面上。
\textsuperscript{19}
洪水在地上一再猛涨,天下所有的高山都没了顶;
\textsuperscript{20}
洪水高出淹没的群山十有五肘。
\textsuperscript{21}
凡地上行动而有血肉的生物:飞禽、牲畜、野兽、在地上爬行的爬虫,以及所有的人全灭亡了;
\textsuperscript{22}
凡在旱地上以鼻呼吸的生灵都死了。
\textsuperscript{23}
这样,天主消灭了在地面上的一切生物,由人以至于牲畜、爬虫以及天空中的飞鸟,这一切都由地上消灭了,只剩下\uline{诺厄}和同他在方舟内的人物。
\textsuperscript{24}
洪水在地上泛滥了一百五十天。
\footnote{据古\uwave{希伯来}人的宇宙观:天的上边(1:7;\uwave{咏}104:3-13,148:4),地的下面都为水所包围;地下的水也叫深渊(\uwave{依}51:10;\uwave{咏}36:6;\uwave{亚}7:4)。关于洪水泛滥的日期,17节为四十天,24节为一百五十天,大概由两种不同的文献而来。}

\textbf{第八章 }
\textbf{洪水退落 }
\textsuperscript{1}
天主想起了\uline{诺厄}和同他在方舟内的一切野兽和牲畜,遂使风吹过大地,水渐渐退落;
\textsuperscript{2}
深渊的泉源和天上的水闸已关闭,雨也由天上停止降落,
\textsuperscript{3}
于是水逐渐由地上退去;过了一百五十天,水就低落了。
\textsuperscript{4}
% TODO \uline{阿辣}\uline{辣特}----下划线太长,两个词,变成了一个词。
七月十七日,方舟停在\uline{阿辣}\uline{辣特}山上。
\textsuperscript{5}
洪水继续减退,直到十月;十月一日,许多山顶都露出来。
\textsuperscript{6}
过了四十天,\uline{诺厄}开了在方舟上做的窗户,
\textsuperscript{7}
放了一只乌鸦;乌鸦飞去又飞回,直到地上的水都干了。
\textsuperscript{8}
\uline{诺厄}等待了七天,又放出了一只鸽子,看看水是否已由地面退尽。
\textsuperscript{9}
但是,因为全地面上还有水,鸽子找不着落脚的地方,遂飞回方舟;\uline{诺厄}伸手将它接入方舟内。
\textsuperscript{10}
再等了七天,他由方舟中又放出一只鸽子,
\textsuperscript{11}
傍晚时,那只鸽子飞回他那里,看,嘴里衔着一根绿的橄榄树枝;\uline{诺厄}于是知道,水已由地上退去。
\textsuperscript{12}
\uline{诺厄}又等了七天再放出一只鸽子;这只鸽子没有回来。
\footnote{通观有关洪水的记载(6:5-7,7:1-8:12),似乎全世界都为洪水所淹没,人类除\uline{诺厄}一家外全都消灭了。但若注意近东古代史家的渲染夸大的作风,“全地”、“天下”或类似的词句,仅指作者所知道的地方(\uwave{创}41:54、57;\uwave{宗}2:5等)。由此可知洪水的泛滥仅是局部的,而未遍及于全世界。淹死的人也只是作者所知道的人民,而不是全人类。细察本书作者的目的,只是记载启示的历史,或天主在世建立神国的历史,所以与启示或与\uline{以色列}人无关系的历史与人物一概不提。}

\textbf{诺厄出方舟 }
\textsuperscript{13}
\uline{诺厄}六百零一岁,正月初一,地上的水都干了,\uline{诺厄}就撤开方舟的顶观望,看见地面已干。
\textsuperscript{14}
二月二十七日,大地全干了。
\textsuperscript{15}
天主于是吩咐\uline{诺厄}说:
\textsuperscript{16}
“你和你的妻子、儿子及儿媳,同你由方舟出来;
\textsuperscript{17}
所有同你在方舟内的有血肉的生物:飞禽、牲畜和各种地上的爬虫,你都带出来,叫他们在地上滋生,在地上生育繁殖。”
\textsuperscript{18}
\uline{诺厄}遂同他的儿子、妻子及儿媳出来;
\textsuperscript{19}
所有的爬虫、飞禽和地上所有的动物,各依其类出了方舟。
\textsuperscript{20}
\uline{诺厄}给上主筑了一座祭坛,拿各种洁净的牲畜和洁净的飞禽,献在祭坛上,作为全燔祭。
\textsuperscript{21}
上主闻到了馨香,心里说:“我再不为人的缘故咒骂大地,因为人心的思念从小就邪恶;我也不再照我所作的打击一切生物了,
\textsuperscript{22}
只愿大地存在之日,稼穑寒暑,冬夏昼夜,循环不息。”
\footnote{由洪水之罚,作者教训人几端重要的道理:罪恶使大地回到了原始的混沌状态;罪恶连累了无灵的受造之物(6:13;\uwave{罗}8:19-22);方舟为圣教会的预象(\uwave{伯}前3:20、21);祭献的举行使人再蒙受天主的祝福;生命之可贵;天主同\uline{诺厄}所立的盟约也及于天地万物。}

\textbf{第九章 }
\textbf{人类复兴 }
\textsuperscript{1}
天主祝福\uline{诺厄}和他的儿子们说:“你们要滋养繁殖,充满大地。
\textsuperscript{2}
地上的各种野兽,天空的各种飞鸟,地上的各种爬虫和水中的各种游鱼,都要对你们表示惊恐畏惧:这一切都已交在你们手中。
\textsuperscript{3}
凡有生命的动物,都可作你们的食物;我将这一切赐给你们,有如以前赐给你们蔬菜一样;
\textsuperscript{4}
凡有生命,带血的肉,你们不可吃;
\textsuperscript{5}
并且,我要追讨害你们生命的血债:向一切野兽追讨,向人,向为弟兄的人,追讨人命。
\textsuperscript{6}
凡流人血的,他的血也要为人所流,因为人是造天主的肖像造的。
\textsuperscript{7}
你们要生育繁殖,在地上滋生繁衍。”
\footnote{天主祝福他们传生人类的话,像祝福原祖一样(1:28-30)。起初天主似乎禁止人吃肉,只准吃蔬菜果品(1:29),现今都准许了,但不准吃带血的肉。按古人的思想,血是生命之所在,是生活的动力。这生命直接来自天主(\uwave{申}12:16、23,15:23;\uwave{肋}3:17,7:26,17:10-14;\uwave{宗}15:29),为此禁止吃血。5、6两节为日后报复法的根据(\uwave{户}35:19;\uwave{出}21:23-25;\uwave{申}19:18-21)。}

\textbf{天主与诺厄立约 }
\textsuperscript{8}
天主对\uline{诺厄}和他的儿子们说:
\textsuperscript{9}
“看,我现在与你们和你们未来的后裔立约,
\textsuperscript{10}
并与同你们在一起的一切生物:飞鸟、牲畜和一切地上野兽,即凡由方舟出来的一切地上生物立约。
\textsuperscript{11}
我与你们立约:凡有血肉的,以后决不再受洪水湮灭,再没有洪水来毁灭大地。”
\textsuperscript{12}
天主说:“这是我在我与你们以及同你们在一起的一切生物之间,立约的永远标记:
\textsuperscript{13}
我把虹霓放在云间,作我与大地之间立约的标记。
\textsuperscript{14}
几时我兴云遮盖大地,云中要出现虹霓,
\textsuperscript{15}
那时我便想起我与你们以及各种属血肉的生物之间所立的盟约:这样水就不会再成为洪水,毁灭一切血肉的生物。
\textsuperscript{16}
几时虹霓在云间出现,我一看见,就想起在天主与地上各种属血肉的生物之间所立的永远盟约。”
\textsuperscript{17}
天主对\uline{诺厄}说:“这就是我在我与地上一切有血肉的生物之间,所立的盟约的标记。”
\footnote{此段所说的盟约原是天主无条件赐恩的许诺;天主以此诺言保证自然界从今以后不再受洪水之害。虹霓的自然现象在洪水前虽已有,但从今以后当作天主诺言的保证。}

\textbf{诺厄的诅咒与祝福 }
\textsuperscript{18}
\uline{诺厄}的儿子由方舟出来的,有\uline{闪}、\uline{含}、和\uline{耶斐特}。含是\uline{客纳罕}的父亲。
\textsuperscript{19}
这三人是\uline{诺厄}的儿子;人类就由这三人分布天下。
\textsuperscript{20}
\uline{诺厄}原是农夫,遂开始种植葡萄园。
\textsuperscript{21}
一天他喝酒喝醉了,就在自己的帐幕内脱去了衣服。
\textsuperscript{22}
\uline{客纳罕}的父亲\uline{含}看见了父亲赤身露体,遂去告诉外面的两个兄弟。
\textsuperscript{23}
\uline{闪}和\uline{耶斐特}二人于是拿了件外衣,搭在肩上,倒退着走进去,盖上父亲的裸体。他们的脸背着,没有看见父亲的裸体。
\textsuperscript{24}
\uline{诺厄}醒了后,知道了小儿对他作的事,
\textsuperscript{25}
就说:“\uline{客纳罕}是可咒骂的,给兄弟当最下贱的奴隶。”
\footnote{\uline{诺厄}诅咒了\uline{客纳罕}而未诅咒\uline{含},一、因天主早祝福了\uline{含}(1节);二、因\uline{客纳罕}的后裔日后放荡无耻。\uline{闪}特受祝福,因他是\uline{亚巴郎}的祖宗,因\uline{亚巴郎}的后裔(基督),万民将获得祝福。\uline{闪}受的祝福也及于\uline{耶斐特},到新约时代也及于\uline{含}和普世万民(\uwave{宗}2:5、9-11;\uwave{路}3:6)。}
\textsuperscript{26}
又说:“上主,闪的天主,应受赞美,\uline{客纳罕}应作他的奴隶。
\textsuperscript{27}
愿天主扩展\uline{耶斐特},使他住在\uline{闪}的帐幕内;\uline{客纳罕}应作他的奴隶。”
\textsuperscript{28}
洪水以后,\uline{诺厄}又活了三百五十年。
\textsuperscript{29}
\uline{诺厄}共活了九百五十岁死了。

\textbf{第十章 }
\textbf{诺厄三子的后裔 }
\textsuperscript{1}
以下是\uline{诺厄}的儿子\uline{闪}、\uline{含}、和\uline{耶斐特}的后裔。洪水以后,他们都生了子孙。

\textsuperscript{2}
\uline{耶斐特}的之孙:\uline{哥默尔}、\uline{玛哥格}、\uline{玛待}、\uline{雅汪}、\uline{突巴耳}、\uline{默舍客}和\uline{提辣斯}。
\textsuperscript{3}
\uline{哥默尔}的子孙:\uline{阿市}\uline{革纳次}、\uline{黎法特}和\uline{托加尔玛}。
\textsuperscript{4}
\uline{雅汪}的子孙:\uline{厄里沙}、\uline{塔尔史士}、\uline{基延}和\uline{多丹}。
\textsuperscript{5}
那些分布于岛上的民族,就是出于这些人:以上这些人按疆域、语言、宗教和国籍,都属\uline{耶斐特}的子孙。

\textsuperscript{6}
\uline{含}的子孙:\uline{雇士}、\uline{米兹辣殷}、\uline{普特}、和\uline{客纳罕}。
\textsuperscript{7}
\uline{雇士}的子孙:\uline{色巴}、\uline{哈威拉}、\uline{撒贝拉}、\uline{辣阿玛}和\uline{撒贝特加}。\uline{辣阿玛}的之孙:\uline{舍巴}和\uline{德丹}。
\footnote{古地理学家多以本章民族的名单为一种极宝贵的文献。各民族的分布是:北有\uline{耶斐特}的后裔,南有\uline{含}的后裔,在二者之间为\uline{闪}的后裔。有许多名字至今已不可考。}
\textsuperscript{8}
\uline{雇士}生\uline{尼默洛得},他是世上第一个强人。
\textsuperscript{9}
他在上主面前是个有本事的猎人,为此有句俗话说:“如在上主面前,有本领的猎人\uline{尼默洛得}。”
\textsuperscript{10}
他开始建国于\uline{巴比伦}、\uline{厄勒客}和\uline{阿加得},都在\uline{史纳尔}地域。
\textsuperscript{11}
他由那地方去了\uline{亚述},建设了\uline{尼尼微}、\uline{勒曷波特}城、\uline{加拉}
\textsuperscript{12}
和在\uline{尼尼微}与\uline{加拉}之间的\uline{勒森}(\uline{尼尼微}即是那大城)。
\footnote{\uline{尼默洛得}(实体书上是:尼默洛特,有误?)的故事,是上古巨人的故事之一,参见6章注一。}
\textsuperscript{13}
\uline{米兹辣殷}生\uline{路丁}人、\uline{阿纳明}人、\uline{肋哈宾}人、\uline{纳斐突歆}人、
\textsuperscript{14}
\uline{帕特洛斯}人、\uline{加斯路}人和\uline{加非}\uline{托尔}人。\uline{培肋}\uline{舍特}人即出自此族。
\textsuperscript{15}
\uline{客纳罕}生长子\uline{漆冬},以后生\uline{赫特}、
\textsuperscript{16}
\uline{耶步斯}人、\uline{阿摩黎}人、\uline{基尔加}\uline{士}人、
\textsuperscript{17}
\uline{希威}人、\uline{阿尔克}人、\uline{息尼}人、
\textsuperscript{18}
\uline{阿尔瓦得}人、\uline{责玛}\uline{黎}人和\uline{哈玛}\uline{特}人;以后,\uline{客纳罕}的宗教分散了,
\textsuperscript{19}
一致\uline{客纳罕}人的边疆,自\uline{漆冬}经过\uline{革辣尔}直到\uline{迦萨},又经过\uline{索多玛}、\uline{哈摩辣}、\uline{阿德玛}和\uline{责波殷},直到\uline{肋沙}。
\textsuperscript{20}
以上这些人按疆域、语言、宗族和国籍,都属\uline{含}的子孙。

\textsuperscript{21}
\uline{耶斐特}的长兄,即\uline{厄贝尔}所有子孙的祖先\uline{闪},也生了儿子。
\textsuperscript{22}
\uline{闪}的子孙:\uline{厄蓝}、\uline{亚述}、\uline{阿帕革}\uline{沙得}、\uline{路得}和\uline{阿兰}。
\textsuperscript{23}
\uline{阿兰}的之孙:\uline{伍兹}、\uline{胡耳}、\uline{革特尔}和\uline{玛士}。
\textsuperscript{24}
\uline{阿帕革}\uline{沙得}生\uline{舍拉};\uline{舍拉}生\uline{厄贝尔}。
\textsuperscript{25}
\uline{厄贝尔}生了两个儿子:一个名叫\uline{培肋格},因为在他的时代世界分裂了;他的兄弟名叫\uline{约刻堂}。
\textsuperscript{26}
\uline{约刻堂}生\uline{阿耳}\uline{摩达得}、\uline{舍肋夫}、\uline{哈匝玛委特}、\uline{耶辣}、
\textsuperscript{27}
\uline{哈多兰}、\uline{乌匝耳}、\uline{狄刻拉}、
\textsuperscript{28}
\uline{敖巴耳}、\uline{阿彼玛耳}、\uline{舍巴}、
\textsuperscript{29}
\uline{敖非尔}、\uline{哈威拉}和\uline{约巴布}:以上都是\uline{约刻堂}的子孙。
\textsuperscript{30}
他们居住的地域,从\uline{默沙}经过\uline{色法尔}直到东面的山地:
\textsuperscript{31}
以上这些人按疆域、语言、宗族和国籍,都属\uline{闪}的子孙:
\textsuperscript{32}
以上这些人按他们的出身和国籍,都是\uline{诺厄}子孙的家族;洪水以后,地上的民族都是由他们分出来的。
\footnote{本章的主旨有二:一、指出\uline{希伯来}人所知道的主要民族和他们的关系;二、说明\uline{以色列}人在他们中的地位。此外值得注意的是:作者将民族的分布情况说明之后,将范围逐渐缩小,一直缩到\uline{以色列}人的历史。作者列出\uline{诺厄}的后代之后,只说\uline{闪}的嫡系,再后只说\uline{特辣黑}的一支。选民的始祖\uline{亚巴郎}即由此支而来。}

\textbf{第十一章 }
\textbf{巴贝耳塔 }
\textsuperscript{1}
当时全世界只有一种语言和一样的话。
\textsuperscript{2}
当人们由东方迁移的时候,在\uline{史纳尔}地方找到了一块平原,就在那里住下了。
\textsuperscript{3}
他们彼此说:“来,我们这砖,用火烧透。”他们遂拿砖当石,拿沥青代灰泥。
\textsuperscript{4}
然后彼此说:“来,让我们建造一城一塔,塔顶摩天,好给我们作纪念,免得我们在全地面上分散了!”
\textsuperscript{5}
上主遂下来,要看看世人所造的城和塔。
\textsuperscript{6}
上主说:“看,他们都是一个民族,都说一样的语言。他们如今就开始做这事;以后他们所想做的,就没有不成功的了。
\textsuperscript{7}
来,我们下去,混乱他们的语言,使他们彼此语言不通。”
\textsuperscript{8}
于是上主将他们分散到全地面,他们遂停止建造那城。
\textsuperscript{9}
为此人称那地为“巴贝耳”,因为上主在那里混乱了全地的语言,且从那里将他们分山到全地面。
\footnote{原祖由于骄傲违背了天主的命令,洪水以后的人类也犯了同样的罪,因此也受了天主的惩罚。再说,人由于骄傲,彼此不和,必分离四散;分离即久,语言必渐分歧。人类的合一,只有赖\uline{基督}的神国方可实现(\uwave{宗}2:5-21;\uwave{默}7:9、10;\uwave{若}11:52)。古人所建的城和塔。即含有骄傲的意思(4-7节),因此日后的先知多以此城象征世上的恶势力(\uwave{哈}1:11;\uwave{依}11:11;\uwave{达}1,2)。}

\textbf{闪族的家谱 }
\textsuperscript{10}
以下是\uline{闪}的后裔:洪水后两年,\uline{闪}正一百岁,生了\uline{阿帕革}\uline{沙得};
\textsuperscript{11}
生\uline{阿帕革}\uline{沙得}后,\uline{闪}还活了五百年,也生了其他的儿女。
\textsuperscript{12}
\uline{阿帕革}\uline{沙得}三十五岁时,生了\uline{舍拉};
\textsuperscript{13}
生\uline{舍拉}后,\uline{阿帕革}\uline{沙得}还活了四百零三年,也生了其他的儿女。
\textsuperscript{14}
\uline{舍拉}三十岁时,生了\uline{厄贝尔};
\textsuperscript{15}
生\uline{厄贝尔}后,\uline{舍拉}还活了四百零三年,也生了其他的儿女。
\textsuperscript{16}
\uline{厄贝尔}三十四岁时,生了\uline{培肋格};
\textsuperscript{17}
生\uline{培肋格}后,\uline{厄贝尔}还活了四百三十年,也生了其他的儿女。
\textsuperscript{18}
\uline{培肋格}三十岁时,生了\uline{勒伍};
\textsuperscript{19}
生\uline{勒伍}后,\uline{培肋格}还活了二百零九年,也生了其他的儿女。
\textsuperscript{20}
\uline{勒伍}三十二岁时,生了\uline{色鲁格};
\textsuperscript{21}
生\uline{色鲁格}后,\uline{勒伍}还活了二百零七年,也生了其他的儿女。
\textsuperscript{22}
\uline{色鲁格}三十岁时,生了\uline{纳曷尔};
\textsuperscript{23}
生\uline{纳曷尔}后,\uline{色鲁格}还活了二百零七年,也生了其他的儿女。
\textsuperscript{24}
\uline{纳曷尔}活到二十九岁时,生了\uline{特辣黑};
\textsuperscript{25}
生\uline{特辣黑}后,\uline{纳曷尔}还活了一百一十九年,也生了其他的儿女。
\textsuperscript{26}
\uline{特辣黑}七十岁时,生了\uline{亚巴郎}、\uline{纳曷尔}、\uline{哈朗}。
\footnote{\uline{闪}族的十大祖先,相似洪水前的十大祖先(5章)。这个族谱,决不是全人类的整个历史,而只是启示历史的系统而已。}

\textbf{特辣黑的后裔 }
\textsuperscript{27}
以下是\uline{特辣黑}的后裔:\uline{特辣黑}生了\uline{亚巴郎}、\uline{纳曷尔}、\uline{哈朗};\uline{哈朗}生了\uline{罗特}。
\textsuperscript{28}
\uline{哈朗}在他的出生地,\uline{加色丁}人的\uline{乌尔},死在他父亲\uline{特辣黑}面前。
\textsuperscript{29}
\uline{亚巴郎}和\uline{纳曷尔}都娶了妻子,\uline{亚巴郎}的妻子名叫\uline{撒辣依};\uline{纳曷尔}的妻子名叫\uline{米耳加},她是\uline{哈朗}的女儿;\uline{哈朗}是\uline{米耳加}和\uline{依色加}的父亲。
\textsuperscript{30}
\uline{撒辣依}不生育,没有子女。
\textsuperscript{31}
\uline{特辣黑}带了自己的儿子\uline{亚巴郎}和孙子,即\uline{哈朗}的儿子\uline{罗特},并儿媳,即\uline{亚巴郎}的妻子\uline{撒辣依},一同由\uline{加色丁}的\uline{乌尔}出发,往\uline{客纳罕}地去;他们到了\uline{哈兰},就在那里住下了。
\textsuperscript{32}
\uline{特辣黑}死于\uline{哈兰},享寿二百零五岁。
\footnote{按作者的意思,真宗教的信仰只保存在\uline{闪}族的伟大后裔\uline{亚巴郎}那一支内,他是选民的始祖,众信友之父(\uwave{世}32:13;\uwave{依}51:1、2;\uwave{罗}4:11)。}

\begin{center}
	\textbf{后编 }
	\textbf{圣组史(12-50)}
\end{center}

\textbf{第十二章 }
\textbf{亚巴郎蒙召 }
\textsuperscript{1}
上主对\uline{亚巴郎}说:“离开你的故乡、你的家族和父家,往我指给你的地方去。
\textsuperscript{2}
我要使你成为一个大民族,我必祝福你,使你成名,成为一个福源。
\textsuperscript{3}
我要祝福那祝福你的人,咒骂那咒骂你的人;地上万民都要因你获得祝福。”
\textsuperscript{4}
\uline{亚巴郎}遂照上主的吩咐起了身,\uline{罗特}也同他一起走了。\uline{亚巴郎}离开\uline{哈兰}时,已七十五岁。
\textsuperscript{5}
他带了妻子\uline{撒辣依}、他兄弟的儿子\uline{罗特}和他在\uline{哈兰}积蓄的财物,获得的仆婢,一同往\uline{客纳罕}地去,终于到了\uline{客纳罕}地。
\textsuperscript{6}
\uline{亚巴郎}经过那地,直到了\uline{舍根}地\uline{摩勒}橡树区;当时\uline{客纳罕}人尚住在那地方。
\textsuperscript{7}
上主显现给\uline{亚巴郎}说:“我要将这地方赐给你的后裔。”\uline{亚巴郎}就在那里给显现于他的上主,筑了一座祭坛。
\textsuperscript{8}
从那里又迁移到\uline{贝特耳}东面山区,在那里搭了帐幕,西有\uline{贝特耳},东有\uline{哈依};他在那里又为上主筑了一座祭坛,呼求上主的名。
\textsuperscript{9}
以后\uline{亚巴郎}渐渐移往\uline{乃革布}区。
\footnote{\uline{亚巴郎}因坚信天主的话,遵命起身,遂蒙了天主祝福,使他的子孙成为一大民族,而且因了他子孙中所出生的\uline{默西亚},为了那些效法他信德的人,\uline{亚巴郎}成了他们幸福的泉源,因为他保存了真宗教和对\uline{默西亚}的希望,因而万民籍\uline{默西亚}得到了救赎,即如圣\uline{保禄}所说:“\uline{亚巴郎}的祝福在\uline{基督}\uline{耶稣}内普及于万民”(\uwave{迦}3:14;\uwave{希}11:8-12)。}

\textbf{亚巴郎去埃及 }
\textsuperscript{10}
其时那地方起了饥荒,\uline{亚巴郎}遂下到埃及,寄居在那里,因为那地方饥荒十分严重。
\textsuperscript{11}
当他要进\uline{埃及}时,对妻子\uline{撒辣依}说:“我知道你是个貌美的女人;
\textsuperscript{12}
\uline{埃及}人见了你,必要说:这是他的妻子;他们定要杀我,让你活着。
\textsuperscript{13}
所以请你说:你是我的妹妹,这样我因了你而必获优待,赖你的情面,保全我的生命。”
\textsuperscript{14}
果然,当\uline{亚巴郎}一到了\uline{埃及},\uline{埃及}人就注意了这女人实在美丽。
\textsuperscript{15}
法郎的朝臣也看见了她,就在法郎前赞她美丽;这女人就被带入法郎的宫中。
\textsuperscript{16}
\uline{亚巴郎}因了她果然蒙了优待,得了些牛羊、公驴、仆婢、母驴和骆驼。
\textsuperscript{17}
但是,上主为了\uline{亚巴郎}的妻子\uline{撒辣依}的事,降下大难打击了法郎和他全家。
\footnote{\uline{亚巴郎}为保护自己的生命财产的所言所行,虽不可取法,但由此可知,圣经中不拘善恶都记载下来,连圣祖的毛病罪过都一一记叙;而且圣经不但对启示的道理有所进展,而且对道德的概念也有所演进。}
\textsuperscript{18}
法郎遂叫\uline{亚巴郎}来说:“你对我作的是什么事?为什么你没有告诉我,她是你的妻子?
\textsuperscript{19}
为什么你说:她是我的妹妹,以致我娶了她做我的妻子?现在,你的妻子在这里,你带她去吧!”
\textsuperscript{20}
法郎于是吩咐人送走了\uline{亚巴郎}和他的妻子以及他所有的一切。

\textbf{第十三章 }
\textbf{亚巴郎回贝特耳 }
\textsuperscript{1}
\uline{亚巴郎}带了妻子和他所有的一切,与\uline{罗特}一同由\uline{埃及}上来,往\uline{乃革布}去。
\textsuperscript{2}
\uline{亚巴郎}有许多牲畜和金银。
\textsuperscript{3}
他由\uline{乃革布}逐渐往\uline{贝特耳}移动帐幕,到了先前他在\uline{贝特耳}与\uline{哈依}之间,支搭帐幕的地方,
\textsuperscript{4}
亦即他先前筑了祭坛,呼求上主之名的地方。

\textbf{亚巴郎与罗特分离 }
\textsuperscript{5}
与\uline{亚巴郎}同行的\uline{罗特},也有羊群、牛群和帐幕,
\textsuperscript{6}
那地方容不下他们住在一起,因为他们的产业太多,无法住在一起。
\textsuperscript{7}
牧放\uline{亚巴郎}牲畜的人与牧放\uline{罗特}牲畜的人,时常发士口角,——当时\uline{客纳罕}人和\uline{培黎齐}人尚住在那里。
\textsuperscript{8}
\uline{亚巴郎}遂对\uline{罗特}说:“在我与你,我的牧人与你的牧人之间,请不要发士口角,因为我们是至亲。
\textsuperscript{9}
所有的地方不都是在你面前吗?请你与我分开。你若往左,我就往右;你若往右,我就往左。”
\textsuperscript{10}
\uline{罗特}举目看见\uline{约但}河整个平原,直到\uline{左哈尔}一带全有水灌溉,——这是在上主消灭\uline{索多玛}和\uline{哈摩辣}以前的事,——有如上主的乐园,有如\uline{埃及}地。
\textsuperscript{11}
\uline{罗特}选了\uline{约但}河的整个平原,遂向东方迁移;这样,他们就彼此分开了:
\textsuperscript{12}
\uline{亚巴郎}住在\uline{客纳罕}地;\uline{罗特}住在平原的城市中,渐渐移动帐幕,直到\uline{索多玛}。
\textsuperscript{13}
\uline{索多玛}人在上主面前罪大恶极。

\textsuperscript{14}
\uline{罗特}与\uline{亚巴郎}分离以后,上主对\uline{亚巴郎}说:“请你举起眼来,由你所在的地方,向东西南北观看;
\textsuperscript{15}
凡你看见的地方,我都要永远赐给你和你的后裔。
\textsuperscript{16}
我要使你的后裔有如地上的灰尘;如果人能数清地上的灰尘,也能数清你的后裔。
\textsuperscript{17}
你起来,纵横走遍这地,因为我要将这地赐给你。”
\textsuperscript{18}
于是\uline{亚巴郎}移动了帐幕,来到\uline{赫贝龙}的\uline{玛默勒}橡树区居住,在那里给上主筑了一座祭坛。
\footnote{论及\uline{亚巴郎}的宽宏大量,金口圣\uline{若望}说:“长辈对晚辈,老者对少年的\uline{罗特},叔父对侄子,言谈有如兄弟,如平辈,让侄子任意选择。”本章给予的教训是:\uline{罗特}喜爱优裕的生活,没有躲避恶劣的环境,而遭受了重罚;\uline{亚巴郎}的慷慨却受到了厚报。当知本章为18:20、21,19:4-19的前奏。}
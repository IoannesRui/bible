% 分栏
\setlength\columnsep{0.6cm}
\begin{multicols}{2}

% 标题
\chapter*{若望三书}

% 右页眉
\rhead{若望三书(若三)}

% 正文
\textbf{致候辞\quad}
\textsuperscript{1}
我长老致书给可爱的\uline{加约},就是我在真理内所爱的。
\textsuperscript{2}
亲爱的,祝你诸事顺利,并祝你健康,就如你的灵魂常顺利一样。

\textbf{赞美加约\quad}
\textsuperscript{3}
有些弟兄来,证明你在真理内,就是说你怎样在真理内生活,我很是高兴。
\renewcommand\thefootnote{\ding{\numexpr171+\value{footnote}}}
\footnote{“有些兄弟”指传教士(5-7节);他们来到\uline{若望}前称道\uline{加约}。}
\textsuperscript{4}
我听说我的孩子们在真理内生活,我没有比这再大的喜乐了。
\textsuperscript{5}
亲爱的,凡你对弟兄,尤其对旅客所行的,都是真信徒的行为。
\textsuperscript{6}
他们在教会前证明了你的爱德。你若以相称天主的态度,帮助他们走上旅途,就是做了善事,
\textsuperscript{7}
因为他们出发是为主的名字,并没有从外教人接受什么;
\textsuperscript{8}
所以我们应当款待这样的人,为叫我们成为与真理合作的人。
\footnote{“与真理合作的人”即扶助宣传福音的人(\uwave{路}10:7、8;\uwave{格}前9:12-14;\uwave{弟}前5:17、18)。}

\textbf{责备狄约勒斐称赞德默特琉\quad}
\textsuperscript{9}
我给教会写过信,但是那在他们中间爱作首领的\uline{狄约勒斐},却不承认我们。
\textsuperscript{10}
为此,我若来到,必要指摘他所行的事,就是他用恶言恶语诽谤我们的事;但这为他还不够:他自己不款待弟兄们,连那愿意款待的,他也加以阻止,甚而将他们逐出教会。
\footnote{\uline{狄约勒斐}大概是当时教会的领袖:(一)他否认\uline{若望}的权威,并且出言不逊;(二)不招待传教士;(三)不但阻止人招待,甚至处罚那些招待了传教士的人。}
\textsuperscript{11}
亲爱的,你不要效法恶,但要效法善:那行善的是出于天主,那作恶的是没有见过天主。
\textsuperscript{12}
众人和真理本身都给\uline{德默特琉}作证,我们自己也给他作证,而你也知道我们所作的证是真实的。
\footnote{关于所称赞的\uline{德默特琉},仅见于此处。}

\textbf{结尾语\quad}
\textsuperscript{13}
我本来有许多事要写给你,但是我不愿意以笔墨给你写;
\textsuperscript{14}
只希望快见到你,我们好亲口面谈。
\textsuperscript{15}
祝你平安!朋友都问候你。请你也一一问候各位朋友。
\end{multicols}
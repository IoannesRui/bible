% 标题
\chapter*{约纳}

% 右页眉
\rhead{约纳(纳)}

% 正文
\textbf{第一章\quad约纳违命远逃\quad}
\textsuperscript{1}
那时,上主的话传给\uline{阿米泰}的儿子\uline{约纳}说:
\footnote{见\uwave{列}下14:25。}
\textsuperscript{2}
“你起身往\uline{尼尼微}大城去,向他们宣布:他们的邪恶已达到我前。”
\textsuperscript{3}
\uline{约纳}却起身,想躲开上主的面,逃到\uline{塔尔史士}去;遂下到\uline{约培},找到一只要开往\uline{塔尔史士}的船,缴了船费,上了船,同他们往\uline{塔尔史士}去,好躲开上主的面。
\footnote{\uline{约纳}是背道而驰,因为\uline{尼尼微}位于东方,\uline{塔尔史士}则在西方(大约在\uline{西班牙})。关于\uline{约培},见\uline{宗}9:36等。}

\textbf{海上忽起风暴\quad}
\textsuperscript{4}
但是上主却使海上起了大风,海中风浪大作,那只船眼看就要被击破。
\textsuperscript{5}
水手们都惊惶起来,每人呼求自己的神,并将船上的货物抛在海里,为减轻载重。\uline{约纳}却下到船舱,躺下沉睡了。
\textsuperscript{6}
船长走到他跟前,向他说:“怎么,你还在睡觉?起来,呼求你的神吧!你的神也许会眷念我们,使我们不致丧亡。”

\textbf{约纳自认为祸因\quad}
\textsuperscript{7}
他们彼此说:“来,我们抽签,以便知道,我们遇到这场灾祸,是因谁的缘故。”他们便抽签,\uline{约纳}竟抽中了。
\footnote{见\uwave{撒}上10:20、21,14:41、42;\uwave{箴}16:33,18:18等处。}
\textsuperscript{8}
他们向他说:“请告诉我们:我们遇到这场灾祸是因什么缘故?你是干什么的?从哪里来?你是什么地方的人?属于哪一个民族?”
\textsuperscript{9}
他回答他们说:“我是个\uline{希伯来}人,我敬畏的,是那创造海洋和陆地的上天的上主天主。”
\textsuperscript{10}
那些人都很害怕,就对他说:“怎么你作了这事?”人们都知道,他是从天主面前逃跑的,因为他已经告诉了他们。

\textbf{约纳被抛入海风暴平息\quad}
\textsuperscript{11}
他们又向他说:“我们该怎样处置你,才能使海为我们而平静?”因为,海越来越汹涌了。
\textsuperscript{12}
他对他们说:“你们举起我,将我抛在海里,海就会为你们平静下来,因为我知道,这场大风暴,临到你们身上,只是因了我的缘故。”
\textsuperscript{13}
众人虽然尽力摇橹,想回到海岸,却是不能,因为海上风暴,越来越汹涌。
\textsuperscript{14}
所以他们便呼求上主说:“上主!求你不要因这一个人的性命,使我们全都丧亡;不要将无辜者的血,归在我们身上,因为你是上主,就按你的意愿作吧!”
\textsuperscript{15}
他们于是举起\uline{约纳},将他抛在海里,海遂平静。
\textsuperscript{16}
众人都极其敬畏上主,遂向上主献祭,许下誓愿。
\footnote{从14,16两节已看出作者的主题:大公主义。}

\textbf{第二章\quad大鱼吞约纳\quad}
\textsuperscript{1}
上主安排了一条大鱼,
\footnote{古代的学者认为这大鱼是鲸鱼,近代的学者则以为是鲛,即俗称的鲨鱼。}
吞了\uline{约纳};\uline{约纳}在鱼腹里,三天三夜。
\textsuperscript{2}
\uline{约纳}从鱼腹里,祈求上主,他的天主。

\textbf{约纳的感恩诗\quad}
\textsuperscript{3}
他说:“我在患难中,呼求上主,他便应允了我;我从阴府的深处呼求,你便俯听了我的呼声。
\textsuperscript{4}
你将我抛入海心深处,大水包围了我;你的波涛和巨浪漫过了我。
\textsuperscript{5}
我曾说:我虽从你面前被抛弃,但我仍要瞻仰你的圣殿。
\textsuperscript{6}
大水围困我,危及我的性命;深渊包围我,海草缠住我的头。
\textsuperscript{7}
我下沉直到礁底,大地的门闩永为我关闭。上主,我的天主!你却从坑里救出了我的性命。
\textsuperscript{8}
当我奄奄一息时,我记起了上主;我的祈祷达于你前,达于你的圣殿中。
\textsuperscript{9}
敬奉虚无偶像的人,实在是舍弃了慈爱的根源。
\textsuperscript{10}
至于我,我要在颂谢的歌声中,向你献祭,偿还我许的誓愿。救恩属于上主!”
\footnote{这首诗与许多《圣咏》很相类似,但与\uline{约纳}的事迹没有多大关系;由此可知,似非出于\uline{约纳}先知的手笔。}

\textbf{约纳获救\quad}
\textsuperscript{11}
当时,上主命令那鱼,那鱼便将\uline{约纳}吐在陆地上。

\textbf{第三章\quad约纳从命前往尼尼微\quad}
\textsuperscript{1}
上主的话再次传给\uline{约纳}说:
\textsuperscript{2}
“你起身往\uline{尼尼微}大城去,向他们宣告我晓谕你的事。”
\textsuperscript{3}
\uline{约纳}便依从上主的话,起身去了\uline{尼尼微}。\uline{尼尼微}在天主前是一座大城,需要三天的行程。
\footnote{“在天主前是一座大城”一句,是一种\uline{希伯来}文的渲染写法,言其极大之意。“需要三天的行程”,即言:谁若愿意走\uline{尼尼微}城的主要广场和街道,需要三天的时间。}
\textsuperscript{4}
\uline{约纳}开始进城,行了一天的路程,宣布说:“还有四十天,\uline{尼尼微}就要毁灭了。”

\textbf{尼尼微的忏悔与获救\quad}
\textsuperscript{5}
\uline{尼尼微}人便信仰了天主,立即宣布禁食,从大到小,都身披苦衣。
\textsuperscript{6}
当这消息传到\uline{尼尼微}王那里,他便起来,离开自己的宝座,脱去长服,披上苦衣,坐在灰土中;
\textsuperscript{7}
然后命人以君王的谕令和其大臣的名义,在\uline{尼尼微}宣布说:“人、牲畜、牛羊、都不可吃什么;不可牧放,也不可喝水。
\textsuperscript{8}
人和牲畜,都应身披苦衣,人要恳切呼求天主,更要转离自己的邪路,放弃手中的暴行。
\textsuperscript{9}
谁知道天主也许会转意怜悯,收回自己的烈怒,使我们不致灭亡。”
\footnote{从宗教的历史文件,可以证明:公祷、禁食、穿苦衣等,连外教人也有这些风俗。“\uline{尼尼微}人便信仰了天主”一句(5节),又点出了作者的主题。}
\textsuperscript{10}
天主看到他们所行的事,看到他们离开了自己的邪路,遂怜悯他们,不将已宣布的灾祸,降在他们身上。

\textbf{第四章\quad约纳对天主仁慈表示不满\quad}
\textsuperscript{1}
\uline{约纳}因此很不高兴,遂发起怒来。
\textsuperscript{2}
他恳求天主说:“上主,当我还在故乡时,我岂不是已想到这事?所以,我预先要逃往\uline{塔尔史士}去,因为我知道你是慈悲的,宽仁的天主;是缓于发怒,富于慈爱,怜悯而不愿降灾祸的天主。
\footnote{见\uwave{出}34:6、7;\uwave{岳}2:13等处。}
\textsuperscript{3}
上主,现在,求你从我身上收去我的性命,因为,我死了比活着还好!”
\textsuperscript{4}
上主说:“你的忿怒合理吗?”

\textbf{约纳怜惜蓖麻\quad}
\textsuperscript{5}
\uline{约纳}出了城,坐在城东,在那里为自己搭了一个棚,坐在棚荫下,要看看那城究竟要发生什么事。
\textsuperscript{6}
上主天主安排了一株蓖麻,使它长的高过\uline{约纳},为他的头遮荫,消除他的烦恼。\uline{约纳}很喜爱这株蓖麻。
\textsuperscript{7}
但第二天曙光升起时,天主安排了一个虫子咬死蓖麻,蓖麻便枯萎了。
\textsuperscript{8}
当太阳升起时,天主又安排了炎热的东风,太阳射在\uline{约纳}头上,使他无法忍受,遂要求死去,说:“我死了比活着还好!”
\textsuperscript{9}
天主向\uline{约纳}说:“你为这株蓖麻发怒合理吗?”他回答说:“我发怒以至于死,是合理的!”

\textbf{上主训戒约纳\quad}
\textsuperscript{10}
上主说:“你为这株蓖麻,并没有劳过力,也没有使它生长,还怜惜它;它不过是一夜生出,一夜死去的植物;
\textsuperscript{11}
对\uline{尼尼微}这座大城,其中有十二万多不能分辨自己左右手的人,且有许多牲畜,我就不该怜惜他们么?”
\footnote{最后两节点出了本书的主题:天主对万众的博爱。见\uwave{智}11:24、25等处。}